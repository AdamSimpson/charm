\section{Launching Programs with \kw{charmrun}}
\label{charmrun}

When compiling \charmpp{} programs, the charmc linker produces
both an executable file and an utility called {\tt charmrun},
which is used to load the executable onto the parallel machine.

To run a \charmpp{} program named ``pgm'' on four processors, type:
\begin{alltt}
charmrun pgm +p4
\end{alltt}

Execution on platforms which use platform specific launchers, (i.e.,
{\bf aprun}, {\bf ibrun}), can proceed without charmrun, or charmrun can be used
in coordination with those launchers via the {\tt ++mpiexec} (see
\ref{mpiexec} parameter.

Programs built using the network version of \charmpp{} can be run
alone, without charmrun.  This restricts you to using the processors
on the local machine, but it is convenient and often useful for
debugging.  For example, a \charmpp{} program can be run on one
processor in the debugger using:

\begin{alltt}
gdb pgm
\end{alltt}

If the program needs some environment variables
to be set for its execution on compute nodes
(such as library paths), they can be set in
.charmrunrc under home directory. charmrun
will run that shell script before running the executable.

\section[Command Line Options]{Command Line Options}
\label{command line options}
\index{command line options}

A \charmpp{} program accepts the following command line options:
\begin{description}

\item[{\tt +pN}] Run the program with N processors. The default is 1.

\item[{\tt +ss}] Print summary statistics about chare creation.  This option
prints the total number of chare creation requests, and the total number of
chare creation requests processed across all processors.

\item[{\tt +cs}] Print statistics about the number of create chare messages
requested and processed, the number of messages for chares requested and
processed, and the number of messages for branch office chares requested and
processed, on a per processor basis.  Note that the number of messages
created and processed for a particular type of message on a given node
may not be the same, since a message may be processed by a different
processor from the one originating the request.

\item[{\tt user\_options}] Options that are be interpreted by the user
program may be included mixed with the system options.
However, {\tt user\_options} cannot start with +.
The {\tt user\_options} will be passed as arguments to the user program
via the usual {\tt argc/argv} construct to the {\tt main}
entry point of the main chare.
\charmpp{} system options will not appear in {\tt argc/argv}.

\end{description}



\subsection{Additional Network Options}
\label{network command line options}

The following {\tt ++} command line options are available in
the network version:
\begin{description}

\item[{\tt ++local}] Run charm program only on local machines. No
 remote shell invocation is needed in this case. It starts node programs
 right on your local machine. This could be useful if you just want to
 run small program on only one machine, for example, your laptop.


\item[{\tt ++mpiexec}]
\label{mpiexec}
Use the cluster's {\tt mpiexec} job launcher instead of the built in ssh
method.

This will pass {\tt -n \$P} to indicate how many processes to
launch. An executable named something other than {\tt mpiexec} can be
used with the additional argument {\tt ++remote-shell} {\it runmpi},
with `runmpi' replaced by the necessary name.

Use of this option can potentially provide a few benefits:

\begin{itemize}
\item Faster startup compared to the SSH approach charmrun would
  otherwise use
\item No need to generate a nodelist file
\item Multi-node job startup on clusters that do not allow connections
  from the head/login nodes to the compute nodes
\end{itemize}

At present, this option depends on the environment variables for some
common MPI implementations. It supports OpenMPI ({\tt OMPI\_COMM\_WORLD\_RANK} and
{\tt OMPI\_COMM\_WORLD\_SIZE}) and M(VA)PICH ({\tt MPIRUN\_RANK} and {\tt
  MPIRUN\_NPROCS} or {\tt PMI\_RANK} and {\tt PMI\_SIZE}).

\item[{\tt ++debug}] Run each node under gdb in an xterm window, prompting
the user to begin execution.

\item[{\tt ++debug-no-pause}] Run each node under gdb in an xterm window
immediately (i.e. without prompting the user to begin execution).

If using one of the {\tt ++debug} or {\tt ++debug-no-pause} options,
the user must ensure the following:
\begin{enumerate}

\item The {\tt DISPLAY} environment variable points to your terminal.
SSH's X11 forwarding does not work properly with \charmpp{}.

\item The nodes must be authorized to create windows on the host machine (see
man pages for {\tt xhost} and {\tt xauth}).

\item {\tt xterm}, {\tt xdpyinfo},  and {\tt gdb} must be in
the user's path.

\item The path must be set in the {\tt .cshrc} file, not the {\tt .login}
file, because {\tt ssh} does not run the {\tt .login} file.

\end{enumerate}

\item[{\tt ++scalable-start}]   Scalable start, or SMP-aware startup. It is useful for scalable process launch on multi-core systems since it creates only one ssh session per node and spawn all clients from the same ssh session

\item[{\tt ++batch}]            Ssh a set of node programs at a time, avoiding overloading Charmrun pe.  In this strategy, the nodes assigned to a charmrun are divided into sets of fixed size. Charmrun performs ssh to the nodes in the current set, waits for the clients to connect back and then performs ssh on the next set. We call the number of nodes in one ssh set as batch size.

\item[{\tt ++maxssh}] Maximum number of {\tt ssh}'s to run at a
time. For backwards compatibility, this option is also available as {\tt ++maxrsh}.

\item[{\tt ++nodelist}] File containing list of nodes.


\item[{\tt ++ppn}]              number of pes per node

\item[{\tt ++help}]             print help messages

\item[{\tt ++runscript}]        script to run node-program with. The specified run script is invoked with the node program and parameter. For example:

\begin{alltt}
./charmrun +p4 ./pgm 100 2 3 ++runscript ./set\_env\_script
\end{alltt}

In this case, the {\tt set\_env\_script} is invoked on each node before launching {\tt pgm}.

\item[{\tt ++xterm}]            which xterm to use

\item[{\tt ++in-xterm}]         Run each node in an xterm window

\item[{\tt ++display}]          X Display for xterm

\item[{\tt ++debugger}]         which debugger to use

\item[{\tt ++remote-shell}]     which remote shell to use

\item[{\tt ++useip}]            Use IP address provided for charmrun IP

\item[{\tt ++usehostname}]      Send nodes our symbolic hostname instead of IP address



\item[{\tt ++server-auth}]      CCS Authentication file

\item[{\tt ++server-port}]      Port to listen for CCS requests

\item[{\tt ++server}]           Enable client-server (CCS) mode

\item[{\tt ++nodegroup}]        which group of nodes to use

\item[{\tt ++verbose}]          Print diagnostic messages

\item[{\tt ++timeout}]          seconds to wait per host connection

\item[{\tt ++p}]                number of processes to create

\end{description}

\subsection{Multicore Options}

On multicore platforms, operating systems (by default) are free to move
processes and threads among cores to balance load. This however sometimes can
degrade the performance of Charm++ applications due to the extra overhead of
moving processes and threads, especailly when Charm++ applications has already
implemented its own dynamic load balancing.

Charm++ provides the following runtime options to set the processor affinity
automatically so that processes or threads no longer move. When cpu affinity
is supported by an operating system (tested at Charm++ configuration time),
same runtime options can be used for all flavors of Charm++ versions including
network and MPI versions, smp and non-smp versions.

\begin{description}

\item[{\tt +setcpuaffinity}]             set cpu affinity automatically for processes (when Charm++ is based on non-smp versions) or threads (when smp)

\item[{\tt +excludecore <core \#>}]       does not set cpu affinity for the given core number. One can use this option multiple times to provide a list of core numbers to avoid.

\item[{\tt +pemap L[-U[:S[.R]+O]][,...]}] Bind the execution threads to
  the sequence of cores described by the arguments using the operating
  system's CPU affinity functions.

A single number identifies a particular core. Two numbers separated by
a dash identify an inclusive range (\emph{lower bound} and \emph{upper
  bound}). If they are followed by a colon and another number (a
\emph{stride}), that range will be stepped through in increments of
the additional number. Within each stride, a dot followed by a
\emph{run} will indicate how many cores to use from that starting
point. A plus represents the offset to the previous core number. 
Multiple {\tt +offset} flags are supported, e.g., 0-7+8+16 equals 0,8,16,1,9,17.

For example, the sequence {\tt 0-8:2,16,20-24} includes cores 0, 2, 4,
6, 8, 16, 20, 21, 22, 23, 24. On a 4-way quad-core system, if one
wanted to use 3 cores from each socket, one could write this as {\tt
0-15:4.3}. {\tt +ppn 10 +pemap 0-11:6.5+12} equals {\tt +ppn 10 +pemap 
0,12,1,13,2,14,3,15,4,16,6,18,7,19,8,20,9,21,10,22}

\item[{\tt +commap p[,q,...]}] Bind communication threads to the
  listed cores, one per process.

\item[{\tt +nodesize } ] The number of processes within one physical node
that use pxshm for intra-node communication.

\end{description}

\subsection{IO buffering options}
\label{io buffer options}
There may be circumstances where a \charmpp{} application may want to take
or relinquish control of stdout buffer flushing. Most systems default to
giving the \charmpp{} runtime control over stdout but a few default to
giving the application that control. The user can override these system
defaults with the following runtime options:

\begin{description}
\item[{\tt +io\_flush\_user}]     User (application) controls stdout flushing
\item[{\tt +io\_flush\_system}]   The \charmpp{} runtime controls flushing
\end{description}


\section{Nodelist file}

For network of workstations,
the list of machines to run the program can be specified in a file.
Without a nodelist file, \charmpp{} runs the program only on the
local machine.

The format of this file
allows you to define groups of machines, giving each group a name.
Each line of the nodes file is a command.  The most important command
is:

\begin{alltt}
host <hostname> <qualifiers>
\end{alltt}

which specifies a host.  The other commands are qualifiers: they modify
the properties of all hosts that follow them.  The qualifiers are:


\begin{tabbing}
{\tt group <groupname>}~~~\= - subsequent hosts are members of specified group\\
{\tt login <login>  }     \> - subsequent hosts use the specified login\\
{\tt shell <shell>  }     \> - subsequent hosts use the specified remote
shell\\
%{\tt passwd <passwd>}     \> - subsequent hosts use the specified password\\
{\tt setup <cmd>  }       \> - subsequent hosts should execute cmd\\
{\tt pathfix <dir1> <dir2>}         \> - subsequent hosts should replace dir1 with dir2 in the program path\\
{\tt cpus <n>}            \> - subsequent hosts should use N light-weight processes\\
{\tt speed <s>}           \> - subsequent hosts have relative speed rating\\
{\tt ext <extn>}          \> - subsequent hosts should append extn to the pgm name\\
\end{tabbing}

{\bf Note:}
By default, charmrun uses a remote shell ``ssh'' to spawn node processes
on the remote hosts. The {\tt shell} qualifier can be used to override
it with say, ``rsh''. One can set the {\tt CONV\_RSH} environment variable
or use charmrun option {\tt ++remote-shell} to override the default remote
shell for all hosts with unspecified {\tt shell} qualifier.

All qualifiers accept ``*'' as an argument, this resets the modifier to
its default value.  Note that currently, the passwd, cpus, and speed
factors are ignored.  Inline qualifiers are also allowed:

\begin{alltt}
host beauty ++cpus 2 ++shell ssh
\end{alltt}

Except for ``group'', every other qualifier can be inlined, with the
restriction that if the ``setup'' qualifier is inlined, it should be
the last qualifier on the ``host'' or ``group'' statement line.

Here is a simple nodes file:

\begin{alltt}
        group kale-sun ++cpus 1
          host charm.cs.illinois.edu ++shell ssh
          host dp.cs.illinois.edu
          host grace.cs.illinois.edu
          host dagger.cs.illinois.edu
        group kale-sol
          host beauty.cs.illinois.edu ++cpus 2
        group main
          host localhost
\end{alltt}

This defines three groups of machines: group kale-sun, group kale-sol,
and group main.  The ++nodegroup option is used to specify which group
of machines to use.  Note that there is wraparound: if you specify
more nodes than there are hosts in the group, it will reuse
hosts. Thus,

\begin{alltt}
        charmrun pgm ++nodegroup kale-sun +p6
\end{alltt}

uses hosts (charm, dp, grace, dagger, charm, dp) respectively as
nodes (0, 1, 2, 3, 4, 5).

If you don't specify a ++nodegroup, the default is ++nodegroup main.
Thus, if one specifies

\begin{alltt}
        charmrun pgm +p4
\end{alltt}

it will use ``localhost'' four times.  ``localhost'' is a Unix
trick; it always find a name for whatever machine you're on.

Using ``ssh'', the user will have to setup
password-less login to remote hosts using
public key authentication based on a key-pair and adding public keys to
``.ssh/authorized\_keys'' file. See ``ssh'' documentation for more information.
If ``rsh'' is used for remote
login to the compute nodes, the user is required to set up remote login permissions on all nodes
using the ``.rhosts'' file in their home directory.

In a network environment, {\tt charmrun} must
be able to locate the directory of the executable.  If all workstations
share a common file name space this is trivial.  If they don't, {\tt charmrun}
will attempt to find the executable in a directory with the same path
from the {\bf \$HOME} directory.  Pathname resolution is performed as
follows:
\begin{enumerate}
	\item The system computes the absolute path of {\tt pgm}.
	\item If the absolute path starts with the equivalent of {\bf \$HOME}
	or the current working directory, the beginning part of the
        path
	is replaced with the environment variable {\bf \$HOME} or the
	current working directory. However, if {\tt ++pathfix dir1 dir2} is
        specified in the nodes file (see above), the part of
        the path matching {\tt dir1} is replaced with {\tt dir2}.
	\item The system tries to locate this program (with modified
	pathname and appended extension if specified) on all nodes.
\end{enumerate}

