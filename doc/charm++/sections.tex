\charmpp{} supports defining and addressing subsets of the elements of a chare
array.  This entity is called a chare array section (section). These section
elements are addressed via a section proxy.
\charmpp{} also supports array sections which are a subset of array elements in
multiple chare arrays of the same type \ref{cross array section}.
Multicast operations, a broadcast to all members of a section, are directly supported in array section proxy with
an unoptimized direct-sending implementation.
Section reduction is not directly supported by the section proxy. 
However, an optimized section multicast/reduction 
library called ''CkMulticast'' is provided as a separate library module,
which can be plugged in as a delegation of a section proxy for performing
section-based multicasts and reductions using optimized spanning trees. 

\section{Section Creation}

For each chare array "A" declared in a ci file, a section proxy 
of type "CProxySection\_A" is automatically generated in the decl and def 
header files. 
In order to create an array section, the user needs to provide array indexes 
of all the array section members through either explicit enumeration, or an index range expression.
You can create an array section proxy in your application by 
invoking ckNew() function of the CProxySection.
For example, for a 3D array:

\begin{alltt}
  CkVec<CkArrayIndex3D> elems;    // add array indices
  for (int i=0; i<10; i++)
    for (int j=0; j<20; j+=2)
      for (int k=0; k<30; k+=2)
         elems.push_back(CkArrayIndex3D(i, j, k));
  CProxySection_Hello proxy = CProxySection_Hello::ckNew(helloArrayID, elems.getVec(), elems.size());
\end{alltt}

Alternatively, one can do the same thing by providing the index range [lbound:ubound:stride] 
for each dimension:

\begin{alltt}
  CProxySection_Hello proxy = CProxySection_Hello::ckNew(helloArrayID, 0, 9, 1, 0, 19, 2, 0, 29, 2);
\end{alltt}

The above codes create a section proxy that contains array elements of 
[0:9, 0:19:2, 0:29:2].

For user-defined array index other than CkArrayIndex1D to CkArrayIndex6D,
one needs to use the generic array index type: CkArrayIndex.

\begin{alltt}
  CkArrayIndex *elems;    // add array indices
  int numElems;
  CProxySection_Hello proxy = CProxySection_Hello::ckNew(helloArrayID, elems, numElems);
\end{alltt}


\section{Section Multicasts: via CkMulticast}
\label {array_section_multicast}

Once you have the array section proxy, you can broadcast to all the 
section members, or send messages to one member using its offset index within the section, like these:

\begin{alltt}
  CProxySection_Hello proxy;
  proxy.someEntry(...)          // section broadcast
  proxy[0].someEntry(...)       // send to the first element in the section.
\end{alltt}

You can send the section proxy in a message to another processor, and still 
safely invoke the entry functions on the section proxy.

In the broadcast example above, for a section with k members, a total
number of k messages will be sent, one to each member, which is
inefficient when several members are on a same processor, in which
case only one message needs to be sent to that processor and delivered
to all section members on that processor locally. To support this
optimization, a separate library called CkMulticast is provided as a
target for delegation to an optimized implementation. This library
also supports section based reduction.

Note: Use of the bulk array constructor (dimensions given in the CkNew
or CkArrayOptions rather than individual insertion) will allow
construction to race ahead of several other startup procedures, this
creates some limitation on the construction delegation and use of
array section proxies.  For safety, array sections should be
created in a post constructor entry method.

To use the library, you need to compile and install CkMulticast library and 
link your applications against the library using -module:

\begin{alltt}
  # compile and install the CkMulticast library, do this only once
  # assuming a net-linux-x86\_64 build
  cd charm/net-linux-x86\_64/tmp
  make multicast

  # link CkMulticast library using -module when compiling application
  charmc  -o hello hello.o -module CkMulticast -language charm++ 
\end{alltt}

The CkMulticast library is implemented using delegation(Sec. ~\ref{delegation}). 
A special ''CkMulticastMgr'' Chare Group is created as a 
delegation for section multicast/reduction - all the messages sent
by the section proxy will be passed to the local delegation branch.

To use the CkMulticast delegation, one needs to create the CkMulticastMgr Group 
first, and then setup the delegation relationship between the section proxy and 
CkMulticastMgr Group. 
One only needs to create one CkMulticastMgr Group globally.
CkMulticastMgr group can serve all multicast/reduction delegations
for different array sections in an application:

\begin{alltt}
  CProxySection_Hello sectProxy = CProxySection_Hello::ckNew(...);
  CkGroupID mCastGrpId = CProxy_CkMulticastMgr::ckNew();
  CkMulticastMgr *mCastGrp = CProxy_CkMulticastMgr(mCastGrpId).ckLocalBranch();

  sectProxy.ckSectionDelegate(mCastGrp);  // initialize section proxy

  sectProxy.someEntry(...)           //multicast via delegation library as before
\end{alltt}

By default, CkMulticastMgr group builds a spanning tree for multicast/reduction
with a factor of 2 (binary tree).
One can specify a different factor when creating a CkMulticastMgr group.
For example,

\begin{alltt}
  CkGroupID mCastGrpId = CProxy_CkMulticastMgr::ckNew(3);   // factor is 3
\end{alltt}

Note, to use CkMulticast library, all multicast messages must inherit from 
CkMcastBaseMsg, as the following.
Note that CkMcastBaseMsg must come first, this is IMPORTANT for CkMulticast 
library to retrieve section information out of the message.


\begin{alltt}
class HiMsg : public CkMcastBaseMsg, public CMessage_HiMsg
\{
public:
  int *data;
\};
\end{alltt}

Due to this restriction, you must define message explicitly for multicast 
entry functions and no parameter marshalling can be used for multicast with 
CkMulticast library.

\section{Section Reductions} 

Since an array element can be a member of multiple array sections, 
it is necessary to disambiguate between which array
section reduction it is participating in each time it contributes to one. For this purpose, a data structure 
called ''CkSectionInfo'' is created by CkMulticastMgr for each 
array section that the array element belongs to.
During a section reduction, the array element must pass the 
\kw{CkSectionInfo} as a parameter in the \kw{contribute()}. 
The \kw{CkSectionInfo} for a section can be retrieved
from a message in a multicast entry point using function call 
\kw{CkGetSectionInfo}:

\begin{alltt}
  CkSectionInfo cookie;

  void SayHi(HiMsg *msg)
  \{
    CkGetSectionInfo(cookie, msg);     // update section cookie every time
    int data = thisIndex;
    mcastGrp->contribute(sizeof(int), &data, CkReduction::sum_int, cookie);
  \}
\end{alltt}

Note that the cookie cannot be used as a one-time local variable in the 
function, the same cookie is needed for the next contribute. This is 
because the cookie includes some context sensive information (e.g., the 
reduction counter). Subsequent invocations of \kw{CkGetSectionInfo()} only updates part of the data in the cookie, rather than creating a brand new one.

Similar to array reduction, to use section based reduction, a
reduction client CkCallback object must be created. You may pass the
client callback as an additional parameter to \kw{contribute}. If
different contribute calls to the same reduction operation pass
different callbacks, some (unspecified, unreliable) callback will be
chosen for use. 

See the following example:

\begin{alltt}
    CkCallback cb(CkIndex_myArrayType::myReductionEntry(NULL),thisProxy); 
    mcastGrp->contribute(sizeof(int), &data, CkReduction::sum_int, cookie, cb);
\end{alltt}

As in an array reduction, users can use built-in reduction 
types(Section~\ref{builtin_reduction}) or define his/her own reducer functions
(Section~\ref{new_type_reduction}).

\section{Section Collectives when Migration Happens}

Using multicast/reduction, you don't need to worry about array migrations.
When migration happens, array element in the array section can still use 
the \kw{CkSectionInfo} it stored previously for doing reduction. 
Reduction messages will be correctly delivered but may not be as efficient 
until a new multicast spanning tree is rebuilt internally 
in \kw{CkMulticastMgr} library. 
When a new spanning tree is rebuilt, a updated \kw{CkSectionInfo} is 
passed along with a multicast message, 
so it is recommended that 
\kw{CkGetSectionInfo()} function is always called when a multicast 
message arrives (as shown in the above SayHi example).

In case when a multicast root migrates, the library must reconstruct the 
spanning tree to get optimal performance. One will get the following
warning message if not doing so:
"Warning: Multicast not optimized after multicast root migrated."
In current implementation, user needs to initiate the rebuilding process
using \kw{resetSection}.

\begin{alltt}
void Foo::pup(PUP::er & p) {
    // if I am multicast root and it is unpacking
   if (ismcastroot && p.isUnpacking()) {
      CProxySection_Foo   fooProxy;    // proxy for the section
      CkMulticastMgr *mg = CProxy_CkMulticastMgr(mCastGrpId).ckLocalBranch();
      mg->resetSection(fooProxy);
        // you may want to reset reduction client to root
      CkCallback *cb = new CkCallback(...);
   }
}
\end{alltt}

\section{Cross Array Sections}
\label{cross array section}
\experimental{}

Cross array sections contain elements from multiple arrays.
Construction and use of cross array sections is similar to normal
array sections with the following restrictions.  

\begin{itemize}

\item Arrays in a section my all be of the same type.

\item Each array must be enumerated by array ID

\item The elements within each array must be enumerated explicitly

\item No existing modules currently support delegation of cross
  section proxies.  Therefore reductions are not currently supported.

\end{itemize}

Note: cross section logic also works for groups with analogous characteristics.

Given three arrays declared thusly:

\begin{alltt}
	  CkArrayID *aidArr= new CkArrayID[3];
	  CProxy\_multisectiontest\_array1d *Aproxy= new CProxy\_multisectiontest\_array1d[3];
	  for(int i=0;i<3;i++)
	    \{
	      Aproxy[i]=CProxy\_multisectiontest\_array1d::ckNew(masterproxy.ckGetGroupID(),ArraySize);	  
	      aidArr[i]=Aproxy[i].ckGetArrayID();
	    \}
\end{alltt}

One can make a section including the  lower half elements of all three
arrays as follows:

\begin{alltt}
	  int aboundary=ArraySize/2;
	  int afloor=aboundary;
	  int aceiling=ArraySize-1;
	  int asectionSize=aceiling-afloor+1;
	  // cross section lower half of each array
	  CkArrayIndex **aelems= new CkArrayIndex*[3];
	  aelems[0]= new CkArrayIndex[asectionSize];
	  aelems[1]= new CkArrayIndex[asectionSize];
	  aelems[2]= new CkArrayIndex[asectionSize];
	  int *naelems=new int[3];
	  for(int k=0;k<3;k++)
	    \{
	      naelems[k]=asectionSize;
	      for(int i=afloor,j=0;i<=aceiling;i++,j++)
	        aelems[k][j]=CkArrayIndex1D(i);
	    \}
	  CProxySection\_multisectiontest\_array1d arrayLowProxy(3,aidArr,aelems,naelems);
\end{alltt}

The resulting cross section proxy, as in the example \uw{arrayLowProxy},
can then be used for multicasts in the same way as a normal array
section.

Note: For simplicity the example has all arrays and sections of uniform
size.  The size of each array and the number of elements in each array
within a section can all be set independently.



