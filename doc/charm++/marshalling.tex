\subsection{Parameter Marshalling}
\label{marshalling}

\experimental{}	
In \charmpp, \index{chare}chares, \index{group}groups and \index{nodegroup}
nodegroups communicate by invoking each others methods. 
The methods may either take several parameters, described here; 
or take a special message object as described in the next section.
Since parameters get marshalled into a message before being
sent across the network, in this manual we use ``message''
to mean either a literal message object or a set of marshalled
parameters.

For example, a chare could have this entry method declaration in 
the interface ({\tt .ci}) file:
\begin{alltt}
  entry void foo(int i,int k);
\end{alltt}
Then invoking foo(2,3) on the chare proxy will eventually
invoke foo(2,3) on the remote chare.

Since \charmpp\ runs on distributed memory machines, we cannot
pass an array via a pointer in the usual \CC\ way.  Instead,
we must specify the length of the array in the interface file, as:
\begin{alltt}
  entry void bar(int n,double arr[n]);
\end{alltt}
Since \CC\ does not recognize this syntax, the array data
must be passed to the chare proxy as a simple pointer.
The array data will be copied and sent to the
destination processor, where the chare will receive the copy
via a simple pointer again.  The remote copy of the data
will be kept until the remote method returns, when
it will be freed.  
This means any modifications made locally after the call will not be 
seen by the remote chare; and the remote chare's modifications
will be lost after the remote method returns-- \charmpp\ always 
uses call-by-value, even for arrays and structures.  

This also means the data must be copied on the sending 
side, and to be kept must be copied again 
at the receive side.  Especially for large arrays, this 
is less efficient than messages, as described in the next section.

Array parameters and other parameters can be combined in arbitrary ways, as:
\begin{alltt}
  entry void doLine(float data[n],int n);
  entry void doPlane(float data[n*n],int n);
  entry void doSpace(int n,int m,int o,float data[n*m*o]);
  entry void doGeneral(int nd,int dims[nd],float data[product(dims,nd)]);
\end{alltt}
The array length expression between the square brackets can be 
any valid C++ expression, and may depend in any way on any of the passed
parameters, global variables, or global data.  The array length expression
is evaluated exactly once per invocation, on the sending side only.
Thus executing the \kw{doGeneral} method above will invoke the 
(user-defined) \kw{product} function exactly once on the sending
processor.

\subsubsection{Marshalling User-Defined Structures and Classes}

For a ``simple'' struct or class without dynamic allocation,
virtual methods, or subclasses, the marshalling system will
copy the type across machines as flat bytes.  Thus these sorts
of structures will work properly as parameters or in arrays
with no further effort:

\begin{alltt}
//Declarations:
class point3d \{
public:
    double x,y,z;    
\};

typedef struct \{
    int refCount;
    char data[17];
\} refChars;

class date \{
public:
    char month,day;
    int year;
    //...non-virtual manipulation routines...
\};

//In the .ci file:
    entry void pointRefOnDate(point3d &p,refChars r[d.year],date &d);
\end{alltt}

Any user-defined types in the argument list must be declared 
before including the ``.decl.h'' file.
As usual in \CC, it is often dramatically more efficient to pass
a large structure by reference (as shown) than by value.

Complicated structures, such as those with dynamically
allocated data or virtual methods can only be passed as
parameters, but never in arrays.  They also must include a pup
routine and \CC\ operator| (see the PUP chapter for details).
\kw{PUP::able} structures may be passed via a pointer,
but will be copied and then dynamically allocated on the receive side.
For historical reasons, pointer-accessible structures cannot appear 
alone in the parameter list (because they are confused with messages).

Large, complicated structures are most efficiently passed 
via messages; not marshalling.
