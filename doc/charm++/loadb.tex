\subsection{Load Balancing Framework}
\label{lbFramework}

In \charmpp{}, objects(except groups) can migrate from processor to 
processor at run-time. Object migration can potentially improve the 
performance of the parallel program by migrating objects from overloaded 
processors to underloaded ones at run-time. However, we can see it is 
non-trivial to decide which objects to move and where to move them in 
order to achieve load balance in a fashion without the knowledge about the 
application. 

 \charmpp{} implements a generic, measurement-based load balancing framework 
which automatically instruments all \charmpp{} objects, collects computation 
load and communication pattern during execution and stores them into a 
\kw{load balancing database}. \charmpp{} then provides a collection of 
\kw{load balancing strategies} whose job is to decide on a new mapping of 
objects to processors based on information from the database.
This is made possible because we can reasonably assume that objects 
in \charmpp{} application tend to exhibit temporal correlation in 
their computation and communication patterns, i.e. future can be to some 
extent predicted using the historical measurement data, allowing effective 
measurement-based load balancing without application-specific knowledge. 

Here are the two terms often used in \charmpp{} load balancing framework:
\begin{itemize}
\item \kw{Load balancer or strategy} takes the measurement
   data and produces the new mapping of the objects. In \charmpp{}, it is
   implemented as Chare Group inherited from BaseLB.
\item \kw{Load balancing database} provides the interface of almost all
   load balancing calls. On each processor, it stores the load 
   balancing instrument data and coordinates the load balancing manager and 
   balancer. it is implemented as a Chare Group.
\end{itemize}

\subsubsection{Available Load Balancing Strategies}
\label{lbStrategy}

Listed below are the available load balancers and their brief 
descriptions:
\begin{itemize}
\item \kw{RandCentLB}:   Randomly assign objects to processors;
\item \kw{RandRefLB}:    Randomly assign objects to processors, then refine;
\item \kw{RefineLB}:     Move a few objects away from the most overloaded processors;
\item \kw{RecBisectBfLB}:        Recursive partition with Breadth first enumeration;
\item \kw{MetisLB}:      Use Metis(tm) to partitioning object communication graph;
\item \kw{CommLB}:       Greedy with communication;
\item \kw{Comm1LB}:      another variation of CommLB;
\item \kw{HeapCentLB}:   Heap Centralized;
\item \kw{GreedyRefLB}:  Apply greedy, then refine;
\item \kw{NeighborLB}:   The neighborhood load balancer
\end{itemize}

It is up to the users to choose which load balancing algorithm for the 
application at compile or run-time. The compiler and run-time options 
is described at section\ref{lbOption}.

\subsubsection{Load Balancing Chare Arrays}
\label{lbarray}

Load balancing Chare arrays is considered easy because
load balancer is well coupled with Chare array implementation - 
chare array when created automatically registers itself to load balancing
framework, instrument of compute wall/cpu time and communication is 
done automatically and APIs are provided for triggering the load balancing; 
thus load balancing with array is prefered in \charmpp{}, although it is doable 
for load balancing individual Chares for experienced users. (This topic is 
currently beyond the scope of this manual.)

To use the load balancer, you must make your array elements migratable
(see migration section above) and choose a \kw{load balancing strategy} 
(see the section\ref{lbStrategy} for a description
of available load balancing strategies).

Considering how and when a load balancing phase starts, we implemented three
methods to meet different needs of the application.
The three methods to use the load balancer with chare arrays are:
{\bf automatic without Sync}, {\bf automatic with Sync} and {\bf manual mode}.
In case of {\em automatic} load balancing, user don't need to worry about 
how and when to start the load balancing phase. Load balancer will start 
itself when it decides it is ready based on the hints given by users.
In the case of {\bf automatic without Sync}, application simply asserts
that load balancer can happen at any time during the execution, and 
the chare arrays are always ready to migrate\footnote{All the temporary 
data of the array in the application has to be properly packed for 
migration.}. 
In this mode, load balancer works in the background and application 
doesn't need to stop waiting for load balancing to complete. In most cases, 
application doesn't want load balancing occurs arbitarily because it knows 
what is the proper time for load balancing\footnote{for example, the 
application with iteration, load balancing best happens after several 
iterations}. {\bf Automatic with Sync} is designed for this scenario. 
In this mode, user insert {\em AtSync()} calls at certain points when load 
balancing is desired. Load balancing takes control after all the 
local array elements reach the sync point, and start load balancing 
from there. The application would take a break and wait for the load 
balancing complete and regain the control from {\em ResumeFromSync()}.
When user wants to have more control on the time and want to avoid
the synchronization of the {\em AtSync()} method, he may want to use 
the {\bf manual} mode to start load balancer himself.

The detailed APIs of these three methods are described as follows:
\begin{itemize}

\item Method 1: By default, the array elements may be asked to migrate at any time
provided that they are not in the middle of executing an entry method. 
The array element's variable \kw{usesAtSync} being false attributes
to this default behaviour. 

\item  Method 2: Using the AtSync method, elements can be migrated at certain
times.  For the AtSync method, set \kw{usesAtSync} to CmiTrue in your 
array element constructor.  When an element is ready to migrate,
call \kw{ArrayElement::AtSync()}.  
After the first \kw{doneInserting} call (which starts the load balancer)
and once all local elements have reached \kw{AtSync},
the load balancer runs and migrates elements.  Once
all migrations are completed, the load balancer calls the 
virtual function \kw{ArrayElement::ResumeFromSync()} for each of the
array element.  One must ensure that no messages are
sent to an array element between its calls to \kw{AtSync} and
\kw{ResumeFromSync}. \kw{AtSync}/\kw{ResumeFromSync} is currently
implemented as a global barrier.

Note: Setting usesAtSync to true makes it mandatory for the array
element to call \kw{AtSync} and implement the \kw{ResumeFromSync}
 function.

\item Method 3: The load balancer can be programmed to be started
manually. 
 To use this feature, the LBDatabase object
can be got explicitly by making the call \kw{LBDatabase::Object()}
For example, 

\begin{alltt}
LBDatabase * myLbdb = LBDatabase::Object()
\end{alltt}

{\em myLbdb-$>$TurnManualLBOn()} will enable the manual switching on of
the Load Balancer, so that no Load Balancer will be started automatically.
{\em TurnManualLBOn()} should be called as early as possible in the
program. It must be ideally called in the constructor of a static array and
definitely before the doneInserting call for a dynamic array.  This call 
has to be made on all processors. It can be called multiple times on one 
pe, but the last one takes effect.   
The function call {\em myLbdb-$>$StartLB()} will start the load balancer. 
This call can be made on only one pe as the start message is broadcast 
to all processors. The load balancing happens in the background and 
won't stop the chares from functioning. 
{\em myLbdb-$>$TurnManualLBOff()} turns off manual load balancing and switches back to the automatic Load balancing mode (using or not using AtSync). 

\end{itemize}


\subsubsection{Compiler and run-time options to use load balancing module}
\label{lbOption}

Load balancing strategies are implemented as libraries in \charmpp{}. This
allow programmers to easily experiment with different existing strategies 
in the library by simply linking a pool of strateges modules and choose 
one to use at run-time via command line options.

Linking a load balancing module is different with activating it:
\begin{itemize}
\item link a LB module: is to link a Load Balancer module(library) at 
   compile time; You can link against multiple LB libraries as candidates.
\item activate a LB: is to actually ask to create a LB strategy and 
   start it. You can only activate one LB either at compile time or at
   run-time and the LB module must be one of the LB module linked at 
   compile time.
\end{itemize}


Below are the descriptions about the compiler and run-time options:

\begin{enumerate}
\item {\bf compile time options:}

\begin{itemize}
\item {\em -module NeighborLB -module CommLB ...}
\item {\em -module EveryLB -lmetis}
\item {\em -balancer CommLB}
\end{itemize}

note:
you can have multiple -module *LBs options. LB modules are
linked into a program, but they are not activated.
Using -balancer to activate one LB as default at compile time. Having -balancer A implies -module A, so you don't have to write -module A, although it does't hurt.
Using EveryLB if you want to link against all existing LBs. One of the load balancer in EveryLB called MetisLB requires METIS library which is located at: charm/net-linux/tmp/libs/conv-libs/metis. You need to compile it first.

\item {\bf run time option:}

\begin{itemize}
\item {\em +balancer CommLB}
\end{itemize}

note:
+balancer activates the LB. +balancer option works only if you already link LB using -module *LB at compile time. +balancer with a wrong LB name will show you all the available LBs linked at compile time.
When you have used -balancer A at compile time, you don't have to use +balancer A to activate it one more time at runtime. However, you can use +balancer B to override -balancer A compile time option and activate another LB - B.

\item {\bf When there is no LB activated}

When you don't activate any of load balancer at compile or runtime, you still 
can run the program. This is done by creating a NullLB which does nothing 
but resume from sync. The NullLB will take over when there is
no LB activated, and will be silent when another LB is doing its job. 

\end{enumerate}


