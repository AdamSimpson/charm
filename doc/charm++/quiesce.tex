\section{Completion Detection}

Completion detection is a method for automatically detecting completion of a
distributed process within an application. This functionality is helpful when
the exact number of messages expected by individual objects is not known. In
such cases, the process must achieve global consensus as to the number of
messages produced and the number of messages consumed.  Completion is reached
within a distributed process when the participating objects have produced and
consumed an equal number of events globally. The number of global events that
will be produced and consumed does not need to be known, just the number of
producers is required.


The completion detection feature is implemented in \charmpp{} as a
module, and therefore is only included when ``{\tt -module completion}'' is
specified when linking your application.

First, the detector should be constructed. This call would typically
belong in application startup code (it initializes the group that
keeps track of completion):

\begin{alltt}
CProxy_CompletionDetector detector = CProxy_CompletionDetector::ckNew();
\end{alltt}

When it is time to start completion detection, invoke the following method of the
library on {\em all} branches of the completion detection group:

\begin{alltt}
void start_detection(int num_producers, CkCallback start, CkCallback finish, int prio_);
\end{alltt}

The \verb|num_producers| parameter is the number of objects (chares)
that will produce elements. So if every chare array element will produce one
event, then it would be the size of the array.

The \verb|start| callback notifies your program that it is safe to
begin producing and consuming (this state is reached when the module
has finished its internal initialization).

The \verb|finish| callback is invoked when completion has been
detected (all objects participating have produced and consumed an
equal number of elements globally).

The \verb|prio| parameter is the priority with which the completion detector will run. 
This feature is still under development, but it should be set below the
application's priority if possible.

For example, the call

\begin{alltt}
detector.start_detection(10, CkCallback(CkIndex_chare1::start_test(0), thisProxy),
                             CkCallback(CkIndex_chare2::finish_test(0), thisProxy), 0);
\end{alltt}

sets up completion detection for 10 producers. Once initialization is done, the callback 
associated with the {\tt start\_test} method will be invoked. Furthermore, when the system
detects completion, the callback associated with {\tt finish\_test} will be invoked. Finally,
the priority given to the completion detection library is set to 0 in this case.

Once initialization is complete (the ``start'' callback is triggered),
make the following call to the library:

\begin{alltt}
void CompletionDetector::produce(int events_produced)
void CompletionDetector::produce() // 1 by default
\end{alltt}

For example, within the code for a chare array object, you might make the following call:
\begin{alltt}
detector.ckLocalBranch()->produce(4);
\end{alltt}

Once all the ``events'' that this chare is going to produce have been sent out,
make the following call:

\begin{alltt}
void CompletionDetector::done(int producers_done)
void CompletionDetector::done() // 1 by default
\end{alltt}

\begin{alltt}
detector.ckLocalBranch()->done();
\end{alltt}

At the same time, objects can also consume produced elements, using the following calls:

\begin{alltt}
void CompletionDetector::consume(int events_consumed)
void CompletionDetector::consume() // 1 by default
\end{alltt}

\begin{alltt}
detector.ckLocalBranch()->consume();
\end{alltt}

Note that an object may interleave calls to {\tt produce()} and {\tt consume()}, i.e.
it could produce a few elements, consume a few, etc. When it is done producing its elements,
it should call {\tt done()}, after which cannot {\tt produce()} any more elements. However,
it can continue to {\tt consume()} elements even after calling {\tt done()}. 
When the library detects that, globally, the number of produced elements equals
the number of consumed elements, and all producers have finished producing
(i.e. called {\tt done()}), it will invoke the \verb|finish| callback.
Thereafter, \verb|start_detection| can be called again to restart the process.

\section{Quiescence Detection}
\label{sec:qd}

In \charmpp, \index{quiescence}quiescence is defined as the state in which no
processor is executing an entry point, no messages are awaiting processing, and
there are no messages in-flight.  \charmpp\ provides two facilities for
detecting quiescence: \kw{CkStartQD} and \kw{CkWaitQD}.  \kw{CkStartQD}
registers with the system a callback that is to be invoked the next time
\index{quiescence}quiescence is detected. Note that if immediate messages are
used, QD cannot be used.  \kw{CkStartQD} has two variants
which expect the following arguments: 

\begin{enumerate}
\item 
A \uw{CkCallback} object. The syntax of this call looks like:
\begin{alltt}
  CkStartQD(const CkCallback& cb);
\end{alltt}

Upon quiescence detection, the specified callback is called with no parameters. Note that
using this variant, you could have your program terminate after quiescence is detected, by
supplying the above method with a CkExit callback (\S~\ref{sec:callbacks/creating}).

\item An index corresponding to the entry function that is to be called,
and a handle to the chare on which that entry function should be called.  The
syntax of this call looks like this:

\begin{alltt}
 CkStartQD(int Index,const CkChareID* chareID);
\end{alltt}

To retrieve the corresponding index of a particular \index{entry method}entry
method, you must use a static method contained within the
(\kw{charmc}-generated) \uw{CkIndex} object corresponding to the
\index{chare}chare containing that entry method.  The syntax of this call is as
follows:

\begin{alltt}
\kw{myIdx}=CkIndex_\uw{ChareClass}::\uw{entryMethod}(\uw{parameters});
\end{alltt}

where \uw{ChareClass} is the \CC{} class of the chare containing
the desired entry method, \uw{entryMethod} is the name of that entry method,
and \uw{parameters} are the parameters taken by the method.
These parameters are only used to resolve the proper \uw{entryMethod};
they are otherwise ignored.

\end{enumerate}

\kw{CkWaitQD}, by contrast, does not register a callback.  Rather,
\kw{CkWaitQD} {\em blocks} and does not return until \index{quiescence}quiescence is
detected.  It takes no parameters and returns no value.  A call to
\kw{CkWaitQD} simply looks like this: 

\begin{alltt}
  CkWaitQD();
\end{alltt}

Note that \kw{CkWaitQD} should only be called from a threaded
\index{entry method}entry method because a call to \kw{CkWaitQD} suspends the
current thread of execution ({\em cf.} \S~\ref{threaded}). 
%If it were called from outside a threaded entry
%method it would suspend the main thread of execution of the processor from
%which \kw{CkWaitQD} was called, and the entire program would come to a grinding
%halt on that processor.

%\function{void CkExitAfterQuiescence()} \index{CkExitAfterQuiescence}
%\desc{This call informs the Charm RTS that computation on all processors
%should terminate as soon as the machine becomes completely idle--that is,
%after all messages and entry methods are finished.  This is the state of 
%quiescence, as described further in Section~\ref{sec:qd}.
%This routine returns immediately.}

