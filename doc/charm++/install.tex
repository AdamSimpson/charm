\charmpp{} can be installed either from the source code or using a precompiled
binary package. Building from the source code provides more flexibility, since one 
can choose the options as desired. However, a precompiled binary may be slightly
easier to get running.
 
\section{Downloading \charmpp{}}

\charmpp{} can be downloaded using one of the following methods:

\begin{itemize}
\item From \charmpp{} website -- The current stable version (source code and
binaries) can be downloaded from our website at {\em http://charm.cs.illinois.edu/software}.
\item From source archive -- The latest development version of \charmpp{} can be downloaded
from our source archive using {\em git clone http://charm.cs.illinois.edu/gerrit/charm}.
\end{itemize}

If you download the source code from the website, you will have to unpack it 
using a tool capable of extracting gzip'd tar files, such as tar (on Unix) 
or WinZIP (under Windows).  \charmpp{} will be extracted to a directory 
called ``charm''. 

\section{Installation}

A typical prototype command for building \charmpp{} from the source code is:
\vspace{5pt}\\
{\bf ./build $<$TARGET$>$ $<$TARGET ARCHITECTURE$>$ [OPTIONS]} where,

\begin{description}
\item [TARGET] is the framework one wants to build such as {\em charm++} or {\em
AMPI}.
\item [TARGET ARCHITECTURE] is the machine architecture one wants to build for
such as {\em netlrts-linux-x86\_64}, {\em pamilrts-bluegeneq} etc.
\item [OPTIONS] are additional options to the build process, e.g. {\em smp} is
used to build a shared memory version, {\em -j8} is given to build in parallel
etc.
\end {description}

In Table~\ref{tab:buildlist}, a list of build commands is provided for some of the commonly 
used systems. Note that, in general, options such as {\em smp},
\verb|--with-production|, compiler specifiers etc can be used with all targets.
It is advisable to build with \verb|--with-production| to obtain the best
performance.  If one desires to perform trace collection (for Projections),
\verb|--enable-tracing --enable-tracing-commthread| should also be passed to the
build command.

Details on all the available alternatives for each of the above mentioned
parameters can be found by invoking \verb|./build --help|. One can also go through the
build process in an interactive manner. Run \verb|./build|, and it will be followed by
a few queries to select appropriate choices for the build one wants to perform.


\begin{table}[ht]
\begin{tabular}{|p{6cm}|p{9cm}|}
\hline
Net with 32 bit Linux & \verb|./build charm++ netlrts-linux --with-production -j8|
\\\hline
Multicore (single node, shared memory) 64 bit Linux & \verb|./build charm++ multicore-linux-x86_64 --with-production -j8|
\\\hline
Net with 64 bit Linux & \verb|./build charm++ netlrts-linux-x86_64 --with-production -j8|
\\\hline
Net with 64 bit Linux (intel compilers) & \verb|./build charm++ netlrts-linux-x86_64 icc --with-production -j8|
\\\hline
Net with 64 bit Linux (shared memory) & \verb|./build charm++ netlrts-linux-x86_64 smp --with-production -j8|
\\\hline
Net with 64 bit Linux (checkpoint restart based fault tolerance) & \verb|./build charm++ netlrts-linux-x86_64 syncft --with-production -j8|
\\\hline
MPI with 64 bit Linux & \verb|./build charm++ mpi-linux-x86_64 --with-production -j8|
\\\hline
MPI with 64 bit Linux (shared memory) & \verb|./build charm++ mpi-linux-x86_64 smp --with-production -j8|
\\\hline
MPI with 64 bit Linux (mpicxx wrappers) & \verb|./build charm++ mpi-linux-x86_64 mpicxx --with-production -j8|
\\\hline
IBVERBS with 64 bit Linux & \verb|./build charm++ verbs-linux-x86_64 --with-production -j8|
\\\hline
Net with 64 bit Windows & \verb|./build charm++ netlrts-win-x86_64 --with-production -j8|
\\\hline
MPI with 64 bit Windows & \verb|./build charm++ mpi-win-x86_64 --with-production -j8|
\\\hline
Net with 64 bit Mac & \verb|./build charm++ netlrts-darwin-x86_64 --with-production -j8|
\\\hline
Blue Gene/Q & \verb|./build charm++ pami-bluegeneq xlc --with-production -j8|
\\\hline
Blue Gene/Q & \verb|./build charm++ pamilrts-bluegeneq xlc --with-production -j8|
\\\hline
Cray XE6 & \verb|./build charm++ gni-crayxe --with-production -j8|
\\\hline
Cray XK7 & \verb|./build charm++ gni-crayxe-cuda --with-production -j8|
\\\hline
Cray XC30 & \verb|./build charm++ gni-crayxc --with-production -j8|
\\\hline
\end{tabular}
\caption{Build command for some common cases}
\label{tab:buildlist}
\end{table}

As mentioned earlier, one can also build \charmpp{} using the precompiled binary
in a manner similar to what is used for installing any common software.


The main directories in a \charmpp{} installation are:

\begin{description}
\item[\kw{charm/bin}]
Executables, such as charmc and charmrun,
used by \charmpp{}.

\item[\kw{charm/doc}]
Documentation for \charmpp{}, such as this
document.  Distributed as LaTeX source code; HTML and PDF versions
can be built or downloaded from our web site.

\item[\kw{charm/include}]
The \charmpp{} C++ and Fortran user include files (.h).

\item[\kw{charm/lib}]
The libraries (.a) that comprise \charmpp{}.

\item[\kw{charm/examples}]
Example \charmpp{} programs.

\item[\kw{charm/src}]
Source code for \charmpp{} itself.

\item[\kw{charm/tmp}]
Directory where \charmpp{} is built.

%\item[\kw{charm/tools}]
%Visualization tools for \charmpp{} programs.

\item[\kw{charm/tests}]
Test \charmpp{} programs used by autobuild.

\end{description}

\section{Reducing disk usage}

The charm directory contains a collection of example-programs and
test-programs.  These may be deleted with no other effects. You may 
also {\tt strip} all the binaries in {\tt charm/bin}.





