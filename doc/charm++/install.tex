\charmpp{} can be installed manually from the source code or using a precompiled
binary package. We also provide a source code package for the
\emph{Spack} package manager.
Building from the source code provides more flexibility, since one
can choose the options as desired. However, a precompiled binary may be slightly
easier to get running.


\section{Manual installation}

\subsection{Downloading \charmpp{}}

\charmpp{} can be downloaded using one of the following methods:

\begin{itemize}
\item From \charmpp{} website -- The current stable version (source code and
binaries) can be downloaded from our website at {\em http://charm.cs.illinois.edu/software}.
\item From source archive -- The latest development version of \charmpp{} can be downloaded
from our source archive using {\em git clone http://charm.cs.illinois.edu/gerrit/charm}.
\end{itemize}

If you download the source code from the website, you will have to unpack it
using a tool capable of extracting gzip'd tar files, such as tar (on Unix)
or WinZIP (under Windows).  \charmpp{} will be extracted to a directory
called ``charm''.

\subsection{Installation}
\label{sec:manualinstall}

A typical prototype command for building \charmpp{} from the source code is:
\vspace{5pt}\\
{\bf ./build $<$TARGET$>$ $<$TARGET ARCHITECTURE$>$ [OPTIONS]} where,

\begin{description}
\item [TARGET] is the framework one wants to build such as {\em charm++} or {\em
AMPI}.
\item [TARGET ARCHITECTURE] is the machine architecture one wants to build for
such as {\em netlrts-linux-x86\_64}, {\em pamilrts-bluegeneq} etc.
\item [OPTIONS] are additional options to the build process, e.g. {\em smp} is
used to build a shared memory version, {\em -j8} is given to build in parallel
etc.
\end {description}

In Table~\ref{tab:buildlist}, a list of build commands is provided for some of the commonly
used systems. Note that, in general, options such as {\em smp},
\verb|--with-production|, compiler specifiers etc can be used with all targets.
It is advisable to build with \verb|--with-production| to obtain the best
performance.  If one desires to perform trace collection (for Projections),
\verb|--enable-tracing --enable-tracing-commthread| should also be passed to the
build command.

Details on all the available alternatives for each of the above mentioned
parameters can be found by invoking \verb|./build --help|. One can also go through the
build process in an interactive manner. Run \verb|./build|, and it will be followed by
a few queries to select appropriate choices for the build one wants to perform.


\begin{table}[ht]
\begin{tabular}{|p{6cm}|p{9cm}|}
\hline
Net with 32 bit Linux & \verb|./build charm++ netlrts-linux --with-production -j8|
\\\hline
Multicore (single node, shared memory) 64 bit Linux & \verb|./build charm++ multicore-linux-x86_64 --with-production -j8|
\\\hline
Net with 64 bit Linux & \verb|./build charm++ netlrts-linux-x86_64 --with-production -j8|
\\\hline
Net with 64 bit Linux (intel compilers) & \verb|./build charm++ netlrts-linux-x86_64 icc --with-production -j8|
\\\hline
Net with 64 bit Linux (shared memory) & \verb|./build charm++ netlrts-linux-x86_64 smp --with-production -j8|
\\\hline
Net with 64 bit Linux (checkpoint restart based fault tolerance) & \verb|./build charm++ netlrts-linux-x86_64 syncft --with-production -j8|
\\\hline
MPI with 64 bit Linux & \verb|./build charm++ mpi-linux-x86_64 --with-production -j8|
\\\hline
MPI with 64 bit Linux (shared memory) & \verb|./build charm++ mpi-linux-x86_64 smp --with-production -j8|
\\\hline
MPI with 64 bit Linux (mpicxx wrappers) & \verb|./build charm++ mpi-linux-x86_64 mpicxx --with-production -j8|
\\\hline
IBVERBS with 64 bit Linux & \verb|./build charm++ verbs-linux-x86_64 --with-production -j8|
\\\hline
OFI with 64 bit Linux & \verb|./build charm++ ofi-linux-x86_64 --with-production -j8|
\\\hline
Net with 64 bit Windows & \verb|./build charm++ netlrts-win-x86_64 --with-production -j8|
\\\hline
MPI with 64 bit Windows & \verb|./build charm++ mpi-win-x86_64 --with-production -j8|
\\\hline
Net with 64 bit Mac & \verb|./build charm++ netlrts-darwin-x86_64 --with-production -j8|
\\\hline
Blue Gene/Q (bgclang compilers) & \verb|./build charm++ pami-bluegeneq --with-production -j8|
\\\hline
Blue Gene/Q (bgclang compilers) & \verb|./build charm++ pamilrts-bluegeneq --with-production -j8|
\\\hline
Cray XE6 & \verb|./build charm++ gni-crayxe --with-production -j8|
\\\hline
Cray XK7 & \verb|./build charm++ gni-crayxe-cuda --with-production -j8|
\\\hline
Cray XC40 & \verb|./build charm++ gni-crayxc --with-production -j8|
\\\hline
\end{tabular}
\caption{Build command for some common cases}
\label{tab:buildlist}
\end{table}

As mentioned earlier, one can also build \charmpp{} using the precompiled binary
in a manner similar to what is used for installing any common software.

When a Charm++ build folder has already been generated, it is possible to
perform incremental rebuilds by invoking \verb|make| from the \verb|tmp| folder
inside it. For example, with a {\em netlrts-linux-x86\_64} build, the path
would be \verb|netlrts-linux-x86_64/tmp|. On Linux and macOS, the tmp symlink
in the top-level charm directory also points to the tmp directory of the most
recent build.

Alternatively, CMake can be used for configuring and building Charm++. You can
use \verb|cmake-gui| or \verb|ccmake| for an overview of available options.
Note that some are only effective when passed with \verb|-D| from the
command line while configuring from a blank slate. To build with all defaults,
\verb|cmake .| is sufficient, though invoking CMake from a separate location
(ex: \verb|mkdir mybuild && cd mybuild && cmake ../charm|) is recommended.

\section{Installation through the Spack package manager}

\charmpp{} can also be installed through the Spack package manager (\url{https://spack.io/}).

A basic command to install \charmpp{} through Spack is the following:

\begin{alltt}
	$ spack install charm
\end{alltt}

By default, the mpi network backend with smp support is built. You can specify
other backends by providing the \texttt{backend} parameter to spack. It is
also possible to specify other options, as listed in
Section~\ref{sec:manualinstall}, by adding them to the Spack command prepended
by a '\texttt{+}'. For example, to build the netlrts version of \charmpp{} with
the integrated OpenMP support, you can use the following command:

\begin{alltt}
	$ spack install charm backend=netlrts +omp
\end{alltt}

To disable an option, prepend it with a '\texttt{\textasciitilde}'. For example,
to build \charmpp{} with smp support disabled, you can use the following command:

\begin{alltt}
	$ spack install charm ~smp
\end{alltt}

By default, the newest released version of \charmpp{} is built. You can select
another version with the '\texttt{@}' option (for example, \texttt{@6.8.1}). To
build the current git version of \charmpp{}, specify the \texttt{@develop}
version:

\begin{alltt}
	$ spack install charm@develop
\end{alltt}

\section{\charmpp{} installation directories}

The main directories in a \charmpp{} installation are:

\begin{description}
\item[\kw{charm/bin}]
Executables, such as charmc and charmrun,
used by \charmpp{}.

\item[\kw{charm/doc}]
Documentation for \charmpp{}, such as this
document.  Distributed as LaTeX source code; HTML and PDF versions
can be built or downloaded from our web site.

\item[\kw{charm/include}]
The \charmpp{} C++ and Fortran user include files (.h).

\item[\kw{charm/lib}]
The static libraries (.a) that comprise \charmpp{}.

\item[\kw{charm/lib\_so}]
The shared libraries (.so/.dylib) that comprise \charmpp{},
if \charmpp{} is compiled with the \texttt{--build-shared} option.

\item[\kw{charm/examples}]
Example \charmpp{} programs.

\item[\kw{charm/src}]
Source code for \charmpp{} itself.

\item[\kw{charm/tmp}]
Directory where \charmpp{} is built.

\item[\kw{charm/tests}]
Test \charmpp{} programs used by autobuild.

\end{description}

\section{Reducing disk usage}

The charm directory contains a collection of example-programs and
test-programs.  These may be deleted with no other effects. You may
also {\tt strip} all the binaries in {\tt charm/bin}.
