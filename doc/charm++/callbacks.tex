\subsection{Callbacks}

\label{callbacks}

A callback is a generic way to transfer control back to a client
after a \charmpp{} library has finished.  For example, after finishing a reduction,
you might want the results passed to some chare's entry method.
To do this, you create an object of type \kw{CkCallback} with
the chare's \kw{CkChareID} and entry method index, then pass the
callback object to the reduction library.


\subsubsection{Client Interface}

\index{CkCallback}

You can create a \kw{CkCallback} object in a number of ways,
depending on what you want to have happen when the callback is
finally invoked.  The callback will be invoked with a \charmpp{}
message; but the message type will depend on the \charmpp{} library that 
actually invokes the callback.  Check the library documentation
to see what kind of message the library will send to your callback.
In any case, you are required to free the message passed to you via
the callback.

The callbacks that go to chares require an ``entry method index'',
an integer that identifies which entry method will be called.
You can get an entry method index using the syntax:

\begin{alltt}
\kw{myIdx}=CkIndex_\uw{ChareName}::\uw{EntryMethod}(\uw{parameters});
\end{alltt}

Here, \uw{ChareName} is the name of the chare (group, or array) containing
the desired entry method, \uw{EntryMethod} is the name of that entry method,
and \uw{parameters} are the parameters taken by the method.
These parameters are only used to resolve the proper \uw{EntryMethod};
they are otherwise ignored.  An entry method index is the \charmpp{}
version of a function pointer.


There are a number of ways to build callbacks, depending on what you
want to have happen when the callback is invoked:

\begin{enumerate}
\item \kw{CkCallback(CkCallbackFn fn,void *param)} When invoked, the
callback will pass \uw{param} and the result message to the given C function,
which should have a prototype like:

\begin{alltt}
void \uw{myCallbackFn}(void *param,void *message)
\end{alltt}

This function will be called on the processor where the callback was created,
so \uw{param} is allowed to point to heap-allocated data.  Of course, you
are required to free any storage referenced by \uw{param}.

\item \kw{CkCallback(CkCallback::ignore)} When invoked, the callback
will do nothing.  This can be useful if the \charmpp{} library requires a callback,
but you don't care when it finishes, or will find out some other way.

\item \kw{CkCallback(CkCallback::ckExit)} When invoked, the callback
will call CkExit(), ending the Charm++ program.

\item \kw{CkCallback(int ep,const CkChareID \&id)} When invoked, the 
callback will send its message to the given entry method of the given
Chare.  Note that a chare proxy will also work in place of a chare id:

\begin{alltt}
	CkCallback myCB(CkIndex_myChare::myEntry(NULL),myChareProxy);
\end{alltt}

\item \kw{CkCallback(int ep,const CkArrayID \&id)} 
When invoked,
the callback will broadcast its message to the given entry method
of the given array.  An array proxy will work in the place of an array id.

\item \kw{CkCallback(int ep,const CkArrayIndex \&idx,const CkArrayID \&id)}
When invoked,
the callback will send its message to the given entry method
of the given array element. 

\item \kw{CkCallback(int ep,const CkGroupID \&id)} 
When invoked,
the callback will broadcast its message to the given entry method
of the given group.

\item \kw{CkCallback(int ep,int onPE,const CkGroupID \&id)}
When invoked,
the callback will send its message to the given entry method
of the given group member. 

\end{enumerate}

One final type of callback, a \kw{CkCallback(CkCallback::resumeThread)}, 
can only be used from within threaded entry methods.  This type of callback
is typically hidden within a thread-capable library, so is discussed further
in the library section.


\subsubsection{Library Interface}

\label{libraryInterface}

Here, a ``library'' is simply any \charmpp{} code which can be called from several
different places.  From the point of view of a library, a \kw{CkCallback}
is a destination for the library's result.  \kw{CkCallback} objects can
be freely copied, marshalled, or even sent in messages.

Postponing the discussion on threads for a moment, the only thing you can do 
with a CkCallback is to move it around or send a message to it:

\begin{alltt}
//Main library entry point, called by asynchronous users:
void myLibrary(...library parameters...,const CkCallback \&cb) 
\{
  ..start some parallel computation, dragging cb along...
\}

//Internal library routine, called when computation is done
void myLibraryDone(...parameters...,const CkCallback \&cb)
\{
  ...prepare a return message...
  cb.send(msg);
\}
\end{alltt}

A \kw{CkCallback} will accept any message type, or even NULL.  The
message is immediately sent to the user's client function or entry point,
so you {\em do} need to document the type of message you will send to the 
callback so the user knows what to expect.

As an alternative to ``send'', the callback can be used in a {\em contribute}
collective operation. This will internally invoke the ``send'' method on the
callback when the contribute operation has finished.

Threaded entry methods can be suspended and resumed through the 
%are a bit more complicated as they must be suspended while waiting
%for the invoked operation to finish. 
{\em CkCallbackResumeThread} class. {\em CkCallbackResumeThread}
is derived from {\em CkCallback} and has specific functionality for threads.
This class automatically suspends the thread when the destructor of the callback is called. 
A suspended threaded client will resume when the ``send'' method is
invoked on the associated callback.
It can be used in situations when the return value is not needed, and only the
synchronization is important. For example:

\begin{alltt}
// Call the "doWork" method and wait until it has completed
void mainControlFlow() \{
  ...perform some work...
  // call a library
  doWork(...,CkCallbackResumeThread());
  // or send a broadcast to a chare collection
  myProxy.doWork(...,CkCallbackResumeThread());
  // The thread is suspended until doWork calls 'send' on the callback
  
  ...some more work...
\}
\end{alltt}

Alternatively, if doWork returns a value of interest, this can be retrieved by
passing a pointer to {\em CkCallbackResumeThread}. This pointer will be modified
by {\em CkCallbackResumeThread} to point to the incoming message. Notice that
the input pointer has to be cast to {\em (void*\&)}:

\begin{alltt}
// Call the "doWork" method and wait until it has completed
void mainControlFlow() \{
  ...perform some work...
  MyMessage *mymsg;
  myProxy.doWork(...,CkCallbackResumeThread((void*\&)mymsg));
  // The thread is suspended until doWork calls send on the callback

  ...some more work using "mymsg"...
\}
\end{alltt}

Notice that the instance of {\em CkCallbackResumeThread} is constructed
as an anonymous parameter to the ``doWork'' call. This insures that the callback
is destroyed as soon as the function returns, thereby suspending the thread.

It is also possible to allocate a {\em CkCallbackResumeThread} on the heap or on
the stack. We suggest that programmers avoid such usage, and favor the anonymous instance construction
shown above. For completeness, we still present the code for heap and stack
allocation of CkCallbackResumeThread callbacks below.

For heap allocation, the user must explicitly ``delete'' the callback in order to
suspend the thread.

\begin{alltt}
// Call the "doWork" method and wait until it has completed
void mainControlFlow() \{
  ...perform some work...
  CkCallbackResumeThread cb = new CkCallbackResumeThread();
  myProxy.doWork(...,cb);
  ...do not suspend yet, continue some more work...
  delete cb;
  // The thread suspends now

  ...some more work after the thread resumes...
\}
\end{alltt}

For a callback that is allocated on the stack, its destructor will be called only
when the callback variable goes out of scope. In this
situation, the function ``thread\_delay'' can be invoked on the callback to
force the thread to suspend. This also works for heap allocated callbacks.

\begin{alltt}
// Call the "doWork" method and wait until it has completed
void mainControlFlow() \{
  ...perform some work...
  CkCallbackResumeThread cb;
  myProxy.doWork(...,cb);
  ...do not suspend yet, continue some more work...
  cb.thread\_delay();
  // The thread suspends now

  ...some more work after the thread is resumed...
\}
\end{alltt}

{\em N.B.}: a {\em CkCallbackResumeThread} can be used to suspend a thread
only once.

{\em Deprecated usage}: in the past, ``thread\_delay'' was used to retrieve the
incoming message from the callback. While that is still allowed for backward
compatibility, its usage is deprecated. The old usage is subject to memory
leaks and dangling pointers.

