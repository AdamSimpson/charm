\charmpp has been augmented with support for partitioning. The key idea is to
divide the allocated set of nodes into subsets that run independent \charmpp
instances. These \charmpp instances (called partitions from now on) have a
unique identifier, can be programmed to do different tasks, and can interact
with each other. Addition of the partitioning scheme does not affect the
existing code base or codes that do not want to use partitioning. Some of the use
cases of partitioning are replicated NAMD, replica based fault tolerance,
studying mapping performance etc. In some aspects, partitioning is similar to 
disjoint communicator creation in MPI. 

\section{Overview}
\charmpp stack has three components - Charm++, Converse and machine layer. In
general, machine layer handles the exchange of messages among nodes, and
interacts with the next layer in the stack - Converse. Converse is responsible
for scheduling of tasks (including user code) and is used by Charm++ to execute
the user application. Charm++ is the top-most level in which the applications
are written. During partitioning, Charm++ and machine layers are unaware of
the partitioning. Charm++ assumes its partition to be the entire world, whereas
machine layer considers the whole set of allocated nodes as one partition.
During start up, converse divides the allocated set of nodes into partitions, in
each of which Charm++ instances are run. It performs the necessary translations
as interactions happen with Charm++ and the machine layer.  The partitions can 
communicate with each other using the Converse API described later.

\section{Ranking}
\charmpp stack assigns a rank to every processing element (PE). In the non-partitioned
version, a rank assigned to a PE is same at all three layers of \charmpp
stack. This rank also (generally) coincides with the rank provided to processors/cores 
by the underlying job scheduler. The importance of these ranks derive from the 
fact that they are used for multiple purposes. Partitioning leads to segregation of the 
notion of ranks at different levels of Charm++ stack. What used to be the PE is now a 
local rank within a partition running a Charm++ instance. Existing methods such as CkMyPe,
CkMyNode, CmiMyPe, CmiMyNode etc continue to provide these local ranks. Hence, existing 
codes do not require any change as long as inter-partition interaction is not required. 

On the other hand, machine layer is provided with the target ranks that
are globally unique. These ranks can be obtained using functions with Global
suffix such as CmiNumNodesGlobal, CmiMyNodeGlobal, CmiMyPeGlobal etc.

Converse, which operates at a layer between Charm++ and machine layer,
performs the required transitions. It maintains relevant information for any
conversion. Information related to partitions can be obtained using Converse
level functions such as CmiMyPartition, CmiPartitionSize, CmiNumPartitions
etc. If required, one can also obtain the mapping of a local rank to a global
rank using functions such as CmiGetPeGlobal, CmiGetNodeGlobal,  CmiGetPeLocal,
CmiGetNodeLocal etc.       


\section{Startup and Partitioning}

Any user who wants to run multiple partitions in one single job can provide a
run time parameter \emph{+partitions <part\_numbers>} or \emph{+replicas
<replica\_number>}.
Converse startup will pick this up, and initialize the necessary data
structure for such a division, and other conversions.  Currently, we have
implemented a simple blocked-division scheme that divides the available
resources among partitions. The entire partitioning step, in this sense, is
transparent to the user. In future, we will provide hooks for each of these
key functions – initial partitioning, rank conversion, and global coordination
– that will enable the users to fully customize replica creation as per their
requirements.

\section{Redirecting output from individual partitions}
Output to standard output (stdout) from various partitions can be directed 
to separate files by passing the target path as a command line option. The run
time parameter \emph{+stdout <path>} is to be used for this purpose. The
\emph{<path>} may
contain the C format specifier \emph{\%d}, which will be replaced by the
partition number. In case, \emph{\%d} is specified multiple times, only the first
three instances from the left will be replaced by the partition number (and rest
will contain garbage values). In case, a format specifier is not specified, 
the partition number will be appended as a suffix to the specified path. Example usage:

\begin{itemize}
\item \emph{+stdout out/\%d/log} will write to \emph{out/0/log, out/1/log,
out/2/log,} $\cdots$.
\item \emph{+stdout log} will write to \emph{log.0, log.1, log.2,} $\cdots$.
\item \emph{+stdout out/\%d/log\%d} will write to \emph{out/0/log0, out/1/log1,
out/2/log2,} $\cdots$.
\end{itemize}

\section{Inter-partition Communication}

A new API was added to Converse to enable sending messages from one replica to
another. Currently, following functions are available for the same
\begin{itemize}
\item CmiInterSyncSend(local\_rank, partition, size, message)        
\item CmiInterSyncSendAndFree(local\_rank, partition, size, message)
\item CmiInterSyncNodeSend(local\_node, partition, size, message)        
\item CmiInterSyncNodeSendAndFree(local\_node, partition, size, message)
\end{itemize}

Users who have coded in Converse will find these functions to be very similar
to basic Converse functions for send – CmiSyncSend and CmiSyncSendAndFree.
Given the local rank of a PE and the partition it belongs to, these two
functions will pass the message to the machine layer. CmiInterSyncSend does
not return till “message” is ready for reuse. CmiInterSyncSendAndFree passes
the ownership of “message” to Charm++ RTS, which will free the message when
the send is complete. Each converse message contains a message header, which
makes those messages active – they contain information about their handlers.
These handlers can be registered using existing API in Charm++ -
CmiRegisterHandler.  CmiInterNodeSend and CmiInterNodeSendAndFree are
counterparts to these functions that allow sending of a message to a node (in
    SMP mode).


