\section{Zero Copy Message Send API}

\label{nocopyapi}
Apart from using messages, \charmpp{} also provides a zero copy message send
API to avoid copies for entry method invocations which use parameter marshalling
instead of messages. This makes use of onesided communication by using the
underlying Remote Direct Memory Access (RDMA) enabled network.
For large arrays (few 100 KBs or more), the cost of copying during marshalling
the message can be quite high. Using this API can help not only save
the expensive copy operation but also reduce the application's memory footprint
by avoiding data duplication. Saving these costs for large arrays proves
to be a significant optimization in achieving faster message send times.
On the other hand, using the zero copy message send API for small arrays can lead
to a drop in performance due to the overhead associated with onesided communication.

\vspace{0.1in}
\noindent
To send an array using the zero copy message send API, specify the array parameter
in the .ci file with the rdma specifier.

\begin{alltt}
entry void foo (int size, rdma int arr[size]);
\end{alltt}

While calling the entry method from the .C file, wrap the array i.e the
pointer in an rdma wrapper.

\begin{alltt}
arrayProxy[0].foo(500000, rdma(arrPtr));
\end{alltt}

Until the RDMA operation is completed, it is not safe to modify the buffer.
To be notified on completion of the RDMA operation, pass an optional callback object
in the rdma wrapper associated with the specific rdma array.

\begin{alltt}
CkCallback cb(CkIndex_Foo::rdmaSent(NULL), thisProxy[thisIndex]);
arrayProxy[0].foo(500000, rdma(arrPtr, cb));
\end{alltt}

The callback will be invoked on completion of the RDMA operation associated with the
corresponding array. Inside the callback, it is safe to overwrite the buffer sent
via the zero copy API and this buffer can be accessed by dereferencing the CkDataMsg
received in the callback.

\begin{alltt}
//called when RDMA operation is completed
void rdmaSent(CkDataMsg *m)
\{
  //get access to the pointer and free the allocated buffer
  void *ptr = *((void **)(m->data));
  free(ptr);
  delete m;
\}
\end{alltt}

The RDMA call is associated with an rdma array rather than the entry method.
In the case of sending multiple rdma arrays, each RDMA call is independent of the other.
Hence, the callback applies to only the array it is attached to and not to all the rdma
arrays passed in an entry method invocation. On completion of the RDMA call for each
array, the corresponding callback is separately invoked.

As an example, for an entry method with two rdma array parameters, each called with the same
callback, the callback will be invoked twice : on completing the transfer of each of the two
RDMA parameters.

\vspace{0.1in}
\noindent
For multiple arrays to be sent via RDMA, declare the entry method in the .ci file as:

\begin{alltt}
entry void foo (int size1, rdma int arr1[size1], int size2, rdma double arr2[size2]);
\end{alltt}

In the .C file, it is also possible to have different callbacks associated with each rdma array.
\begin{alltt}
CkCallback cb1(CkIndex_Foo::rdmaSent1(NULL), thisProxy[thisIndex]);
CkCallback cb2(CkIndex_Foo::rdmaSent2(NULL), thisProxy[thisIndex]);
arrayProxy[0].foo(500000, rdma(arrPtr1, cb1), 1024000, rdma(arrPtr2, cb2));
\end{alltt}

This API is demonstrated in \examplerefdir{rdma/simpleRdma} and \testrefdir{pingpong}

\vspace{0.1in}
\noindent
It should be noted that calls to entry methods with rdma specified parameters are
currently only supported for point to point operations and not for collective operations.
Additionally, there is also no support for migration of chares that have pending RDMA transfer
requests.

\vspace{0.1in}
\noindent
It should also be noted that the benefit of this API can be seen for large arrays on
only RDMA enabled networks. On networks which do not support RDMA, the API is functional
but doesn't show any performance benefit as it behaves like a regular entry method that
copies its arguments.
