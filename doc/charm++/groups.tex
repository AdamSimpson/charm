\subsection{Group Objects}

A {\sl group\footnote{Originally called {\em Branch Office Chare} or 
{\em Branched Chare}}} \index{group}is a collection of chares where 
there exists \index{chare}one chare (or {\sl branch}) on each
processor.   Each branch has its own data members.  Groups have
a definition syntax similar to normal chares, except that they must
inherit from the system defined class \keyword{Group}, rather than
\keyword{Chare}.

In the interface file, we declare

\begin{tabbing}
~~~~ \=~~~~ \=~~~~ \=~~~~ \=~~~~ \=~~~~ \=~~~~ \=~~~~ \=~~~~ \=~~~~ \kill
\> \kw{group} \uw{GroupType} \{ \\
\> \>  // Interface specifications as for normal chares \\
\> \};
\end{tabbing}

In the {\tt .h} file, we define \uw{GroupType} as follows:

\begin{tabbing}
~~~~ \=~~~~ \=~~~~ \=~~~~ \=~~~~ \=~~~~ \=~~~~ \=~~~~ \=~~~~ \=~~~~ \kill
\> \kw{class} \uw{GroupType} : \kw{public Group} [,other superclasses
] \{ \\
\> \> // Data and member functions as in C++ \\
\> \> // Entry functions as for normal chares \\
\> \};
\end{tabbing}

A group is identified by a globally unique group identifier, whose type is
\kw{CkGroupID}. \index{CkGroupID}This identifier is common to all of the 
group's branches and can be obtained from the variable \keyword{thisgroup},
\index{thisgroup}which is a public local variable of the \keyword{Group} 
superclass.  For groups, \kw{thishandle} \index{thishandle} is the handle of 
the particular branch in which the function is executing: it is 
a normal chare handle.

Groups can be used to implement data-parallel operations easily.  In
addition to sending messages to a particular branch of a group, one
can broadcast messages to all branches of a group.  There can be many
instances corresponding to a group type.  Each instance has a
different group handle, and its own set of branches.

\subsubsection{Group Creation}

\noindent {\bf Chare Group Declaration}:

\noindent Given a {\tt .ci} file as follows:

\begin{verbatim}
group G {
  entry G(M1 *);
  entry void someEntry(M2 *);
};
\end{verbatim}

\noindent and the following {\tt .h} file:

\begin{verbatim}
class G : public Group {
  public:
    G(M1 *);
    void someEntry(M2 *);
};
\end{verbatim}

we can create a \index{group}group in a manner similar to a regular \index{chare}chare.  Note
the difference in how the \index{virtual handle}virtual handle is created.

\begin{verbatim}
M1 *m1 = new M1;
CProxy_G *pG = new CProxy_G(m1);
  // or
CkGroupID gid = CProxy_G::ckNew(m1);
CProxy_G g(gid);
\end{verbatim}

\subsubsection{Method Invocation on Groups}

Before sending a message to a \index{group}group via an entry
method, we need to get a proxy of that group using the group identifier (a
\index{CkGroupID}\kw{CkGroupID}). The syntax for obtaining the proxy or a proxy
pointer is:

\begin{tabbing} ~~~~ \=~~~~ \=~~~~ \=~~~~ \=~~~~ \=~~~~ \=~~~~ \=~~~~ \=~~~~
\=~~~~ \kill \> \kw{CProxy}\_\uw{groupType} {\it groupProxy}({\it groupID}); \\
\> \> or, \\ \> \kw{CProxy}\_\uw{groupType} *{\it groupProxyPointer} = \kw{new
CProxy}\_\kw{groupType}({\it groupID}); \end{tabbing}

The first approach creates a proxy to the group represented by {\it groupID}
while the second creates a pointer named {\it groupProxyPointer} to a proxy to
the group represented by {\it groupID}. 

A message may be sent to a particular \index{branch}branch of group using the
notation:

\begin{tabbing} ~~~~ \=~~~~ \=~~~~ \=~~~~ \=~~~~ \=~~~~ \=~~~~ \=~~~~ \=~~~~
\=~~~~ \kill \> {\it groupProxy}$.$\uw{EntryMethod}({\it MessagePointer}, {\it
Processor}) \\ \> \> or, \\ \> {\it groupProxyPointer}$->$\uw{EntryMethod}({\it
MessagePointer}, {\it Processor}) \end{tabbing}

This sends the message in {\it MessagePointer} to the \index{branch}branch of
the group represented by {\it groupID} which is on processor number {\it
Processor} at the entry method \uw{EntryMethod}, which must be a valid entry
method of that group type. This call is asynchronous and non-blocking; it
returns immediately after sending the message.

A message may be broadcast \index{broadcast} to all branches of a branched
chare (i.e., to all processors) using the notation :

\begin{tabbing} ~~~~ \=~~~~ \=~~~~ \=~~~~ \=~~~~ \=~~~~ \=~~~~ \=~~~~ \=~~~~
\=~~~~ \kill \> {\it groupProxy}$.$\uw{EntryMethod}({\it MessagePointer}) \\ \>
{\it groupProxyPointer}$->$\uw{EntryMethod}({\it MessagePointer}) \end{tabbing}

This sends the message in {\it MessagePointer} to all branches of the group at
the entry method {\sf EntryMethod}, which must be a valid entry method of that
group type. This call is asynchronous and non-blocking; it returns immediately
after sending the message.

Note that the programmer relinquishes control of a message after sending it.
Further access to the message field can cause runtime errors.


Sequential objects, chares and other groups can access public members of the
\index{branch}branch of a group \index{group} {\it on their processor} using
the following notation:

((\uw{GroupType}*)\kw{CkLocalBranch}(\kw{CkGroupID} {\it groupID}))-$>${\it
DataMember}, and \\ ((\uw{GroupType}*)\kw{CkLocalBranch}(\kw{CkGroupID} {\it
groupID}))-$>$\uw{method}().  \index{CkLocalBranch}

Thus a dynamically created \index{chare}chare can call a public method of a
group without needing to know which processor it actually resides: the method
executes in the local \index{branch}branch of the group. 

\index{CkLocalBranch}\kw{CkLocalBranch} returns a generic ({\tt void
*}) pointer.  It needs to be
cast to a pointer of appropriate classe before invoking methods or
accessing data members. \charmpp\ provides another way to do this using
the generated
\index{proxy}{\em proxy} classes. One may call the static method
\kw{ckLocalBranch} of the proxy class of appropriate
group to get the correct type of pointer.  For
example, method \uw{foo} can to be invoked on the local \index{branch}branch of
a group \uw{G} with \uw{gid} as CkGroupID as:

(\kw{CProxy}\_\uw{G}::\kw{ckLocalBranch}({\it gid}))-$>$\uw{foo}(...);\\
\index{ckLocalBranch}









