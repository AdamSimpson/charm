\documentclass[11pt]{article}

\newif\ifpdf
\ifx\pdfoutout\undefined
  \pdffalse
\else
  \pdfoutput=1
  \pdftrue
\fi

\ifpdf
  \pdfcompresslevel=9
  \usepackage[pdftex,colorlinks=true,plainpages=false]{hyperref}
\else
\fi

\usepackage{fullpage}
\pagestyle{headings}
\setlength{\parskip}{0.1in}
\setlength{\textheight}{9.5in}
\setlength{\textwidth}{6.5in}
\setlength{\parindent}{0in}
\setlength{\topmargin}{-.5in}
\parskip 0.1in

%
% Constants
%
\newcommand{\version}{5.0}		%%% The current version number
\newcommand{\prevversion}{4.9}	%%% The previous version number

%
% Commands
%
\newcommand{\zap}[1]{ }
\newcommand{\fcmd}{\bf}		%%% Font for Charm commands
\newcommand{\fparm}{\it\sf}	%%% Font for parameters to Charm commands
\newcommand{\fexec}{\bf}	%%% Font for compile/execute cmds/options
\newcommand{\atitle}[1]{{\it #1}}
\newcommand{\keyword}[1]{{\textbf{#1}}}
\newcommand{\userword}[1]{{\fparm \textsf{#1}}}
\newcommand{\constraint}[1]{Note: {\it #1}}
\newcommand{\note}[1]{Note: {\it #1}}

%
% Conveniences
%
\newcommand{\uw}[1]{\userword{#1}}
\newcommand{\kw}[1]{\keyword{#1}}
\newcommand{\CLocalBranch}{\keyword{CLocalBranch}}

%
%       \CC gives "C++" that looks good.
%
\newcommand{\CC}{C\kern -0.0em\raise 0.5ex\hbox{\normalsize++}}
\newcommand{\emCC}{C\kern -0.0em\raise 0.4ex\hbox{\normalsize\em++}}
\newcommand{\charmpp}{{\sc Charm++}}

\makeindex

\begin{document}

\begin{titlepage}
\vspace*{2in}
\Huge
\begin{center}
The \\
\charmpp \\
Programming Language \\
Manual\\
\vspace*{0.5in}
Version 5.0\\
\vspace*{0.7in}
\today
\end{center}
\normalsize
\end{titlepage}

\vspace*{2.5in}
\Large
The Charm software was developed as a group effort.
The earliest prototype, Chare Kernel(1.0), was developed by Wennie Shu and Kevin
Nomura working with Laxmikant Kale. 
The second prototype, Chare Kernel(2.0), 
a complete re-write with major design changes,
was developed by a team consisting of
Wayne Fenton, Balkrishna Ramkumar, Vikram Saletore, Amitabh B. Sinha
and Laxmikant Kale. The translator for Chare Kernel(2.0) was written by
Manish Gupta.
Charm(3.0), with significant design changes, was developed by a team
consisting of Attila Gursoy, Balkrishna Ramkumar, Amitabh B.  Sinha and
Laxmikant Kale, with a new translator written by Nimish Shah.  
The Charm++ implementation was done by Sanjeev Krishnan.
Charm(4.0) included Charm++ and was released in fall 1993. 
Charm software (4.5), has been developed by
Attila Gursoy, Sanjeev Krishnan, Milind Bhandarkar, Joshua Yelon, Narain
Jagathesan and Laxmikant Kale.
The current version of Charm Kernel was rewritten from scratch by
Milind Bhandarkar.
\normalsize


\newpage
\tableofcontents

\newpage
\chapter{Introduction}

%update as you wish

This manual describes \charmpp, an object oriented portable parallel
programming language based on \CC. Its program structure, execution
model, interface language constructs and runtime system calls are
described here\footnote{For a description of the underlying design
philosophy please refer to the following papers :\\
    L. V. Kale and Sanjeev Krishnan,
    {\em ``\charmpp : Parallel Programming with Message-Driven Objects''},
    in ``Parallel Programming Using \CC'',
    MIT Press, 1995. \\
    L. V. Kale and Sanjeev Krishnan,
    {\em ``\charmpp : A Portable Concurrent Object Oriented System
    Based On \CC''},
    Proceedings of the Conference on Object Oriented Programming,
    Systems, Languages and Applications (OOPSLA), September 1993.
}.
% Change to citations in appendices. 

\charmpp\ has continuously evolved since the OOPSLA 1993 paper.  The earlier
versions modified the \CC\ syntax to support \charmpp\ primitives, and
contained a full-fledged \charmpp\ translator that parsed the \charmpp\
syntactic extensions as well as the \CC\ syntax to produce a \CC\ program,
which was later compiled using a \CC\ compiler.  The current version does not
augment the C++ syntax, and does not use a \charmpp\ translator as in previous
versions. Instead, the older constructs are converted to calls into the runtime
library, several new constructs are added, and minimal language constructs are
used to describe the interfaces.

\charmpp\ is an explicitly parallel language based on \CC\ with a runtime
library for supporting parallel computation called the Charm kernel.  It
provides a clear separation between sequential and parallel objects.  The
execution model of \charmpp\ is message driven, thus helping one write programs
that are latency-tolerant.  \charmpp\ supports dynamic load balancing while
creating new work as well as periodically, based on object migration.  Several
dynamic load balancing strategies are provided.  \charmpp\ supports both
irregular as well as regular, data-parallel applications.  It is based on the
{\sc Converse} interoperable runtime system for parallel programming.

Currently the parallel platforms supported by \charmpp\ are the IBM SP, SGI
Origin2000, Cray T3E, Intel Paragon, a single workstation or a network of
workstations from Sun Microsystems (Solaris), IBM RS-6000 (AIX) SGI (IRIX 5.3
or 6.4), HP (HP-UX), and Intel x86 (Linux, Windows NT).  \charmpp\ programs can
run without changing the source on all these platforms.  Please see the {\sl
Charm++/Converse Installation and Usage Manual} for details about installing,
compiling and running \charmpp\ programs.

For a description of the C-based {\sc Charm} parallel programming system,
please refer to the {\sl Charm Programming Language Manual} and the {\sl
Tutorial Introduction to Charm}\footnote{{\sc Charm} is no longer actively
supported and maintained, and these manuals are kept only for offering the
historical perspectives.}.

\section{Overview}

%update as you wish

\charmpp\ is an object oriented parallel language. What sets \charmpp\ apart
from traditional programming models such as message passing and shared variable
programming is that the execution model of \charmpp\ is message-driven.
Therefore, computations in \charmpp\ are triggered based on arrival of
associated messages. These computations in turn can fire off more messages to
other (possibly remote) processors that trigger more computations on those
processors.

At the heart of any \charmpp\ program is a scheduler that repetitively chooses
a message from the available pool of messages, and executes the computations
associated with that message.

The programmer-visible entities in a \charmpp\ program are:

\begin{itemize}
\item Concurrent Objects : called {\em chares}\footnote{
      Chare (pronounced {\bf ch\"ar}, \"a as in c{\bf a}rt) is Old 
      English for chore.
      }
\item Communication Objects : Messages
\item Chare Groups and Nodegroups
\item Chare Arrays
\item Readonly data
\end{itemize}

\charmpp\ starts a program by creating a single \index{chare} instance of each
{\em mainchare} on processor 0, and invokes constructor methods of these chare.
Typically, these chares then creates a number of other \index{chare} chares,
possibly on other processors, which can simultaneously work to solve the
problem at hand.

Each \index{chare}chare contains a number of \index{entry method}{\em entry
methods}, which are methods that can be invoked from remote processors. The
\charmpp\ runtime system needs to be explicitly told about these methods, via
an {\em interface} in a separate file.  The syntax of this interface
specification file is described in the later sections.

\charmpp\ provides system calls to asynchronously create remote \index{chare}
chares and to asynchronously invoke entry methods on remote chares by sending
\index{message} messages to those chares. This asynchronous
\index{message}message passing is the basic interprocess communication
mechanism in \charmpp. However, \charmpp\ also permits wide variations on this
mechanism to make it easy for the programmer to write programs that adapt to
the dynamic runtime environment.  These possible variations include
prioritization (associating priorities with method invocations), conditional
\index{message packing}message packing and unpacking (for reducing messaging
overhead), \index{quiescence}quiescence detection (for detecting completion of
some phase of the program), and dynamic load balancing (during remote object
creation). In addition, several libraries are built on top of \charmpp\ that
can simplify otherwise arduous parallel programming tasks.

The following sections provide detailed information about various features of
\charmpp\ programming system.

\section{History}

%update with history up to 5.0 version

The {\sc Charm} software was developed as a group effort of the Parallel
Programming Laboratory at the University of Illinois at Urbana-Champaign.
Researchers at the Parallel Programming Laboratory keep \charmpp\ updated for
the new machines, new programming paradigms, and for supporting and simplifying
development of emerging applications for parallel processing.  The earliest
prototype, Chare Kernel(1.0), was developed in the late eighties. It consisted
only of basic remote method invocation constructs available as a library.  The
second prototype, Chare Kernel(2.0), a complete re-write with major design
changes.  This included C language extensions to denote Chares, messages and
asynchronous remote method invocation.  {\sc Charm}(3.0) improved on this
syntax, and contained important features such as information sharing
abstractions, and chare groups (called Branch Office Chares).  {\sc Charm}(4.0)
included \charmpp\ and was released in fall 1993.  \charmpp\ in its initial
version consisted of syntactic changes to \CC\ and employed a special
translator that parsed the entire \CC\ code while translating the syntactic
extensions.  {\sc Charm}(4.5)  had a major change that resulted from a
significant shift in the research agenda of the Parallel Programming
Laboratory. The message-driven runtime system code of the \charmpp\ was
separated from the actual language implementation, resulting in an
interoperable parallel runtime system called {\sc
Converse}. The \charmpp\ runtime system was
retargetted on top of {\sc Converse}, and popular programming paradigms such as
MPI and PVM were also implemented on {\sc Converse}. This allowed
interoperability between these paradigms and \charmpp. This release also
eliminated the full-fledged \charmpp\ translator by replacing syntactic
extensions to \CC\ with \CC\ macros, and instead contained a small language and
a translator for describing the interfaces of \charmpp\ entities to the runtime
system.  This version of \charmpp, which, in earlier releases was known as {\em
Interface Translator \charmpp}, is the default version of \charmpp\ now, and
hence referred simply as {\bf \charmpp}.  In early 1999, the runtime system of
\charmpp\ was formally named the Charm Kernel, and was rewritten in \CC.
Several new features were added. The interface language underwent significant
changes, and the macros that replaced the syntactic extensions in original
\charmpp, were replaced by natural \CC\ constructs.


\section{Notation Used}

Small code samples used to illustrate syntax specifications throughout
this document will use the following typeface conventions:

\begin{itemize}
\item Language keywords appear as boldface words: \kw{chare}.
\item User-defined types and function names appear in a sans serif font:
\uw{chareName}. 
\item User-defined variables appear italicized: {\it myChare}.
\item All other code appears in the same font as the regular text of
this document.
\end{itemize}

Longer code samples of actual code will appear in the standard
typewriter font. 





\newpage
\section{\charmpp{} Overview}

We think that \charmpp\ is easy to use if you are familiar with object-based
programming. (But of course that is our opinion, if your opinion differs,
you are encouraged to let us know the reasons, and features that you would
like to see in \charmpp.) Object-based programming is built around the
concept of ``encapsulation'' of data. As implemented in \CC, data
encapsulation is achieved by grouping together data and methods (also known
as functions, subroutines, or procedures) inside of an object.

A class is a blueprint for an object.  The encapsulated data is said to be
``private'' to the object, and only the methods of that class can manipulate
that data. A method that has the same name as the class is a ``blessed''
method, called a ``Constructor'' for that class.  A constructor method is
typically responsible for initializing the encapsulated data of an object.
Each method, including the constructor can optionally be supplied data in
the form of parameters (or arguments). In \CC, one can create objects with
the {\tt new} operator that returns a pointer to the object. This pointer
can be used to refer to the object, and call methods on that object.

\charmpp{} is built on top of \CC, and also based on ``encapsulation''.
Similar to \CC, \charmpp\ entities can contain private data, and public
methods. The major difference is that these methods can be invoked from
remote processors asynchronously.  Asynchronous method invocation means that
the caller does not wait for the method to be actually executed and does not
wait for the method's return value. Therefore, \charmpp\ methods (called
entry methods) do not have a return value\footnote{Asynchronous remote
method invocation is the core of \charmpp. However, to simplify programming,
\charmpp\ makes use of the interoperable nature of its runtime system, and
combines seamlessly with user-level threads to also support synchronous
method execution, albeit with a slight overhead of thread creation and
scheduling.}. Since the actual \charmpp\ object on which the method is being
invoked may be on a remote processor\footnote{With its own, different address
space}, the \CC\ way of referring to an object, via a pointer, is not valid
in \charmpp.  Instead, we refer to a remote chare via a ``proxy''.

Those familiar with various component models\footnote{Such as CORBA} in the
distributed computing world will recognize ``proxy'' to be a dummy, standin
entity that refers to an actual entity.  For each chare type, a ``proxy''
class exists\footnote{The proxy class is generated by the ``interface
translator'' based on a description of the entry methods}.  The methods of
this ``proxy'' class correspond to the remote methods of actual class, and
act as ``forwarders''. That is, when one invokes a method on a proxy to a
remote object, the proxy forwards this method invocation to the actual
remote object. All entities that are created and manipulated remotely in
\charmpp\ have such proxies. Proxies for each type of entity in \charmpp\
have some differences among the features they support, but the basic syntax
and semantics remains the same-- that of invoking methods on the remote
object by invoking methods on proxies.


\subsection{\charmpp\ Execution Model}

A \charmpp\ program consists of a number of \charmpp\ objects distributed
across the available number of processors. Thus, the basic unit of parallel
computation in \charmpp\ programs is the {\em chare}\index{chare}, a \charmpp\
object that can be created on any available processor and can be accessed from
remote processors.  A \index{chare}chare is similar to a process, an actor, an
ADA task, etc.  \index{chare}Chares are created dynamically, and many chares
may be active simultaneously.  Chares send \index{message}{\em messages} to one
another to invoke methods asynchronously.  Conceptually, the system maintains a
``work-pool'' consisting of seeds for new \index{chare}chares, and
\index{message}messages for existing chares. The runtime system (called {\em
Charm Kernel}) may pick multiple items, non-deterministically, from this pool
and execute them.  

Methods of a \index{chare}chare that can be remotely invoked are called
\index{entry method}{\em entry} methods.  Entry methods may take marshalled
parameters, or a pointer to a message object.  Since \index{chare}chares can
be created on remote processors, obviously some constructor of a chare needs
to be an entry method.  Ordinary entry methods\footnote{``Threaded'' or
``synchronous'' methods are different.} are completely non-preemptive--
\charmpp\ will never interrupt an executing method to start any other work,
and all calls made are asynchronous.

\charmpp\ provides dynamic seed-based load balancing. Thus location (processor
number) need not be specified while creating a remote \index{chare}chare. The
Charm Kernel will then place the remote chare on a least loaded processor. Thus
one can imagine chare creation as generating only a seed for the new chare,
which may {\em take root} on the most {\em fertile} processor. Charm Kernel
identifies a \index{chare}chare by a {\em ChareID}.  Since user code does not
need to name a chares' processor, chares can potentially migrate from one
processor to another.  (This behavior is used by the dynamic load-balancing
framework for chare containers, such as arrays.)

Other \charmpp\ objects are collections of chares. They are: {\em
chare-arrays}, \index{group}{\em chare-groups}, and \index{nodegroup}{\em
chare-nodegroups}, referred to as {\em arrays}, {\em groups}, and {\em
nodegroups} throughout this manual. An array is a collection of arbitrary
number of migratable chares, indexed by some index type, and mapped to
processors according to a user-defined map group. A group (nodegroup) is a
collection of chares, one per processor (SMP node), that is addressed using
a unique system-wide name.

Every \charmpp\ program must have at least one \kw{mainchare}.  Each
\kw{mainchare} is created by the system on processor 0 when the \charmpp\
program starts up.  Execution of a \charmpp\ program begins with the Charm
Kernel constructing all the designated \kw{mainchare}s.  Typically, the
\kw{mainchare} constructor starts the computation by creating arrays, other
chares, and groups.  It can also be used to initialize shared \kw{readonly}
objects.

The only method of communication between processors in \charmpp\ is
asynchronous \index{entry method} entry method invocation on remote chares.
For this purpose, Charm Kernel needs to know the types of
\index{chare}chares in the user program, the methods that can be invoked on
these chares from remote processors, the arguments these methods take as
input etc. Therefore, when the program starts up, these user-defined
entities need to be registered with Charm Kernel, which assigns a unique
identifier to each of them. While invoking a method on a remote object,
these identifiers need to be specified to Charm Kernel. Registration of
user-defined entities, and maintaining these identifiers can be cumbersome.
Fortunately, it is done automatically by the \charmpp\ interface translator.
The \charmpp\ interface translator generates definitions for {\em proxy}
objects. A proxy object acts as a {\em handle} to a remote chare. One
invokes methods on a proxy object, which in turn carries out remote method
invocation on the chare.

In addition, the \charmpp\ interface translator provides ways to enhance the
basic functionality of Charm Kernel using user-level threads and futures. These
allow entry methods to be executed in separate user-level threads.  These
\index{threaded} {\em threaded} entry methods may block waiting for data by
making {\em synchronous} calls to remote object methods that return results in
messages.

\charmpp\ program execution is terminated by the \kw{CkExit} call.  Like the
\kw{exit} system call, \kw{CkExit} never returns. The Charm Kernel ensures
that no more messages are processed and no entry methods are called after a
\kw{CkExit}. \kw{CkExit} need not be called on all processors; it is enough
to call it from just one processor at the end of the computation.


\subsection{Entities in \charmpp\ programs}

This section describes various entities in a typical \charmpp\ program.

\subsubsection{Sequential Objects}

A \charmpp\ program typically consists mostly of ordinary sequential \CC
code and objects. Such entities are only accessible locally, are not known
to the \charmpp\ runtime system, and thus need not be mentioned in the
module interface files. 

\charmpp\ does not affect the syntax or semantics of such \CC\ entities,
except that changes to global variables (or static data members of a class)
on one node will not be visible on other nodes.  Global data changes
must be explicitly sent between processors.  For processor- and
thread-private storage, refer to the ``Global Variables'' section
of the Converse manual.


\subsubsection{Messages}

Messages supply data arguments to the asynchronous remote method invocation.
These objects are treated differently from other objects in \charmpp\ by the
runtime system, and therefore they must be specified in the interface file
of the module.  With parameter marshalling, the system creates and handles
the message completely internally. Other messages are instances of \CC\
classes that are subclassed from a special class that is generated by the
\charmpp\ interface translator.  Another variation of communication objects
is conditionally packed and unpacked. This variation should be used when one
wants to send messages that contain pointers to the data rather than the
actual data to other processors. This type of communication objects contains
two static methods: \kw{pack}, and \kw{unpack}. The third variation of
communication objects is called {\em varsize} messages. Varsize messages is
an effective optimization on conditionally packed messages, and can be
declared with special syntax in the interface file.

\subsubsection{Chares}

Chares are the most important entities in a \charmpp\ program. These concurrent
objects are different from sequential \CC\ objects in many ways. Syntactically,
Chares are instances of \CC\  classes that are derived from a system-provided
class called \kw{Chare}. Also, in addition to the usual \CC\ private and public
data and method members, they contain some public methods called {\em entry
methods}. These entry methods do not return anything (they are {\tt void}
methods), and take at most one argument, which is a pointer to a message.
Chares are {\em accessed} using a proxy (an object of a specialized class
generated by the \charmpp\ interface translator) or using a handle (a \kw{
CkChareID} structure defined in \charmpp), rather than a pointer as in \CC.
Semantically, they are different from \CC\ objects because they can be created
asynchronously from remote processors, and their entry methods also could be
invoked asynchronously from the remote processors. Since the constructor method
is invoked from remote processor (while creating a chare), every chare should
have its constructors as entry methods (with at most one message pointer
parameter). These chares and their entry methods have to be specified in the
interface file.

\subsubsection{Chare Arrays}

Chare arrays are collections of chares. However, unlike chare groups or
nodegroups, arrays are not constrained by characteristics of the underlying
parallel machine such as number of processors or nodes. Thus, chare arrays
can have any number of {\em elements}. The array elements themselves are
chares, and methods can be invoked on individual array elements as usual.  
Each element of an array has a globally unique index, and messages are
addressed to that index.

Unlike other entities in \charmpp\, the dynamic load balancing framework (LB
Framework) treats array elements as objects that can be migrated across
processors. Thus, the runtime system keeps track of computational load
across the system, and also the time spent in execution of entry methods on
array elements, and then employs one of several strategies to redistribute
array elements across the available processors.

\subsubsection{Chare Groups}

Chare Groups\footnote{ These were called Branch Office Chares (BOC) in earlier
versions of Charm.} are a special type of concurrent objects.  Each chare group
is a collection of chares, with one representative (group member) on each
processor. All the members of a chare group share a globally unique name
(handle, defined by Charm kernel to be of type \kw{CkGroupID}). An entire chare
group could be addressed using this global handle, and an individual member of
a chare group can be addressed using the global handle, and a processor number.
Chare groups are instances of \CC\ classes subclassed from a system-provided
class called \kw{Group}. The Charm kernel has to be notified that these chares
are semantically different, and therefore chare groups have a different
declaration in the interface specification file.

\subsubsection{Chare Nodegroups}

Chare nodegroups are very similar to chare groups except that instead of having
one group member on each processor, the nodegroup has one member on each shared
memory multiprocessor node. Note that \charmpp\ (and its underlying runtime
system Converse) distinguish between processors and nodes. A node consists of
one or more processors that share an address space. The last few years have
seen emergence of fast SMP systems of small (2-4 processors) to large (32-64
processors) number of processors per node. A network of such SMP nodes is the
most general model of parallel computers, making pure distributed and pure
shared memory systems mere special cases. \charmpp\ is built on top of this
machine abstraction, and Chare nodegroups embody this abstraction in a higher
level language construct. Semantically, methods invoked on a nodegroup member
could be executed on any processor within that node. This fact can be utilized
for supporting load balance across processors within a node. However, this also
means that different processors within a node could be executing methods of the
same nodegroup member simultaneously, thus leading to common problems
associated with shared address space programming. However, \charmpp\ eases such
problems by allowing the programmer to specify an entry method of a nodegroup
to be {\em exclusive}, thus guaranteeing that no other {\em exclusive} method
of that nodegroup member can execute simultaneously within the node.



\section{The \charmpp\ Language}
  \section{Modules}

\subsection{Structure of a \charmpp\ Program}

A \charmpp\ program is structurally similar to a \CC{} program.  Most of a
\charmpp\ program {\em is} \CC{} code \footnote{\bf Constraint: The \CC{} code
cannot, however, contain global or static variables.}. The main syntactic units
in a \charmpp\ program are class definitions. A \charmpp\ program can be
distributed across several source code files.

There are five disjoint categories of objects (classes) in \charmpp:

\begin{itemize}
\item Sequential objects: as in \CC{}
\item Chares (concurrent objects) \index{chare}
\item Chare Groups \index{chare groups} (a form of replicated objects)
\index{group}
\item Chare Arrays \index{chare arrays} (an indexed collection of chares)
\index{array}
\item Messages (communication objects)\index{message}
\end{itemize}

The user's code is written in \CC{} and interfaces with the \charmpp\ system as
if it were a library containing base classes, functions, etc.  A translator is
used to generate the special code needed to handle \charmpp\ constructs.  This
translator generates \CC{} code that needs to be compiled with the user's code.

Interfaces to the \charmpp\ objects (such as messages, chares, readonly
variables etc.) \index{message}\index{chare}\index{readonly} have to be
declared in \charmpp\ interface files. Typically, such entities are grouped
\index{module} into {\em modules}. A \charmpp\ program may consists of multiple
modules.  One of these modules is declared to be a \kw{mainmodule}. All the
modules that are ``reachable'' from the \kw{mainmodule} via the \kw{extern}
construct are included in a \charmpp\ program.

The \charmpp\ interface file has the suffix ``.ci''.  The \charmpp\ interface
translator parses this file and produces two files (with suffixes ``.decl.h''
and ``.def.h'', {\em for each modules declared in the ``.ci'' file}), that
contain declarations (interface) and definitions (implementation)of various
translator-generated entities. If the name of a module is \uw{MOD}, then the
files produced by the \charmpp\ interface translator are named \uw{MOD.decl.h}
and \uw{MOD.def.h}\footnote{Note that the interface file for module \uw{MOD}
need not be named \uw{MOD.ci}. Indeed one ``.ci'' file may contain interface
declarations for multiple modules, and the translator will produce one pair of
declaration and definition files for each module.}.  We recommend that the
declarations header file be included at the top of the header file (\uw{MOD.h})
for module \uw{MOD}, and the definitions file be included at the bottom of the
code for module (\uw{MOD.C}) \footnote{In the earlier version of interface
translator, these files used to be suffixed with ``.top.h'' and ``.bot.h'' for
this reason.}.

A simple \charmpp\ program is given below:

\begin{alltt}
///////////////////////////////////////
// File: pgm.ci

mainmodule Hello \{
  readonly CkChareID mid;
  mainchare HelloMain \{
    entry HelloMain(); // implicit CkArgMsg * as argument
    entry void Wait(void);
  \};
  group HelloGroup \{
    entry HelloGroup(void);
  \};
\};

////////////////////////////////////////
// File: pgm.h
#include "Hello.decl.h" // Note: not pgm.decl.h

class HelloMain: public Chare \{
  public:
    HelloMain(CkArgMsg *);
    void Wait(void);
  private:
    int count;
\};

class HelloGroup: public Group \{
  public:
    HelloGroup(void);
\};

/////////////////////////////////////////
// File: pgm.C
#include "pgm.h"

CkChareID mid;

HelloMain::HelloMain(CkArgMsg *msg) \{
  delete msg;
  count = 0;
  mid = thishandle;
  CProxy_HelloGroup::ckNew(); // Create a new "HelloGroup"
\}

void HelloMain::Wait(void) \{
  count++;
  if (count == CkNumPes()) \{ // Wait for all group members to finish the printf
    CkExit();
  \}
\}

HelloGroup::HelloGroup(void) \{
  ckout << "Hello World from processor " << CkMyPe() << endl;
  CProxy_HelloMain pself(mid);
  pself.Wait();
\}

#include "Hello.def.h" // Include the Charm++ object implementations

/////////////////////////////////////////
// File: Makefile

pgm: pgm.ci pgm.h pgm.C
      charmc -c pgm.ci
      charmc -c pgm.C
      charmc -o pgm pgm.o -language charm++

\end{alltt}

\uw{HelloMain} is designated a \kw{mainchare}. Thus the Charm Kernel starts
execution of this program by creating an instance of \uw{HelloMain} on
processor 0. The HelloMain constructor creates a chare group \uw{HelloGroup},
and stores a handle to itself and returns. The call to create the group returns
immediately after directing Charm Kernel to perform the actual creation and
invocation.  Shortly after, the Charm Kernel will create an object of type
\uw{HelloGroup} on each processor, and call its constructor. The constructor
will then print ``Hello World...'' and then call the \uw{Wait} method of
\uw{HelloMain}. The Wait method calls CkExit() after all group members have
called it (i.e., they have finished printing ``Hello World...''), and the
\charmpp program exits.

\subsection{Functions in the ``decl.h'' and ``def.h'' files}

The \texttt{decl.h} file provides declarations for the proxy classes of the
concurrent objects declared in the ``.ci'' file (from which the \texttt{decl.h}
file is generated). So the \uw{Hello.decl.h} file will have the declaration of
the class CProxy\_HelloMain. Similarly it will also have the declaration for
the HelloGroup class. 

This class will have functions to create new instances of the chares and
groups, like the function \kw{ckNew}. For \uw{HelloGroup} this function creates
an instance of the class \uw{HelloGroup} on all the processors. 

The proxy class also has functions corresponding to the entry methods defined
in the ``.ci'' file. In the above program the method wait is declared in
\uw{CProxy\_HelloMain} (proxy class for \uw{HelloMain}).

The proxy class also provides static registration functions used by the
\charmpp{} runtime.  The \texttt{def.h} file has a registration function
(\uw{\_\_registerHello} in the above program) which calls all the registration
functions corresponding to the readonly variables and entry methods declared in
the module.

	
  \subsection{Messages}

\label{messages}
Although \charmpp supports automated parameter marshalling for entry methods,
you can also manually handle the process of packing and unpacking parameters by
using messages. 
%By using messages, you can potentially improve performance by
%avoiding unnecessary copying. 
A message encapsulates all the parameters sent to an
entry method.  Since the parameters are already encapsulated,
sending messages is often more efficient than parameter marshalling, and
can help to avoid unnecessary copying.
Moreover, assume that the receiver is unable to process the contents of the
message at the time that it receives it. For example, consider a 
tiled matrix multiplication program, wherein each chare receives an $A$-tile
and a $B$-tile before computing a partial result for $C = A \times B$. If we 
were using parameter marshalled entry methods, we would have to copy the first
tile received by a chare in order to save it. Then, upon receiving the second
tile, the chare would use the second tile and the first (saved) tile to 
compute a partial result. However, using messages, we would just save a {\em pointer} 
to the message encapsulating the tile received first, instead of the tile data itself.

\vspace{0.1in}
\noindent
{\bf Managing the memory buffer associated with a message.}
As suggested in the example above, the biggest difference between marshalled parameters and messages
is that an entry method invocation is assumed to {\em keep} the message that it
is passed. That is, the \charmpp{} runtime system assumes that code in the body of the invoked
entry method will explicitly manage the memory associated with the message that it is passed. Therefore,
in order to avoid leaking memory, the body of an entry method must either \kw{delete} the message that
it is receives, or save a pointer to it, and \kw{delete} it a later point in the execution of the code.
%is code written for the body of an 
%either store the passed message or explicitly {\em delete} it, or else the message
%will never be destroyed, wasting memory.

Moreover, in the \charm{} execution model, once you pass a message buffer to the runtime system (via 
an asynchronous entry method invocation), you should {\em not} reuse the buffer. That is, after you have
passed a message buffer into an asynchronous entry method invocation, you shouldn't 
access its fields, or pass that same buffer into a second entry method invocation. Note that this rule
doesn't preclude the {\em single reuse} of an input message -- consider an entry method invocation
$i_1$, which receives as input the message buffer $m_1$. Then, $m_1$ may be passed to an 
asynchronous entry method invocation $i_2$. However, once $i_2$ has been issued with $m_1$ as its input
parameter, $m_1$ cannot be used in any further entry method invocations.
%message buffer, that message buffer may in turn be passed to an entry method invocation that accepts a 
%message of the same type. However, 
 
%Thus each entry method must be passed a {\em new} message.

Several kinds of message are available.
Regular \charmpp{} messages are objects of
\textit{fixed size}. One can have messages that contain pointers or variable
length arrays (arrays with sizes specified at runtime) and still have these
pointers to be valid when messages are sent across processors, with some
additional coding.  Also available is a mechanism for assigning
\textit{priorities} to a message regardless of its type.
A detailed discussion of priorities appears later in this section.

\subsection{Message Declaration and Definition}

Like all other entities involved in asynchronous method invocation, messages
need to be declared in the {\tt .ci} file. In the {\tt .ci} file (the
interface file), a message is declared as: 

\begin{alltt}
 message MessageType;
\end{alltt}

%A message that contains variable length arrays is declared as:
%
%\begin{alltt}
% message MessageType \{
%   type1 var_name1[];
%   type2 var_name2[];
%   type3 var_name3[];
%\};
%\end{alltt}
%
If the name of the message class is \uw{MessageType}, the class must inherit 
publicly from a class whose name is \uw{CMessage\_MessageType}. This class
is generated by the charm translator. Therefore, the definition of the \uw{MessageType} class is
as follows:

\begin{alltt}
 class MessageType : public CMessage_MessageType \{
    // List of data and function members as in \CC
 \};
\end{alltt}


\subsubsection{Message Creation and Deletion}

\label{memory allocation}

\index{message}Messages are allocated using the \CC\ \kw{new} operator:

\begin{alltt}
 MessageType *msgptr =
  new [(int sz1, int sz2, ... , int priobits=0)] MessageType[(constructor arguments)];
\end{alltt}

The arguments enclosed within the square brackets are optional, and 
are used when allocating messages
with variable length arrays or prioritized messages. We will discuss the
creation of such messages shortly.


Arguments \uw{sz1, sz2, ...}
denote the size (in number of elements) of the memory blocks that need to be
allocated and assigned to the pointers that the message contains. The
\uw{priobits} argument denotes the size of a bitfield (number of bits) that
will be used to store the message priority.   

Suppose we have a fixed size message of type \uw{MyMessage}, which has already been
declared in a {\tt .ci} file. The definition of the message type would then take the
following form:

\begin{alltt}
class MyMessage : public CMessage_MyMessage \{
  // Since this is a fixed size message it can only have 
  // data members whose sizes can be determined at compile-time

  int a;
  double x[NDIMS];
  UserDefinedType u1;
  AnotherUserDefinedType u2[5];

  // Define constructors, methods, etc.
\};
\end{alltt}

Then we would create a new message by doing the following:

\begin{alltt}
Message *msg = new Message;
\end{alltt}

To allocate a message whose class declaration is:

\begin{alltt}
class VarsizeMessage : public CMessage_VarsizeMessage \{
 public:
  int *firstArray;
  double *secondArray;
\};
\end{alltt}

do the following:

\begin{alltt}
VarsizeMessage *msg = new (10, 20) VarsizeMessage;
\end{alltt}

This allocates a \uw{VarsizeMessage}, in which \uw{firstArray} points to an
array of 10 ints and \uw{secondArray} points to an array of 20 doubles.  This
is explained in detail in later sections. 

To add a \index{priority}priority bitfield to this message, 

\begin{alltt}
VarsizeMessage *msg = new (10, 20, sizeof(int)*8) VarsizeMessage;
\end{alltt}

Note, you must provide number of bits which is used to store the priority as
the \uw{priobits} parameter. The section on prioritized execution describes how
this bitfield is used.

In Section~\ref{message packing} we explain how messages can contain arbitrary
pointers, and how the validity of such pointers can be maintained across
processors in a distributed memory machine.

When a message \index{message} is sent to a \index{chare}chare, the programmer
relinquishes control of it; the space allocated to the message is freed by the
system.  When a message is received at an entry point it is not freed by the
runtime system.  It may be reused or deleted by the programmer.  Messages can
be deleted using the standard \CC{} \kw{delete} operator.  

There are no limitations of the methods of message classes except that the
message class may not redefine operators \texttt{new} or \texttt{delete}.


\subsubsection{Messages with Variable Length Arrays}

\label{varsize messages}
\index{variable size messages}
\index{varsize message}

An ordinary message in \charmpp\ is a fixed size message that is allocated
internally with an envelope which encodes the size of the message. Very often,
the size of the data contained in a message is not known until runtime. One can
use packed\index{packed messages} messages to alleviate this problem.  However,
it requires multiple memory allocations (one for the message, and another for
the buffer.) This can be avoided by making use of a \emph{varsize} message.
In \emph{varsize} messages, the space required for these variable length arrays
is allocated with the message such that it is contiguous to the message.

Such a message is declared as 

\begin{alltt}
 message mtype \{
   type1 var_name1[];
   type2 var_name2[];
   type3 var_name3[];
 \};
\end{alltt}

in \charmpp\ interface file. The class \uw{mtype} has to inherit from
\uw{CMessage\_mtype}. In addition, it has to contain variables of corresponding
names pointing to appropriate types. If any of these variables (data members)
are private or protected, it should declare class \uw{CMessage\_mtype} to be a
``friend'' class. Thus the \uw{mtype} class declaration should be similar to:

\begin{alltt}
class mtype : public CMessage_mtype \{
 private:
   type1 *var_name1;
   type2 *var_name2;
   type3 *var_name3;
   friend class CMessage_mtype;
\};
\end{alltt}

\small
\hrule

\noindent\textbf{An Example}

Suppose a \charmpp\ message contains two variable length arrays of types
\texttt{int} and \texttt{double}:

\begin{alltt} 
class VarsizeMessage: public CMessage_VarsizeMessage \{
  public:
    int lengthFirst;
    int lengthSecond;
    int* firstArray;
    double* secondArray;
    // other functions here
\};
\end{alltt}

Then in the \texttt{.ci} file, this has to be declared as: 

\begin{alltt}
message VarsizeMessage \{
  int firstArray[];
  double secondArray[];
\};
\end{alltt}

We specify the types and actual names of the fields that
contain variable length arrays. The dimensions of these arrays are NOT
specified in the interface file, since they will be specified in the
constructor of the message when message is created. In the {\tt .h} or {\tt .C}
file, this class is declared as:

\begin{alltt} 

class VarsizeMessage : public CMessage_VarsizeMessage \{ 
  public: 
    int lengthFirst;
    int lengthSecond;
    int* firstArray;
    double* secondArray;
    // other functions here
\};
\end{alltt}

The interface translator generates the \uw{CMessage\_VarsizeMessage} class,
which contains code to properly allocate, pack and unpack the
\uw{VarsizeMessage}.


One can allocate messages of the \uw{VarsizeMessage} class as follows:

\begin{alltt}
// firstArray will have 4 elements
// secondArray will have 5 elements 
VarsizeMessage* p = new(4, 5, 0) VarsizeMessage;
p->firstArray[2] = 13;     // the arrays have already been allocated 
p->secondArray[4] = 6.7; 
\end{alltt}

Another way of allocating a varsize message is to pass a \uw{sizes} in an array
instead of the parameter list. For example,

\begin{alltt}
int sizes[2];
sizes[0] = 4;               // firstArray will have 4 elements
sizes[1] = 5;               // secondArray will have 5 elements 
VarsizeMessage* p = new(sizes, 0) VarsizeMessage;
p->firstArray[2] = 13;     // the arrays have already been allocated 
p->secondArray[4] = 6.7; 
\end{alltt}

\hrule
\normalsize

No special handling is needed for deleting varsize messages.

\subsubsection{Message Packing}

\label{message packing}
\index{message packing}

The \charmpp{} interface translator generates implementation for three static
methods for the message class \uw{CMessage\_mtype}. These methods have the
prototypes:

\begin{alltt}
    static void* alloc(int msgnum, size_t size, int* array, int priobits);
    static void* pack(mtype*);
    static mtype* unpack(void*);
\end{alltt}

One may choose not to use the translator-generated methods and may override
these implementations with their own \uw{alloc}, \uw{pack} and \uw{unpack}
static methods of the \uw{mtype} class.  The \kw{alloc} method will be called
when the message is allocated using the \CC\ \kw{new} operator. The programmer
never needs to explicitly call it.  Note that all elements of the message are
allocated when the message is created with \kw{new}. There is no need to call
\kw{new} to allocate any of the fields of the message. This differs from a
packed message where each field requires individual allocation. The \kw{alloc}
method should actually allocate the message using \kw{CkAllocMsg}, whose
signature is given below:

\begin{alltt}
void *CkAllocMsg(int msgnum, int size, int priobits); 
\end{alltt}  


For varsize messages, these static methods \texttt{alloc}, \texttt{pack}, and 
\texttt{unpack} are
generated by the interface translator.  For example, these
methods for the \kw{VarsizeMessage} class above would be similar to:

\begin{alltt}
// allocate memory for varmessage so charm can keep track of memory
static void* alloc(int msgnum, size_t size, int* array, int priobits)
\{
  int totalsize, first_start, second_start;
  // array is passed in when the message is allocated using new (see below).
  // size is the amount of space needed for the part of the message known
  // about at compile time.  Depending on their values, sometimes a segfault
  // will occur if memory addressing is not on 8-byte boundary, so altered
  // with ALIGN8
  first_start = ALIGN8(size);  // 8-byte align with this macro
  second_start = ALIGN8(first_start + array[0]*sizeof(int));
  totalsize = second_start + array[1]*sizeof(double);
  VarsizeMessage* newMsg = 
    (VarsizeMessage*) CkAllocMsg(msgnum, totalsize, priobits);
  // make firstArray point to end of newMsg in memory
  newMsg->firstArray = (int*) ((char*)newMsg + first_start);
  // make secondArray point to after end of firstArray in memory
  newMsg->secondArray = (double*) ((char*)newMsg + second_start);

  return (void*) newMsg;
\}

// returns pointer to memory containing packed message
static void* pack(VarsizeMessage* in)
\{
  // set firstArray an offset from the start of in
  in->firstArray = (int*) ((char*)in->firstArray - (char*)in);
  // set secondArray to the appropriate offset
  in->secondArray = (double*) ((char*)in->secondArray - (char*)in);
  return in;
\}

// returns new message from raw memory
static VarsizeMessage* VarsizeMessage::unpack(void* inbuf)
\{
  VarsizeMessage* me = (VarsizeMessage*)inbuf;
  // return first array to absolute address in memory
  me->firstArray = (int*) ((size_t)me->firstArray + (char*)me);
  // likewise for secondArray
  me->secondArray = (double*) ((size_t)me->secondArray + (char*)me);
  return me;
\}
\end{alltt}

The pointers in a varsize message can exist in two states.  At creation, they
are valid \CC\ pointers to the start of the arrays.  After packing, they become
offsets from the address of the pointer variable to the start of the pointed-to
data.  Unpacking restores them to pointers. 

\subsubsection{Custom Packed Messages}

\index{packed messages}

In many cases, a message must store a {\em non-linear} data structure using
pointers.  Examples of these are binary trees, hash tables etc. Thus, the
message itself contains only a pointer to the actual data. When the message is
sent to the same processor, these pointers point to the original locations,
which are within the address space of the same processor. However, when such a
message is sent to other processors, these pointers will point to invalid
locations.

Thus, the programmer needs a way to ``serialize'' these messages
\index{serialized messages}\index{message serialization}{\em only if} the
message crosses the address-space boundary.  \charmpp{} provides a way to do
this serialization by allowing the developer to override the default
serialization methods generated by the \charmpp{} interface translator.
Note that this low-level serialization has nothing to do with parameter
marshalling or the PUP framework described later.

Packed messages are declared in the {\tt .ci} file the same way as ordinary
messages:

\begin{alltt}
message PMessage;
\end{alltt}

Like all messages, the class \uw{PMessage} needs to inherit from
\uw{CMessage\_PMessage} and should provide two {\em static} methods: \kw{pack}
and \kw{unpack}. These methods are called by the \charmpp\ runtime system, when
the message is determined to be crossing address-space boundary. The prototypes
for these methods are as follows:

\begin{alltt}
static void *PMessage::pack(PMessage *in);
static PMessage *PMessage::unpack(void *in);
\end{alltt}

Typically, the following tasks are done in \kw{pack} method:

\begin{itemize}

\item Determine size of the buffer needed to serialize message data.

\item Allocate buffer using the \kw{CkAllocBuffer} function. This function
takes in two parameters: input message, and size of the buffer needed, and
returns the buffer.

\item Serialize message data into buffer (alongwith any control information
needed to de-serialize it on the receiving side.

\item Free resources occupied by message (including message itself.)  

\end{itemize}

On the receiving processor, the \kw{unpack} method is called. Typically, the
following tasks are done in the \kw{unpack} method:

\begin{itemize}

\item Allocate message using \kw{CkAllocBuffer} function. {\em Do not
use \kw{new} to allocate message here. If the message constructor has
to be called, it can be done using the in-place \kw{new} operator.}

\item De-serialize message data from input buffer into the allocated message.

\item Free the input buffer using \kw{CkFreeMsg}.

\end{itemize}

Here is an example of a packed-message implementation:

\begin{alltt}
// File: pgm.ci
mainmodule PackExample \{
  ...
  message PackedMessage;
  ...
\};

// File: pgm.h
...
class PackedMessage : public CMessage_PackedMessage
\{
  public:
    BinaryTree<char> btree; // A non-linear data structure 
    static void* pack(PackedMessage*);
    static PackedMessage* unpack(void*);
    ...
\};
...

// File: pgm.C
...
void*
PackedMessage::pack(PackedMessage* inmsg)
\{
  int treesize = inmsg->btree.getFlattenedSize();
  int totalsize = treesize + sizeof(int);
  char *buf = (char*)CkAllocBuffer(inmsg, totalsize);
  // buf is now just raw memory to store the data structure
  int num_nodes = inmsg->btree.getNumNodes();
  memcpy(buf, &num_nodes, sizeof(int));  // copy numnodes into buffer
  buf = buf + sizeof(int);               // don't overwrite numnodes
  // copies into buffer, give size of buffer minus header
  inmsg->btree.Flatten((void*)buf, treesize);    
  buf = buf - sizeof(int);              // don't lose numnodes
  delete inmsg;
  return (void*) buf;
\}

PackedMessage*
PackedMessage::unpack(void* inbuf)
\{
  // inbuf is the raw memory allocated and assigned in pack
  char* buf = (char*) inbuf;
  int num_nodes;
  memcpy(&num_nodes, buf, sizeof(int));
  buf = buf + sizeof(int);
  // allocate the message through Charm RTS
  PackedMessage* pmsg = 
    (PackedMessage*)CkAllocBuffer(inbuf, sizeof(PackedMessage));
  // call "inplace" constructor of PackedMessage that calls constructor
  // of PackedMessage using the memory allocated by CkAllocBuffer,
  // takes a raw buffer inbuf, the number of nodes, and constructs the btree
  pmsg = new ((void*)pmsg) PackedMessage(buf, num_nodes);  
  CkFreeMsg(inbuf);
  return pmsg;
\}
... 
PackedMessage* pm = new PackedMessage();  // just like always 
pm->btree.Insert('A');
...
\end{alltt}


While serializing an arbitrary data structure into a flat buffer, one must be
very wary of any possible alignment problems.  Thus, if possible, the buffer
itself should be declared to be a flat struct.  This will allow the \CC\
compiler to ensure proper alignment of all its member fields.



\subsubsection{Immediate Messages}

Immediate messages are special messages that skip the Charm scheduler, they
can be executed in an ``immediate'' fashion even in the middle of 
a normal running entry method. 
They are supported only in nodegroup.




  \subsection{Sequential Objects}
\index{sequential objects} 

These are the same as \CC{} classes and functions.  All \CC{} features can
be used.  However, care needs to be taken when sequential objects
interact with \charmpp\ objects.

  \subsection{Chare Objects}

\index{chare}Chares are concurrent objects with methods that can be invoked
remotely. These methods are known as \index{entry method}entry methods. All
chares must have a constructor that is an entry method, and may have any
number of other entry methods. All chare classes and their entry methods are
declared in the interface (\texttt{.ci}) file:

\begin{alltt}
chare ChareType
\{
    entry ChareType(\uw{parameters1});
    entry void EntryMethodName(\uw{parameters2});
\};
\end{alltt}

Although it is {\em declared} in an interface file, a chare is a \CC{} object and must
have a normal \CC{} {\em implementation} (definition) in addition. A chare
class {\tt ChareType} must inherit from the class {\tt CBase\_ChareType}, which
is a special class that is generated by the \charmpp translator from the
interface file.

To be concrete, the \CC{} definition of the \index{chare}chare above might have 
the following definition in a \texttt{.h} file:

\begin{alltt}
   class ChareType : public CBase\_ChareType \{
       // Data and member functions as in C++
       // One or more entry methods definitions of the form:
       public:
           ChareType(\uw{parameters2});
           void EntryMethodName2(\uw{parameters2});
   \};
\end{alltt}

\index{chare}
Each chare encapsulates data associated with medium-grained units of work in a
parallel application.
Chares can be dynamically created on any processor; there may
be thousands of chares on a processor. The location of a chare is
usually determined by the dynamic load balancing strategy. However,
once a chare commences execution on a processor, it does not migrate
to other processors\footnote{Except when it is part of an array.}.  
Chares do not have a default ``thread of
control'': the entry methods \index{entry methods} in a
chare execute in a message driven fashion upon the arrival of a 
message\footnote{Threaded methods augment this behavior since they execute in
a separate user-level thread, and thus can block to wait for data.}.

The entry method definition specifies a function that is executed {\em without
interruption} when a message is received and scheduled for processing. Only one
message per chare is processed at a time.  Entry methods are defined exactly as
normal \CC{} function members, except that they must have the return value
\kw{void} (except for the constructor entry method which may not have a return
value, and for a {\em synchronous} entry method, which is invoked by a {\em
threaded} method in a remote chare). Each entry method can either take no
arguments, take a list of arguments that the runtime system can automatically
pack into a message and send (see section~\ref{marshalling}), or take a single
argument that is a pointer to a \charmpp message (see section~\ref{messages}).

A chare's entry methods can be invoked via {\it proxies} (see
section~\ref{proxies}). Proxies to a chare of type {\tt chareType} have type
{\tt CProxy\_chareType}. By inheriting from the CBase parent class, each chare
gets a {\tt thisProxy} member variable, which holds a proxy to itself. This
proxy can be sent to other chares, allowing them to invoke entry methods on this
chare.

\zap{
Each chare instance is identified by a {\em handle} \index{handle}
which is essentially a global pointer, and is unique across all
processors.  The handle of a chare has type \kw{CkChareID}.  The
variable \kw{thishandle} holds the handle of the
chare whose entry function or public function is currently executing.
\kw{thishandle} is a public instance variable of the chare object
which is inherited from the system-defined superclass
\kw{CBase}\_\uw{ClassType}.
Following the older syntax, chares are also allowed to inherit directly
for the superclass \kw{Chare} instead of \kw{CBase}\_\uw{ClassType}, although
this form is not suggested.
\kw{thishandle} can be used to set fields in a message. This  
mechanism allows chares to send their handles to other chares.
}

\subsubsection{Chare Creation}

\label{chare creation}

Once you have declared and defined a chare class, you will want to create some
chare objects to use. Chares are created by the {\tt ckNew} method, which is a
static method of the chare's proxy class:

\begin{alltt}
   CProxy_chareType::ckNew(\uw{parameters}, int destPE);
\end{alltt}

The {\tt parameters} correspond to the parameters of the chare's constructor.
Even if the constructor takes several arguments, all of the arguments should be
passed in order to {\tt ckNew}. If the constructor takes no arguments, the
parameters are omitted. By default, the new chare's location is determined by
the runtime system. However, this can be overridden by passing a value for
{\tt destPE}, which specifies the PE where the chare will be created.

The \index{chare}chare creation method deposits the \index{seed}{\em seed} for
a chare in a pool of seeds and returns immediately. The \index{chare}chare will
be created later on some processor, as determined by the dynamic \index{load
balancing}load balancing strategy (or by {\tt destPE}).
When a \index{chare}chare is created, it is
initialized by calling its \index{constructor}constructor \index{entry
method}entry method with the parameters specified by {\tt ckNew}.

Suppose we have declared a chare class {\tt C} with a constructor that takes two
arguments, an {\tt int} and a {\tt double}.

\begin{enumerate}
\item{This will create a new \index{chare}chare of type \uw{C} on {\em any}
processor and return a proxy to that chare:}

\begin{alltt}
   CProxy_C chareProxy = CProxy_C::ckNew(1, 10.0);
\end{alltt} 

\item{This will create a new \index{chare}chare of type \uw{C} on processor
\kw{destPE} and return a proxy to that chare:}

\begin{alltt}
   CProxy_C chareProxy = CProxy_C::ckNew(1, 10.0, destPE);
\end{alltt}

\end{enumerate}

For an example of chare creation in a full application, see
{\tt examples/charm++/fib} in the \charmpp software distribution, which
calculates fibonacci numbers in parallel.

\subsubsection{Method Invocation on Chares}

A message \index{message} may be sent to a \index{chare}chare through a proxy
object using the notation:

\begin{tabbing}
chareProxy.EntryMethod(\uw{parameters})
\end{tabbing}

This invokes the entry method \uw{EntryMethod} on the chare referred
to by the proxy \uw{chareProxy}. This call
is asynchronous and non-blocking; it returns immediately after sending the
message. 


\subsubsection{Local Access}

You can get direct access to a local chare using the
proxy's \kw{ckLocal} method, which returns an ordinary \CC\ pointer
to the chare if it exists on the local processor, and NULL otherwise.

\begin{alltt}
C *c=chareProxy.ckLocal();
if (c==NULL) //...is remote-- send message
else //...is local-- directly use members and methods of c
\end{alltt}


  \subsection{Group Objects}
\label{sec:group}

A \kw{group}\footnote{Originally called {\em Branch Office Chare} or 
{\em Branched Chare}} \index{group}is a collection of chares where 
there exists \index{chare}one chare (or {\sl branch}) on each
processor.   Each branch has its own data members.  Groups have
a definition syntax similar to normal chares,
and they have to inherit from the system defined class \kw{CBase}\_\uw{ClassName}.

In the interface file, we declare

\begin{alltt}
 group GroupType \{
   // Interface specifications as for normal chares
 \};
\end{alltt}

In the \texttt{.h} file, we define \uw{GroupType} as follows:

\begin{alltt}
 class GroupType : public CBase\_GroupType \{
  // Data and member functions as in C++
  // Entry functions as for normal chares
 \};
\end{alltt}

A group is identified by a globally unique group identifier, whose type is
\kw{CkGroupID}. This identifier is common to all of the group's branches and
can be obtained from the variable \kw{thisgroup}, which is a public local
variable of the \kw{Group} superclass.  For groups, \kw{thishandle} is the
handle of the particular branch in which the function is executing: it is a
normal chare handle.

Groups can be used to implement data-parallel operations easily.  In addition
to sending messages to a particular branch of a group, one can broadcast
messages to all branches of a group.  There can be many instances corresponding
to a group type.  Each instance has a different group handle, and its own set
of branches.

\subsubsection{Group Creation}

Given a \texttt{.ci} file as follows:

\begin{alltt}
group G \{
  entry G(\uw{parameters1});
  entry void someEntry(\uw{parameters2});
\};
\end{alltt}

and the following \texttt{.h} file:

\begin{alltt}
class G : public CBase\_G \{
  public:
    G(\uw{parameters1});
    void someEntry(\uw{parameters2});
\};
\end{alltt}

we can create a \index{group}group in a manner similar to a regular
\index{chare}chare. 

\begin{alltt}
CProxy_G groupProxy = CProxy_G::ckNew(\uw{parameters1});
or
CkGroupID groupId = CProxy_G::ckNew(\uw{parameters1});
CProxy_G groupProxy(groupId);
\end{alltt}

\subsubsection{Method Invocation on Groups}

Before sending a message to a \index{group}group via an entry
method, we need to get a proxy of that group.

A message may be sent to a particular \index{branch}branch of group using the
notation:

\begin{alltt}
 groupProxy[Processor].EntryMethod(\uw{parameters});
\end{alltt}

This sends the given parameters to the \index{branch}branch of
the group referred to by \uw{groupProxy} which is on processor number
\uw{Processor} at the entry method \uw{EntryMethod}, which must be a valid
entry method of that group type. This call is asynchronous and non-blocking; it
returns immediately after sending the message.

A message may be broadcast \index{broadcast} to all branches of a group
(i.e., to all processors) using the notation :

\begin{alltt}
 groupProxy.EntryMethod(\uw{parameters});
\end{alltt}

This sends the given parameters to all branches of the group at
the entry method \uw{EntryMethod}, which must be a valid entry method of that
group type. This call is asynchronous and non-blocking; it returns immediately
after sending the message.


Sequential objects, chares and other groups can gain access to the local
(i.e., on their processor) group object using:

\begin{alltt}
GroupType *g=groupProxy.ckLocalBranch();
\end{alltt}

This call returns a regular \CC\ pointer to the actual object (not a proxy)
referred to by the proxy \uw{groupProxy}.  Once a proxy to the
local branch of a group is obtained, that branch can be accessed as a regular
\CC\ object.  Its public methods can return values, and its public data is 
readily accessible.

Thus a dynamically created \index{chare}chare can call a public method of a
group without needing to know which processor it actually resides: the method
executes in the local \index{branch}branch of the group.

One very nice use of Groups is to reduce the number of messages sent between
processors by collecting the data from all the chares on a single processor
before sending that data to the mainchare.  To do this, create basic chares to
break up the work of a problem.  Also, create a group.  When a particular chare
finishes its work, it reports its findings to the local branch of the group.
When all the chares on one processor are complete, the local branch of the
group can then report to the main chare.  This reduces the number of messages
sent to main from the number of chares created to the number of processors. 








  \subsection{Nodegroup Objects}

{\it Node groups} \index{node groups} \index{nodegroup} are very
similar to the group objects already discussed in that node groups are
collections of chares as well.  Node groups, however, have one chare per node
rather than one chare per processor.  So, each node contains a branch of the
node group, each containing one set of data members.  When an entry method of a
node group is executed, it runs on only one processor within each node.

Node groups have a definition syntax that is very similar to groups.  Rather
than inheriting from the system defined class, \kw{Group}, node groups inherit
from \kw{Nodegroup}.  For example, in the interface file, we declare:

\begin{alltt}
 nodegroup NodegroupType \{
  // Interface specifications as for normal chares
 \};
\end{alltt}

In the {\tt .h} file, we define \uw{NodeGroupType} as follows:

\begin{alltt}
 class NodeGroupType : public Nodegroup [,other superclasses ] \{
  // Data and member functions as in \CC{}
  // Entry functions as for normal chares
 \};
\end{alltt}

Like groups, nodegroups are identified by a globally unique identifier of type
\index{CkGroupID}\kw{CkGroupID}.  Just like with groups, this identifier is
common to all branches of the nodegroup and can be obtained from the variable
\index{thisgroup}\kw{thisgroup}, and once again, \index{thishandle}
\kw{thishandle} is the handle of the particular branch in which the function is
executing.

Node groups may possess \index{exclusive}\kw{exclusive} entry methods.  These
are entry methods that will not run while other other exclusive entry methods
of that node group are running on the same node.  For instructions for making
an entry method exclusive, refer to section \ref{attributes}.

For certain applications, node groups can be used in the place of regular
groups to cut down on messaging overhead when shared memory access is possible.
For example, consider a parallel program that does one calculation that can be
decomposed into several mutually exclusive subcalculations.  The program
distributes the work amongst all of the processors, the subresults are all
stored in the local branch of a group, and when the local branch has recieved
all of its results, it relays everything to one particular processor where the
subresults are put together into the final result.  When normal groups are
used, the number of messages sent is $O$(\# of processors).  However, if node
groups are used, a number of message sends will be replaced by local memory
accesses if there is more than one processor per node.  Instead, the number of
messages sent is $O$(\# of nodes).

Just like groups, there can be many instances corresponding to a single node
group type, and each instance has a different group handle, and its own set of
branches.


\subsubsection{Method Invocation on Nodegroups}

Methods can be invoked either on a particular \index{branch}branch of a
\index{nodegroup}nodegroup by specifying a {\em node number} as a method
parameter. In the absence of such a parameter, the call is treated as broadcast
on a \index{nodegroup}nodegroup, i.e. executed by all nodes. When a method is
invoked on a particular \index{branch}branch of a \index{nodegroup}nodegroup,
it may be executed by {\em ANY} processor in that node. Thus two invocations of
a specific method on a particular \index{branch}branch of a
\index{nodegroup}nodegroup may be carried out {\em simultaneously} by two
different processors of the node. If that method contains code that should be
executed by only one processor at a time, the method should be flagged
\index{exclusive}\kw{exclusive} in the interface file. If a method \uw{M} of a
nodegroup \uw{NG} is marked exclusive, it means that while that method is being
executed by any processor within a node, no other processor within the same
node may execute any other {\em exclusive} method of that
\index{nodegroup}nodegroup \index{branch}branch.  Other processors are free to
execute other {\em non-exclusive} methods of that nodegroup
\index{branch}branch, however.

The local \index{branch}branch of a \kw{nodegroup} can be accessed using
\kw{CkLocalNodeBranch()} function. Thus data members could be accessed/modified
or methods could be invoked on a \index{branch}branch of a \kw{nodegroup} using
this function. Note that such accesses are {\em not} thread-safe by default.
Concurrent invocation of a method on a \index{nodegroup}nodegroup by different
processors within a node may result in unpredictable runtime behavior.  One way
to avoid this is to use node-level locks (described in Converse manual.)

\kw{CkLocalNodeBranch} returns a generic ({\tt void *}) pointer, similar to
\kw{CkLocalBranch}.  Also, the static method \kw{ckLocalNodeBranch} of the
proxy class of appropriate \index{nodegroup}nodegroup can be called to get the
correct type of pointer.

  \subsection{Basic Arrays}

Arrays \index{arrays} are arbitrarily-sized collections of chares.  The
entire array has a globally unique identifier of type \kw{CkArrayID}, and
each element has a unique index of type \kw{CkArrayIndex}.  A \kw{CkArrayIndex}
can be a single integer (i.e. 1D array), several integers (i.e. a
multidimentional array), or an arbitrary string of bytes (e.g. a binary tree
index).

Array elements can be dynamically created and destroyed on any processor,
and messages for the elements will still arrive properly.
Array elements can be migrated at any time, allowing arrays to be efficiently
load balanced.  Array elements can also receive array broadcasts and
contribute to array reductions.

\subsubsection{Declaring a 1D Array}

You can declare a one-dimentional \index{array}\index{chare array}chare array
as:

\begin{alltt}
//In the .ci file:
array [1D] A \{
  entry A(\uw{parameters1});
  entry void someEntry(\uw{parameters2});
\};
\end{alltt}

Just as every Chare inherits from the system class \kw{Chare}, every 
array element inherits from the system class \kw{ArrayElement} (or one
of its subclasses, \kw{ArrayElement1D}, \kw{ArrayElement2D}, or 
\kw{ArrayElement3D}). Just as a Chare inherits ``thisChare'', each
array element inherits ``thisArrayID'', the \kw{CkArrayID} of its array,
and ``thisIndex'', the element's array index.

\begin{alltt}
class A : public ArrayElement1D \{
  public:
    A(\uw{parameters1});
    A(CkMigrateMessage *);

    void someEntry(\uw{parameters2});
\};
\end{alltt}

Note \uw{A}'s odd migration constructor, which is normally empty:

\begin{alltt}
//In the .C file:
A::A(void)
\{
  //...your constructor code...
\}
A::A(CkMigrateMessage *m) \{ \}
\end{alltt}

Read the section ``Migratable Array Elements'' for more
information on the \kw{CkMigrateMessage} constructor. 


\subsubsection{Creating a Simple Array}

You always create an array using the \kw{CProxy\_Array::ckNew}
routine.  This returns a proxy object, which can
be kept, copied, or sent in messages.
To create a 1D \index{array}array containing elements indexed 
(0, 1, ..., \uw{num\_elements}-1), use:

\begin{alltt}
CProxy_A1 a1 = CProxy_A1::ckNew(\uw{parameters},num_elements);
\end{alltt}

For creating higher-dimensional arrays, or for more options
when creating the array, see section~\ref{advanced array create}.


\subsubsection{Messages}

An array proxy responds to the appropriate index call--
for 1D arrays, use [i] or (i); for 2D use (x,y); for 3D
use (x,y,z); and for user-defined types use [f] or (f).

To send a \index{Array message} message to an array element, index the proxy 
and call the method name:

\begin{alltt}
a1[i].doSomething(\uw{parameters});
a3(x,y,z).doAnother(\uw{parameters});
aF[CkArrayIndexFoo(...)].doAgain(\uw{parameters});
\end{alltt}

You may invoke methods on array elements that have not yet
been created-- by default, the system will buffer the message until the
element is created\footnote{However, the element must eventually be 
created-- i.e., within a 3-minute buffering period.}.

Messages are not guarenteed to be delivered in order.
For example, if I invoke a method A, then method B;
it is possible for B to be executed before A.

\begin{alltt}
a1[i].A();
a1[i].B();
\end{alltt}

Messages sent to migrating elements will be delivered after
the migrating element arrives.  It is an error to send 
a message to a deleted array element.


\subsubsection{Broadcasts}
To \index{Array broadcast} broadcast a message to all the current elements of an array, 
simply omit the index, as:

\begin{alltt}
a1.doIt(\uw{parameters}); //<- invokes doIt on each array element
\end{alltt}

The broadcast message will be delivered to every existing array 
element exactly once.  Broadcasts work properly even with ongoing
migrations, insertions, and deletions.


\subsubsection{Reductions on Chare Arrays}
A \index{array reduction}reduction applies a single operation (e.g. add,
max, min, ...) to data items scattered across many processors and
collects the result in one place.  \charmpp{} supports reductions on the
elements of a Chare array.

The data to be reduced comes from each array element, 
which must call the \kw{contribute} method:

\begin{alltt}
ArrayElement::contribute(int nBytes,const void *data,CkReduction::reducerType type);
\end{alltt}

Reductions are described in more detail in Section~\ref{reductions}.


\subsubsection{Destroying Arrays}
To destroy an array element-- detach it from the array,
call its destructor, and release its memory--invoke its 
\kw{Array destroy} method, as:

\begin{alltt}
a1[i].destroy();
\end{alltt}

You must ensure that no messages are sent to a deleted element. 
After destroying an element, you may insert a new element at
its index.




\subsection{Advanced Arrays}

The basic array features described above (creation, messaging,
broadcasts, and reductions) are needed in almost every
\charmpp{} program.  The more advanced techniques that follow
are not universally needed; but are still often useful.


\subsubsection{Declaring 2D, 3D, or User-defined Index Arrays}

\charmpp{} contains direct support for multidimentional and
even user-defined index arrays.  These arrays can be declared as:

\begin{alltt}
//In the .ci file:
message MyMsg;
array [1D] A1 \{ entry A1(); entry void e(\uw{parameters});\}
array [2D] A2 \{ entry A2(); entry void e(\uw{parameters});\}
array [3D] A3 \{ entry A3(); entry void e(\uw{parameters});\}
array [Foo] AF \{ entry AF(); entry void e(\uw{parameters});\}
\end{alltt}

The last declaration expects an array index of type \kw{CkArrayIndex}\uw{Foo},
which must be defined before including the \texttt{.decl.h} file 
(see ``User-defined array index type'' below).  

\begin{alltt}
//In the .h file:
class A1:public ArrayElement1D \{ public: A1()\{\} ...\};
class A2:public ArrayElement2D \{ public: A2()\{\} ...\};
class A3:public ArrayElement3D \{ public: A3()\{\} ...\};
class AF:public ArrayElementT<foo> \{ public: AF()\{\} ...\};
\end{alltt}

A 1D array element can access its index via its inherited ``thisIndex''
field; a 2D via ``thisIndex.x'' and ``thisIndex.y'', and a 3D via
``thisIndex.x'', ``thisIndex.y'', and ``thisIndex.z''.  A user-defined
index array can access its index as ``thisIndex''.


\subsubsection{Advanced Array Creation}
\label{advanced array create}
There are several ways to control the array creation process.
You can adjust the map and bindings before creation, change
the way the initial array elements are created, create elements
explicitly during the computation, and create elements implicitly,
``on demand''.  

You can create all your elements using any one of these methods,
or create different elements using different methods.  
An array element has the same syntax and semantics no matter
how it was created.


\subsubsection{Advanced Array Creation: CkCreateOptions}
\index{CkCreateOptions}
\label{CkCreateOptions}

The array creation method \kw{ckNew} actually takes a parameter
of type \kw{CkCreateOptions}.  This object describes several
optional attributes of the new array.

The most common form of \kw{CkCreateOptions} is to set the number
of initial array elements.  A \kw{CkCreateOptions} object will be 
constructed automatically in this special common case.  Thus
the following code segments all do exactly the same thing:

\begin{alltt}
//Implicit CkCreateOptions
  a1=CProxy_A1::ckNew(\uw{parameters},nElements);

//Explicit CkCreateOptions
  a1=CProxy_A1::ckNew(\uw{parameters},CkCreateOptions(nElements));

//Separate CkCreateOptions
  CkCreateOptions opts(nElements);
  a1=CProxy_A1::ckNew(\uw{parameters},opts);
\end{alltt}

Note that the ``numElements'' in an array element is simply
the numElements passed in when the array was created.
The true number of array elements may grow or shrink during 
the course of the computation, so numElements can become 
out of date. 


\subsubsection{Advanced Array Creation: Map Object}
\index{array map}
\label{array map}

You can use \kw{CkCreateOptions} to specify a ``map object''
for an array.  The map object is used by the array manager
to determine the ``home'' processor of each element.  The
home processor is the processor responsible for maintaining
the location of the the element.

There is a default map object, which maps 1D array indices
in a round-robin fashion to processors, and maps other array
indices based on a hash function.

A custom map object is implemented as a group which inherits from
\kw{CkArrayMap} and defines these virtual methods:

\begin{alltt}
class CkArrayMap : public Group
\{
public:
  //...
  
  //Return an ``arrayHdl'', given some information about the array
  virtual int registerArray(int numInitialElements,CkArrayID aid);
  //Return the home processor number for this element of this array
  virtual int procNum(int arrayHdl,const CkArrayIndex &element);
\}
\end{alltt}

Once you've instantiated a custom map object, you can use it to
control the location of a new array's elements using the
\kw{setMap} method of the \kw{CkCreateOptions} object described above.
For example, if you've declared a map object named ``blockMap'':

\begin{alltt}
//Create the map group
  CProxy_blockMap myMap=CProxy_blockMap::ckNew();
//Make a new array using that map
  CkCreateOptions opts(nElements);
  opts.setMap(myMap);
  a1=CProxy_A1::ckNew(\uw{parameters},opts);
\end{alltt}



\subsubsection{Advanced Array Creation: Initial Elements}
\index{array initial}
\label{array initial}

The map object described above can also be used to create
the initial set of array elements in a distributed fashion.
An array's initial elements are created by its map object,
by making a call to \kw{populateInitial} on each processor.

You can create your own set of elements by creating your
own map object and overriding this virtual function of \kw{CkArrayMap}:

\begin{alltt}
  virtual void populateInitial(int arrayHdl,int numInitial,
	void *msg,CkArrMgr *mgr)
\end{alltt}

In this call, \kw{arrayHdl} is the value returned by \kw{registerArray},
\kw{numInitial} is the number of elements passed to \kw{CkCreateOptions},
\kw{msg} is the constructor message to pass, and \kw{mgr} is the
array to create.

\kw{populateInitial} creates new array elements using the method
\kw{void CkArrMgr::insertInitial(CkArrayIndex idx,void *ctorMsg)}.
For example, to create one row of 2D array elements on each processor,
you would write:

\begin{alltt}
void xyElementMap::populateInitial(int arrayHdl,int numInitial,
	void *msg,CkArrMgr *mgr)
\{
  if (numInitial==0) return; //No initial elements requested
	
  //Create each local element
  int y=CkMyPe();
  for (int x=0;x<numInitial;x++) \{
    mgr->insertInitial(CkArrayIndex2D(x,y),CkCopyMsg(&msg));
  \}
  mgr->doneInserting();
  CkFreeMsg(msg);
\}
\end{alltt}

Thus calling \kw{ckNew(10)} on a 3-processor machine would result in
30 elements being created.


\subsubsection{Advanced Array Creation: Bound Arrays}
\experimental{}
\index{bound arrays} \index{bindTo}
\label{bound arrays}
You can ``bind'' a new array to an existing array
using the \kw{bindTo} method of \kw{CkCreateOptions}.  Bound arrays
act like separate arrays in all ways except for migration--
corresponding elements of bound arrays always migrate together.
For example, this code creates two arrays A and B which are
bound together-- A[i] and B[i] will always be on the same processor.

\begin{alltt}
//Create the first array normally
  aProxy=CProxy_A::ckNew(\uw{parameters},nElements);
//Create the second array bound to the first
  CkCreateOptions opts(nElements);
  opts.bindTo(aProxy);
  bProxy=CProxy_B::ckNew(\uw{parameters},nElements);
\end{alltt}

Bound arrays are often useful if A[i] and B[i] perform different 
aspects of the same computation, and thus will run most efficiently 
if they lie on the same processor.  Bound array elements are guarenteed
to always be able to interact using \kw{ckLocal} (see 
section~\ref{ckLocal for arrays}), although the local pointer must
be refreshed after any migration.

An arbitrary number of arrays can be bound together--
in the example above, we could create yet another array
C and bind it to A or B.  The result would be the same
in either case-- A[i], B[i], and C[i] will always be
on the same processor.

There is no relationship between the types of bound arrays--
it is permissible to bind arrays of different types or of the
same type.  It is also permissible to have different numbers
of elements in the arrays, although elements of A which have
no corresponding element in B obey no special semantics.
Any method may be used to create the elements of any bound
array.


\subsubsection{Advanced Array Creation: Dynamic Insertion}

In addition to creating initial array elements using ckNew,
you can also
create array elements during the computation.

You insert elements into the array by indexing the proxy
and calling insert.  The insert call optionally takes 
parameters, which are passed to the constructor; and a
processor number, where the element will be created.
Array elements can be inserted in any order from 
any processor at any time.  Array elements need not 
be contiguous.

If using \kw{insert} to create all the elements of the array,
you must call \kw{CProxy\_Array::doneInserting} before using
the array.

\begin{alltt}
//In the .C file:
int x,y,z;
CProxy_A1 a1=CProxy_A1::ckNew();  //Creates a new, empty 1D array
for (x=...) \{
   a1[x  ].insert(\uw{parameters});  //Bracket syntax
   a1(x+1).insert(\uw{parameters});  // or equivalent parenthesis syntax
\}
a1.doneInserting();

CProxy_A2 a2=CProxy_A2::ckNew();   //Creates 2D array
for (x=...) for (y=...)
   a2(x,y).insert(\uw{parameters});  //Can't use brackets!
a2.doneInserting();

CProxy_A3 a3=CProxy_A3::ckNew();   //Creates 3D array
for (x=...) for (y=...) for (z=...)
   a3(x,y,z).insert(\uw{parameters});
a3.doneInserting();

CProxy_AF aF=CProxy_AF::ckNew();   //Creates user-defined index array
for (...) \{
   aF[CkArrayIndexFoo(...)].insert(\uw{parameters}); //Use brackets...
   aF(CkArrayIndexFoo(...)).insert(\uw{parameters}); //  ...or parenthesis
\}
aF.doneInserting();

\end{alltt}

The \kw{doneInserting} call starts the the reduction manager (see ``Array
Reductions'') and load balancer (see ``Load Balancer'')-- since
these objects need to know about all the array's elements, they
must be started after the initial elements are inserted.
You may call \kw{doneInserting} multiple times, but only the first
call actually does anything.  You may even \kw{insert} or \kw{destroy}
elements after a call to \kw{doneInserting}, with different semantics-- 
see the reduction manager and load balancer sections for details.

If you do not specify one, the system will choose a procesor to 
create an array element on based on the current map object.



\subsubsection{Advanced Array Creation: Demand Creation}

Normally, invoking an entry method on a nonexistant array
element is an error.  But if you add the attribute
\index{createhere} \index{createhome}
\kw{[createhere]} or \kw{[createhome]} to an entry method,
 the array manager will 
``demand create'' a new element to handle the message.  

With \kw{[createhome]}, the new element
will be created on the home processor, which is most efficient when messages for
the element may arrive from anywhere in the machine. With \kw{[createhere]},
the new element is created on the sending processor, which is most efficient
if when messages will often be sent from that same processor.

The new element is created by calling its default (taking no
paramters) constructor, which must exist and be listed in the .ci file.
A single array can have a mix of demand-creation and
classic entry methods; and demand-created and normally 
created elements.



\subsubsection{User-defined array index type}

\index{Array index type, user-defined}
\charmpp{} array indices are arbitrary collections of integers.
To define a new array index, you create an ordinary C++ class 
which inherits from \kw{CkArrayIndex} and sets the ``nInts'' member
to the length, in integers, of the array index.

For example, if you have a structure or class named ``foo'', you 
can use a \uw{foo} object as an array index by defining the class:

\begin{alltt}
#include <charm++.h>
class CkArrayIndexFoo:public CkArrayIndex \{
    foo f;
public:
    CkArrayIndexFoo(const foo \&in) 
    \{
        f=in;
        nInts=sizeof(f)/sizeof(int);
    \}
    //Not required, but convenient: cast-to-foo operators
    operator foo &() \{return f;\}
    operator const foo &() const \{return f;\}
\};
\end{alltt}

Note that \uw{foo}'s size must be an integral number of integers--
you must pad it with zero bytes if this is not the case.
Also, \uw{foo} must be a simple class-- it cannot contain 
pointers, have virtual functions, or require a destructor.
Finally, there is a \charmpp\ configuration-time option called
CK\_ARRAYINDEX\_MAXLEN \index{CK\_ARRAYINDEX\_MAXLEN} 
which is the largest allowable number of 
integers in an array index.  The default is 3; but you may 
override this to any value by passing ``-DCK\_ARRAYINDEX\_MAXLEN=n'' 
to the \charmpp\ build script as well as all user code. Larger 
values will increase the size of each message.

You can then declare an array indexed by \uw{foo} objects with

\begin{alltt}
//in the .ci file:
array [Foo] AF \{ entry AF(); ... \}

//in the .h file:
class AF:public ArrayElementT<foo>
\{ public: AF() \{\} ... \}

//in the .C file:
    foo f;
    CProxy_AF a=CProxy_AF::ckNew();
    a[CkArrayIndexFoo(f)].insert();
    ...
\end{alltt}

Note that since our CkArrayIndexFoo constructor is not declared
with the explicit keyword, we can equivalently write the last line as:

\begin{alltt}
    a[f].insert();
\end{alltt}

When you implement your array element class, as shown above you 
can inherit from
\kw{ArrayElementT}, a class templated by the index type \uw{foo}.
The array index (an object of type \uw{foo}) is then accessible as ``thisIndex'':
For example:

\begin{alltt}

//in the .C file:
AF::AF()
\{
    foo myF=thisIndex;
    functionTakingFoo(myF);
\}
\end{alltt}


\subsubsection{Migratable Array Elements}
Array objects can \index{migrate}migrate from one PE to another.
For example, the load balancer (see section ``Load Balancing Chare Arrays'')
might migrate array elements to better balance the load between
processors.  For an array element to migrate, it must implement
a pack/unpack or ``pup'' method:

\begin{alltt}
//In the .h file:
class A2:public ArrayElement2D \{
private: //My data members:
    int nt;
    unsigned char chr;
    float flt[7];
    int numDbl;
    double *dbl;
public:	
    //...other declarations

    virtual void pup(PUP::er \&p);
\};

//In the .C file:
void A2::pup(PUP::er \&p)
\{
    ArrayElement2D::pup(p); //<- MUST call superclass's pup routine
    p(nt);
    p(chr);
    p(flt,7);
    p(numDbl);
    if (p.isUnpacking()) dbl=new double[numDbl];
    p(dbl,numDbl);
\}
\end{alltt}

See the \index{PUP} section ``PUP'' for more details on pup routines
and the \kw{PUP::er} type.

The system uses one pup routine to do both packing and unpacking by
passing different types of \kw{PUP::er}s to it.  You can determine
what type of \kw{PUP::er} has been passed to you with the
\kw{isPacking()}, \kw{isUnpacking()}, and \kw{isSizing()} calls.

An array element can migrate by calling the \kw{migrateMe}(\uw{destination
processor}) member function-- this call must be the last action
in an element entry point.  The system can also migrate array elements
for load balancing (see the ``Load Balancing Chare Arrays'' chapter).

To migrate your array element to another processor, the \charmpp{}
runtime will:

\begin{itemize}
\item Call your \kw{ckAboutToMigrate} method
\item Call your \uw{pup} method with a sizing \kw{PUP::er} to determine how 
big a message it needs to hold your element.
\item Call your \uw{pup} method again with a packing \kw{PUP::er} to pack 
your element into a message.
\item Call your element's destructor (killing off the old copy).
\item Send the message (containing your element) across the network.
\item Call your element's migration constructor on the new processor.
\item Call your \uw{pup} method on with an unpacking \kw{PUP::er} to unpack 
the element.
\item Call your \kw{ckJustMigrated} method
\end{itemize}

Migration constructors, then, are normally empty-- all the unpacking
and allocation of the data items is done in the element's \uw{pup} routine.
Deallocation is done in the element destructor as usual.


\subsubsection{Load Balancing Chare Arrays}
\charmpp{} includes a run-time load balancer which works
on array elements.  The idea is to run programs more efficiently
by migrating array elements away from overloaded processors.
To use the load balancer, you must make your elements migratable
(see migration section above) and chose a load balancing 
``strategy'' (see the load balancing section for a description
of load balancing strategy routines).

There are two ways to use the load balancer-- the AtSync
method, for elements that only want to be migrated at certain
times; and the Anytime method, for elements that can migrate
at any time.

The Anytime method is the default.  Your elements may be
asked to migrate any time they are not currently executing
an entry method.

For the AtSync method, set \kw{usesAtSync} to true in your 
array element constructor.  When an element is ready to migrate,
call \kw{ArrayElement::AtSync}.  Once all elements have reached \kw{AtSync}, 
and after the first DoneInserting call,
the load balancer runs and may migrate elements.  Once
all migrations are complete, the load balancer calls 
\kw{ArrayElement::ResumeFromSync}.  You must ensure no messages are
sent to an array element between its calls to \kw{AtSync} and
\kw{ResumeFromSync}. \kw{AtSync}/\kw{ResumeFromSync} is currently
a implemented as a global barrier.


\subsubsection{Local Access}
\experimental{}
\index{ckLocal for arrays}
\label{ckLocal for arrays}
You can get direct access to a local array element using the
proxy's \kw{ckLocal} method, which returns an ordinary \CC\ pointer
to the element if it exists on the local processor; and NULL if
the element does not exist or is on another processor.

\begin{alltt}
A1 *a=a1[i].ckLocal();
if (a==NULL) //...is remote-- send message
else //...is local-- directly use members and methods of a
\end{alltt}

Note that if the element migrates or is deleted, any pointers 
obtained with \kw{ckLocal} are no longer valid.  It is best,
then, to either avoid \kw{ckLocal} or else call \kw{ckLocal} 
each time the element may have migrated; e.g., at the start 
of each entry method.


\subsubsection{Array Section}
\experimental{}

\charmpp{} now supports array section. Array section is a subset of array 
elements in a chare array. You can build a special proxy for a section and do 
multicast via the proxy. Section reduction is not directly supported in the 
section proxy. However, an optimized section multicast/reduction library called 
CkMulticast is provided as a separate library module. 

For each chare array "A" declared in a ci file, the definition of section proxy 
of type "CProxySection\_A" is automatically generated. You can create an array 
section proxy in your application by invoking ckNew() to CProxySection:

\begin{alltt}
  CkArrayIndexMax *elems;    // add array indices
  int numElems;
  CProxySection_Hello proxy = CProxySection_Hello::ckNew(helloArrayID, elems, numElems);
\end{alltt}

Once you have the array section proxy, you can do multicast to all the 
section members, or send messages to one member using its index that
is local to the section, like these:

\begin{alltt}
  CProxySection_Hello proxy;
  proxy.someEntry(...)          // multicast
  proxy[0].someEntry(...)       // send to the first element in the section.
\end{alltt}

You can move the section proxy in a message to another processor, and still 
safely invoke the entry functions in the section proxy.

In the multicast example above, for a section with k members, total number 
of k messages will be sent to all the memebers separately, which is considered 
inefficient when multiple section members are at same processor, in which case 
the messages can be combined into one. To optimize the communication and in 
order to support the section reduction, a separate library called CkMulticast 
is provided.

To use the library, you need to compile and install CkMulticast library and 
link your applications against the library using -module:

\begin{alltt}
  # compile and install the CkMulticast library, do this only once
  cd charm/net-linux/tmp/libs/ck-libs/multicast
  make

  # link CkMulticast library using -module
  charmc  -o hello hello.o -module CkMulticast -language charm++ 
\end{alltt}

CkMulticast library is implemented as a delegation of communication for a 
section proxy. Once an array section proxy is delegated, all the messages sent
from the section proxy will be routed to the local delegation branch and 
handled by it. 

To use the CkMulticast delegation, you need to create the CkMulticastMgr Group 
first, and setup the delegation relationship between the section proxy and 
CkMulticastMgr Group. You only need to create one CkMulticastMgr Group though,
it can serve as multicast/reduction delegation for all array sections you 
create:

\begin{alltt}
  CProxySection_Hello sectProxy = CProxySection_Hello::ckNew(...);
  CkGroupID mCastGrpId = CProxy_CkMulticastMgr::ckNew();
  CkMulticastMgr *mcastGrp = CProxy_CkMulticastMgr(mCastGrpId).ckLocalBranch();

  sectProxy.ckDelegate(mCastGrpId);  //section proxy knows who is the delegation
  mcastGrp->setSection(sectProxy);   //delegation knows whom to delegate

  sectProxy.someEntry(...)           //multicast via delegation library as before
\end{alltt}

Note, to use CkMulticast library, all multicast messages must inherit from 
CkMcastBaseMsg, as following:

\begin{alltt}
class HiMsg : public CkMcastBaseMsg, public CMessage_HiMsg
\{
public:
  int *data;
\};
\end{alltt}

Due to this restriction, you need to define message explicitly for multicast 
entry functions and no parameter marshalling can be used for multicast with 
CkMulticast library.

To use section reduction, the root of the reduction array element need to 
 register a reduction callback function to the CkMulticastMgr delegation:

\begin{alltt}
  CProxySection_Hello sectProxy;
  CkMulticastMgr *mcastGrp = CProxy_CkMulticastMgr(mCastGrpId).ckLocalBranch();
  mcastGrp->setReductionClient(sectProxy, callback, userData);
\end{alltt}

When an array element in a section contributes to the reduction, it needs to 
retrieve the section cookie from the multicast message received, and use the 
cookie when talking to the delegation:

\begin{alltt}
  CkSectionCookie cookie;

  void SayHi(HiMsg *msg)
  \{
    CkGetSectionCookie(cookie, msg);     // update section cookie every time
    int data = thisIndex;
    mcastGrp->contribute(sizeof(int), &data, CkReduction::sum_int, cookie);
  \}
\end{alltt}

Note, cookie is retrieved from the multicast message and contains information
about the multicast spanning tree information and reduction counter.
You need to keep this cookie for next uses(i.e. define cookie outside of the 
entry function instead of a local variable). Using multicast/reduction, you
don't need to worry about array migrations, the CkMulticast library can 
automatically update multicast spanning tree for efficient communication.


  \section{Read-only Data}
\label{readonly}

Since \charmpp\ does not allow global variables,
it provides a special mechanism for sharing
data amongst all objects. {\it Read-only} variables of simple data types or
compound data types including messages and arrays are used to share information
that is obtained only after the program begins execution and does not change
after they are initialized in the dynamic scope of the {\tt main} function of
the \kw{mainchare}. They are broadcast to every PE by the \charmpp\ runtime,
and can be accessed in the same way as \CC ``global'' variables on any PE.
Compound data structures containing pointers can be made available as read-only
variables using read-only messages(see section~\ref{messages}) or read-only arrays(see section~\ref{basic arrays}.  Note that memory
has to be allocated for read-only messages by using \kw{new} to create the
message in the {\tt main} function of the \kw{mainchare}. 

Read-only variables are declared by using the type modifier \kw{readonly},
which is similar to \kw{const} in \CC. Read-only data is specified in the {\tt
.ci} file (the interface file) as: 

\begin{alltt}
 readonly Type ReadonlyVarName;
\end{alltt}

The variable \uw{ReadonlyVarName} is declared to be a read-only variable of
type \uw{Type}. \uw{Type} must be a single token and not a type expression.

\begin{alltt}
 readonly message MessageType *ReadonlyMsgName;
\end{alltt}

The variable \uw{ReadonlyMsgName} is declared to be a read-only message of type
are pointers to message types. In this case, the message will be initialized on
\uw{MessageType}. Pointers are not allowed to be readonly variables unless they
every PE.

\begin{alltt}
 readonly Type ReadonlyArrayName [arraysize];
\end{alltt}

The variable \uw{ReadonlyArrayName} is declared to be a read-only array of type
\uw{Type} with \uw{arraysize} elements. \uw{Type} must be a single token and
not a type expression. The value of \uw{arraysize} must be known at compile
time.

Read-only variables must be declared either as global or as public class static
data in the C/\CC\ implementation files, and these declarations have the usual
form:

\begin{alltt}
 Type ReadonlyVarName;
 MessageType *ReadonlyMsgName;
 Type ReadonlyArrayName [arraysize];
\end{alltt}

Similar declarations preceded by \kw{extern} would appear in the {\tt
.h} file. 

{\it Note:}  The current \charmpp\ translator cannot prevent
assignments to read-only variables.  The user must make sure that no
assignments occur in the program outside of the mainchare constructor.

For concrete examples for using read-only variables, please refer to Travelling
Salesman Problem (TSP) in
\examplerefdir{hello}, and GaussSeidel
elimination in \examplerefdir{gaussSeidel3D}.
    
  \subsection{Quiescence Detection}

In \charmpp, \index{quiescence}quiescence is defined as the state in which no
processor is executing an entry point, and no messages are awaiting processing.

\charmpp\ provides two facilities for detecting quiescence: \kw{CkStartQd} and
\kw{CkWaitQd}. \index{CkStartQd} \index{CkWaitQd}

\kw{CkStartQd} registers with the system a callback that should be made the
next time \index{quiescence}quiescence is detected.  \kw{CkStartQd} takes two
parameters: an index corresponding to the entry function that is to be called,
and a handle to the chare on which that entry function should be called.  The
syntax of this call looks like this:

\begin{tabbing}
~~~~ \=~~~~ \=~~~~ \=~~~~ \=~~~~ \=~~~~ \=~~~~ \=~~~~ \=~~~~ \=~~~~ \kill
\> \kw{CkStartQd}(\kw{int} {\it Index}, \kw{CkChareID} {\it chareID});
\end{tabbing}

To retrieve the corresponding index of a particular \index{entry method}entry
method, you must use a static method contained within the
\index{CProxy}\kw{CProxy} object corresponding to the \index{chare}chare
containing that entry method.  The syntax of this call is as follows:

\begin{tabbing}
~~~~ \=~~~~ \=~~~~ \=~~~~ \=~~~~ \=~~~~ \=~~~~ \=~~~~ \=~~~~ \=~~~~ \kill
\kw{CProxy}\_\uw{ChareName}::\kw{ckIdx}\_\uw{EntryMethod}(\uw{Msg}
*{\it Message});
\end{tabbing}

where {\it chareID} is the name of the chare identifier of the chare containing
the desired entry method, \uw{EntryMethod} is the name of that entry method,
and {\it Message} is a pointer to the kind of message that the desired entry
method takes as a parameter. To make this look a little cleaner, we have
provided a simple macro called \kw{EntryIndex}, which can be used in the place
of this convoluted looking static method call.
\kw{EntryIndex}\index{EntryIndex} takes as parameters the type of chare in
which the entry method is located, the name of the entry method itself, and the
type of message that the entry method takes as a parameter. For example:

\begin{tabbing}
~~~~ \=~~~~ \=~~~~ \=~~~~ \=~~~~ \=~~~~ \=~~~~ \=~~~~ \=~~~~ \=~~~~ \kill
\> \kw{EntryIndex}(\uw{ChareName}, \uw{EntryName}, \uw{MsgName});
\end{tabbing}

Note that ChareName, EntryName, and MsgName are {\bf NOT} variables or
constants. This is text that the preprocessor uses to fill in portions of the
previously mentioned static method call.  Additionally, this macro method will
not work with templated chares (refer to Section ~\ref{inheritance and
templates} for details on templated chares).

\index{CkWaitQd}\kw{CkWaitQd}, however, does not register a callback.  Rather,
\kw{CkWaitQd} blocks and does not return until \index{quiescence}quiescence is
detected.  It takes no parameters and returns no value.  A call to
\kw{CkWaitQd} simply looks like this: 

\begin{tabbing}
~~~~ \=~~~~ \=~~~~ \=~~~~ \=~~~~ \=~~~~ \=~~~~ \=~~~~ \=~~~~ \=~~~~ \kill
\> \kw{CkWaitQd}();
\end{tabbing}

Keep in mind that \kw{CkWaitQd} should only be called from threaded
\index{entry method}entry methods because a call to \kw{CkWaitQd} suspends the
current thread of execution, and if it were called outside of a threaded entry
method it would suspend the main thread of execution of the processor from
which \kw{CkWaitQd} was called and the entire program would come to a grinding
halt on that processor.
    
  \subsection{Terminal I/O}

\index{input/output}
\charmpp\ provides both C and \CC\ style methods of doing terminal I/O.  

In place of C-style printf and scanf, \charmpp\ provides
\kw{CkPrintf} and \kw{CkScanf}.  These functions have
interfaces that are identical to their C counterparts, but there are some
differences in their behavior that should be mentioned.

A recent change to \charmpp\ is to also support all forms of printf,
cout, etc. in addition to the special forms shown below.  The special
forms below are still useful, however, since they obey well-defined
(but still lax) ordering requirements.

\function{int CkPrintf(format [, arg]*)} \index{CkPrintf} \index{input/output}
\desc{This call is used for atomic terminal output. Its usage is similar to
\texttt{printf} in C.  However, \kw{CkPrintf} has some special properties
that make it more suited for parallel programming on networks of
workstations.  \kw{CkPrintf} routes all terminal output to the \kw{charmrun},
which is running on the host computer.  So, if a
\index{chare}chare on processor 3 makes a call to \kw{CkPrintf}, that call
puts the output in a TCP message and sends it to host
computer where it will be displayed.  This message passing is an asynchronous
send, meaning that the call to \kw{CkPrintf} returns immediately after the
message has been sent, and most likely before the message has actually
been received, processed, and displayed. \footnote{Because of
communication latencies, the following scenario is actually possible:
Chare 1 does a \kw{Ckprintf} from processor 1, then creates chare 2 on
processor 2.  After chare 2's creation, it calls \kw{CkPrintf}, and the
message from chare 2 is displayed before the one from chare 1.}
}

\function{void CkError(format [, arg]*))} \index{CkError} \index{input/output} 
\desc{Like \kw{CkPrintf}, but used to print error messages on \texttt{stderr}.}

\function{int CkScanf(format [, arg]*)} \index{CkScanf} \index{input/output}
\desc{This call is used for atomic terminal input. Its usage is similar to
{\tt scanf} in C.  A call to \kw{CkScanf}, unlike \kw{CkPrintf},
blocks all execution on the processor it is called from, and returns
only after all input has been retrieved.
}

For \CC\ style stream-based I/O, \charmpp\ offers 
\kw{ckout} and \kw{ckerr} in the place of cout, and cerr.  The
\CC\ streams and their \charmpp\ equivalents are related in the same
manner as printf and scanf are to \kw{CkPrintf} and \kw{CkScanf}.  The
\charmpp\ streams are all used through the same interface as the \CC\ 
streams, and all behave in a slightly different way, just like C-style
I/O.

  \section{Other Calls}

\label{other Charm++ calls}

The following calls provide information about the machines upon which the
parallel program is executing.  Processing Element refers to a single CPU.
Node refers to a single machine-- a set of processing elements which share
memory (i.e. an address space).  Processing Elements and Nodes are numbered,
starting from zero.

Thus if a parallel program is executing on one 4-processor workstation and one
2-processor workstation, there would be 6 processing elements (0, 1 ,2, 3, 4,
and 5) but only 2 nodes (0 and 1).  A given node's processing elements are
numbered sequentially.

\function{int CkNumPes()} \index{CkNumPes}
\desc{returns the total number of processors, across all nodes.}

\function{int CkMyPe()} \index{CkMyPe}
\desc{returns the processor number on which the call was made.}

\function{int CkMyRank()} \index{CkMyRank}
\desc{returns the rank number of the processor on which the call was made.
Processing elements within a node are ranked starting from zero.}

\function{int CkMyNode()} \index{CkMyNode}
\desc{returns the address space number (node number) on which the call was made.}

\function{int CkNumNodes()} \index{CkMyNodes}
\desc{returns the total number of address spaces.}

\function{int CkNodeFirst(int node)} \index{CkNodeFirst}
\desc{returns the processor number of the first processor in this address space.}

\function{int CkNodeSize(int node)} \index{CkNodeSize}
\desc{returns the number of processors in the address space on which the call was made.}

\function{int CkNodeOf(int pe)} \index{CkNodeOf}
\desc{returns the node number on which the call was made.}

\function{int CkRankOf(int pe)} \index{CkRankOf}
\desc{returns the rank of the given processor within its node.}

The following calls provide commonly needed functions.

\function{void CkAbort(const char *message)} \index{CkAbort}
\desc{Cause the program to abort, printing the given error message.}

\function{void CkExit()} \index{CkExit}
\desc{This call informs the Charm kernel that computation on all processors
should terminate.  After the currently executing entry method completes, no
more messages or entry methods will be called.  \kw{CkExit} should be the last
call of the entry method from which it was called.}

\function{double CkTimer()} \index{CkTimer} \index{timers}
\desc{returns the current value of the system timer in milliseconds. The system
timer is started when the program begins execution. This timer measures process
time (user and system).}

\function{double CkWallTimer()} \index{CkWallTimer} \index{timers}
\desc{returns the elapsed time since the program has started from the wall
clock timer.}

    

\newpage
\section{Inheritance and Templates in Charm++}

\label{inheritance and templates}

\charmpp\ supports inheritance among \charmpp\ objects such as
chares, groups, and messages. This, along with facilities for generic
programming using \CC\ style templates for \charmpp\ objects, is a
major enhancement over the previous versions of \charmpp.

\subsection{Chare Inheritance}

\index{inheritance}

Chare inheritance makes it possible to remotely invoke methods of a base
chare \index{base chare} from a proxy of a derived
chare.\index{derived chare} Suppose a base chare is of type 
\uw{BaseChare}, then the derived chare of type \uw{DerivedChare} needs to be
declared in the \charmpp\ interface file to be explicitly derived from
\uw{BaseChare}. Thus, the constructs in the \texttt{.ci} file should look like:

\begin{alltt}
  chare BaseChare \{
    entry BaseChare(someMessage *);
    entry void baseMethod(void);
    ...
  \}
  chare DerivedChare : BaseChare \{
    entry DerivedChare(otherMessage *);
    entry void derivedMethod(void);
    ...
  \}
\end{alltt}

Note that the access specifier \kw{public} is omitted, because \charmpp\
interface translator only needs to know about the public inheritance,
and thus \kw{public} is implicit. A Chare can inherit privately from other
classes too, but the \charmpp\ interface translator does not need to know
about it, because it generates support classes ({\em proxies}) to remotely
invoke only \kw{public} methods.

The class definitions of both these chares should look like:

\begin{alltt}
  class BaseChare : public Chare \{
    // private or protected data
    public:
      BaseChare(someMessage *);
      void baseMethod(void);
  \};
  class DerivedChare : public BaseChare \{
    // private or protected data
    public:
      DerivedChare(otherMessage *);
      void derivedMethod(void);
  \};
\end{alltt}

Now, it is possible to create a derived chare, and invoke methods of base
chare from it, or to assign a derived chare proxy to a base chare proxy
as shown below:

\begin{alltt}
  ...
  otherMessage *msg = new otherMessage();
  CProxy_DerivedChare *pd = new CProxy_DerivedChare(msg);
  pd->baseMethod();     // OK
  pd->derivedMethod();  // OK
  ...
  Cproxy_BaseChare *pb = pd;
  pb->baseMethod();    // OK
  pb->derivedMethod(); // COMPILE ERROR
\end{alltt}

Note that \CC\ calls the default constructor \index{default constructor} of the
base class from any constructor for the derived class where base class
constructor is not called explicitly. Therefore, one should always provide a
default constructor for the base class, or explicitly call another base
class constructor.

Multiple inheritance \index{multiple inheritance} is also allowed for Chares
and Groups. Often, one should make each of the base classes inherit
``virtually'' from \kw{Chare} or \kw{Group}, so that a single copy of
\kw{Chare} or \kw{Group} exists for each multiply derived class.

Entry methods are inherited in the
same manner as methods of sequential \CC{} objects.  
To make an entry method virtual, just add the keyword \kw{virtual}
to the corresponding chare method-- no change is needed in the interface file.
Pure virtual entry methods also require no special description
in the interface file.


\subsection{Inheritance for Messages}

\index{message inheritance}

Messages cannot inherit from other messages.  A message can, however,
inherit from a regular \CC\ class.  For example:

\begin{alltt}
//In the .ci file:
  message BaseMessage1;
  message BaseMessage2;

//In the .h file:
  class Base \{
    // ...
  \};
  class BaseMessage1 : public Base, public CMessage_BaseMessage1 \{
    // ...
  \};
  class BaseMessage2 : public Base, public CMessage_BaseMessage2 \{
    // ...
  \};
\end{alltt}

Messages cannot contain virtual methods
or virtual base classes unless you use a packed message.
Parameter marshalling has complete support for inheritance, virtual
methods, and virtual base classes via the PUP::able framework.


% ( I think the following is now a lie  OSL 7/5/2001 )  
%Similar to Chares, messages can also be derived from base messages. One needs
%to specify this in the \charmpp\ interface file similar to the Chare
%inheritance specification (that is, without the \kw{public} access specifier.)
%Message inheritance makes it possible to send a message of derived type to the
%method expecting a base class message.


\subsection{Generic Programming Using Templates}

\index{templates}

One can write ``templated'' code for Chares, Groups, Messages and other
\charmpp\  entities using familiar \CC\ template syntax (almost). The \charmpp\
interface translator now recognizes most of the \CC\ templates syntax,
including a variety of formal parameters, default parameters, etc. However, not
all \CC\ compilers currently recognize templates in ANSI drafts, therefore the
code generated by \charmpp\ for templates may not be acceptable to some current
\CC\ compilers

\zap{
\newcommand{\longcompilerfootnote}{\footnote{ Most modern \CC\
    compilers belong to one of the two camps. One that supports
    Borland style template instantiation, and the other that supports
    AT\&T Cfront style template instantiation. In the first, code is
    generated for the source file where the instantiation is seen.
    GNU \CC\ falls in this category.  In the second, which template is
    to be instantiated, and where the templated code is seen is noted
    in a separate area (typically a local directory), and then just
    before linking all the template instantiations are
    generated. Solaris CC 5.0 belongs to this category. For templates
    to work for compilers in the first category such as for GNU \CC\
    all the templated code needs to be visible to the compiler at the
    point of instantiation, that is, while compiling the source file
    containing the template instantiation. For a variety of reasons,
    \charmpp\ interface translator cannot generate all the templated
    code in the declarations file {\tt *.decl.h}, which is included in
    the source file where templates are instantiated. Thus, for
    \charmpp\ generated templates to work for GNU \CC\ even parts of
    the definitions file {\tt *.def.h} should be included in the \CC\
    source file. }}
}

Since \CC\ compilers require that
the template definitions (in \emph{addition} to the template
declarations) be available in all sources which use them, you will
need to include the templated Charm definitions in your header file.
That is, given a module {\tt stlib}, in addition to having a line {\tt
  \#include "stlib.decl.h"} in your header file (e.g. {\tt stlib.h}),
you also need the following lines towards the end of the file:

\begin{alltt}
#define CK_TEMPLATES_ONLY
#include "stlib.def.h"
#undef CK_TEMPLATES_ONLY
\end{alltt}

This has the effect of including into the header file only those
declarations which relate to templates.  You will \emph{still} need to
include the file {\tt stlib.def.h} \emph{again} in your implementation
sources (i.e., {\tt stlib.C}) in order to pick up the rest of the
(non-template-related) definitions.  Note that for completely
template-based libraries, this means that you might need to create an
implementation file {\tt stlib.C} when you otherwise wouldn't solely
for the purpose of making sure that the non-template definitions in
{\tt stlib.def.h} are included and compiled.

The \charmpp\ interface file should contain the template
definitions as well as the instantiation. For example, if a message
class \uw{TMessage} is templated with a formal type parameter 
\uw{DType}, then every instantiation of \uw{TMessage} should be specified
in the \charmpp\ interface file. An example will illustrate this better:
\index{template}

\begin{alltt}
  template <class DType=int, int N=3> message TMessage;
  message TMessage<>; // same as TMessage<int,3>
  message TMessage<double>; // same as TMessage<double, 3>
  message TMessage<UserType, 1>;
\end{alltt}

Note the use of default template parameters. It is not necessary for
template definitions and template instantiations to be part of the
same module.  Thus, templates could be defined in one module, and
could be instantiated in another module \index{module}, as long as the
module defining a template is imported into the other module using the
\kw{extern module} construct. Thus it is possible to build a standard
\charmpp\ template library. Here we give a flavor of possibilities:

\begin{alltt}
module SCTL \{
  template <class dtype> message  Singleton;
  template <class dtype> group Reducer \{
    entry Reducer(void);
    entry void submit(Singleton<dtype> *);
  \}
  template <class dtype> chare ReductionClient \{
    entry void recvResult(Singleton<dtype> *);
  \}
\};

module User \{
  extern module SCTL;
  message Singleton<int>;
  group Reducer<int>;
  chare RedcutionClient<int>;
  chare UserClient : ReductionClient<int> \{
    entry UserClient(void);
  \}
\};
\end{alltt}

The \uw{Singleton} message is a template for storing one element of any
\uw{dtype}. The \uw{Reducer} is a group template for a spanning-tree reduction,
which is started by submitting data to the local branch. It also contains a
public method to register the \uw{ReductionClient} (or any of its derived
types), which acts as a callback to receive results of a reduction.


\newpage
\appendix

\section{Compiling, Running and Debugging Charm++/Converse Programs}


\section{\charmpp\ Keywords}
The following is the complete list of keywords in \charmpp:

\begin{enumerate}
\item array
\item char
\item chare
\item class
\item double
\item entry
\item exclusive
\item extern
\item float
\item group
\item int
\item long
\item mainchare
\item mainmodule
\item message
\item module
\item nodegroup
\item packed
\item readonly
\item short
\item stacksize
\item sync 
\item template
\item thisgroup
\item thishandle
\item threaded
\item unsigned 
\item varsize
\item virtual
\item void
\end{enumerate}

The following is the complete list of system operators and calls in \charmpp.
Currently no user-defined functions may have one of these names.

\begin{enumerate}
\item CkAllocBuffer
\item CkAllocMsg
\item CkAllocSysMsg
\item CkArgMsg
\item CkArrayID
\item CkBroadcastMsgBranch
\item CkBroadcastMsgNodeBranch
\item CkChareID
\item CkCopyMsg
\item CkCreateChare
\item CkCreateGroup
\item CkCreateNodeGroup
\item CkError
\item CkExit
\item CkFreeMsg
\item CkFreeSysMsg
\item CkGetChareID
\item CkGetGroupID
\item CkGetNodeGroupID(void);
\item CkGetRefNum
\item CkGetSrcNode
\item CkGetSrcPe
\item CkLocalBranch(int groupID);
\item CkLocalNodeBranch(int groupID);
\item CkMyPe
\item CkMyRank
\item CkMyNode
\item CkNodeFirst
\item CkNodeOf
\item CkNodeSize
\item CkNumNodes
\item CkNumPes
\item CkPrintf
\item CkPriorityPtr
\item CkQdMsg
\item CkRankOf
\item CkRegisterChare
\item CkRegisterEp
\item CkRegisterMainChare
\item CkRegisterMsg
\item CkRegisterReadonly
\item CkRegisterReadonlyMsg
\item CkRemoteCall
\item CkRemoteBranchCall
\item CkRemoteNodeBranchCall
\item CkScanf
\item CkSendMsg
\item CkSendMsgBranch
\item CkSendMsgNodeBranch
\item CkSendToFuture
\item CkSetQueueing
\item CkSetRefNum
\item CkStartQD
\item CkTimer
\item CkWaitQd
\item CkWallTimer
\end{enumerate}


\section{Syntax Changes from \charmpp\ 4.9}

The following changes are required to make older \charmpp\ 4.9
programs run with the new translator and runtime system in
\charmpp\ 5.0.

\begin{itemize}

\item Replace all references to {\tt *.top.h} and {\tt *.bot.h} to
{\tt *.decl.h} and {\tt *.def.h} respectively. This should be done in
Makefile, and everywhere these two types of files are included.

\item Change all X.ci files to include a top-level enclosure of module X {...}.

\item Change \kw{chare} \uw{main} to \kw{mainchare} \uw{main}.

\item Replace \kw{chare\_object} by \kw{Chare}.

\item Replace \kw{groupmember} by \kw{Group}.

\item Replace \kw{comm\_object} by \uw{CMessage\_<msgName>} in all
message declarations in {\tt *.h} files. 

\item Remove all \kw{operator new} methods of messages.

\item Remove \kw{MsgIndex(..)} parameter to \kw{new} for message allocation.

\item Remove all the empty messages (used only for triggering
computations.) from {\tt *.h}, {\tt *.ci}, {\tt *.C} files. All the
entry methods that take these empty messages as parameters should be
made methods with \kw{void} parameter. This should be done in all {\tt
*.ci}, {\tt *.h}, and {\tt *.C} files.  In these methods, there may be
a {\tt delete msg}. Remove that. 

\item Check for \kw{mainhandle} in the source. If it is there, declare
it as a \kw{readonly} variable in {\tt .ci} file, and initialize it in
the \kw{mainchare}'s constructor, so that it is available to all the
processors during the run.

\item All the \kw{packmessage} declarations in {\tt *.ci} files should
be changed to \kw{message [packed]}. 

\item All \kw{CPrintf}, \kw{CScanf}, and \kw{CError} should be changed
to \kw{CkPrintf}, \kw{CkScanf}, and \kw{CkError}. 

\item All \kw{CharmExit} should be changed to \kw{CkExit}.

\item Replace \kw{CMyPe} by \kw{CkMyPe}. Replace \kw{CNumPes} by \kw{CkNumPes}.

\item Change \kw{ChareIDType} to \kw{CkChareID}.

\item Change signature of \uw{M::pack} and \uw{M::unpack} for all
messages \uw{M} in {\tt *.h}, {\tt *.C} to\\
\verb+static void *pack(M* msg)+\\
and\\
\verb+static M* unpack(void *buf)+\\
and change the code
accordingly. This is a significant change because {\tt pack} and {\tt unpack}
used to be instance methods, and now they are class static
methods.  Avenues of optimizations open with this change, but one need
not explore those in the interest of time immediately. Further, one
should make sure that the performance of the new scheme is at least as
good as the old one. 

\item Replace all \kw{new\_group} by \uw{CProxy\_<grpName>::ckNew}.

\item Replace \kw{new\_chare2} by proxy creation.

\item Replace \kw{CSendMsg} by temporary proxy creation based on
ChareID and invoking appropriate method on it. One optimization is to
create a proxy immediately after a ChareID is received, and reusing it
everytime.

\item Similarly replace \kw{CSendMsgBranch} by temporary proxy
creation based on the groupID, and invoking appropriate method on it,
with processor number as the second parameter. Once again, there is an
opportunity for optimization here. 

\item Similarly replace \kw{CBroadcastMsgBranch} by temporary proxy
creation based on the groupID, and invoking appropriate method on it,
without any second parameter. Once again, there is an opportunity for
optimization here. 

\item Replace \kw{CStartQuiescence} by \kw{CkStartQD}.

\item Replace \kw{GetEntryPtr} by appropriate static method
(\kw{ckIdx\_*}) calls. 

\item Replace \kw{CLocalBranch} macros by \kw{ckLocalBranch} instance
method on temporarily created proxy. 

\item Change \kw{CPriorityPtr} to \kw{CkPriorityPtr}, also cast it
explicitly to {\tt (int *)}. 

\item Replace \kw{QuiescenceMessage} by \kw{CkQDMsg}. Remove all
extern declarations of \kw{CkQDMsg} from {\tt *.ci} files.

\end{itemize}


\section{Structured Dagger}
\label{sec:sdag}

\charmpp\ is based on the Message-Driven parallel programming paradigm.  The
message-driven programming style avoids the use of blocking receives and
allows overlap of computation and communication by scheduling computations
depending on availability of data.  This programing style enables \charmpp\
programs to tolerate communication latencies adaptively. Threads suffer from
loss of performance due to context-switching overheads and limited scalability
due to large and unpredictable stack memory requirements, when used in a
data-driven manner to coordinate a sequence of remotely triggered actions.

The need to sequence remotely triggered actions
arises in many situations. Let us consider an example:

%\begin{figure}[ht]
\begin{center}
\begin{alltt}
      class compute_object : public Chare \{
      private:
      int         count;
      Patch       *first, *second;
      public:
      compute_object(MSG *msg) \{
      count = 2; MyChareID(\&chareid);
      PatchManager->Get(msg->first_index, recv_first, \&thishandle,NOWAIT);
      PatchManager->Get(msg->second_index, recv_second, \&thishandle,NOWAIT);
      \}
      void recv_first(PATCH_MSG *msg) \{
       first = msg->patch;
       filter(first);
       if (--count == 0 ) computeInteractions(first,second);
      \} 
      void recv_second(PATCH_MSG *msg)\{
       second = msg->patch;
       filter(second);
       if (--count == 0) computeInteractions(first,second);
      \}
     \}
\end{alltt}
\end{center}
%\caption{Compute Object in a Molecular Dynamics Application}
%\label{figchareexample}
%\end{figure}


Consider an algorithm for computing cutoff-based pairwise interactions
between atoms in a molecular dynamics application, where interaction
between atoms is considered only when they are within some cutoff
distance of each other.  This algorithm is based on a combination of
task and spatial decompositions of the molecular system. The bounding
box for the molecule is divided into a number of cubes ({\em Patches})
each containing some number of atoms.  Since each patch contains a
different number of atoms and these atoms migrate between patches as
simulation progresses, a dynamic load balancing scheme is used. In
this scheme, the task of computing the pairwise interactions between
atoms of all pairs of patches is divided among a number of {\em
Compute Objects}. These compute objects are assigned at runtime to
different processors. The initialization message for each compute
object contains the indices of the patches. The patches themselves are
distributed across processors. Mapping information of patches to
processors is maintained by a replicated object called {\em
PatchManager}.  Figure~\ref{figchareexample} illustrates the \charmpp\
implementation of the compute object. Each compute object requests
information about both patches assigned to it from the
PatchManager. PatchManager then contacts the appropriate processors
and delivers the patch information to the requesting compute
object. The compute object, after receiving information about each
patch, determines which atoms in a patch do not interact with atoms in
another patch since they are separated by more than the cut-off
distance. This is done in method {\tt filter}.  Filtering could be
done after both patches arrive. However, in order to increase
processor utilization, we do it immediately after any patch
arrives. Since the patches can arrive at the requesting compute object
in any order, the compute object has to buffer the received patches,
and maintain state information using counters or flags.  This example
has been chosen for simplicity in order to demonstrate the necessity
of counters and buffers.  In general, a parallel algorithm may have
more interactions leading to the use of many counters, flags, and
message buffers, which complicates program development significantly.

Threads are typically used to perform the abovementioned sequencing.
Lets us code our previous example using threads.

%\begin{figure}[ht]
\begin{center}
\begin{alltt}
void compute_thread(int first_index, int second_index)
\{
    getPatch(first_index);
    getPatch(second_index);
    threadId[0] = createThread(recvFirst);
    threadId[1] = createThread(recvSecond);
    threadJoin(2, threadId);
    computeInteractions(first, second);
  \}
  void recvFirst(void)
  \{
    recv(first, sizeof(Patch), ANY_PE, FIRST_TAG);
    filter(first);
  \}
  void recvSecond(void)
  \{
    recv(second, sizeof(Patch), ANY_PE, SECOND_TAG);
    filter(second);
  \}
\end{alltt}
\end{center}
%\caption{Compute Thread in a Molecular Dynamics Application}
%\label{figthreadexample}
%\end{figure}

Contrast the compute chare-object example in figure~\ref{figchareexample} with
a thread-based implementation of the same scheme in
figure~\ref{figthreadexample}. Functions \uw{getFirst}, and \uw{getSecond} send
messages asynchronously to the PatchManager, requesting that the specified
patches be sent to them, and return immediately. Since these messages with
patches could arrive in any order, two threads, \uw{recvFirst} and
\uw{recvSecond}, are created. These threads block, waiting for messages to
arrive. After each message arrives, each thread performs the filtering
operation. The main thread waits for these two threads to complete, and then
computes the pairwise interactions. Though the programming complexity of
buffering the messages and maintaining the counters has been eliminated in this
implementation, considerable overhead in the form of thread creation, and
synchronization in the form of {\em join} has been added. Let us now code the
same example in \sdag. It reduces the parallel programming complexity without
adding any significant overhead.

%\begin{figure}[ht]
\begin{center}
\begin{alltt}
  array[1D] compute_object \{
    entry void recv_first(Patch *first);
    entry void recv_second(Patch *first);
    entry void compute_object(MSG *msg)\{
      atomic \{
         PatchManager->Get(msg->first_index,\dots);
         PatchManager->Get(msg->second_index,\dots);
      \}
      overlap \{
        when recv_first(Patch *first) atomic \{ filter(first); \}
        when recv_second(Patch *second) atomic \{ filter(second); \}
      \}
      atomic \{ computeInteractions(first, second); \}
    \}
  \}
\end{alltt}
\end{center}
%\caption{\sdag\ Implementation of the Compute Object}
%\label{figsdagexample}
%\end{figure}

\sdag\ is a coordination language built on top of \charmpp\ that supports the
sequencing mentioned above, while overcoming limitations of thread-based
languages, and facilitating a clear expression of flow of control within the
object without losing the performance benefits of adaptive message-driven
execution.  In other words, \sdag\ is a structured notation for specifying
intra-process control dependences in message-driven programs. It combines the
efficiency of message-driven execution with the explicitness of control
specification. \sdag\ allows easy expression of dependences among messages and
computations and also among computations within the same object using
when-blocks and various structured constructs.  \sdag\ is adequate for
expressing control-dependencies that form a series-parallel control-flow graph.
\sdag\ has been developed on top of \charmpp\. \sdag\ allows \charmpp\ entry
methods (in chares, groups or arrays) to specify code (a when-block body) to be
executed upon occurrence of certain events.  These events (or guards of a
when-block) are entry methods of the object that can be invoked remotely. While
writing a \sdag\ program, one has to declare these entries in \charmpp\
interface file. The implementation of the entry methods that contain the
when-block is written using the \sdag\ language. Grammar of \sdag\ is given in
the EBNF form below.

\subsection{Usage}

You can use SDAG to implement entry methods for any chare, chare array, group,
or nodegroup. Any entry method implemented using SDAG must be implemented in the
interface (.ci) file for its class. An SDAG entry method consists of a series of
SDAG constructs of the following kinds:

\begin{itemize}
    \item {\tt atomic} blocks: Atomic blocks simply contain sequential \CC code.
        They're called atomic because the code within them executes without
        interruption from incoming messages. Typically atomic blocks hold the
        code that actually deals with incoming messages in a {\tt when}
        statement, or to do local operations before a message is sent or after
        it's received.
    \item {\tt overlap} blocks: Overlap blocks contain a series of SDAG
        statements within them which can occur in any order. Commonly these
        blocks are used to hold a series of {\tt when} triggers which can be
        received and processed in any order. Flow of control doesn't leave the
        overlap block until all the statements within it have been processed.
    \item {\tt when} statements: These statement, also called triggers, indicate
        that we expect an incoming message of a particular type, and provide
        code to handle that message when it arrives. They commonly occur inside
        of {\tt overlap} blocks, loops, and other control flow statements.
    \item {\tt forall} loops: These loops are used when each iteration of a loop
        can be performed in parallel. This is in contrast to a regular {\tt for}
        loop, in which each iteration is executed sequentially.
    \item {\tt if}, {\tt for}, and {\tt while} statements: these statements have
        the same meaning as the normal {\tt if}, {\tt for}, and {\tt while}
        loops in sequential \CC programs. This allows the programmer to use
        common control flow constructs outside the context of atomic blocks.
\end{itemize}

\sdag{} code can be inserted into the .ci file for any array, group, or chare's entry methods.

If you've added \sdag\ code to your class, you must link in the code by:
\begin{itemize}
  \item Adding ``{\it className}\_SDAG\_CODE'' inside the class declaration
     in the .h file.  This macro defines the entry points and support
     code used by \sdag{}.  Forgetting this results in a compile error
     (undefined sdag entry methods referenced from the .def file).
  \item Adding a call to the routine ``\_\_sdag\_init();'' from every constructor,
     including the migration constructor.  Forgetting this results in
     using uninitalized data, and a horrible runtime crash.
  \item Adding a call to the pup routine ``\_\_sdag\_pup(p);'' from your pup routine.
     Forgetting this results in failure after migration.
\end{itemize}

For example, an array named ``Foo'' that uses sdag code might contain:

\begin{alltt}
class Foo : public CBase_Foo \{
public:
    Foo_SDAG_CODE
    Foo(...) \{
       __sdag_init();
       ...
    \}
    Foo(CkMigrateMessage *m) \{
       __sdag_init();
    \}
    
    void pup(PUP::er &p) \{
       CBase_Foo::pup(p);
       __sdag_pup(p);
    \}
\};
\end{alltt}

For more details regarding \sdag{}, look at the example located in the 
{\tt examples/charm++/hello/sdag} directory in the \charmpp\ distribution.


\subsection{Grammar}

\subsubsection{Tokens}

\begin{alltt}
  <ident> = Valid \CC{} identifier 
  <int-expr> = Valid \CC{} integer expression 
  <\CC{}-code> = Valid \CC{} code 
\end{alltt}

\subsubsection{Grammar in EBNF Form}

\begin{alltt}
<sdag> := <class-decl> <sdagentry>+ 

<class-decl> := "class" <ident> 

<sdagentry> := "sdagentry" <ident> "(" <ident> "*" <ident> ")" <body> 

<body> := <stmt> 
        | "\{" <stmt>+ "\}" 

<stmt> := <overlap-stmt> 
        | <when-stmt> 
        | <atomic-stmt> 
        | <if-stmt> 
        | <while-stmt> 
        | <for-stmt> 
        | <forall-stmt> 

<overlap-stmt> := "overlap" <body> 

<atomic-stmt> := "atomic" "\{" <\CC-code> "\}" 

<if-stmt> := "if" "(" <int-expr> ")" <body> [<else-stmt>] 

<else-stmt> := "else" <body> 

<while-stmt> := "while" "(" <int-expr> ")" <body> 

<for-stmt> := "for" "(" <c++-code> ";" <int-expr> ";" <c++-code> ")" <body> 

<forall-stmt> := "forall" "[" <ident> "]" "(" <range-stride> ")" <body> 

<range-stride> := <int-expr> ":" <int-expr> "," <int-expr> 

<when-stmt> := "when" <entry-list>  <body> 

<entry-list> := <entry> 
              | <entry> [ "," <entry-list> ] 

<entry> := <ident> [ "[" <int-expr> "]" ] "(" <ident> "*" <ident> ")" 
  
\end{alltt}



\chapter{Further Information}

\section{Related Publications}
\label{publications}

For starters, see the publications and reports as well
as related manuals that can be found on the Parallel Programming
Laboratory website: {\tt http://charm.cs.uiuc.edu/}. 

\section{Associated Tools and Libraries}

Several tools and libraries are provided for \charmpp. {\bf
Projections} is an automatic performance analysis tool which provides
the user with information about the parallel behavior of \charmpp\ programs. The purpose of implementing \charmpp standard
libraries is to reduce the time needed to develop parallel
applications with the help of a set of efficient and re-usable modules.

\subsection{Projections}
{\bf Projections} is a
performance visualization and feedback tool. The system has a much
more refined understanding of user computation than is possible in
traditional tools.

Projections displays information about the request for creation and
the actual creation of tasks in \charmpp\ programs. Projections also
provides the function of post-mortem clock
synchronization. Additionally, it can also automatically partition
the execution of the running program into logically separate units,
and automatically analyzes each individual partition. 

Future versions will be able to provide recommendations/suggestions
for improving performance as well.

\subsection{Communication}
Communication optimizations tend to be specific to a particular
architecture or an application. To improve portability and to reduce the
cost of developing parallel applications a mechanism to integrate these
different optimizations should exist. Moreover, it should be possible to
automatically adapt the strategy to the situation at hand. The
communication library integrates the different strategies to perform
each-to-many multicast, including tree-based multicast, grid -based
multicast and hypercube-based (dimensional exchange) schemes. The
framework provided is flexible enough to absorb new strategies and
communication patterns. It also provides the capability to do dynamic
switching of strategies. This helps the library to adapt itself to the
existing environment.

\section{Contacts}
\label{Distribution}

While we can promise neither bug-free software nor immediate solutions   
to all problems, \charmpp\ is a stable system and it is our intention to
keep it as up-to-date and usable as our resources will allow
by responding quickly to questions and bug reports.  To that
end, there are mechanisms in place for contacting Charm users
and developers. 

Our software is made available for research use and evaluation.
For the latest software distribution, further information about {\sc
Charm}/\charmpp\ and information on how to contact the Parallel
Programming laboratory, see our website at {\it
http://charm.cs.uiuc.edu/}.  The software is also available by
anonymous ftp, from a.cs.uiuc.edu, under the directory
pub/research-groups/CHARM.  

If retrieval of a publication via these channels is not possible,
please send electronic mail to {\tt kale@cs.uiuc.edu} or postal mail to:

{\bf 
\begin{tabbing}
\hspace{0.5in}\=\hspace{0.3in}\=\hspace{0.3in}\=\hspace{0.3in}\= \kill
\> Laxmikant Kale \\
\> Department of Computer Science \\
\> University of Illinois \\
\> 1304 West Springfield Avenue \\
\> Urbana, IL 61801 \\
\end{tabbing}
}

A mailing list exists for announcements about software releases and
updates relating to \charmpp/{\sc Converse}.  To subscribe, send
e-mail to: ???????.


\newpage
\addtocontents{toc}{\contentsline {section}{\numberline {}References}{46}}
\bibliographystyle{plain}
\bibliography{group}

\newpage
\addtocontents{toc}{\contentsline {section}{\numberline {}Index}{47}}
\include{index}

\end{document}
