Codes and libraries written in \charmpp{} and MPI can also be used in an
interoperable manner.  Currently, this functionality is supported only if
\charmpp{} is built using MPI, PAMILRTS, or GNI as the network layer (e.g.
mpi-linux-x86\_64 build).  An example program to demonstrate the interoperation
is available in examples/charm++/mpi-coexist. In the following text, we will
refer to this example program for the ease of understanding.

\section{Control Flow and Memory Structure}
The control flow and memory structure of a \charmpp{}-MPI interoperable program  is
similar to that of a pure MPI program that uses external MPI libraries. The
execution of program begins with pure MPI code's {\em main}. At some point after
MPI\_Init() has been invoked, the following function call should be made to initialize
\charmpp{}: \\

{\bf void CharmLibInit(MPI\_Comm newComm, int argc, char **argv)}\\

\noindent If Charm++ is build on top of MPI, {\em newComm} is the MPI
communicator that \charmpp{} will use for the setup and communication. All the
MPI ranks that belong to {\em newComm} should call this function collectively.  A collection of
MPI ranks that make the CharmLibInit call defines a new \charmpp{} instance.
Different MPI ranks that belong to different communicators can make this call
independently, and separate \charmpp{} instances that are not aware of each
other will be created. This results in a space division. As of now, a particular MPI
rank can only be part of one unique \charmpp{} instance. For PAMILRTS and GNI,
the newComm argument is ignored. These layers do not support the space division
of given processors, but require all processors to make the CharmLibInit call.
The mode of interoperating here is called time division, and can be used with
MPI-based Charm++ also if the size of newComm is same as MPI\_COMM\_WORLD.
Arguments {\em argc and argv} should contain the information required by
\charmpp{} such as the load balancing strategy etc.

During the initialization, the control is transferred from MPI to \charmpp{}
RTS on the MPI ranks that made the call. Along with basic setup, \charmpp{} RTS also invokes
the constructors of all mainchares during initialization. Once the initial set up
is done, control is transferred back to MPI as if returning from a function call.
Since \charmpp{} initialization is made via a function call from the pure MPI
program, \charmpp{} resides in the same memory space as the pure MPI program. This
makes transfer of data to/from MPI from/to \charmpp{} convenient (using pointers).

\section{Writing Interoperable \charmpp{} Libraries}
Minor modifications are required to make a \charmpp{} program interoperable with a pure
MPI program:
\begin{itemize}
\item If the interoperable \charmpp{} library does not contain a main
module, the Charm++ RTS provides a main module and the control is returned back to MPI after the
initialization is complete. In the other case, the library should explicitly call CkExit to mark the end of
initialization  and the user should provide {\em -nomain-module} link time flag when
building the final executable.

\item {\em CkExit} should be used the same way {\em return} statement is used for returning
back from a function call. {\em CkExit} should be called only once from one
of the processors. This unique call marks the transfer of control from Charm++ RTS to
MPI.
\item Include {\em mpi-interoperate.h} - if not included in the files that call
{\em CkExit}, invoking {\em CkExit} will result
in unspecified behavior.
\item Since the CharmLibInit call invokes the constructors of mainchares, the
constructors of mainchares should only perform basic set up such as creation of chare
arrays etc, i.e. the set up should not result in invocation of actual work, which
should be done using interface functions (when desired from the pure MPI
program). However, if the main module is provided by the library, CharmLibInit
behaves like a regular Charm++ startup and execution which is stopped when CkExit is
explicitly called by the library.
One can also avoid use of mainchares, and perform the necessary
initializations in an interface function as demonstrated in the interoperable
library examples/charm++/mpi-coexist/libs/hello.
\item Interface functions - Every library needs to define interface function(s)
that can be invoked from pure MPI programs, and transfers the control to the
\charmpp{} RTS. The interface functions are simple functions whose task is to
start work for the \charmpp{} libraries. Here is an example interface function for the
{\em hello} library.
\begin{alltt}
void HelloStart(int elems)
\{
  if(CkMyPe() == 0) \{
    CkPrintf("HelloStart - Starting lib by calling constructor of MainHello\\n");
    CProxy\_MainHello mainhello = CProxy\_MainHello::ckNew(elems);
  \}
  StartCharmScheduler(-1);
\}
\end{alltt}

This function creates a new chare (mainHello) defined in the {\em hello} library which
subsequently results in work being done in {\em hello} library.
More examples of such interface functions can
be found in hi (HiStart) and kNeighbor (kNeighbor) directories in
examples/charm++/mpi-coexist/libs. Note that a scheduler call {\em
StartCharmScheduler()} should be made from the interface functions to start the
message reception by \charmpp{} RTS.
\end{itemize}

\section{Writing Interoperable MPI Programs}
An MPI program that invokes \charmpp{} libraries should include {\em mpi-interoperate.h}.
As mentioned earlier, an initialization call, {\em CharmLibInit} is
required after invoking MPI\_Init to perform the initial set up of \charmpp{}.
It is advisable to call an MPI\_Barrier after a control transfer between \charmpp{}
and MPI to avoid any side effects. Thereafter, a \charmpp{} library can be invoked at
any point using the interface functions. One may look at
examples/charm++/mpi-coexist/multirun.cpp for a working example. Based on the
way interfaces are defined, a library can be invoked multiple times. In the end,
one should call {\em CharmLibExit} to free resources reserved by \charmpp{}.

\section{Compilation and Execution}
An interoperable \charmpp{} library can be compiled as usual using {\em charmc}.
Instead of producing an executable in the end, one should create a library (*.a)
as shown in examples/charm++/mpi-coexist/libs/hi/Makefile. The compilation
process of the MPI program, however, needs modification. One has to include the
charm directory (-I\$(CHARMDIR)/include) to help the compiler find the location of
included {\em mpi-interoperate.h}. The linking step to create the executable
should be done using {\em charmc}, which in turn uses the compiler used to build
charm. In the linking step, it is required to pass {\em -mpi} as an argument
because of which {\em charmc} performs the linking for interoperation. The charm
libraries, which one wants to be linked, should be passed using {\em -module}
option. Refer to examples/charm++/mpi-coexist/Makefile to view a working
example. For execution on BG/Q systems, the following additional argument should be
added to the launch command: {\em --envs PAMI\_CLIENTS=MPI,Converse}.

\section{User Driven Mode}
In addition to the above technique for interoperation, one can also interoperate
with \charmpp{} in user driven mode. User driven mode is intended for cases
where the developer has direct control over the both the \charmpp{} code and
the non-\charmpp{} code, and would like a more tightly coupled relation between
the two. When executing in user driven mode, {\em main} is called on every rank
as in the above example. To initialize the \charmpp{} runtime, a call to
{\em CharmInit} should be called on every rank:

{\bf void CharmInit(int argc, char **argv)} \\

{\em CharmInit} starts the \charmpp{} runtime in user driven mode, and executes
the constructor of the main chare. Control returns to user code when a call to
{\em CkExit} is made. Once control is returned, user code can do other work as
needed, including creating chares, and invoking entry methods on proxies. Any
messages created by the user code will be sent/received the next time the user
calls {\em StartCharmScheduler}. Calls to {\em StartCharmScheduler} allow the
\charmpp{} runtime to resume sending and processing messages, and control
returns to user code when {\em CkExit} is called. The \charmpp{} scheduler can
be started and stopped in this fashion as many times as necessary.
{\em CharmLibExit} should be called by the user code at the end of execution.

A small example of user driven interoperation can be found in
examples/charm++/user-driven-interop.
