\section{\kw{initnode} and \kw{initproc} Routines}
\section{Other Calls}

\label{other Charm++ calls}

The following calls provide information about the machines upon which the
parallel program is executing.  Processing Element refers to a single CPU.
Node refers to a single machine-- a set of processing elements which share
memory (i.e. an address space).  Processing Elements and Nodes are numbered,
starting from zero.

Thus if a parallel program is executing on one 4-processor workstation and one
2-processor workstation, there would be 6 processing elements (0, 1 ,2, 3, 4,
and 5) but only 2 nodes (0 and 1).  A given node's processing elements are
numbered sequentially.

\function{int CkNumPes()} \index{CkNumPes}
\desc{returns the total number of processors, across all nodes.}

\function{int CkMyPe()} \index{CkMyPe}
\desc{returns the processor number on which the call was made.}

\function{int CkMyRank()} \index{CkMyRank}
\desc{returns the rank number of the processor on which the call was made.
Processing elements within a node are ranked starting from zero.}

\function{int CkMyNode()} \index{CkMyNode}
\desc{returns the address space number (node number) on which the call was made.}

\function{int CkNumNodes()} \index{CkMyNodes}
\desc{returns the total number of address spaces.}

\function{int CkNodeFirst(int node)} \index{CkNodeFirst}
\desc{returns the processor number of the first processor in this address space.}

\function{int CkNodeSize(int node)} \index{CkNodeSize}
\desc{returns the number of processors in the address space on which the call was made.}

\function{int CkNodeOf(int pe)} \index{CkNodeOf}
\desc{returns the node number on which the call was made.}

\function{int CkRankOf(int pe)} \index{CkRankOf}
\desc{returns the rank of the given processor within its node.}

The following calls provide commonly needed functions.

\function{void CkAbort(const char *message)} \index{CkAbort}
\desc{Cause the program to abort, printing the given error message.}

\function{void CkExit()} \index{CkExit}
\desc{This call informs the Charm kernel that computation on all processors
should terminate.  After the currently executing entry method completes, no
more messages or entry methods will be called.  \kw{CkExit} should be the last
call of the entry method from which it was called.}

\function{double CkTimer()} \index{CkTimer} \index{timers}
\desc{returns the current value of the system timer in milliseconds. The system
timer is started when the program begins execution. This timer measures process
time (user and system).}

\function{double CkWallTimer()} \index{CkWallTimer} \index{timers}
\desc{returns the elapsed time since the program has started from the wall
clock timer.}



\section{The Sequence of \charmpp{} Program Startup}

%describe the order in which entities are constructed on PE 0 and other PEs
%what assumptions can user program make about entity availability:
%ie groups are available in any chare array constructor, but not vice versa etc.

The \charmpp{} program starts with the following sequence:
\begin{enumerate}
\item Modules Registration: the modules are registered in the same order with
their specified order at the linking stage of the program compilation.
For example, if "-module A -module B" is specified for \charmpp{} program
linking, then module A is registered before module B at runtime.

\item \kw{initnode},\kw{initproc} Calls: all those calls are invoked before the
creation of other \charmpp{} data structures and the invocation of every
mainchares from different modules.

\item \kw{readonly} Variables: those variables are initialized in the mainchare following the program order as written on PE 0. After the initialization, they
are broadcasted to every other PEs making them available in the constructors
of \charmpp{} objects such as Group objects etc..

\item \kw{Group} and \kw{NodeGroup} Creation: on PE 0, constructors of these
objects are invoked in the program order. However, on every other PEs, their
creation are triggered by messages. Since the message order is not guaranteed
in \charmpp{} program, constructors should \textbf{not} depend on other Group
or NodeGroup objects on a PE. In addition, since those objects are initialized
after the initialization of readonly variabales, readonly variables can be used
in their constructors.

\item \charmpp{} Array Creation: the order of calling Array constructors follows
the same mechanism with that of Group and NodeGroup as described above.
Therefore, the creation of one array should \textbf{not} depend on other arrays.
As Array objects are initialized last, their constructors can use 
readonly variables and local branches to Group or NodeGroup objects.
\end{enumerate}
