\section{Interface Description Language (IDL)}

The Interface Description Language(IDL) is the language used to
describe the interfaces that client objects call and server object
implementations provide.

A typical simple IDL interface file follows:

\begin{verbatim}
module myModule {
  interface myClass1 {
    attribute long myAttr1;
    attribute int myAttr2;
    oneway void myMethod1(
        in char c,
        in short s,
        in long l,
        in unsigned short us,
        in unsigned long ul,
        out float f,
        inout double d,
 	out int a[10]);
  };
};
\end{verbatim}

An interface can have attributes as well as operations.  Array
parameters in operation declaration are supported. Templates and String
types are currently not supported.

\noindent {\bf Parameter declaration}:

Parameter declaration attributes include:
\begin{itemize}
\item {\bf in} - the parameter is passed from client to server
\item {\bf out} - the parameter is passed from server to client
\item {\bf inout} - the parameter is passed in both directions
\end{itemize}

\noindent {\bf How to write client/server side code }:

To define the classes declared in the {\tt .idl} file, one must write
{\tt .h} and {\tt .C} file for the classes as usual. For example, in the
{\tt .idl} file, we have:

\begin{verbatim}
module myModule {
  interface myClass1 {
    oneway void myMethod1( in char c);
  };
};
\end{verbatim}

In the  {\tt .h} file:

\begin{verbatim}
class myClass1 {
  public:
    void myMethod1( char );
    myClass1();
    ~myClass1();
};
\end{verbatim}

We implement these methods in the {\tt .C} file.

Instead of using {\tt myClass1} directly, one should declare and use the
class by using the name {\tt CImyClass1}. For example, in {\tt main.C}: 

\begin{verbatim}
main::main(CkArgMsg* m)
{
  CImyClass1 a;
  // create the object
  a.ciSetProc(CI_PE_ANY).ciCreate();
  // call method
  a.myMethod1('a');
  // delete the object
  a.ciDelete();
}
\end{verbatim}

