\charmpp{} offers a couple of checkpoint/restart mechanisms. Each
of these targets a specific need in parallel programming. However,
both of them are based on the same infrastructure.

Traditional chare-array-based \charmpp{} applications, including AMPI
applications, can be checkpointed to storage buffers (either files or
memory regions) and be restarted later from those buffers. The basic
idea behind this is straightforward: checkpointing an application is
like migrating its parallel objects from the processors onto buffers,
and restarting is the reverse. Thanks to the migration utilities like
PUP methods (Section~\ref{sec:pup}), users can decide what data to
save in checkpoints and how to save them. However, unlike migration
(where certain objects do not need a PUP method), checkpoint requires
all the objects to implement the PUP method.

The two checkpoint/restart schemes implemented are:
\begin{itemize}
\item Shared filesystem: provides support for \emph{split execution}, where the
execution of an application is interrupted and later resumed.
\item Double local-storage: offers an online \emph{fault
tolerance} mechanism for applications running on unreliable machines.
\end{itemize}

\section{Split Execution}

There are several reasons for having to split the execution of an
application. These include protection against job failure, a single
execution needing to run beyond a machine's job time limit, and
resuming execution from an intermediate point with different
parameters. All of these scenarios are supported by a mechanism to
record execution state, and resume execution from it later. 

Parallel machines are assembled from many complicated components, each
of which can potentially fail and interrupt execution
unexpectedly. Thus, parallel applications that take long enough to run
from start to completion need to protect themselves from losing work
and having to start over. They can achieve this by periodically taking
a checkpoint of their execution state from which they can later
resume.

Another use of checkpoint/restart is where the total execution
time of the application exceeds the maximum allocation time for a job
in a supercomputer. For that case, an application may checkpoint
before the allocation time expires and then restart from the
checkpoint in a subsequent allocation.

A third reason for having a split execution is when an application
consists in \emph{phases} and each phase may be run a different number
of times with varying parameters. Consider, for instance, an
application with two phases where the first phase only has a possible
configuration (it is run only once). The second phase may have several
configuration (for testing various algorithms). In that case, once the
first phase is complete, the application checkpoints the
result. Further executions of the second phase may just resume from
that checkpoint.

An example of \charmpp{}'s support for split execution can be seen
in \testrefdir{chkpt/hello}.

\subsection{Checkpointing}

\label{sec:diskcheckpoint}
	The API to checkpoint the application is:

\begin{alltt} 
  void CkStartCheckpoint(char* dirname,const CkCallback& cb);
\end{alltt}

The string {\it dirname} is the destination directory where the
checkpoint files will be stored, and {\it cb} is the callback function
which will be invoked after the checkpoint is done, as well as when
the restart is complete. Here is an example of a typical use:

\begin{alltt} 
  . . .  CkCallback cb(CkIndex_Hello::SayHi(),helloProxy);
  CkStartCheckpoint("log",cb);
\end{alltt}

A chare array usually has a PUP routine for the sake of migration.
The PUP routine is also used in the checkpointing and restarting
process.  Therefore, it is up to the programmer what to save and
restore for the application. One illustration of this flexbility is a
complicated scientific computation application with 9 matrices, 8 of
which hold intermediate results and 1 that holds the final results
of each timestep.  To save resources, the PUP routine can well omit
the 8 intermediate matrices and checkpoint the matrix with the final
results of each timestep.

Group and nodegroup objects (Section~\ref{sec:group}) are normally not
meant to be migrated. In order to checkpoint them, however, the user
has to write PUP routines for the groups and declare them as {\tt
[migratable]} in the .ci file. Some programs use {\it mainchares} to
hold key control data like global object counts, and thus mainchares
need to be checkpointed too. To do this, the programmer should write a
PUP routine for the mainchare and declare them as {\tt [migratable]}
in the .ci file, just as in the case of Group and NodeGroup.

The checkpoint must be recorded at a synchronization point in the
application, to ensure a consistent state upon restart. One easy way
to achieve this is to synchronize through a reduction to a single
chare (such as the mainchare used at startup) and have that chare make
the call to initiate the checkpoint.

After {\tt CkStartCheckpoint} is executed, a directory of the
designated name is created and a collection of checkpoint files are
written into it.

\subsection{Restarting}

The user can choose to run the \charmpp{} application in restart mode,
i.e., restarting execution from a previously-created checkpoint. The
command line option {\tt +restart DIRNAME} is required to invoke this
mode. For example:

\begin{alltt}
  > ./charmrun hello +p4 +restart log
\end{alltt}

Restarting is the reverse process of checkpointing. \charmpp{} allows
restarting the old checkpoint on a different number of physical
processors.  This provides the flexibility to expand or shrink your
application when the availability of computing resources changes.

Note that on restart, if an array or group reduction client was set to a static
function, the function pointer might be lost and the user needs to
register it again. A better alternative is to always use an entry method
of a chare object. Since all the entry methods are registered
inside \charmpp{} system, in the restart phase, the reduction client
will be automatically restored.

After a failure, the system may contain fewer or more
processors. Once the failed components have been repaired, some
processors may become available again. Therefore, the user may need
the flexibility to restart on a different number of processors than in
the checkpointing phase. This is allowable by giving a different {\tt
+pN} option at runtime. One thing to note is that the new load
distribution might differ from the previous one at checkpoint time, so
running a load balancer (see Section~\ref{loadbalancing}) after
restart is suggested.

If restart is not done on the same number of processors, the
processor-specific data in a group/nodegroup branch cannot (and
usually should not) be restored individually. A copy from processor 0
will be propagated to all the processors.

\subsection{Choosing What to Save}

In your programs, you may use chare groups for different types of
purposes.  For example, groups holding read-only data can avoid
excessive data copying, while groups maintaining processor-specific
information are used as a local manager of the processor. In the
latter situation, the data is sometimes too complicated to save and
restore but easy to re-compute. For the read-only data, you want to
save and restore it in the PUP'er routine and leave empty the
migration constructor, via which the new object is created during
restart.  For the easy-to-recompute type of data, we just omit the
PUP'er routine and do the data reconstruction in the group's migration
constructor.

A similar example is the program mentioned above, where there are two
types of chare arrays, one maintaining intermediate results while the
other type holds the final result for each timestep. The programmer
can take advantage of the flexibility by leaving PUP'er routine empty
for intermediate objects, and do save/restore only for the important
objects.

\section{Online Fault Tolerance}
\label{sec:MemCheckpointing}
As supercomputers grow in size, their reliability decreases
correspondingly. This is due to the fact that the ability to assemble
components in a machine surpasses the increase in reliability per
component. What we can expect in the future is that applications will
run on unreliable hardware.

The previous disk-based checkpoint/restart can be used as a fault
tolerance scheme. However, it would be a very basic scheme in that
when a failure occurs, the whole program gets killed and the user has
to manually restart the application from the checkpoint files.  The
double local-storage checkpoint/restart protocol described in this
subsection provides an automatic fault tolerance solution. When a
failure occurs, the program can automatically detect the failure and
restart from the checkpoint.  Further, this fault-tolerance protocol
does not rely on any reliable external storage (as needed in the previous
method).  Instead, it stores two copies of checkpoint data to two
different locations (can be memory or local disk).  This double
checkpointing ensures the availability of one checkpoint in case the
other is lost.  The double in-memory checkpoint/restart scheme is
useful and efficient for applications with small memory footprint at
the checkpoint state.  The double in-disk variant stores checkpoints
into local disk, thus can be useful for applications with large memory
footprint.
%Its advantage is to reduce the recovery
%overhead to seconds when a failure occurs.
%Currently, this scheme only supports Chare array-based Charm++ applications.


\subsection{Checkpointing}

The function that application developers can call to record a checkpoint in a
chare-array-based application is:
\begin{alltt}
      void CkStartMemCheckpoint(CkCallback &cb)
\end{alltt}
where {\it cb} has the same meaning as in
section~\ref{sec:diskcheckpoint}.  Just like the above disk checkpoint
described, it is up to the programmer to decide what to save.  The
programmer is responsible for choosing when to activate checkpointing
so that the size of a global checkpoint state, and consequently the
time to record it, is minimized.

In AMPI applications, the user just needs to call the following
function to record a checkpoint:
\begin{alltt}
      void AMPI_MemCheckpoint()
\end{alltt}

\subsection{Restarting}

When a processor crashes, the restart protocol will be automatically
invoked to recover all objects using the last checkpoints. The program
will continue to run on the surviving processors. This is based on the
assumption that there are no extra processors to replace the crashed
ones.

However, if there are a pool of extra processors to replace the
crashed ones, the fault-tolerance protocol can also take advantage of
this to grab one free processor and let the program run on the same
number of processors as before the crash.  In order to achieve
this, \charmpp{} needs to be compiled with the macro option {\it
CK\_NO\_PROC\_POOL} turned on.

\subsection{Double in-disk checkpoint/restart}

A variant of double memory checkpoint/restart, {\it double in-disk
checkpoint/restart}, can be applied to applications with large memory
footprint.  In this scheme, instead of storing checkpoints in the
memory, it stores them in the local disk.  The checkpoint files are
named ``ckpt[CkMyPe]-[idx]-XXXXX'' and are stored under the /tmp
directory.

Users can pass the runtime option {\it +ftc\_disk} to activate this
mode.  For example:

\begin{alltt}
   ./charmrun hello +p8 +ftc_disk
\end{alltt} 

\subsection{Building Instructions}
In order to have the double local-storage checkpoint/restart
functionality available, the parameter \emph{syncft} must be provided
at build time:

\begin{alltt}
   ./build charm++ net-linux-x86_64 syncft
\end{alltt} 

At present, only a few of the machine layers underlying the \charmpp{}
runtime system support resilient execution. These include the
TCP-based \texttt{net} builds on Linux and Mac OS X. For clusters overbearing 
job-schedulers that kill a job if a node goes down, the way to demonstrate the killing 
of a process is show in Section~\ref{ft:inject} . 
\charmpp{} runtime system can automatically detect failures and restart from checkpoint.

\subsection{Failure Injection}
To test that your application is able to successfully recover from
failures using the double local-storage mechanism, we provide a
failure injection mechanism that lets you specify which PEs will fail
at what point in time. You must create a text file with two
columns. The first colum will store the PEs that will fail. The second
column will store the time at which the corresponding PE will
fail. Make sure all the failures occur after the first checkpoint. The
runtime parameter \emph{kill\_file} has to be added to the command
line along with the file name:

\begin{alltt}
   ./charmrun hello +p8 +kill_file <file>
\end{alltt} 

An example of this usage can be found in the \texttt{syncfttest}
targets in \testrefdir{jacobi3d}.

\subsection{Failure Demonstration}
\label{ft:inject}
For HPC clusters, the job-schedulers usually kills a job if a node goes down. To demonstrate
restarting after failures on such clusters, \tt{CkDieNow()} function can be used. You just need to place it at any place
in the code. When it is called by a processor, the processor will hang and stop responding to any communicataion.
A spare processor will replace the crashed processor and continue execution after getting the checkpoint of the crashed processor. 
To make it work, you need to add the command line option
\emph{+wp}, the number following that option is the working processors and the remaining 
are the spare processors in the system.
