\subsection{Checkpoint/Restart}
\index{Checkpoint/Restart}
\label{sec:checkpoint}

\charmpp{} offers fault tolerance capabilities such as 
checkpoint/restart. Usual Chare array-based \charmpp{} application 
can be checkpointed to disk files and later on restarting from the files.

The basic idea behind this is straightforward: Checkpointing an 
application is like migrating its parallel objects from the processors
onto disks, and restarting is the reverse. Thanks to the migration 
utilities like PUP'ing(Section~\ref{sec:pup}), users can decide what 
data to save in checkpoints and how to save them.

\subsubsection{Checkpointing}
	The API to checkpoint the application is:

\begin{alltt} 
  void CkStartCheckpoint(char* dirname,const CkCallback& cb);
\end{alltt}

The string {\it dirname} is the destination directory where the checkpoint
files will be stored, and {\it cb} is the callback function which will be
invoked after the checkpoint is done, as well as when the restart is
complete. Here is an example of a typical use:

\begin{alltt} 
  . . .
  CkCallback cb(CkIndex_Hello::SayHi(),helloProxy);
  CkStartCheckpoint("log",cb);
\end{alltt}

A chare array usually has a PUP routine for the sake of migration. 
The PUP routine is also used in the checkpointing and restarting process.
Therefore, it is up to the programmer what to save and restore for
the application. One illustration of this flexbility is a complicated
scientific computation application with 9 matrices, 8 of which holding 
the intermediate results and 1 holding the final results of each timestep.
To save resource, the PUP routine can well omit the 8 intermediate matrices
and checkpoint the matrix with final results of each timestep. 

Group and nodegroup objects(Section~\ref{sec:group}) are normally not 
meant to be migrated. In order to checkpoint them, however, the user 
wants to write PUP routines for the groups and declare them as 
{\tt [migratable]} in the .ci file.

After {\tt CkStartCheckpoint} is executed, a directory of the designated
name is created and a collection of checkpoint files are written into it. 


\subsubsection{Restarting}

The user can choose to run the \charmpp{} application in restart mode, i.e.,
restarting execution from last checkpoint. The command line option {\tt
-restart DIRNAME} is required to invoke this mode. For example:

\begin{alltt}
  > ./charmrun hello +p4 +restart log
\end{alltt}

Restarting is the reverse process of checkpointing. \charmpp{} allows 
restarting the old checkpoint on different number of physical processor.
This provides the flexibility to expand or shrink your application when
the availability of computing resource changes. 

Note that on restart, if the old reduction client was set to a static 
function, the function pointer might be lost and the user needs to register
it again. A better alternative is to always use entry method of a chare
object. Since all the entry methods are registered inside \charmpp{} system,
in restart phase, the reduction client will be automatically restored.