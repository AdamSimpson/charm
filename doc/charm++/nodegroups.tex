\subsection{NodeGroup Objects}

The {\em node group} construct \index{node groups} \index{nodegroup} \index{Nodegroup} is 
similar to the group construct discussed above. 
That is, a node group is a collection of chares that can be addressed via globally unique
identifier. 
%already discussed in that node groups are
%collections of chares as well.  
However, a node group has one chare per {\em process}, or {\em logical node}, rather than one chare per PE.
%rather than one chare per processor.  
Therefore, each logical node hosts a single branch of the
node group.  When an entry method of a
node group is executed on one of its branches, it executes on {\em some} PE within the node.

\subsubsection{NodeGroup Declaration} 

Node groups are defined in a similar way to groups.  \footnote{As with groups,
older syntax allows node groups to inherit from \kw{NodeGroup} instead of a
specific, generated ``\uw{CBase\_}'' class.} For example, in the interface file, we declare:

\begin{alltt}
 nodegroup NodeGroupType \{
  // Interface specifications as for normal chares
 \};
\end{alltt}

In the {\tt .h} file, we define \uw{NodeGroupType} as follows:

\begin{alltt}
 class NodeGroupType : public CBase_NodeGroupType \{
  // Data and member functions as in \CC{}
  // Entry functions as for normal chares
 \};
\end{alltt}

Like groups, NodeGroups are identified by a globally unique identifier of type
\index{CkGroupID}\kw{CkGroupID}.  Just as with groups, this identifier is
common to all branches of the NodeGroup, and can be obtained from the inherited
data member \index{thisgroup}\kw{thisgroup}.
There can be many instances corresponding to a single NodeGroup
type, and each instance has a different identifier, and its own set of
branches.


%, and once again, \index{thishandle}
%\kw{thishandle} is the handle of the particular branch in which the function is
%executing.


\subsubsection{Method Invocation on NodeGroups}

As with chares, chare arrays and groups, entry methods are invoked on
NodeGroup branches via proxy objects. 
An entry method may be invoked on a {\em particular} \index{branch}branch of a
\index{nodegroup}nodegroup by specifying a {\em logical node number} argument
to the square bracket operator of the proxy object. A broadcast is expressed
by omitting the square bracket notation. For completeness, example syntax for these
two cases is shown below:

\begin{alltt}
 // Invoke `someEntryMethod' on the i-th logical node of
 // a NodeGroup whose proxy is `myNodeGroupProxy':
 myNodeGroupProxy[i].someEntryMethod(\uw{parameters});

 // Invoke `someEntryMethod' on all logical nodes of
 // a NodeGroup whose proxy is `myNodeGroupProxy':
 myNodeGroupProxy[i].someEntryMethod(\uw{parameters});
\end{alltt}

%In the absence of such a parameter, the call is treated as a broadcast
%to all branches of the NodeGroup of the a \index{nodegroup}nodegroup, i.e. executed by all nodes. 
It is worth restating that when an entry method is
invoked on a particular \index{branch}branch of a \index{nodegroup}nodegroup,
it may be executed by {\em any} PE in that logical node. Thus two invocations of
a single entry method on a particular \index{branch}branch of a
\index{nodegroup}NodeGroup may be executed {\em concurrently} by two
different PEs in the logical node. If this may cause data races in your
program, please consult \S~\ref{sec:nodegroups/exclusive} (below).

%If that method contains code that should be
%executed by only one processor at a time, the method should be flagged
%\index{exclusive}\kw{exclusive} in the interface file. 

\subsubsection{NodeGroups and \kw{exclusive} Entry Methods}
\ref{sec:nodegroups/exclusive}

Node groups may have \index{exclusive}\kw{exclusive} entry methods.  The execution of an \kw{exclusive}
entry method invocation is mutually exclusive with those of all other \kw{exclusive} entry methods invocations.
That is, an \kw{exclusive}
entry method invocation is not executed on a logical node as long as another \kw{exclusive} entry 
method is executing on it.
More explicitly, if a method \uw{M} of a
nodegroup \uw{NG} is marked exclusive, it means that while an instance of that method is being
executed by a PE within a logical node, no other PE within that
logical node will execute any other {\em exclusive} methods.
%of that
%\index{nodegroup}nodegroup \index{branch}branch.  
However, PEs in the logical node may still 
execute {\em non-exclusive} entry method invocations.
%on that l
%\index{branch}branch, however.
%of that node group are running on the same node.  
An entry method can be marked exclusive by tagging it with the \kw{exclusive} attribute,
as explained in \S~\ref{attributes}.


\subsubsection{Accessing the Local Branch of a NodeGroup}

The local \index{branch}branch of a \kw{NodeGroup} \uw{NG}, and hence its
member fields and methods, can be accessed through the method \kw{NG*
CProxy\_NG::ckLocalBranch()} of its proxy. Note that accessing data members of
a NodeGroup branch in this manner is {\em not} thread-safe by default, although
you may implement your own mutual exclusion schemes to ensure safety.
%accesses are {\em not} thread-safe by default.  Concurrent invocation of a
%method on a \index{nodegroup}nodegroup by different processors within a node
%may result in unpredictable runtime behavior.  
One way to ensure safety is to use node-level locks, which are described in the
Converse manual.

%For certain applications, node groups can be used in the place of regular
%groups to mitigate messaging overhead when sharing of address spaces between 
%PEs is possible.
%For example, consider a parallel program that does one calculation that can be
%decomposed into several mutually exclusive subcalculations.  The program
%distributes the work amongst all of the processors, the subresults are all
%stored in the local branch of a group, and when the local branch has recieved
%all of its results, it relays everything to one particular processor where the
%subresults are put together into the final result.  When normal groups are
%used, the number of messages sent is $O$(\# of processors).  However, if node
%groups are used, a number of message sends will be replaced by local memory
%accesses if there is more than one processor per node.  Instead, the number of
%messages sent is $O$(\# of nodes).
NodeGroups can be used in a similar way to groups so as to implement lower-level
optimizations such as data sharing and message reduction.


