%\subsection{\kw{initnode} and \kw{initproc} routines}

\index{initcall}
\label{initcall}
Some registration routines need be executed exactly once
before the computation begins. You may choose to 
declare a regular  \CC\ subroutine \kw{initnode} in the .ci file
to ask \charmpp{} to execute the routine exactly once on {\em every logical node} 
before the computation begins, or to declare a regular  \CC\ subroutine 
\kw{initproc} to be executed exactly once on {\em every processor}.

\begin{alltt}
module foo \{
    initnode void fooNodeInit(void);
    initproc void fooProcInit(void);
    chare bar \{
        ...
        initnode void barNodeInit(void);
        initproc void barProcInit(void);
    \};
\};
\end{alltt}

This code will execute the routines \uw{fooNodeInit} and static 
\uw{bar::barNodeInit} once on every logical node and \uw{fooProcInit}
and \uw{bar::barProcInit} on every processor before the main computation 
starts.
Initnode calls are always executed before initproc calls.
Both init calls (declared as static member functions) can be used in chares, 
chare arrays and groups.

Note that these routines should only implement registration and startup functionality, 
and not parallel computation, since
the \charmpp{} run time system will not have started up fully when they are invoked. 
%does not start yet ---
In order to begin the parallel computation, you should 
use a mainchare instead, which gets executed on only PE 0.
%to begin the computation.  
%Initcall routines are typically
%used to do special registrations and global variable setup
%before the computation actually begins.

