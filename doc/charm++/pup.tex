\subsection{PUP}

The \index{PUP} PUP framework is a generic way to describe the data in an object.
The \charmpp\ system can use this description to pack the object 
into a message, and unpack the message into a new object on another 
processor.
The name thus is a contraction of the words Pack and UnPack (PUP).

Like many \CC\ concepts, the PUP framework is easier to use than 
describe: 

\begin{alltt}
class foo \{
 private:
    double a;
    int x;
    char y;
    unsigned long z;
    float q[3];
 public:
    ...other methods...

    //pack/unpack routine: describe my fields to charm++
    void pup(PUP::er &p) \{
      p|a;
      p|x; p|y; p|z;
      p(q,3);
    \}
\};
PUPmarshall(foo); //<- for passing foo's as marshalled parameters
\end{alltt}

This class's pup routine describes the fields of a \uw{foo} to \charmpp\.
This will allow \charmpp\ to: marshall parameters of type \uw{foo},
send \uw{foo} objects in messages, translate \uw{foo}s across processors,
inspect and modify \uw{foo} objects in the debugger, and checkpoint 
calculations involving \uw{foo}s.

The \kw{PUPmarshall(\uw{foo})} statement allows foo objects to be
marshalled properly when passed as parameters to \charmpp\ entry methods.

As shown, data is described by passing it to an object of type 
\index{PUP::er} \kw{PUP::er}.  You can apply a \kw{PUP::er} to any 
primitive variable, as ``p(a)''.
For an array of primitive variables, you apply the array and an integer
that gives the length of the array, as in ``p(q,3)''.  You can pass a 
simple run of bytes 
by calling the \kw{PUP::er} with a ``void *'' and number of bytes
(bytes, however, prevents cross-platform operation).

The system uses the one pup routine to do both packing and unpacking by
passing different types of \kw{PUP::er}s to it.  You can determine
what type of \kw{PUP::er} has been passed to you with the
\kw{p.isPacking()}, \kw{p.isUnpacking()}, and \kw{p.isSizing()} methods.
The \kw{p.isDeleting()} method returns true if the pupped object
will be deleted after packing.

Note that the same data must be passed to the \kw{PUP::er} 
in the same order regardless of the pack direction.
This is the fundamental beauty of the \kw{PUP::er} framework.

The \kw{PUP::er} overhead is very small-- one virtual function call
for each item or array to be packed/unpacked.  The actual packing/unpacking is
normally a simple memory-to-memory binary copy. 

Please note that if your object contains Structured Dagger code (see section ``Structured Dagger'') you must use the following syntax to correctly pup the object:

\begin{alltt}
class bar \{
 private:
    int a,b;
 public:
    bar_SDAG_CODE 
    ...other methods...

    virtual void pup(PUP::er& p) \{
      __sdag_pup(p);
      ...pup other data here...
    \}
\};
\end{alltt}


\subsubsection{C and Fortran bindings}
C and Fortran programmers can use a limited subset of the
\kw{PUP::er} capability.  The routines all take a 
handle named \kw{pup\_er}.  The routines 
have the prototype:
\begin{alltt}
void pup\_\kw{type}(pup\_er p,\kw{type} *val);
void pup\_\kw{type}s(pup\_er p,\kw{type} *vals,int nVals);
\end{alltt}
The first call is for use with a single element;
the second call is for use with an array.
The supported types are char, short, int, long,
uchar, ushort, uint, ulong, float, and double,
which all have the usual C meanings.

A byte-packing routine
\begin{alltt}
void pup\_bytes(pup\_er p,void *data,int nBytes);
\end{alltt}
is also provided, but its use is discouraged
for cross-platform puping.

\kw{pup\_isSizing}, \kw{pup\_isPacking}, \kw{pup\_isUnpacking},
and \kw{pup\_isDeleting} calls are also available.
Since C and Fortran have no destructors, you should 
actually deallocate all data when passed a deleting \kw{pup\_er}.

C and Fortran users cannot use \kw{PUP::able} objects, 
seeking, or write custom \kw{PUP::er}s. Using the \CC\
interface is recommended.

\subsubsection{Dynamic Allocation}

If your class has fields that are dynamically allocated, when unpacking
these need to be allocated (in the usual way) before you pup them.
Deallocation should be left to the class destructor as usual.

\begin{alltt}
class bar \{
private:
    foo *f; /*Heap-allocated foo object*/
    int n;
    double *arr;/*malloc'd array of n doubles*/
public:
    ...other methods...
    //Destructor is the usual
    ~bar() \{free(arr);delete f;\}
    
    //pack/unpack: describe my fields to Charm++
    void pup(PUP::er &p) \{
      if (p.isUnpacking()) 
         f=new foo;
      f->pup(p);//pup f's fields
      
      p|n;//pup the array length n
      if (p.isUnpacking()) 
         arr=(double *)malloc(sizeof(double)*n);
      p(arr,n);//pup array data
    \}
\};
PUPmarshall(bar); //<- for passing foo's as marshalled parameters
\end{alltt}


\subsubsection{Automatic allocation via PUP::able}
If the class \uw{foo} above might have been a subclass, instead of
simply using \uw{new foo} above we would have had to allocate 
an object of the appropriate subclass.  Since determining the
proper subclass and calling the appropriate constructor yourself can be 
difficult, the PUP framework provides a scheme for automatically
determining and dynamically allocating subobjects of the appropriate type.

Your superclass must inherit from \kw{PUP::able}, which provides 
the basic machinery used to move the class.  
A concrete superclass and all its concrete subclasses require these
four features:

\begin{itemize}
\item A line declaring \kw{PUPable \uw{className};} in the .ci file.
This registers the class's constructor.

\item A call to the macro \kw{PUPable\_decl(\uw{className})} in the
class's declaration, in the header file.  This adds a virtual 
method to your class to allow \kw{PUP::able} to determine your class's type.

\item A migration constructor---a constructor that takes \kw{CkMigrateMessage *}.
This is used to create the new object on the receive side, immediately
before calling the new object's \kw{pup} routine.

\item A working, virtual \kw{pup} method.  You can omit this if your
class has no data that needs to be packed.
\end{itemize}

An abstract superclass---a superclass that will never actually be 
packed---only needs to inherit from \kw{PUP::able} and include a 
\kw{PUPable\_abstract(\uw{className})} macro in their body.  The 
.ci file, \kw{PUPable\_decl} macro, and constructor are not needed.

For example, if \uw{parent} is a concrete superclass and \uw{child} its
subclass,

\begin{alltt}
//In the .ci file:
   PUPable parent;
   PUPable child; //Could also have said ``PUPable parent, child;''

//In the .h file:
class parent : public PUP::able \{
    ... data members ...
public:
    ... other methods ...
    parent() \{...\}
    
    //PUP::able support: decl, migration constructor, and pup
    PUPable\_decl(parent);  
    parent(CkMigrateMessage *m) : PUP::able(m) \{\}
    virtual void pup(PUP::er &p) \{
        PUP::able::pup(p);//Call base class
        ... pup data members as usual ...
    \}  
\};
class child : public parent \{
    ... more data members ...
public:    ... more methods, possibly virtual ...
    child() \{...\}
    
    //PUP::able support: decl, migration constructor, and pup
    PUPable\_decl(child);  
    child(CkMigrateMessage *m) : parent(m) \{\}
    virtual void pup(PUP::er &p) \{
        parent::pup(p);//Call base class
        ... pup child's data members as usual ...
    \}  
\};

\end{alltt}

These declarations supercede the \kw{PUPmarshall} declaration 
described earlier.

With these declarations, then, we can automatically 
allocate and pup a pointer to a parent or child
using the vertical bar \kw{PUP::er} syntax, which on the receive
side will create a new object of the appropriate type:

\begin{alltt}
class keepsParent \{
    parent *obj; //May actually point to a child class (or be NULL)
public:
    ...
    ~keepsParent() \{
        delete obj;
    \}
    void pup(PUP::er &p) 
    \{
        p|obj;
    \}
\};
PUPmarshall(keepsParent);
\end{alltt}

This will properly pack, allocate, and unpack obj whether
it is actually a parent or child object.  The child class 
can use all the usual \CC\ features, such as virtual functions
and extra private data.

If obj is NULL when packed, it will be restored to NULL when unpacked.
For example, if the nodes of a binary tree are \kw{PUP::able},
one may write a recursive pup routine for the tree quite easily:

\begin{alltt}
// In the .ci file:
    PUPable treeNode;

// In the .h file
class treeNode : public PUP::able \{
    treeNode *left;//Left subtree
    treeNode *right;//Right subtree
    ... other fields ...
public:
    treeNode(treeNode *l=NULL, treeNode *r=NULL);
    ~treeNode() \{delete left; delete right;\}
    
    // The usual PUP::able support:
    PUPable\_decl(treeNode);
    treeNode(CkMigrateMessage *m) : PUP::able(m) \{ left=right=NULL; \}
    void pup(PUP::er &p) \{
        PUP::able::pup(p);//Call base class
        p|left;
        p|right;
        ... pup other fields as usual ...
    \}
\};
\end{alltt}

This same implementation will also work properly even if the tree's
internal nodes are actually subclasses of treeNode.

Finally, a \kw{PUP::able} superclass like \uw{parent} above 
must normally be passed around via a pointer or reference, because the object
might actually be some subclass like \uw{child}.  Because
pointers and references cannot be passed across processors,
for parameter marshalling you must use the special templated 
smart pointer classes \kw{CkPointer} and \kw{CkReference},
which only need to be listed in the .ci file.

A \kw{CkReference} is a read-only reference to a \kw{PUP::able} object---it
is only valid for the duration of the method call.  A \kw{CkPointer}
transfers ownership of the unmarshalled \kw{PUP::able} to the method, so the 
pointer can be kept and the object used indefinitely.  

For example, if the entry method \uw{bar} needs a \kw{PUP::able} \uw{parent}
object for in-call processing, you would use a \kw{CkReference} like this:

\begin{alltt}
// In the .ci file:
    entry void barRef(int x,CkReference<parent> p);

// In the .h file:
    void barRef(int x,parent &p) \{
      // can use p here, but only during this method invocation
    \}
\end{alltt}

If the entry method needs to keep its parameter, use a \kw{CkPointer} like this:
\begin{alltt}
// In the .ci file:
    entry void barPtr(int x,CkPointer<parent> p);

// In the .h file:
    void barPtr(int x,parent *p) \{
      // can keep this pointer indefinitely, but must eventually delete it
    \}
\end{alltt}

Both \kw{CkReference} and \kw{CkPointer} are read-only from the send 
side---unlike messages, which are consumed when sent, the same object 
can be passed to several parameter marshalled entry methods.
In the example above, we could do:

\begin{alltt}
   parent *p=new child;
   someProxy.barRef(x,*p);
   someProxy.barPtr(x,p); // Makes a copy of p
   delete p; // We allocated p, so we destroy it.
\end{alltt}


\subsubsection{Common PUP::ers}
The most common \kw{PUP::er}s used are \kw{PUP::sizer},
\kw{PUP::toMem}, and \kw{PUP::fromMem}.  These are sizing,
packing, and unpacking \kw{PUP::er}s, respectively.

\kw{PUP::sizer} simply sums up the sizes of the native
binary representation of the objects it is passed.
\kw{PUP::toMem} copies the binary representation of the
objects passed into a preallocated contiguous memory buffer.
\kw{PUP::fromMem} copies binary data from a contiguous memory
buffer into the objects passed.  All three support the
\kw{size} method, which returns the number of bytes used
by the objects seen so far.

Other common \kw{PUP::er}s are \kw{PUP::toDisk}, 
\kw{PUP::fromDisk}, and \kw{PUP::xlater}.  The first
two are simple filesystem variants of the \kw{PUP::toMem} 
and \kw{PUP::fromMem} classes; \kw{PUP::xlater} translates
binary data from an unpacking PUP::er into the machine's
native binary format, based on a \kw{machineInfo} structure
that describes the format used by the source machine.


\subsubsection{PUP::seekBlock}
It may rarely occur that you require items to be unpacked
in a different order than they are packed.  That is, you
want a seek capability.  \kw{PUP::er}s support a limited 
form of seeking.

To begin a seek block, create a \kw{PUP::seekBlock} object
with your current PUP::er and the number of ``sections'' to 
create.  Seek to a (0-based) section number
with the seek method, and end the seeking with the endBlock method.
For example, if we have two objects A and B, where A's pup
depends on and affects some object B, we can pup the two with:

\begin{alltt}
void pupAB(PUP::er &p)
\{
  ... other fields ...
  PUP::seekBlock s(p,2); //2 seek sections
  if (p.isUnpacking()) 
  \{//In this case, pup B first
    s.seek(1);
    B.pup(p);
  \}
  s.seek(0);
  A.pup(p,B);
  
  if (!p.isUnpacking()) 
  \{//In this case, pup B last
    s.seek(1);
    B.pup(p);
  \}
  s.endBlock(); //End of seeking block
  ... other fields ...
\};
\end{alltt}

Note that without the seek block, A's fields would be unpacked
over B's memory, with disasterous consequences.
The packing or sizing path must traverse the seek sections
in numerical order; the unpack path may traverse them in any
order.  There is currently a small fixed limit of 3 on the 
maximum number of seek sections.


\subsubsection{Writing a PUP::er}
System-level programmers may occasionally find it useful to define
their own \kw{PUP::er} objects.  The system \kw{PUP::er} class is 
an abstract base class that funnels all incoming pup requests
to a single subroutine:

\begin{alltt}
    virtual void bytes(void *p,int n,size\_t itemSize,dataType t);
\end{alltt}

The parameters are, in order, the field address, the number of items,
the size of each item, and the type of the items. The \kw{PUP::er}
is allowed to use these fields in any way.  However, an isSizing
or isPacking PUP::er may not modify the referenced user data; 
while an isUnpacking PUP::er may not read the original values of 
the user data.  If your PUP::er is not clearly packing (saving values
to some format) or unpacking (restoring values), declare it as 
sizing \kw{PUP::er}.



