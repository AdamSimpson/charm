\section{Quick BigSim Tutorial}
\label{sec:bigtutor}

This is a step-by-step quick tutorial for simple usage of BigSim simulation framework and visualizing its Projections output logs.
For more information, please refer to BigSim and Projections manuals.

\begin{enumerate}

\item download latest version of Charm from website or git repository:\\
        cd $\sim$ \\
        git clone git://charm.cs.illinois.edu/charm.git\\

\item build charm (and AMPI) with bigemulator and bigsim (replace ``linux" with ``darwin" for mac):\\
        cd charm\\
        ./build charm++ netlrts-linux-x86\_64 bigemulator bigsim\\
        ./build AMPI netlrts-linux-x86\_64 bigemulator bigsim\\

\item compile your code using charm or AMPI compilers located in ``netlrts-linux-x86\_64-bigemulator-bigsim/bin", for example:\\
        cd tests/ampi/jacobi3d; make\\

\item run your application emulating the target machine, for example:\\
        ./charmrun +p1 jacobi 4 4 2 5 +vp32 +x32 +y1 +z1 +cth1 +wth1 +bglog\\

\item download BigSim's simulator\\
        cd $\sim$\\
        git clone git://charm.cs.illinois.edu/BigFastSim\\

\item build BigFastSim:\\
        cd BigFastSim/Release\\
        vim makefile  \#change CHARMPATH=\$(HOME)/charm/netlrts-linux-x86\_64-bigemulator-bigsim/\\
        make\\

\item copy simulator to trace files' directory:\\
        cd $\sim$/charm/tests/ampi/jacobi3d\\
        cp $\sim$/BigFastSim/Release/seqSimulator .\\

\item run the simulator with projections output: (to see other options such as changing latency and bandwidth run ``./seqSimulator -help")\\
        ./seqSimulator -tproj\\

\item download and make Projections:\\
        git clone git://charm.cs.illinois.edu/projections.git\\
        cd projections\\
        ant\\

\item run Projections:\\
        ./bin/projections64 \#open tproj.sts file\\

\end{enumerate}

After opening the symbol file (file/open tproj.sts), you can use different features of Projections such as tools/Timelines.
