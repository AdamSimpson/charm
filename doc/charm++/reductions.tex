\section{Reduction Clients}

\label{reductionClients}

After the data is reduced, it is passed to you via a callback object,
as described in section~\ref{callbacks}.  The message passed to
the callback is of type \kw{CkReductionMsg}. Unlike typed reductions
briefed in Section~\ref{reductions}, here we discuss callbacks that take 
\kw{CkReductionMsg*} argument.
The important members of \kw{CkReductionMsg} are
\kw{getSize()}, which returns the number of bytes of reduction data; and
\kw{getData()}, which returns a ``void *'' to the actual reduced data.

The callback to be invoked when the reduction is complete is specified
as an additional parameter to \kw{contribute}. It is an error for chare array elements
to specify different callbacks to the same reduction contribution.
\begin{alltt}
    double forces[2]=get_my_forces();
    // When done, broadcast the CkReductionMsg to ``myReductionEntry''
    CkCallback cb(CkIndex_myArrayType::myReductionEntry(NULL), thisProxy);
    contribute(2*sizeof(double), forces,CkReduction::sum_double, cb);
\end{alltt}

In the case of the reduced version used for synchronization purposes, the
callback parameter will be the only input parameter:
\begin{alltt}
    CkCallback cb(CkIndex_myArrayType::myReductionEntry(NULL), thisProxy);
    contribute(cb);
\end{alltt}

and the corresponding callback function:

\begin{alltt}
void myReductionEntry(CkReductionMsg *msg)
\{
  int reducedArrSize=msg->getSize() / sizeof(double);
  double *output=(double *) msg->getData();
  for(int i=0 ; i<reducedArrSize ; i++)
  \{
   // Do something with the reduction results in each output[i] array element
   .
   .
   .
  \}
  delete msg;
\}
\end{alltt}

(See \examplerefdir{reductions/simple\_reduction} for a complete example).

If the target of a reduction is an entry method defined by a
\emph{when} clause in SDAG(Section~\ref{sec:sdag}), one may wish to set a
reference number (or tag) that SDAG can use to match the resulting
reduction message. To set the tag on a reduction message, call the
\kw{CkCallback::setRefNum(CMK\_REFNUM\_TYPE refnum)} method on the callback passed
to the {\tt contribute()} call.

\section{Defining a New Reduction Type}

\label{new_type_reduction}

It is possible to define a new type of reduction, performing a 
user-defined operation on user-defined data.  This is done by 
creating a {\em reduction function}, which 
combines separate contributions 
into a single combined value.

The input to a reduction function is a list of \kw{CkReductionMsg}s.
A \kw{CkReductionMsg} is a thin wrapper around a buffer of untyped data
to be reduced.  
The output of a reduction function is a single CkReductionMsg
containing the reduced data, which you should create using the
\kw{CkReductionMsg::buildNew(int nBytes,const void *data)} method.  

Thus every reduction function has the prototype:
\begin{alltt}
CkReductionMsg *\uw{reductionFn}(int nMsg,CkReductionMsg **msgs);
\end{alltt}

For example, a reduction function to add up contributions 
consisting of two machine {\tt short int}s would be:

\begin{alltt}
CkReductionMsg *sumTwoShorts(int nMsg,CkReductionMsg **msgs)
\{
  //Sum starts off at zero
  short ret[2]={0,0};
  for (int i=0;i<nMsg;i++) \{
    //Sanity check:
    CkAssert(msgs[i]->getSize()==2*sizeof(short));
    //Extract this message's data
    short *m=(short *)msgs[i]->getData();
    ret[0]+=m[0];
    ret[1]+=m[1];
  \}
  return CkReductionMsg::buildNew(2*sizeof(short),ret);
\}
\end{alltt}

The reduction function must be registered with \charmpp{} 
using \kw{CkReduction::addReducer(reducerFn fn, bool streamable)} from
an \kw{initnode} routine (see section~\ref{initnode} for details
on the \kw{initnode} mechanism). It takes a required parameter,
\kw{reducerFn fn}, a function pointer to the reduction function, and
an optional parameter \kw{bool streamable}, which indicates if the
function is streamable or not (see section~\ref{streamable_reductions}
for more information). \kw{CkReduction::addReducer} returns a
\kw{CkReduction::reducerType} which you can later
pass to \kw{contribute}.  Since \kw{initnode} routines are executed
once on every node, you can safely store the \kw{CkReduction::reducerType}
in a global or class-static variable.  For the example above, the reduction
function is registered and used in the following manner:

\begin{alltt}
//In the .ci file:
  initnode void registerSumTwoShorts(void);

//In some .C file:
/*global*/ CkReduction::reducerType sumTwoShortsType;
/*initnode*/ void registerSumTwoShorts(void)
\{
  sumTwoShortsType=CkReduction::addReducer(sumTwoShorts);
\}

//In some member function, contribute data to the customized reduction:
  short data[2]=...;
  contribute(2*sizeof(short),data,sumTwoShortsType);
\end{alltt}
Note that typed reductions briefed in Section~\ref{reductions}
can also be used for custom reductions. The target reduction client 
can be declared as in Section~\ref{reductions} but the reduction functions
will be defined as explained above.\\
Note that you cannot call \kw{CkReduction::addReducer}
from anywhere but an \kw{initnode} routine.\\
(See \examplerefdir{barnes-charm} for a complete example).


\subsection{Streamable Reductions}

\label{streamable_reductions}

For custom reductions over fixed sized messages, it is often desirable
that the runtime process each contribution in a streaming fashion,
i.e. as soon as a contribution is received from a chare array element,
that data should be combined with the current aggregation of other
contributions on that PE. This results in a smaller memory footprint
because contributions are immediately combined as they come in rather
than waiting for all contributions to be received. Users can write
their own custom streamable reducers by reusing the message memory of
the zeroth message in their reducer function by passing it as the last
argument to \kw{CkReduction::buildNew}:

\begin{alltt}
CkReductionMsg *sumTwoShorts(int nMsg,CkReductionMsg **msgs)
\{
  // reuse msgs[0]'s memory:
  short *retData = (short*)msgs[0]->getData();
  for (int i=1;i<nMsg;i++) \{
    //Sanity check:
    CkAssert(msgs[i]->getSize()==2*sizeof(short));
    //Extract this message's data
    short *m=(short *)msgs[i]->getData();
    retData[0]+=m[0];
    retData[1]+=m[1];
  \}
  return CkReductionMsg::buildNew(2*sizeof(short), retData, sumTwoShortsReducer, msgs[0]);
\}
\end{alltt}

Note that \emph{only message zero} is allowed to be reused.
For reducer functions that do not operate on fixed sized messages,
such as set and concat, streaming would result in quadratic memory
allocation and so is not desirable. Users can specify that a custom
reducer is streamable when calling \kw{CkReduction::addReducer} by
specifying an optional boolean parameter (default is false):

\begin{alltt}
static void initNodeFn(void) {
    sumTwoShorts = CkReduction::addReducer(sumTwoShorts, /* streamable = */ true);
}
\end{alltt}

