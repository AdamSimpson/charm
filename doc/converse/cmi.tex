\chapter{Converse Machine Interface}

\internal{
\section{Initialization and Wrap-Up}

\function{void ConverseInit(char *argv[])}
\index{ConverseInit}
\desc{This function initializes the machine interface. It should be
called prior to any other Converse functions. 
Multiple calls to this function in a process should
be avoided. \param{argv} is in similar format as passed to
\param{main(argc, argv)}. 
They would be utilized by \param{ConverseInit()} to initialize
machine specific parameters such as number of processors.}

\function{void ConverseExit(void)}
\index{ConverseExit}
\desc{This function frees the resources acquired by Converse and wraps up. 
Any other Converse function should not be called after a call to this function.
\note{It does not terminate the calling
process. A separate call to \param{exit()} is needed after 
\param{ConverseExit()} to achieve this.}}
}


\section{Message Handler Calls}


\function {\#define CmiMsgHeaderSizeBytes}
\index{CmiMsgHeaderSizeBytes}
\desc{This constant, defined in {\tt converse.h}, denotes
the size of the message header in bytes.
Every message buffer must allocate these many bytes at its beginning
for the header.}
 
\function {void CmiSetHandler(int *MessageBuffer, int HandlerId)}
\index{CmiSetHandler}
\desc{This macro sets the handler field of a message to \param{HandlerId}.}
 
\function {int CmiGetHandler(int *MessageBuffer)}
\index{CmiGetHandler}
\desc{This call returns the handler field of a message \param{MessageBuffer}.}
 
\function {HANDLER CmiGetHandlerFunction(int *MessageBuffer)}
\index{CmiGetHandlerFunction}
\desc{This call returns the handler function pointer for a message 
\param{MessageBuffer}. This
usually involves looking up a table using the handler-id stored in
the message header at the sending processor. HANDLER is defined as 
{\tt typedef void (*HANDLER)(void *)} .}
\index{HANDLER}

\function {int CmiRegisterHandler(HANDLER HandlerFunction)}
\index{CmiRegisterHandler}
\desc{This call registers a message handler with the CMI and returns a
handler index which can be subsequently used to specify the handler
for a message. \note{This function should be called
in the same order for all handlers in all the processes belonging to an 
application so that the handler-ids remain consistent across processes.
For this purpose, it is useful to register all handlers at the initialization
stage of an application.}}


\section{Point-To-Point Communication}

\internal{
\function{void *CmiGetSpecificMsg(int HandlerId)}
\index{CmiGetSpecificMsg}
\desc{This call waits until a message for the specified handler is
available, and returns a pointer to the message buffer. Ownership of
the message buffer is maintained with the CMI (e.g. another call to
scheduler or \param{CmiSpecificMsg()} can overwrite the
contents of this buffer.).  The message handler should explicitly call
\param{CmiGrabBuffer()} to acquire ownership of the message buffer.}
}

\internal{
\function{void *CmiGetMsg(void)}
\desc{Retrieves a message from the network's message queue and
returns a pointer to the message buffer. If network's message queue is
empty, this function returns \param{(void *) 0}. Ownership of the
message buffer is maintained with CMI. Another call to \param{CmiGetMsg()}
or \param{CmiSpecificMsg()} could overwrite the contents of buffer.
The calling function should
explicitly call \param{CmiGrabBuffer()} to acquire ownership 
the message buffer.} 
}


\function{void CmiSyncSend(unsigned int destPE, unsigned int size, void *msg)}
\index{CmiSyncSend}
\desc{Sends \param{msg} of size \param{size} bytes to processor
\param{destPE}. Message buffer for \param{msg} could be reused after
the call returns.
The first CmiMsgHeaderSizeBytes bytes of \param{msg} must be set aside
for use by the system.  (This applies to all the calls below that send
or broadcast a message.)}

\function{void CmiSyncSendAndFree(unsigned int destPE, unsigned int size, void *msg)}
\index{CmiSyncSendAndFree}
\desc{Sends \param{msg} of size \param{size} bytes to processor
\param{destPE} and frees the message buffer for \param{msg}.}

\function{CmiCommHandle CmiAsyncSend(unsigned int destPE, unsigned int size, void *msg)}
\index{CmiAsyncSend}
\desc{Initiates an asynchronous send of \param{msg} of length
\param{size} bytes to processor \param{destPE} and returns a
communication handle which could be used to enquire the status of this
communication. Message buffer for \param{mesg} should not be reused or freed
until communication is complete.}

\function{int CmiAsyncMsgSent(CmiCommHandle handle)}
\index{CmiAsyncMsgSent}
\desc{Returns the status of asynchronous send specified by
communication handle \param{handle}.}

\function{void CmiReleaseCommHandle(CmiCommHandle handle)}
\index{CmiReleaseCommHandle}
\desc{Releases the communication handle \param{handle} and
associated resources. It does not free the message buffer.
\param{handle} could be reused by CMI for another communication after
this call succeeds.}

\internal{
\function{CommHandle CmiVectorSend(int destPE, int HandlerId, int len, int sizes[], void *DataArray[])}
\desc{Initiates an asynchronous send of data to processor
\param{destPE}.  The data consists of \param{len} pieces residing in
different areas of memory, which are logically concatenated.  The
\param{DataArray} array contains pointers to the pieces; the size of
\param{DataArray[i]} is taken from \param{sizes[i]}. \param{HandlerId}
is inserted at the appropriate place in the combined message and the
corresponding handler will be invoked on the receiving side. This
function returns a communication handle which could be used to enquire
about the status of communication using \param{CmiAsyncMsgSent()}.
Individual pieces of data as well as the arrays \param{sizes} and
\param{DataArray} should not be overwritten or freed before the
communication is complete.}
}

\function{void CmiGrabBuffer(void **pbuf)}
\index{CmiGrabBuffer}
\desc{Transfers the ownership of the buffer pointed to by
\param{*pbuf} to the calling procedure. On machines where 
\param{*pbuf} points to a
system buffer which cannot be allocated to the user, 
CMI copies the buffer contents to newly allocated user
space and updates \param{*pbuf} to point to the new buffer.
In either case, the user should eventually free the buffer by calling
\param{CmiFree}.}

\function{int CmiDeliverMsgs(int MaxMsgs)}
\index{CmiDeliverMsgs}
\desc{Retrieves messages from the network message queue and invokes 
corresponding handler functions for arrived messages. This function 
returns after either the network message queue becomes empty or after
\param{MaxMsgs} messages have been retrieved and their handlers called. 
It returns the difference between total messages delivered and \param{MaxMsgs}.
The handler is given a pointer to the message as  its parameter.
Normally, the delivered message is owned by CMI,  and so the handler
should not free it, or store it for later use.  
If the handler wishes to  use the message later (for example, by
enqueing the message with the scheduler), it must call
\param{CmiGrabBuffer()} to
acquire the  ownership of the message buffer.}

\function{void CmiDeliverSpecificMsg(int HandlerId)}
\index{CmiDeliverSpecificMsg}
\desc{Retrieves messages from the network queue and delivers the first message
with its handler field equal to \param{HandlerId}. This functions queues
the remaining messages retrieved from the network. It returns after the 
invoked handler function returns.}

\internal{
\section{Global Pointer}

\function{int CmiGptrCreate(GlobalPtr *gptr, void *lptr, unsigned int size)}
\desc{This function creates a global pointer by initializing contents of
\param{*gptr} to point to memory on the local processor pointed to by
\param{lptr} of \param{size} bytes. \param{*gptr} could then be sent to other 
processors, and could be used by \param{CmiGet()} and \param{CmiPut()}
to read and write this memory by remote processors. This functions returns
a positive integer on success.}

\function{void *CmiGptrDref(GlobalPtr *gptr)}
\desc{This function returns the address of local memory associated
with global pointer \param{gptr}.}

\function{int CmiSyncGet(GlobalPtr *gptr, void *lptr, unsigned int size)}
\desc{Copies \param{size} bytes from 
memory pointed to by global pointer \param{gptr}
to local memory pointed to by \param{lptr}. 
This is a synchronous operation and the calling processor blocks until
the data is transferred to local memory. This function returns
a positive integer on success.}

\function{CommHandle CmiGet(GlobalPtr *gptr, void *lptr, unsigned int size)}
\desc{Initiates copying of \param{size} bytes from 
memory pointed to by global pointer \param{gptr}
to local memory pointed to by \param{lptr}. 
This function returns a  communication handle which could be used
to  enquire about the status of this operation.}

\function{CommHandle CmiPut(GlobalPtr *gptr, void *lptr, unsigned int size)}
\desc{Initiates copying of \param{size} bytes from a processor's local
memory pointed to by \param{lptr} to the memory pointed to by global
pointer \param{gptr}.  This function returns a  communication handle
which could be used to  enquire about the status of this operation.}
}

\section{Broadcast}

\function{void CmiSyncBroadcast(unsigned int size, void *msg)}
\index{CmiSyncBroadcast}
\desc{Sends \param{msg} of length \param{size} bytes to all processors
excluding the processor on which the caller resides. }

\function{void CmiSyncBroadcastAndFree(unsigned int size, void *msg)}
\index{CmiSyncBroadcastAndFree}
\desc{Sends \param{msg} of length \param{size} bytes to all processors
excluding the processor on which the caller resides. Frees the message buffer
for \param{msg} when the broadcast completes. Therefore \param{msg} must
point to a dynamically allocated buffer.}

\function{void CmiSyncBroadcastAll(unsigned int size, void *msg)}
\index{CmiSyncBroadcastAll}
\desc{Sends \param{msg} of length \param{size} bytes to all processors
including the processor on which the caller resides. This function
does not free the message buffer for \param{msg}.}

\function{void CmiSyncBroadcastAllAndFree(unsigned int size, void *msg)}
\index{CmiSyncBroadcastAllAndFree}
\desc{Sends \param{msg} of length \param{size} bytes to all processors
including the processor on which the caller resides. This function
frees the message buffer for \param{msg} before returning, so
\param{msg} must point to a dynamically allocated buffer.}

\function{CmiCommHandle CmiAsyncBroadcast(unsigned int size, void *msg)}
\index{CmiAsyncBroadcast}
\desc{Initiates asynchronous broadcast of message \param{msg} of
length \param{size} bytes to all processors excluding the processor on
which the caller resides. It returns a communication handle which
could be used to check the status of this send using
\param{CmiAsyncMsgSent()}. \param{msg} should not be overwritten or
freed before the communication is complete.}

\function{CmiCommHandle CmiAsyncBroadcastAll(unsigned int size, void *msg)}
\index{CmiAsyncBroadcastAll}
\desc{Initiates asynchronous broadcast of message \param{msg} of
length \param{size} bytes to all processors including the processor on
which the caller resides. It returns a communication handle which
could be used to check the status of this send using
\param{CmiAsyncMsgSent()}. \param{msg} should not be overwritten or
freed before the communication is complete.}

\section{Timer}

\function{double CmiTimer(void)}
\index{CmiTimer}
\desc{Returns current value of the timer in seconds. This is
typically the time spent since the \param{ConverseInit()} call.
The precision of this timer is the best available on the particular machine,
and usually has at least microsecond accuracy.}

\section{Processor Ids}

\function{int CmiNumPe(void)}
\index{CmiNumPe}
\desc{Returns total number of processors in the machine on which the 
parallel program is being run.}

\function{int CmiMyPe(void)}
\index{CmiMyPe}
\desc{Returns the logical processor identifier of processor on which the 
caller resides. A processor Id is between \param{0} and \param{CmiNumPe()-1}.}

Also see the calls in section 4.2 % put a ref here..??

\section{Memory Management}

\function{void *CmiAlloc(int size)}
\index{CmiAlloc}
\desc{Allocates memory of size \param{size} in heap and returns pointer to 
the usable space. All the converse entities (e.g. messages) should be 
allocated with \param{CmiAlloc} and not with standard C library 
functions such as \param{malloc} because converse uses \param{CmiFree} to free 
these entities internally. \note{Currently this function is a simple wrapper
around \param{malloc}. However, in future, sophisticated memory allocation
strategies will be incorporated into converse.}}

\function{int CmiSize(void *ptr)}
\index{CmiSize}
\desc{This function returns the allocated size of memory pointed to by 
\param{ptr}. \param{ptr} should be returned by an earlier call to 
\param{CmiAlloc}.}

\function{void CmiFree(void *ptr)}
\index{CmiFree}
\desc{This function frees the memory pointed to by \param{ptr}. \param{ptr}
should be returned by an earlier callto \param{CmiAlloc}.}


\section{Input/Output}

\function{void CmiPrintf(char *format, arg1, arg2, ...)}
\index{CmiPrintf}
\desc{This function does an atomic \param{printf()} on \param{stdout}. 
On machine with host, this is implemented on top of the messaging 
layer using asynchronous sends.}

\function{int CmiScanf(char *format, void *arg1, void *arg2, ...)}
\index{CmiScanf}
\desc{This function performs an atomic \param{scanf} from \param{stdin}.
The processor, on which the caller resides, blocks for input. On machines with
host, this is implemented on top of the messaging layer using asynchronous
send and blocking receive.}

\function{void CmiError(char *format, arg1, arg2, ...)}
\index{CmiError}
\desc{This function does an atomic \param{printf()} on \param{stderr}. 
On machines with host, this is implemented on top of the messaging 
layer using asynchronous sends.}

\internal{
\section{Processor Groups}

\function{void CmiPgrpCreate(Pgrp *group)}
\desc{Creates a processor-group with calling processsor as the root processor.}

\function{void CmiPgrpDestroy(Pgrp *group)}
\desc{Frees resources associated with a processor group \param{group}.}

\function{void CmiAddChildren(Pgrp *group, int penum, int size, int procs[])}
\desc{Adds \param{size} processors from array \param{procs[]} to the
processor-group \param{group} as children of processor penum. This function
could be called only by the root processor of processor-group \param{group}.}

\function{CommHandle CmiAsyncMulticast(Pgrp *group, unsigned int size, void *msg)}
\desc{Initiates asynchronous broadcast of message \param{msg} of
length \param{size} bytes to all processors belonging to \param{group}
excluding the processor on which the caller resides. It returns a
communication handle which could be used to check the status of this
send using \param{CmiAsyncMsgSent()}. \param{msg} should not be
overwritten or freed before the communication is complete. \note{Caller
need not belong to \param{group}.}} 

\function{int CmiPgrpRoot(Pgrp *group)}
\desc{Returns the processor id of root of processor-group \param{group}. }

\function{int CmiNumChildren(Pgrp *group, int penum)}
\desc{Returns  number of children of processor \param{penum} 
in the processor-group \param{group}.}

\function{int CmiParent(Pgrp *group, int penum)}
\desc{Returns  processor id of parent of processor \param{penum} 
in the processor-group \param{group}.}

\function{void CmiChildren(Pgrp *group, int node, int *children)}
\desc{Fills in array \param{children} with processor ids of all the
children processor \param{node} in processor-group \param{group}. This
array should atleast be of size \param{CmiNumChildren()}.}
}

\section{Spanning Tree Calls}

Converse defines a spanning tree organization of the processor procNums along
with routines for accessing elements of that tree.  The spanning tree
routines come in handy when writing programs with collective communication
because message communication patterns can be made to proceed along the
spanning tree arcs to avoid bottlenecks at a single procNum.

\function{int CmiSpanTreeRoot()}
\index{CmiSpanTreeRoot}
\desc{Returns the processor number of the root of the spanning tree.}

\function{int CmiSpanTreeParent(int procNum)}
\index{CmiSpanTreeParent}
\desc{This function returns the processor number of the parent of
\param{procNum} in the spanning tree.}

\function{int CmiNumSpanTreeChildren(int procNum)}
\index{CmiNumSpanTreeChildren}
\desc{Returns the number of children of \param{procNum} in the spanning tree.}

\function{void CmiSpanTreeChildren(int procNum, int *children)}
\index{CmiSpanTreeChildren}
\desc{This function fills the array \param{children} with processor
numbers of children of \param{procNum} in the spanning tree.}

