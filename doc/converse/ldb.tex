\chapter{Load Balancer Calls}


\function{CldNewSeedFromLocal()}
\index{CldNewSeedFromLocal}
\desc{}

\function{CldNewSeedFromNet()}
\index{CldNewSeedFromNet}
\desc{}

\function{CldFillLdb()}
\index{CldFillLdb}
\desc{}

\function{CldStripLdb()}
\index{CldStripLdb}
\desc{}

%\internal{
%\function {void ClbInit(void)}
%Routine called at scheduler startup to allow the load balancer to
%register its own message handlers.
%}
%
%\function {void ClbEnqueue(Message *msg, FunctionPtr send\_func)}
%Function called by a handler to let the load balancer know about a
%``seed'' message. If the seed is to be processed locally, the
%load balancer will
%enqueue it in the scheduler's queue to be eventually executed.
%{\sf send\_func} specifies the function
%which should be used to send the seed to another processor
%if it is not to be processed locally.
%
%\function {void ClbEnqueuePrio(Message *msg, FunctionPtr send\_func,
%void *prio, int len)} 
%Same as {\sf ClbEnqueue()}, except that a priority {\sf prio} of length
%{\sf len} bytes is associated with the message.
%
%
%The most general load balancing scheme will have three opportunities
%to migrate load.  The first will be when {\sf ClbEnQueue()} is called.
%This routine will either immediately send the message elsewhere for
%processing, or enqueue it in a local data structure.  A token message
%will then be enqueued with the scheduler to reactivate the load
%balancing in order to execute the seed message at the appropriate
%time.  The second opportunity for load balancing occurs when the load
%balancing module receives a request from the load balancing module on
%another processor for some work.  When this occurs, a message will be
%delivered to the requesting node from the load balance queue, and the
%token message in the scheduler queue will be marked as invalid.  The
%third opportunity occurs when a token message is removed from the
%scheduler queue.  Control will return to the load balancing module,
%which could then choose to send seed messages to another host.

