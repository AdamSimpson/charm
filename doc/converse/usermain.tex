\chapter{Initialization and Completion}

\function{void ConverseInit(char *argv[])}
\index{ConverseInit}
\desc{This function initializes the machine interface. It should be
called prior to any other Converse functions. 
Multiple calls to this function in a process should
be avoided. \param{argv} is in similar format as passed to
\param{main(argc, argv)}. 
It would be utilized by \param{ConverseInit()} to initialize
machine specific parameters such as number of processors.}

\function{void ConverseExit(void)}
\index{ConverseExit}
\desc{This function frees the resources acquired by Converse and wraps up. 
Any other Converse function should not be called after a call to this function.
\note{It does not terminate the calling
process. A separate call to \param{exit()} is needed after 
\param{ConverseExit()} to achieve this.}}

\function{void user\_main(int argc, char **argv)}
\index{user_main}
\desc{This function must be written by the Converse user.  When the
converse program is loaded, this function will be called on ALL
processors.  Typically, this function is structured as follows: first,
one calls ConverseInit.  Second, adds some work to the work-pool by
sending messages or creating threads.  Third, one performs all work in
the work-pool by calling CsdScheduler.  Finally, when all work is done
and CsdScheduler exits, one shuts down the system by calling
ConverseExit.}
