\chapter{Initialization and Completion}

The program utilizing Converse begins executing at {\tt main}, like
any other C program.  However, depending on the machine and the
implementation decisions of the machine's vendor, {\tt main} may be
executed by more than one processor.  On some machines, every
processor executes {\tt main}.  On others, only one processor executes
{\tt main}.  All processors which don't execute {\tt main} are asleep
when the program begins.

There are two ways for the program to start up the Converse system.
The first method is to use ConverseInit:

\function{typedef void (*CmiStartFn)(int argc, char **argv);}
\function{void ConverseInit(int argc, char *argv[], CmiStartFn fn)}
\index{ConverseInit}
\desc{This must be called by every processor which initially executed
{\tt main}.  Initializes the entire Converse system, then causes all
processors to begin executing the function {\tt fn}, which should
perform all your computation.  ConverseInit never returns.  You may
not call any other Converse function until initialization is complete,
that is, until {\tt fn} begins executing. \param{argc} and
\param{argv} are in the same format as the arguments passed to
\param{main(argc, argv)}.}

ConverseInit is the initialization method one would use in most
programs.  However, some programs may find it difficult to work around
the constraint that ConverseInit never returns.  Thus, we provide
ConverseStart, an alternate initialization method.  Currently,
ConverseStart is supported on all machines, however, it is {\em
potentially} possible that it will be unimplementable on some
machines.  Thus, we do not guarantee its presence.  Use it only when
you absolutely need a startup function that returns.

\function{void ConverseStart(int argc, char *argv[], CmiStartFn fn)}
\index{ConverseInit}
\desc{This must be called by every processor which initially executed
{\tt main}.  Initializes the entire Converse system, then causes all
processors {\em which did not initially execute {\tt main}} to begin
executing the function {\tt fn}.  Once these threads have been
launched, ConverseStart returns.  You may not call any other Converse
function until initialization is complete, that is, until {\tt fn}
begins executing and ConverseStart returns. \param{argc} and
\param{argv} are in the same format as the arguments as passed to
\param{main(argc, argv)}.}

Regardless which startup method is used, all processors should call
ConverseExit during termination:

\function{void ConverseExit(void)}
\index{ConverseExit}
\desc{This function frees the resources acquired by Converse and wraps up. 
Any other Converse function should not be called after a call to this
function.}

