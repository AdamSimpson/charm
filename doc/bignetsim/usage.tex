\subsection{Using BigSimulator}

BigSimulator (BigNetSim) has 2 major modes.

\begin{itemize}
\item{ Trace based traffic simulation }
\item{Artificial traffic generation based simulation.
The mode of the simulator is governed by the $USE\_TRANSCEIVER$ parameter
in the netconfig file.  When set to 0, trace based simulation is used,
when set to 1, traffic generation is used.}
\end{itemize}

Trace based simulation.  This is used to study target application performance,
or detailed network performance when loaded by a specific application.

There are two command line parameters for traced based simulation.
\begin{alltt}
  ./charmrun +p2 ./bigsimulator arg1 arg2
\end{alltt}
\begin{alltt}
  arg1 = 0 => Latency only mode
         1 => Detailed contention model
  arg2 = N => starts execution at the time marked by skip point N (0 is start)
\end{alltt}

\subsubsection{Simple Latency Model}
To use the simple latency model, follow the setup procedure above, noting that
the files are located in the trunk/SimpleLatency directory. This will produce
the "bigsimulator" file.

The command line parameters used for this model are different.  The format is
as follows:

\begin{alltt}
  [charmrun +p#] bigsimulator -lat <latency> -bw <bandwidth>
               [-cpp <cost per packet> -psize <packet size>]
               [-winsize <window size>] [-skip] [-print\_params]
\end{alltt}

\begin{alltt}
  Latency (lat)         - type double; in microseconds
  Bandwidth (bw)        - type double; in GB/s
  Cost per packet (cpp) - type double; in microseconds
  Packet size (psize)   - type int; in bytes
  Window size (winsize) - type int; in log entries
\end{alltt}

The implemented equation is: $lat + (N/bw) + cpp \times (N/psize)$

Latency and bandwidth are required.  If cost per packet is given, then packet
size must be given, as well.  Otherwise, cost per packet defaults to 0.0.
Packet size, if given, must be a positive integer.

The -winsize flag allows the user to specify the size of the window (number of
log entries) used when reading in the bgTrace log files.  This is useful if the
log files are large.  If -winsize is not specified, the value defaults to 0,
which indicates that no windowing will be used (i.e., there will be one window
for each time line that is equal to the size of the time line).

As with the second parameter in the examples of part (a) of this section, the
-skip flag indicates that the simulation should skip forward to the time stamp
set during trace creation (see the BigSim tutorial talk from the 2008 Charm++
workshop).  If -skip is not included, then no skipping will occur.

The -print\_params flag is provided for debugging convenience.  When present,
the simple latency model parameters will be displayed during simulation
initialization.

\subsubsection{Artificial Traffic Models}
Artificial traffic generation based simulation is use to study the performance
of interconnects under standard network load schemes.

\begin{alltt}
  ./bigsimulator arg1 arg2 arg3 arg4 arg5 arg6
\end{alltt}
example
\begin{alltt}
  ./bigsimulator 1 2 3 100 2031 0.1
\end{alltt}

\begin{alltt}
  arg1 = 0 => Latency only mode
         1 => Detailed contention model
  arg2 = 1 => deterministic traffic
         2 => poisson traffic
  arg3 = 1 => KSHIFT
         2 => RING
         3 => BITTRANSPOSE
         4 => BITREVERSAL
         5 => BITCOMPLEMENT
         6 => UNIFORM\_DISTRIBUTION
  arg4 = number of packets
  arg5 = message size
  arg6 = load factor
\end{alltt}
