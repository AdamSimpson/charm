\documentclass[10pt,dvips]{article}
\usepackage{../pplmanual,epsfig,graphicx}
%%% Commonly Needed packages
\usepackage{graphicx,color,calc}
\usepackage{fancyvrb}
\usepackage{makeidx}
\usepackage{alltt}
\usepackage[linkbordercolor=(0 0 1),citebordercolor=(0 1 0)]{hyperref}
%%\usepackage{xspace} <- creates problems with other hyperlink packages like "html"

%%% Commands for uniform looks of C++, Charm++, and Projections
\newcommand{\CC}{C\hbox{++}}
\newcommand{\emCC}{C\hbox{\em++}}
\newcommand{\charmpp}{\textsc{Charm++}}
\newcommand{\charmc}{\texttt{charmc}}
\newcommand{\projections}{\textsc{Projections}}
\newcommand{\converse}{\textsc{Converse}}
\newcommand{\ampi}{\textsc{AMPI}}
\newcommand{\tempo}{\textsc{TeMPO}}
\newcommand{\irecv}{\textsl{iRecv}}
\newcommand{\sdag}{\textsl{Structured Dagger}}
\newcommand{\jade}{Jade}

%%% Commands to produce margin symbols
\newcommand{\new}{\marginpar{\fbox{\bf$\mathcal{NEW}$}}}
\newcommand{\important}{\marginpar{\fbox{\bf\Huge !}}}
\newcommand{\experimental}{\marginpar{\fbox{\bf\Huge $\beta$}}}

%%% Commands for manual elements
\newcommand{\zap}[1]{ }
\newcommand{\function}[1]{{\noindent{\textsf{#1}}\\}}
\newcommand{\cmd}[1]{{\noindent{\textsf{#1}}\\}}
\newcommand{\args}[1]{\hspace*{2em}{\texttt{#1}}\\}
\newcommand{\prototype}[1]{\vspace{0.2in}\index{#1}}
\newcommand{\param}[1]{{\texttt{#1}}}
\newcommand{\kw}[1]{{\textsf{#1}\index{#1}}}
\newcommand{\uw}[1]{{\textsl{#1}}}
\newcommand{\desc}[1]{\indent{#1}}
\newcommand{\note}[1]{(\textbf{Note:} #1)}
\newcommand{\term}[1]{{\bf #1}\index{#1}}

\makeindex


\title{\projections{} Manual}
\version{2.1}
\credits{
By Mike DeNardo, Sid Cammeresi, Theckla Loucios, Orion Lawlor, Gengbin Zheng,
Chee Wai Lee, and Sindhura Bandhakavi
}

\begin{document}
\maketitle

\section{Introduction}
\projections{} is a visualization tool to help you understand and analyze what it
happening in your parallel (\charmpp{}) program. This performance visualization and
feedback tool has a much more refined understanding of user computation than is
possible in traditional tools.

\projections{} displays information about the request for creation and the
actual creation of tasks in \charmpp{} programs. \projections{} also provides the
function of post-mortem clock synchronization. Additionally, it can also
automatically partition the execution of the running program into logically
separate units, and automatically analyzes each individual partition.

Future versions will be able to provide recommendations/suggestions for
improving performance as well.

\section{Preparing the \charmpp{} Program}

In order to use \projections{} to analyse your program, you need to apply
necessary compile options and command line options. Also, \charmpp{} allows
you to insert user events that are detectable to \projections{}, by employing
a set of function calls in your program.

\subsection{Tracemodes}
To use \projections{}, you first have to make sure your \charmpp{} program is
compiled with proper \projections{} tracemode by giving corresponding compile
option. Besides compile options, there are a few command line options for
\projections{}.

\

Currently there are 3 tracemodes available:
\begin{itemize}
\item
Projections mode

Compile option: {\tt -tracemode projections}

This tracemode generates log files that contain information about events like
entry method calling and message packing during the execution of the program.
These data are used by \projections{} in visualization and analysis.

When you run your program compiled with this tracemode, a log file will be
created for each processor.  An additional symbol table file will also be
created.  The names of the log files will be NAME.\#.log, where NAME is the
name of your executable and \# is the processor \#.  The name of the symbol
table file is NAME.sts.

Related command line options and their uses:

{\tt +binary-trace}:  generate projections log in binary form

{\tt +gz-trace}:      compress projections log.

\item
Summary mode

Compile option: {\tt -tracemode sum}

In this tracemode, only a rough summary of CPU utilization is created in running
thusede program. A snap shot is taken every "bin-size" period of time. This brief
information can be used to quickly outline the execution of the program. It helps
user to find interesting spots in their program faster.

Similarly, a summary file named NAME.\#.sum will be generated for each processor,
plus a symbol table file NAME.sum.sts, where NAME is the name of the executable.

Related command line options and their uses:

{\tt +binsize TIME}:   set bin size for summary

{\tt +version}:        set summary version to generate

\item
Counter mode \footnote{Currently implemented on Irix only}

Compile option: {\tt -tracemode counter}

This tracemode generates performance counter information like number of cache
misses and number of entry point calls. These data are not used in the current
version of \projections{} but will be supported in very near future.

The counter files created are named NAME.\#.count and the symbol table file
NAME.count.sts respectively, where NAME is the name of the executable.

\end{itemize}

Compile option {\tt +tracemode all} can be used to generate all available types
of tracing data files.

Other command line options valid for all tracemodes and their uses include:

{\tt +logsize SIZE}: to specify a logsize (actually log line number).

{\tt +traceroot DIR}: to specify an existing directory for putting log/sum/count/sts
files.

{\tt +traceoff}: to turn off tracing at the beginning.

\

Also, \charmpp{} allows user to start/stop tracing the execution at certain
point of time by two \charmpp{} function calls {\tt traceBegin()} and {\tt traceEnd()}.
They set/reset a flag to start/stop dumping tracing data from the local processor,
so they should be called on all processors. Obviously tracing should be turned
off at first (by command line option {\tt +traceoff}) if some period of time in
middle of the execution is of interest to the user.

\subsection{User Events}

\projections{} has the ability to display in \projections{} the "user events"
that have been added by the programmer into the code. The user events are
like markers inserted in various places in the program and can then be inspected
to see where they show up in relationship to the other \charmpp{} system events.
This ability allows the user to participate more actively in the performance analysis
of the program.

There are a few \charmpp{} calls that insert user events into the program.

\begin{itemize}
\item
{\tt int traceRegisterUserEvent(char* EventDesc, int EventNum=-1) }

This function registers a user event by associating an integer event number to
the string description. Event number has to be the same on all processors. There
are 3 ways to ensure that, and correspondingly 3 ways to call the function:

\begin{enumerate}
\item
Call {\tt traceRegisterUserEvent} on node 0 in main::main without specifying
an event number, and store returned event number into a readonly variable.
\item
Call {\tt traceRegisterUserEvent} on all nodes without specifying an event
number. The returned value is the common event number that is agreed on system-wide.
\item
Call {\tt traceRegisterUserEvent} and specify the event number on processor 0.
Doing this on other processors would have no effect. Afterwards, the event number
can be used in the following user event calls.
\end{enumerate}

Eg. {\tt traceRegisterUserEvent("Time Step Begin", 10);}

\item
{\tt void traceUserEvent(int EventNum) }

This function works like a marker, indicating an occurrence of the user event
with the given {\tt EventNum}. It can be used to signalize a user event, in
relationship to other system events.

Eg. {\tt traceUserEvent(10);}

\item
{\tt void traceUserBracketEvent(int EventNum, double StartTime, double EndTime) }

This function brackets an event that lasts the period of time between
{\tt StartTime} and {\tt EndTime}. Both times can be obtained from function call
{\tt CmiWallTimer()} in second.

Eg.
\begin{verbatim}
   traceRegisterUserEvent("Critical Code", 20);
   double critStart;  // times of start
   critStart = TraceTimer();
   // do the critical code
   traceUserBracketEvent(20, critStart,CmiWallTimer());
\end{verbatim}

\end{itemize}

\section{Compiling And Running \projections{}}

\subsection{Compiling \projections{}}
\begin{enumerate}
\item[1)]
   Make sure the JDK commands "java", "javac" and "jar"
   are in your path
\item[2)]
   Type "make"
\item[3)]
   The following files will be placed in 'bin':

      {\tt projections}           : Starts projections, for UNIX machines

      {\tt projections.bat}       : Starts projections, for Windows machines

      {\tt projections.jar}       : archive of all the java and image files
\end{enumerate}

\subsection{Running \projections{}}
From any location, type
\begin{alltt}
> PROJECTIONS_LOCATION/bin/projections
\end{alltt}
where {\tt PROJECTIONS\_LOCATION} is the path to the main projections
directory.

\section{Using \projections{}}
When \projections{} starts, you will see the main window with a title and
six buttons:

\begin{figure}[htb]
\center
\epsfig{figure=fig/mainwindow.eps,height=3in}
\caption{\projections{} main window}
\label{mainwindow}
\end{figure}

1. Open File

2. Graphs

3. Timelines

4. Usage Profile

5. Animations

6. View Log Files

7. Histograms

8. Overview

Clicking on any of the buttons will bring up the appropriate next window, as
described in the following sections.  Initially, only the Open File button
will be active.

\subsection{Open File}

   Clicking on the Open File button brings up a dialog box to let you select
   the location of the data you want to look at.  Navigate to the directory
   containing your data and select the *.sts file.  Click on 'OK'.  If you
   have selected a valid file, \projections{} will load in some preliminary data
   from the files and then activate the rest of the buttons in the main window.
   If your file is invalid, you will be shown an error dialog.

\subsection{Graphs}

   The Graphs window is where you can analyze your data by breaking it into
   any number of intervals and look at what goes on in each of those intervals.

   When the Graph Window first appears, a dialog box will also appear, asking
   you what interval settings you want to use.  It will show you the total
   amount of time your program run took (in microseconds) and ask you to enter
   either the interval size you want or the number of intervals you want.
   Entering a number in either box will recalculate the other number, so you
   will know both items. Click on 'OK' when you are satisifed with your choice.
   Your data will then be analyzed.

   The amount of time to analyze your data depends on several factors, including
   the number of processors, number of entries, and number of intervals you have
   selected.  Although a progress meter has not been implemented at this time,
   you can look at the console window to see which log file is being analyzed.
   Even for large amounts of data, this step should not take more than a few
   minutes, though.

   \begin{figure}[htb]
   \center
   \epsfig{figure=fig/graph.eps,height=4.3in}
   \caption{Graph module}
   \label{graph}
   \end{figure}

      The Graph Window has 3 components in its display:
   \begin{enumerate}
   \item[1)]
   Display Panel:
      \begin{itemize}
      \item[-]
        Largest Component in top/left corner
      \item[-]
        Displays title, graph, and axes
      \item[-]
        Allows you to toggle display between a line graph and a bar graph
      \item[-]
        Allows you to scale the graph along the X-axis.  You can either enter
        a scale value $>=$ 1.0 in the text box, or you can use the $<<$ and $>>$
        buttons to increment/decrement the scale by .25.  Clicking on Reset
        sets the scale back to 1.0.  When the scale is greater than 1.0, a
        scrollbar will appear along the bottom of the graph to let you
        scroll back and forth.
      \end{itemize}

   \item[2)]
   Legend Panel:
      \begin{itemize}
      \item[-]
        Top right side of the display
      \item[-]
        Shows what is currently being displayed on the graph and what color it
        is.
      \item[-]
        Click on the 'Select Display Items' button to bring up a window to
        add/remove items from the graph and to change the colors of the items.
        \begin{itemize}
	\item[*]
          The Select Display Items window shows a list of items that you can
          display on the graph.  There are 3 main sections:
            System Usage, System Msgs, and User Entries
          The System Usage and System Msgs are the same for all programs.  The
          User Entries section has program-specific items in it.
	\item[*]
          Click on the checkbox next to an item to have it displayed on the
          graph.
	\item[*]
          Click on the colorbox next to an item to modify its color.
	\item[*]
          Click on 'Select All' to choose all of the items
	\item[*]
          Click on 'Clear All' to remove all of the items
	\item[*]
          Click on 'Apply' to apply you choices/changes to the graph
	\item[*]
          Click on 'Close' to exit
	\end{itemize}
      \end{itemize}
   \item[3)]
   Control Panel:
      \begin{itemize}
      \item[-]
        Bottom of the display
      \item[-]
        Allows you to toggle what is displayed on the X-axis.  You can either
        have the x-axis display the data by interval or by processor.
      \item[-]
        Allows you to toggle what is displayed on the Y-axis.  You can either
        have the y-axis display the data by the number of msgs sent or
        by the amount of time taken.
      \item[-]
        Allows you to change what data is being displayed by iterating through
        the selections.  If you have selected an x-axis type of 'interval',
        that means you are looking at what goes on in each interval for a
        specific processor.  Clicking on the $<<, <, >, >>$ buttons will change
        the processor you are looking at by either -5, -1, +1, or +5.
        Conversely, if you have an x-axis of 'processor', then the iterate
        buttons will change the value of the interval that you are looking at
        for each processor.
      \item[-]
        Allows you to indicate which intervals/processors you want to examine.
        Instead of just looking at one processor or one interval, the box and
        buttons on the right side of this panel let you choose any number or
        processors/intervals to look at.  Just enter the number(s) in the box.
        If you want to look at multiple items, separate them with commas.  If
        your selections include a range of items, you can separate those with
        a dash.

        ex: Want to see processors 1,3,5,7:  Enter 1,3,5,7

	ex: Want to see processors 1,2,3,4:  Enter 1-4

	ex: Want to see processors 1,2,3,7:  Enter 1-3,7

        Clicking on 'Apply' updates the graph with your choices.
        Clicking on 'Select All' chooses the entire range.  When you select more
        than one set of data to display, the graph will show the TOTAL amount of
        the data for all of those items, EXCEPT for the processor usage, which
        shows the average amount.
      \end{itemize}
    \end{enumerate}

\subsection{Timelines}

   The Timeline window lets you look at what a specific processor is doing at
   each moment of the program.

   \begin{figure}[htb]
   \center
   \epsfig{figure=fig/timeline.eps,height=3.8in}
   \caption{Timeline module}
   \label{timeline}
   \end{figure}

      When the Timeline window first appears, a dialog box appears along with it.
   The box asks for the following information:
   \begin{itemize}
   \item[-]
     Processor(s):  Choose which processor(s) you want to see a timeline for.
                    To enter multiple values, separate them with a comma or a
                    dash (for ranges).  (See the Graphs section for examples)
   \item[-]
     Begin Time  : Choose what time you want your timeline to start at.
   \item[-]
     End Time    : Choose what time you want your timeline to end at.
   \item[-]
     Length      : Choose the length of your timeline.
   \end{itemize}

   The dialog box tells you what the valid processor choices are as well as what
   the valid time ranges are.

   Instead of entering a BeginTime, you can have the dialog box choose a
   BeginTime for you based on the occurrence of a specific entry.  To do this,
   you go to the bottom portion of the dialog box and select an entry to find an
   occurrence of.  Then, you choose the processor you want to find an occurrence
   on, and which occurrence you want to find (N).  Click on 'Search for Begin
   Time'.  The dialog box will display a message telling you if your occurrence
   was found and when it was found.  If valid, the time is automatically entered
   as the begin time.

   When you are satisifed with your time and processor ranges, click on 'OK'.
   \projections{} will then get the Timeline data for you.  The time for this step
   depends on the number of items in your time range and the number of
   processors you have chosen.

   The Timeline Window consists of two parts:
   \begin{enumerate}
   \item[1)]
      Display Panel:
      This is where the timelines are displayed and is the largest portion of
      the window.  The time axes are displayed at the top and the bottom of the
      panel, and the units are microseconds.  The left side of the panel shows
      the processor labels.  Underneath each label is a percentage telling you
      what amount of the total time in your timeline was actually spent working
      on this program.

      The timeline itself consists of colored bars for each
      work item.  Placing the cursor over one of these bars will bring up a
      pop-up window telling you the name of that item, the begin time, the end
      time, and the total time.  It will also tell you what amount of time was
      spent packing, how many messages were created during this work item, and
      which processor created this item.  If you click on the item, a window
      will appear telling you similar information to the pop-up window.  This
      window will also list all of the messages created during this work item,
      and it will tell you what time they were sent at and to which entry.

   \item[2)]
      Control Panel:

	It's located at the bottom of the window. Its components are described
	as follows.

      Checkboxes:
      \begin{itemize}
      \item[-]
        Display Pack Times\\
        Lets you toggle display of Time spent packing
      \item[-]
        Display Message Creations\\
        Lets you toggle display of message creations. These are represented
        by little vertical lines at the time a message was created.
      \item[-]
	Display Idle Time\\
        Lets you toggle display of idle time
      \item[-]
        View User Event ({\bf really cool new feature!!!})\\
	Checking this box will bring up a new window showing the string description,
	begin time, end time and duration of all user events on each processor.
	Especially, the user event caught by function {\tt traceUserEvent} has
	a duration of 0.

      \begin{figure}[htb]
      \center
      \epsfig{figure=fig/userevent.eps,height=1.5in}
      \caption{User Event Window}
      \label{userevent}
      \end{figure}

        Like system events, user events are also displayed as thin lines or bars
	above in the disply area.
      \end{itemize}
      Buttons:
      \begin{itemize}
      \item[-]
        Select Ranges\\
        Brings up the initial dialog box
      \item[-]
        Change Colors\\
        Lets you change colors for the work items
      \item[-]
        Scale\\
        Enter a scale $>=$ 1.0 in the box, or click on the $<<$ and $>>$ buttons to
        adjust the scale by 0.25 increments.  Click on Reset to set the scale
        back to 1.0
      \end{itemize}

      Another really cool new feature {\bf zoomability} has been implemented in
      timeline module.

      To determine the exact time of any event on the timeline, move your mouse
      along either the top or bottom axis and a white vertical highlight line
      will show where your cursor is along the timeline.  The "Highlight Time"
      box on the bottom of the Timeline window will show the exact time of your
      cursor.

      To select an area, click on the axis to define the start of the area and
      drag the mouse to the end of the area to be defined.  Two yellow vertical
      lines will bracket the area of interest.  The exact times of the selected
      area will be shown in the "Selection Start Time" text area and the
      "Selection End Time" text area.  The difference between these times is
      shown in ms in the "Selected Length" text area.  Thus, this feature can be
      used to measure the time between two events of interest across processors,
      and is an easy way to measure the time of an entry point (instead of
      getting the "Tool Tip" balloon by putting the cursor in the entry point.

      To zoom into a selected area (instead of using the scale field and/or the
      "<<" and ">>" buttons) simply select an area and click on either the "Zoom
      Selected" or the "Load Selected" buttons.  The difference between these
      two buttons is that the "Load Selected" zooms into the selected area and
      discards any events that are outside the time range.  This is useful
      because it will be a lot faster to zoom because the "Zoom Selected" draws
      all the events on a virtual canvas and then zooms into the canvas.
      However, if the "Load Selected" button is used, to zoom back out, you need
      to start over from "Select Ranges" button and that will slow down the processing.

   \end{enumerate}

\subsection{Usage Profile}

   The Usage Profile window lets you see percentage-wise what each processor
   spends its time on during a specified period.

   When the window first comes up, a dialog box appears asking for the
   processor(s) you want to look at as well as the time range you want to look
   at.  This is similar to the dialog for the Timelines.

   The bottom portion of the Usage Profile window lets you adjust the scales in
   both the X and Y directions.  The X direction is useful if you are looking at
   a large number of processors.  The Y direction is useful if there are
   small-percentage items for a processor.

   The left side of the display shows a scale from 0\% to 100\%.  The main part of
   the display shows the statistics.  Each processor is represented by a
   vertical bar.  The top of the bar always shows the overhead time.  Below that
   is always (if exists) the idle time and then the message packing/unpacking
   times.  The rest of the bar is ordered from the bottom with the largest
   percentage items being closest to the bottom.  If you place the cursor over a
   portion of the bar, a pop-up window will appear telling you the name of the
   item, what percent of the usage it has, and the processor it is on.

\subsection{Animations}

   This window animates the processor usage by displaying
   different colors for different amount of usage.

   The left box allows you to select the real time between frames;
   the right box the processor time between frames.


\subsection{View Log Files}

   This window lets you see a translation of a log file from a bunch of numbers
   to an English version.  A dialog box asks which processor you want to look
   at.  After choosing and pressing OK, the translated version appears.

   Each line has:
   \begin{itemize}
   \item[-] a line number (starting at 0)
   \item[-] the time the event occurred at
   \item[-] a description of what happened.
   \end{itemize}

\subsection{Histograms}

This module allows you to examine the execution time distribution of all your
entry points(EP). It gives a histogram of different number of EP's that have
execution time falling in different time bins. Under the graph there are also
statistics given in numbers.

\subsection{Overview}

Overview gives user an overview of the utilization of all processors during the
execution. Each processor has a row of colored bars in the display, different colors
indicating different utilization at that time. Moving a mouse over the graph
will invoke a display of the processor usage of the specific processor at the
specific time in the status bar below the graph. Vertical and horizontal zoom is
enabled by two zooming bars to the right and lower of the graph.

\end{document}
