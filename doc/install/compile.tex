%%%%%%%%%%%%%%%%%%%%%%%%%%%%%%%%%%%%%%%%%%%%%%%%%%%%%%%%%%%%%%%%%%%%%%%%%%%%
% RCS INFORMATION:
%
%       $RCSfile$
%       $Author$        $Locker$       $State$
%       $Revision$      $Date$
%
%%%%%%%%%%%%%%%%%%%%%%%%%%%%%%%%%%%%%%%%%%%%%%%%%%%%%%%%%%%%%%%%%%%%%%%%%%%%
% DESCRIPTION:
%
%%%%%%%%%%%%%%%%%%%%%%%%%%%%%%%%%%%%%%%%%%%%%%%%%%%%%%%%%%%%%%%%%%%%%%%%%%%%
% REVISION HISTORY:
%
% $Log$
% Revision 1.4  1995-11-16 23:58:29  brunner
% Alphabetized option list
%
% Revision 1.3  1995/11/14  21:23:55  brunner
% Added paragraph on including a user-defined load balance strategy.
%
% Revision 1.2  1995/11/02  22:40:39  brunner
% Includes short description of what charmc does, and description
% of options
%
% Revision 1.1  1995/10/31  22:27:53  milind
% Initial revision
%
%%%%%%%%%%%%%%%%%%%%%%%%%%%%%%%%%%%%%%%%%%%%%%%%%%%%%%%%%%%%%%%%%%%%%%%%%%%%

\section{Compiling Converse, Charm, and Charm++ Programs}

The shell script, {\tt charmc}, is a versatile program to standardize
compiling and linking procedures among various machines and operating
systems.  The script selects the appropriate libraries to compile
Charm, Charm++, or Converse programs.  It also standardizes compiling
and linking procedures and options for sequential C and C++ programs
on the supported systems.  The {\tt charmc} program is executed in the
following way:

\begin{verbatim}
charmc [options] input-files
\end{verbatim}

Files with the following extensions are handled by {\verb+charmc+}
\begin{description}
\item[{\verb+.p+}] Charm code
\item[{\verb+.P+}] Charm++ code
\item[{\verb+.c+}] C code
\item[{\verb+.C+}] C++ code
\item[{\verb+.o+}] Object files
\item[{\verb+.a+}] Library files
\end{description}

\subsection{Compiling with a user-defined load-balance strategy}

The {\tt charmc} script is designed to link in either one of the
standard load balance libraries, or a user written library.  A
user-defined strategy needs to define all of the {\tt Cld} calls
described in the {\em Converse API Reference} manual.  A {\tt .o} file
containing these functions is created, given the name {\tt
libck-ldb-xxx.o}, where {\tt xxx} is the name of the strategy, and
placed in the {\em Converse} library directory.  Then that strategy
can be linked with a user program using the {\tt -balance} option.

\subsection{Command line options for {\tt charmc}}
The following options are supported by charmc:
\begin{description}

\item[{\tt -balance} {\em load-balance-strategy}]

Selects the desired load balancing strategies.  Currently, only {\tt
rand}, {\tt acwn}, and {\tt mngr} are supported. If a user-defined
strategy has been placed in the charm library directory, it may also
be selected with this option.  Default is {\tt -balance rand}.
	
\item[{\tt -c}]

Ignored.  For compatibility with {\tt cc} and {\tt ld}.

\item[{\tt -c++} {\em C++ compiler}]

Forces the specified C++ compiler to be used.

\item[{\tt -cc} {\em C-compiler}]

Forces the specified C compiler to be used.

\item[{\tt -cp} {\em copy-file}]

Creates a copy of the output file in {\em copy-file}.

\item[{\tt -cpp-option} {\em options}]

Options passed to the C pre-processor.

\item[{\tt -D*}]

Passed to C preprocessor, C and C++ compilers.

\item[{\tt -execmode} {\em execution-mode}]

Selects the desired execution mode.  Currently supported modes are
{\tt none}, {\tt summary}, and {\tt projections}. Default is {\tt
-execmode none}.

\item[{\tt -I*}]

Passed to C preprocessor, C and C++ compilers.

\item[{\tt -g}]

Causes compiled files to include debugging information.

\item[{\tt -L*}]

Passed to all linkers.

\item[{\tt -language \{converse|charm|charm++\}}]

Selects compiler and linker options, and appropriate libraries
to compile the selected language.  Default is {\tt -language charm}.

\item[{\tt -ld} {\em linker}]

Forces the specified linker to be used for {\tt -language charm}.

\item[{\tt -ld++} {\em linker}]

Forces the specified linker to be used for {\tt -language charm++}.

\item[{\tt -ld++-option} {\em options}]

Options passed to the linker for {\tt -language charm++}.

\item[{\tt -ld-option} {\em options}]

Options passed to the linker for {\tt -language charm}.

\item[{\tt -ldro-option} {\em options}]

Options passes to the linker when linking {\tt .o} files.

\item[{\tt -machine} {\em machine-type}]

If more than one version of converse/charm/charm++ has been installed,
allows selection of versions other than the default.  The default
version is the version last installed. For a list of machine types
currently supported, see the previous section.

\item[{\tt -memory}]

Currently ignored

\item[{\tt -o} {\em output-file}]

Output file name.

\item[{\tt -queue} {\em queueing strategy}]

Currently ignored

\item[{\tt -save}]

Intermediate files produced by the Charm or Charm++ translator are saved.

\item[{\tt -seq}]

Compiles C and C++ files using the system's sequential/front end
compilers.

\item[{\tt -use-fastest-cc}]

Some environments provide more than one C compiler (cc and gcc, for
example).  This option selects the compiler most likely to produce
faster executing code.

\item[{\tt -use-reliable-cc}]

Some environments provide more than one C compiler (cc and gcc, for
example).  This option selects the compiler most likely to succesfully
compile C code.

\item[{\tt -verbose}]

All commands executed by charmc are echoed to stdout.

\item[{\tt -O}]

Causes files to be compiled with default optimization.  Default.

\item[{\tt -NO}]

Causes no optimization to be used.  Overrides {\tt -O}.

\item[{\tt -O*}]

Passed to C/C++ compiler.

\item[{\tt -l*}]

Specifies libraries which are passed to all linkers.

\item[{\tt -s}]

Passed to C and C++ linkers.

\end{description}



