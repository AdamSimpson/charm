%%%%%%%%%%%%%%%%%%%%%%%%%%%%%%%%%%%%%%%%%%%%%%%%%%%%%%%%%%%%%%%%%%%%%%%%%%%%
% RCS INFORMATION:
%
%       $RCSfile$
%       $Author$        $Locker$                $State$
%       $Revision$      $Date$
%
%%%%%%%%%%%%%%%%%%%%%%%%%%%%%%%%%%%%%%%%%%%%%%%%%%%%%%%%%%%%%%%%%%%%%%%%%%%%
% DESCRIPTION:
%
%%%%%%%%%%%%%%%%%%%%%%%%%%%%%%%%%%%%%%%%%%%%%%%%%%%%%%%%%%%%%%%%%%%%%%%%%%%%
% REVISION HISTORY:
%
% $Log$
% Revision 1.3  1995-11-16 20:58:57  brunner
% Added some paragon information
%
% Revision 1.2  1995/11/16  19:05:03  milind
% *** empty log message ***
%
% Revision 1.1  1995/10/31  22:27:53  milind
% Initial revision
%
%%%%%%%%%%%%%%%%%%%%%%%%%%%%%%%%%%%%%%%%%%%%%%%%%%%%%%%%%%%%%%%%%%%%%%%%%%%%

\section[Executing Charm Programs]{Executing Charm programs}
\label{executing charm programs}

The Charm linker produces one executable file.  On machines with a host
(such as a network of workstations), a link to the proper host program
{\fexec conv-host} is created in the user program directory.  Sample
execution examples are given below (the executable is called {\fparm
pgm}). Exact details will differ from site to site.  The list of Charm
command line options is in Section~\ref{command line options}.

\begin{itemize}

\item \underline{\bf CM-5:} 
	\begin{tabbing}
	{\fexec pgm}
	\end{tabbing}
	will run program {\fparm pgm} interactively from one of the partition 
	managers.  For programs with large resource requirements, use the utilities
	{\fexec jrun} and {\fexec jstat}. See the man pages at your site for
	more information. Also, the {\fexec +pN} option is recognized and runs {\fexec pgm} on {\fexec N} processors.

\item \underline{\bf nCUBE/2:} 
	\begin{tabbing}
	{\fexec xnc -d4 pgm}
	\end{tabbing}
	runs {\fparm pgm} on a hypercube of dimension 4 (i.e. on 16 processors). 

\item \underline{\bf Paragon running SunMos:} 
	\begin{tabbing}
	{\fexec yod -sz 4 pgm}
	\end{tabbing}
	runs {\fparm pgm} on four processors.

\item \underline{\bf Paragon running OSF:} 
	\begin{tabbing}
	{\fexec pexec pgm -sz 4}
	\end{tabbing}
	runs {\fparm pgm} on four processors.

\item \underline{\bf Network of workstations:} 
	\begin{tabbing}
	{\fexec conv-host pgm +p4}
	\end{tabbing}
	executes {\fparm pgm} on 4 nodes.  In a network environment, Charm must
	be able to locate the directory of the executable.  If all workstations
	share a common file name space this is trivial.  If they don't, Charm
	will attempt to find the executable in a directory with the same path
	from the {\bf \$HOME} directory.  Pathname resolution is performed as 
	follows:
	\begin{enumerate}
		\item The system computes the absolute path of {\fexec pgm}.
		\item If the absolute path starts with the equivalent of {\bf \$HOME} 
			or the current working directory, the beginning part of the path 
			is replaced with the environment variable {\bf \$HOME} or the 
			current working directory.
		\item The system tries to locate this program (with modified 
			pathname) on all nodes.
	\end{enumerate}

The list of nodes must be specified in a file, with one node entry per
line, each entry being in the format:

{\fexec nodename [username] [passwd] [setup\_command]}

The {\fparm nodename} must be specified completely (e.g. sparc1.cs.uiuc.edu).
The {\fparm username} is optional; if the login id on that node is the
same as the current login id, it need not be specified.
The {\fparm password} argument is ignored in the current implementation.
The {\fparm setup\_command} is optional; if specified,
the command is executed on the corresponding node before the program 
begins execution. The {\bf *} character may be used as a
placeholder for an absent argument.
The number of nodes specified in the nodes file must not be less
than the number of nodes specified with the {\fexec +p} command
line option. If the number of nodes specified in the nodes file is
less than the number of nodes specified with the {\fexec +p} option,
{\fexec +p} entries in the nodes file will be used in succession, starting
with the first entry in the file.
The name of the nodes fil\index{nodes file}\index{.nodes} itself is
obtained by  {\fexec conv-host} in the following order:
\begin{enumerate}
\item	From the command line option {\fexec ++nodesfile}.
\item	If the {\fexec ++nodesfile} option is not given, the value of the 
environment variable {\fexec NODES} is taken to be the nodes file.
\item	If the environment variable {\fexec NODES} is not set, the file 
{\fparm .nodes}\index{.nodes}\index{nodes file} in the user's home directory is taken 
to be the nodes file.
\item	If the above file does not exist, the file 
{\fparm .nodes}\index{.nodes}\index{nodes file} in the current
directory is used as  the nodes file.
\end{enumerate}
The user is required to set up remote login permissions on all nodes
using the .rhosts file in the home directory.

\end{itemize}

Note that the Charm linker will provide the correct 
executable. The user, however, needs to know how programs are run for
the particular machine.



\subsection[Command Line Options]{Command Line Options}
\label{command line options}
\index{command line options}

A Charm program accepts the following command line options:
\begin{description}
\item[{\fexec +pN}] Run the program with N processors. The default is 1.
Note that on some nonshared memory machines, e.g., nCUBE/2, the user must
specify the number of processors using the command provided for that
machine (e.g. {\fexec xnc} on the nCUBE/2).
In such cases the {\fexec +p} option is ignored.
%\item[{\fexec +mM}] Run the program with M Kwords of memory per
%processor. The default is 50 Kwords of memory per processor.
%\item[{\fexec +mmM}] Run the program with M Kwords of memory for
%processor 0.
\item[{\fexec +ss}] Print summary statistics about chare creation.  This option
prints the total number of chare creation requests, and the total number of
chare creation requests processed across all processors.
\item[{\fexec +cs}] Print statistics about the number of create chare messages
requested and processed, the number of messages for chares requested and 
processed, and the number of messages for branch office chares requested and
processed, on a per processor basis.  Note that the number of messages 
created and processed for a particular type of message on a given node 
may not be the same, since a message may be processed by a different
processor from the one originating the request.
\item[{\fexec +mems}] Print the Memory Usage Statistics at the end, including
the number of memory allocation requests and memory free requests, based on
the size of the memory allocated or freed.
\item[{\fexec user\_options}] Options that are be interpreted by the user
program may be included after all the system options. 
However, {\fexec user\_options} cannot start with +.
The {\fexec user\_options} will be passed as arguments to the user program 
via the usual {\fcmd argc/argv} construct to the {\fcmd main}\index{main}
entry point of the main chare. 
Charm system options will not appear in {\fcmd argc/argv}.
\end{description}

\subsubsection[Additional Uniprocessor Command Line Options]
{Additional Uniprocessor Command Line Options}
\label{uniprocessor command line options}

The uniprocessor versions can be used to simulate multiple
processors on a single workstation\index{uniprocessor command line
options}.  Any number of processors between 1 and 32 can be simulated by
using the {\fexec +p} option, limited only by the available memory on the
uniprocessor workstation.  By default, the uniprocessor versions handle
a single message from each processor, going in order from processor 0
thru $P-1$ (where $P$ is the number of processors) repeatedly.  
%If the
%user supplies the {\fexec +seed} \index{+seed} command line option
%followed by an 
%integer value, the processors will be accessed in a random (but
%deterministic) order.  {\fexec +seed} is only recognized by the
%uniprocessor version.

\subsubsection[Additional Network Command Line Options]
{Additional Network Command Line Options}
\label{network command line options}

The following {\fexec ++} command line options are available in
the network version\index{network command line options}:
\begin{description}
\item[{\fexec ++debug}] Run each node under gdb in an xterm window, prompting
the user to begin execution.
\index{++debug}
\item[{\fexec ++debug-no-pause}] Run each node under gdb in an xterm window
immediately (i.e. without prompting the user to begin execution).
\index{++debug-no-pause}
\item[{\fexec ++maxrsh}] Maximum number of {\fcmd rsh}'s to run at a
time.
\index{++maxrsh}
\item[{\fexec ++resend-wait}] Timeout before retransmitting datagrams
(in msec).
\index{++resend-wait}
\item[{\fexec ++resend-fail}] Timeout before retransmission fails (in
msec).\index{++resend-fail}
This parameter can help the user kill ``runaway'' processes, which may not
be killed otherwise when the user interrupts the program before it 
completes execution.
Currently a bug exists in the network version that may cause programs to
terminate prematurely if this value is set too low and {\fexec scanf} 
operations are being performed.

\item[{\fexec ++nodesfile}] File containing list of nodes.
\index{++nodesfile}\index{.nodes}\index{nodes file}
\end{description}

If using the {\fexec ++debug} option, the user must ensure the
following:
\index{++debug}
\begin{enumerate}
\item {\fexec xterm} and {\fexec gdb} must be in the user's path.
\item The path must be set in the {\fexec .cshrc} file, not the {\fexec .login}
file, because {\fexec rsh} does not run the {\fexec .login} file. 
\item The nodes must be authorized to create windows on the host machine (see
man page for {\fexec xhost}).
\end{enumerate}
