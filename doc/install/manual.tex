%%%%%%%%%%%%%%%%%%%%%%%%%%%%%%%%%%%%%%%%%%%%%%%%%%%%%%%%%%%%%%%%%%%%%%%%%%%%
% RCS INFORMATION:
%
%       $RCSfile$
%       $Author$        $Locker$                $State$
%       $Revision$      $Date$
%
%%%%%%%%%%%%%%%%%%%%%%%%%%%%%%%%%%%%%%%%%%%%%%%%%%%%%%%%%%%%%%%%%%%%%%%%%%%%
% DESCRIPTION:
%
%%%%%%%%%%%%%%%%%%%%%%%%%%%%%%%%%%%%%%%%%%%%%%%%%%%%%%%%%%%%%%%%%%%%%%%%%%%%
% REVISION HISTORY:
%
% $Log$
% Revision 1.4  1996-10-01 18:19:51  milind
% Added a word , various..
%
% Revision 1.3  1995/11/17 17:48:18  brunner
% New title page, but I can't get rid of page number
%
% Revision 1.2  1995/11/01  20:02:08  milind
% *** empty log message ***
%
% Revision 1.1  1995/10/31  22:27:53  milind
% Initial revision
%
%%%%%%%%%%%%%%%%%%%%%%%%%%%%%%%%%%%%%%%%%%%%%%%%%%%%%%%%%%%%%%%%%%%%%%%%%%%%

\documentstyle[11pt,fullpage]{article}
\pagestyle{empty}
\setlength{\textwidth}{6.5in}
\setlength{\textheight}{9in}
\setlength{\parindent}{0in}
\setlength{\topmargin}{-.5in}
\parskip 0.1in
\newcommand{\zap}[1]{ }
\newcommand{\fcmd}{\bf}                 %%% Font for Charm commands
\newcommand{\fparm}{\bf\sf}             %%% Font for parameters to Charm commands
\newcommand{\fexec}{\bf}                %%% Font for compile/execute cmds/options

\begin{document}
 
%\begin{titlepage}
%\vspace*{2in}
%\Huge
%\begin{center}
%Charm/Charm++/Converse \\
%Installation\\
%Manual\\
%\vspace*{0.7in}
%\today
%\end{center}
%\normalsize
%\end{titlepage}

\begin{titlepage}

\title{
\vspace*{3in}
Charm/Charm++/Converse Installation and Usage
}

\author{Parallel Programming Laboratory\\
	Department of Computer Science\\
	University of Illinois at Urbana--Champaign}
\date{\today}

\maketitle

\end{titlepage}

\newpage
\pagestyle{headings}

\section{Introduction}
In this manual, we describe how to install Charm, Charm++ and
Converse, how to compile and execute programs, and the available
command line options.  We also describe various queueing and load balancing
strategies, and the various modes of execution of Charm and Charm++ programs.

\charmpp{} can be installed either from the source code or using a precompiled
binary package. Building from the source code provides more flexibility, since one 
can choose the options as desired. However, a precompiled binary may be slightly
easier to get running.
 
\section{Downloading \charmpp{}}

\charmpp{} can be downloaded using one of the following methods:

\begin{itemize}
\item From \charmpp{} website -- The current stable version (source code and
binaries) can be downloaded from our website at {\em http://charm.cs.illinois.edu/software}.
\item From source archive -- The latest development version of \charmpp{} can be downloaded
from our source archive using {\em git clone http://charm.cs.illinois.edu/gerrit/charm}.
\end{itemize}

If you download the source code from the website, you will have to unpack it 
using a tool capable of extracting gzip'd tar files, such as tar (on Unix) 
or WinZIP (under Windows).  \charmpp{} will be extracted to a directory 
called ``charm''. 

\section{Installation}

A typical prototype command for building \charmpp{} from the source code is:
\vspace{5pt}\\
{\bf ./build $<$TARGET$>$ $<$TARGET ARCHITECTURE$>$ [OPTIONS]} where,

\begin{description}
\item [TARGET] is the framework one wants to build such as {\em charm++} or {\em
AMPI}.
\item [TARGET ARCHITECTURE] is the machine architecture one wants to build for
such as {\em netlrts-linux-x86\_64}, {\em pamilrts-bluegeneq} etc.
\item [OPTIONS] are additional options to the build process, e.g. {\em smp} is
used to build a shared memory version, {\em -j8} is given to build in parallel
etc.
\end {description}

In Table~\ref{tab:buildlist}, a list of build commands is provided for some of the commonly 
used systems. Note that, in general, options such as {\em smp},
\verb|--with-production|, compiler specifiers etc can be used with all targets.
It is advisable to build with \verb|--with-production| to obtain the best
performance.  If one desires to perform trace collection (for Projections),
\verb|--enable-tracing --enable-tracing-commthread| should also be passed to the
build command.

Details on all the available alternatives for each of the above mentioned
parameters can be found by invoking \verb|./build --help|. One can also go through the
build process in an interactive manner. Run \verb|./build|, and it will be followed by
a few queries to select appropriate choices for the build one wants to perform.


\begin{table}[ht]
\begin{tabular}{|p{6cm}|p{9cm}|}
\hline
Net with 32 bit Linux & \verb|./build charm++ netlrts-linux --with-production -j8|
\\\hline
Multicore (single node, shared memory) 64 bit Linux & \verb|./build charm++ multicore-linux-x86_64 --with-production -j8|
\\\hline
Net with 64 bit Linux & \verb|./build charm++ netlrts-linux-x86_64 --with-production -j8|
\\\hline
Net with 64 bit Linux (intel compilers) & \verb|./build charm++ netlrts-linux-x86_64 icc --with-production -j8|
\\\hline
Net with 64 bit Linux (shared memory) & \verb|./build charm++ netlrts-linux-x86_64 smp --with-production -j8|
\\\hline
Net with 64 bit Linux (checkpoint restart based fault tolerance) & \verb|./build charm++ netlrts-linux-x86_64 syncft --with-production -j8|
\\\hline
MPI with 64 bit Linux & \verb|./build charm++ mpi-linux-x86_64 --with-production -j8|
\\\hline
MPI with 64 bit Linux (shared memory) & \verb|./build charm++ mpi-linux-x86_64 smp --with-production -j8|
\\\hline
MPI with 64 bit Linux (mpicxx wrappers) & \verb|./build charm++ mpi-linux-x86_64 mpicxx --with-production -j8|
\\\hline
IBVERBS with 64 bit Linux & \verb|./build charm++ verbs-linux-x86_64 --with-production -j8|
\\\hline
OFI with 64 bit Linux & \verb|./build charm++ ofi-linux-x86_64 --with-production -j8|
\\\hline
Net with 64 bit Windows & \verb|./build charm++ netlrts-win-x86_64 --with-production -j8|
\\\hline
MPI with 64 bit Windows & \verb|./build charm++ mpi-win-x86_64 --with-production -j8|
\\\hline
Net with 64 bit Mac & \verb|./build charm++ netlrts-darwin-x86_64 --with-production -j8|
\\\hline
Blue Gene/Q (bgclang compilers) & \verb|./build charm++ pami-bluegeneq --with-production -j8|
\\\hline
Blue Gene/Q (bgclang compilers) & \verb|./build charm++ pamilrts-bluegeneq --with-production -j8|
\\\hline
Cray XE6 & \verb|./build charm++ gni-crayxe --with-production -j8|
\\\hline
Cray XK7 & \verb|./build charm++ gni-crayxe-cuda --with-production -j8|
\\\hline
Cray XC40 & \verb|./build charm++ gni-crayxc --with-production -j8|
\\\hline
\end{tabular}
\caption{Build command for some common cases}
\label{tab:buildlist}
\end{table}

As mentioned earlier, one can also build \charmpp{} using the precompiled binary
in a manner similar to what is used for installing any common software.

When a Charm++ build folder has already been generated, it is possible to
perform incremental rebuilds by invoking \verb|make| from the \verb|tmp| folder
inside it. For example, with a {\em netlrts-linux-x86\_64} build, the path
would be \verb|netlrts-linux-x86_64/tmp|. On Linux and macOS, the tmp symlink
in the top-level charm directory also points to the tmp directory of the most
recent build.

Alternatively, CMake can be used for configuring and building Charm++. You can
use \verb|cmake-gui| or \verb|ccmake| for an overview of available options.
Note that some are only effective when passed with \verb|-D| from the
command line while configuring from a blank slate. To build with all defaults,
\verb|cmake .| is sufficient, though invoking CMake from a separate location
(ex: \verb|mkdir mybuild && cd mybuild && cmake ../charm|) is recommended.

The main directories in a \charmpp{} installation are:

\begin{description}
\item[\kw{charm/bin}]
Executables, such as charmc and charmrun,
used by \charmpp{}.

\item[\kw{charm/doc}]
Documentation for \charmpp{}, such as this
document.  Distributed as LaTeX source code; HTML and PDF versions
can be built or downloaded from our web site.

\item[\kw{charm/include}]
The \charmpp{} C++ and Fortran user include files (.h).

\item[\kw{charm/lib}]
The static libraries (.a) that comprise \charmpp{}.

\item[\kw{charm/lib\_so}]
The shared libraries (.so/.dylib) that comprise \charmpp{},
if \charmpp{} is compiled with the \texttt{--build-shared} option.

\item[\kw{charm/examples}]
Example \charmpp{} programs.

\item[\kw{charm/src}]
Source code for \charmpp{} itself.

\item[\kw{charm/tmp}]
Directory where \charmpp{} is built.

\item[\kw{charm/tests}]
Test \charmpp{} programs used by autobuild.

\end{description}

\section{Reducing disk usage}

The charm directory contains a collection of example-programs and
test-programs.  These may be deleted with no other effects. You may
also {\tt strip} all the binaries in {\tt charm/bin}.

%%%%%%%%%%%%%%%%%%%%%%%%%%%%%%%%%%%%%%%%%%%%%%%%%%%%%%%%%%%%%%%%%%%%%%%%%%%%
% RCS INFORMATION:
%
%       $RCSfile$
%       $Author$        $Locker$       $State$
%       $Revision$      $Date$
%
%%%%%%%%%%%%%%%%%%%%%%%%%%%%%%%%%%%%%%%%%%%%%%%%%%%%%%%%%%%%%%%%%%%%%%%%%%%%
% DESCRIPTION:
%
%%%%%%%%%%%%%%%%%%%%%%%%%%%%%%%%%%%%%%%%%%%%%%%%%%%%%%%%%%%%%%%%%%%%%%%%%%%%
% REVISION HISTORY:
%
% $Log$
% Revision 1.7  1997-09-18 00:00:59  jyelon
% *** empty log message ***
%
% Revision 1.6  1997/07/22 12:05:25  jyelon
% Rewrote section on charmc.
%
% Revision 1.5  1995/11/17 17:19:37  brunner
% Incorporated proofreading comments from 11-16 afternoon
%
% Revision 1.4  1995/11/16  23:58:29  brunner
% Alphabetized option list
%
% Revision 1.3  1995/11/14  21:23:55  brunner
% Added paragraph on including a user-defined load balance strategy.
%
% Revision 1.2  1995/11/02  22:40:39  brunner
% Includes short description of what charmc does, and description
% of options
%
% Revision 1.1  1995/10/31  22:27:53  milind
% Initial revision
%
%%%%%%%%%%%%%%%%%%%%%%%%%%%%%%%%%%%%%%%%%%%%%%%%%%%%%%%%%%%%%%%%%%%%%%%%%%%%

\section{Compiling Converse, Charm, and Charm++ Programs}

The {\tt charmc} program standardizes compiling and linking procedures
among various machines and operating systems.  The word ``charmc'' is
slightly misleading, this is a general-purpose tool for compiling and
linking, not restricted to charm programs at all.

Charmc can perform the following tasks.  The (simplified) syntax for
each of these modes is shown.  Caution: in reality, one almost always
has to add some command-line options in addition to the simplified
syntax shown below.  The options are described next.

\begin{verbatim}
 * Compile C                    charmc -o pgm.o pgm.c
 * Compile C++                  charmc -o pgm.o pgm.C
 * Compile Charm                charmc -o pgm.o pgm.p
 * Compile old-style Charm++    charmc -o pgm.o pgm.P
 * Link                         charmc -o pgm   obj1.o obj2.o obj3.o...
 * Compile + Link               charmc -o pgm   src1.c src2.C src3.p src4.P
 * Create Library               charmc -o lib.a obj1.o obj2.o obj3.o...
 * CPM preprocessing            charmc -gen-cpm file.c
 * Preprocess modern Charm++    charmc file.ci
\end{verbatim}

Charmc has been given data on how to invoke the compilers on each
different platform.  This data is in a file named {\tt conv-mach.csh},
which can be found in the same directory where {\tt charmc} resides.
Local modifications to the commands or options {\tt charmc} uses may
usually be accomplished by editing {\tt conv-mach.csh}.

Charmc automatically figures out where the charm lib and include
directories are --- at no point do you have to configure this
information.  However, the code that finds the lib and include
directories can be confused if you remove charmc from its normal
directory, or rearrange the directory tree.  Thus, the files in the
charm distribution must be left where they are, relative to each
other.  Use symbolic links if you want to put a copy of charmc into
a local bin directory.

The following command-line options are available to users of charmc:

\begin{description}

\item[{\tt -o} {\em output-file}:]

Output file name.  Note: charmc only ever produces one output file at
a time.  Because of this, you cannot compile multiple source files at
once, unless you then link or archive them into a single output-file.
If exactly one source-file is specified, then an output file will be
selected by default using the obvious rule (eg, if the input file if
pgm.c, the output file is pgm.o).  If multiple input files are
specified, you must manually specify the name of the output file,
which must be a library or executable.

\item[{\tt -c}:]

Ignored.  There for compatibility with {\tt cc}.

\item[{\tt -D*}:]

Defines preprocessor variables from the command line at compile time.

\item[{\tt -I}:]

Add a directory to the search path for preprocessor include files.

\item[{\tt -g}:]

Causes compiled files to include debugging information.

\item[{\tt -L*}:]

Add a directory to the search path for libraries selected by
the {\tt -l} command.

\item[{\tt -l*}:]

Specifies libraries to link in.

\item[{\tt -O}:]

Causes files to be compiled with maximum optimization.

\item[{\tt -NO}:]

If this follows -O on the command line, it turns optimization back off.
This is just a convenience for simple-minded makefiles.

\item[{\tt -s}:]

Strip the executable of debugging symbols.  Only meaningful when
producing an executable.

\item[{\tt -save}:]

Intermediate files produced by the Charm or Charm++ translator are saved.

\item[{\tt -verbose}:]

All commands executed by charmc are echoed to stdout.

\item[{\tt -seq}:]

Indicates that we're compiling sequential code.  On parallel machines
with front ends, this option also means that the code is for the front
end.  This option is only valid with C and C++ files.

\item[{\tt -machine} {\em machine-type}:]

If more than one version of converse/charm/charm++ has been installed,
this option allows selection of versions other than the default.  The
default {\em machine-type} is the version in which the {\tt charmc}
being run resides.  See the previous chapter on installing charm.

\item[{\tt -use-fastest-cc}:]

Some environments provide more than one C compiler (cc and gcc, for
example).  Usually, charmc prefers the less buggy of the two.  This
option causes charmc to switch to the most aggressive compiler,
regardless of whether it's buggy or not.

\item[{\tt -use-reliable-cc}:]

Some environments provide more than one C compiler (cc and gcc, for
example).  Usually, charmc prefers the less buggy of the two, but
not always.  This option causes charmc to switch to the most reliable
compiler, regardless of whether it produces slow code or not.

\item[{\tt -language \{converse|converse++|charm|charm++\}}:]

When linking with charmc, one must specify the ``language''.  This
is just a way to help charmc include the right libraries.  Pick the
``language'' according to this table:

\begin{itemize}
\item{{\bf Charm++} if your program includes Charm++, Charm, C++, and C.}
\item{{\bf Charm} if your program includes Charm and C.}
\item{{\bf Converse++} if your program includes C++ and C.}
\item{{\bf Converse} if your program includes only C.}
\end{itemize}

\item[{\tt -balance} {\em load-balance-strategy}:]

When compiling Charm or Charm++, one must include a load-balancing
library.  There are currently three to choose from: {\tt rand}, {\tt
acwn}, and {\tt mngr} are supported.  Default is {\tt -balance rand}.

\item[{\tt -tracemode} {\em tracing-mode}:]

Selects the desired degree of tracing for Charm and Charm++ programs.
See the Charm manual and the Projections and SummaryTool manuals for
more information.  Currently supported modes are {\tt none}, {\tt
summary}, and {\tt projections}. Default is {\tt -tracemode none}.


\item[{\tt -c++} {\em C++ compiler}:]

Forces the specified C++ compiler to be used.

\item[{\tt -cc} {\em C-compiler}:]

Forces the specified C compiler to be used.

\item[{\tt -cp} {\em copy-file}:]

Creates a copy of the output file in {\em copy-file}.

\item[{\tt -cpp-option} {\em options}:]

Options passed to the C pre-processor.

\item[{\tt -ld} {\em linker}:]

Use this option only when compiling programs that do not include C++
modules.  Forces charmc to use the specified linker.

\item[{\tt -ld++} {\em linker}:]

Use this option only when compiling programs that include C++
modules.  Forces charmc to use the specified linker.

\item[{\tt -ld++-option} {\em options}:]

Options passed to the linker for {\tt -language charm++}.

\item[{\tt -ld-option} {\em options}:]

Options passed to the linker for {\tt -language charm}.

\item[{\tt -ldro-option} {\em options}:]

Options passes to the linker when linking {\tt .o} files.

\item[{\tt -queue} {\em queueing strategy}:]

Currently ignored.

\end{description}

\section{Executing \charmpp{} Programs}
\label{executing charm programs}

When compiling \charmpp{} programs, the charmc linker produces 
both an executable file and a program called {\tt charmrun},
which is used to load the executable onto the parallel machine.

To run a \charmpp{} program named ``pgm'' on four processors, type:
\begin{alltt}
charmrun pgm +p4
\end{alltt}


Programs built using the network version of \charmpp{} can be run
alone, without charmrun.  This restricts you to using the processors
on the local machine, but it is convenient and often useful for
debugging.  For example, a \charmpp{} program can be run on one
processor in the debugger using:

\begin{alltt}
gdb pgm
\end{alltt}

If the program needs some environment variables
to be set for its execution on compute nodes
(such as library paths), they can be set in
.charmrunrc under home directory. charmrun
will run that shell script before running the executable.

\subsection[Command Line Options]{Command Line Options}
\label{command line options}
\index{command line options}

A \charmpp{} program accepts the following command line options:
\begin{description}

\item[{\tt +pN}] Run the program with N processors. The default is 1.

\item[{\tt +ss}] Print summary statistics about chare creation.  This option
prints the total number of chare creation requests, and the total number of
chare creation requests processed across all processors.

\item[{\tt +cs}] Print statistics about the number of create chare messages
requested and processed, the number of messages for chares requested and 
processed, and the number of messages for branch office chares requested and
processed, on a per processor basis.  Note that the number of messages 
created and processed for a particular type of message on a given node 
may not be the same, since a message may be processed by a different
processor from the one originating the request.

\item[{\tt user\_options}] Options that are be interpreted by the user
program may be included mixed with the system options. 
However, {\tt user\_options} cannot start with +.
The {\tt user\_options} will be passed as arguments to the user program 
via the usual {\tt argc/argv} construct to the {\tt main}
entry point of the main chare. 
\charmpp{} system options will not appear in {\tt argc/argv}.

\end{description}



\subsubsection{Additional Network Options}
\label{network command line options}

The following {\tt ++} command line options are available in
the network version:
\begin{description}

\item[{\tt ++local}] Run charm program only on local machines. No 
 remote shell invocation is needed in this case. It starts node programs 
 right on your local machine. This could be useful if you just want to 
 run small program on only one machine, for example, your laptop.


\item[{\tt ++mpiexec}]

Use the cluster's {\tt mpiexec} job launcher instead of the built in rsh/ssh
method.

This will pass {\tt -n \$P} to indicate how many processes to
launch. An executable named something other than {\tt mpiexec} can be
used with the additional argument {\tt ++remote-shell} {\it runmpi},
with `runmpi' replaced by the necessary name.

Use of this option can potentially provide a few benefits:

\begin{itemize}
\item Faster startup compared to the SSH/RSH approach charmrun would
  otherwise use
\item No need to generate a nodelist file
\item Multi-node job startup on clusters that do not allow connections
  from the head/login nodes to the compute nodes
\end{itemize}

At present, this option depends on the environment variables for some
common MPI implementations. It supports OpenMPI ({\tt OMPI\_COMM\_WORLD\_RANK} and
{\tt OMPI\_COMM\_WORLD\_SIZE}) and M(VA)PICH ({\tt MPIRUN\_RANK} and {\tt
  MPIRUN\_NPROCS} or {\tt PMI\_RANK} and {\tt PMI\_SIZE}).

\item[{\tt ++debug}] Run each node under gdb in an xterm window, prompting
the user to begin execution.

\item[{\tt ++debug-no-pause}] Run each node under gdb in an xterm window
immediately (i.e. without prompting the user to begin execution).

If using one of the {\tt ++debug} or {\tt ++debug-no-pause} options,
the user must ensure the following:
\begin{enumerate}

\item The {\tt DISPLAY} environment variable points to your terminal.
SSH's X11 forwarding does not work properly with \charmpp{}.

\item The nodes must be authorized to create windows on the host machine (see
man pages for {\tt xhost} and {\tt xauth}).

\item {\tt xterm}, {\tt xdpyinfo},  and {\tt gdb} must be in
the user's path.

\item The path must be set in the {\tt .cshrc} file, not the {\tt .login}
file, because {\tt rsh} does not run the {\tt .login} file. 

\end{enumerate}

\item[{\tt ++maxrsh}] Maximum number of {\tt rsh}'s to run at a
time.

\item[{\tt ++nodelist}] File containing list of nodes.


\item[{\tt ++ppn}]              number of pes per node

\item[{\tt ++help}]             print help messages

\item[{\tt ++runscript}]        script to run node-program with

\item[{\tt ++xterm}]            which xterm to use

\item[{\tt ++in-xterm}]         Run each node in an xterm window

\item[{\tt ++display}]          X Display for xterm

\item[{\tt ++debugger}]         which debugger to use

\item[{\tt ++remote-shell}]     which remote shell to use

\item[{\tt ++useip}]            Use IP address provided for charmrun IP

\item[{\tt ++usehostname}]      Send nodes our symbolic hostname instead of IP address

\item[{\tt ++server-auth}]      CCS Authentication file

\item[{\tt ++server-port}]      Port to listen for CCS requests

\item[{\tt ++server}]           Enable client-server (CCS) mode

\item[{\tt ++nodegroup}]        which group of nodes to use

\item[{\tt ++verbose}]          Print diagnostic messages

\item[{\tt ++timeout}]          seconds to wait per host connection

\item[{\tt ++p}]                number of processes to create

\end{description}

\subsubsection{Multicore Options}

On multicore platforms, operating systems (by default) are free to move
processes and threads among cores to balance load. This however sometimes can
degrade the performance of Charm++ applications due to the extra overhead of
moving processes and threads, especailly when Charm++ applications has already
implemented its own dynamic load balancing.

Charm++ provides the following runtime options to set the processor affinity
automatically so that processes or threads no longer move. When cpu affinity
is supported by an operating system (tested at Charm++ configuration time),
same runtime options can be used for all flavors of Charm++ versions including
network and MPI versions, smp and non-smp versions.

\begin{description}

\item[{\tt +setcpuaffinity}]             set cpu affinity automatically for processes (when Charm++ is based on non-smp versions) or threads (when smp)

\item[{\tt +excludecore <core \#>}]       does not set cpu affinity for the given core number. One can use this option multiple times to provide a list of core numbers to avoid.

\item[{\tt +pemap L[-U[:S[.R]]][,...]}] Bind the execution threads to
  the sequence of cores described by the arguments using the operating
  system's CPU affinity functions.

A single number identifies a particular core. Two numbers separated by
a dash identify an inclusive range (\emph{lower bound} and \emph{upper
  bound}). If they are followed by a colon and another number (a
\emph{stride}), that range will be stepped through in increments of
the additional number. Within each stride, a dot followed by a
\emph{run} will indicate how many cores to use from that starting
point.

For example, the sequence {\tt 0-8:2,16,20-24} includes cores 0, 2, 4,
6, 8, 16, 20, 21, 22, 23, 24. On a 4-way quad-core system, if one
wanted to use 3 cores from each socket, one could write this as {\tt
  0-15:4.3}.

\item[{\tt +commap p[,q,...]}] Bind communication threads to the
  listed cores, one per process.

\end{description}

\subsection{Nodelist file}

For network of workstations,
the list of machines to run the program can be specified in a file.  
Without a nodelist file, \charmpp{} runs the program only on the 
local machine.

The format of this file
allows you to define groups of machines, giving each group a name.
Each line of the nodes file is a command.  The most important command
is:

\begin{alltt}
host <hostname> <qualifiers>
\end{alltt}

which specifies a host.  The other commands are qualifiers: they modify
the properties of all hosts that follow them.  The qualifiers are:


\begin{tabbing}
{\tt group <groupname>}~~~\= - subsequent hosts are members of specified group\\
{\tt login <login>  }     \> - subsequent hosts use the specified login\\
{\tt shell <shell>  }     \> - subsequent hosts use the specified remote 
shell\\
%{\tt passwd <passwd>}     \> - subsequent hosts use the specified password\\
{\tt setup <cmd>  }       \> - subsequent hosts should execute cmd\\
{\tt pathfix <dir1> <dir2>}         \> - subsequent hosts should replace dir1 with dir2 in the program path\\
{\tt cpus <n>}            \> - subsequent hosts should use N light-weight processes\\
{\tt speed <s>}           \> - subsequent hosts have relative speed rating\\
{\tt ext <extn>}          \> - subsequent hosts should append extn to the pgm name\\
\end{tabbing}

{\bf Note:}
By default, charmrun uses a remote shell ``rsh'' to spawn node processes
on the remote hosts. The {\tt shell} qualifier can be used to override
it with say, ``ssh''. One can set the {\tt CONV\_RSH} environment variable
or use charmrun option {\tt ++remote-shell} to override the default remote 
shell for all hosts with unspecified {\tt shell} qualifier.

All qualifiers accept ``*'' as an argument, this resets the modifier to
its default value.  Note that currently, the passwd, cpus, and speed
factors are ignored.  Inline qualifiers are also allowed:

\begin{alltt}
host beauty ++cpus 2 ++shell ssh
\end{alltt}

Except for ``group'', every other qualifier can be inlined, with the
restriction that if the ``setup'' qualifier is inlined, it should be
the last qualifier on the ``host'' or ``group'' statement line.

Here is a simple nodes file:

\begin{alltt}
        group kale-sun ++cpus 1
          host charm.cs.uiuc.edu ++shell ssh
          host dp.cs.uiuc.edu
          host grace.cs.uiuc.edu
          host dagger.cs.uiuc.edu
        group kale-sol
          host beauty.cs.uiuc.edu ++cpus 2
        group main
          host localhost
\end{alltt}

This defines three groups of machines: group kale-sun, group kale-sol,
and group main.  The ++nodegroup option is used to specify which group
of machines to use.  Note that there is wraparound: if you specify
more nodes than there are hosts in the group, it will reuse
hosts. Thus,

\begin{alltt}
        charmrun pgm ++nodegroup kale-sun +p6
\end{alltt}

uses hosts (charm, dp, grace, dagger, charm, dp) respectively as
nodes (0, 1, 2, 3, 4, 5).

If you don't specify a ++nodegroup, the default is ++nodegroup main.
Thus, if one specifies

\begin{alltt}
        charmrun pgm +p4
\end{alltt}

it will use ``localhost'' four times.  ``localhost'' is a Unix
trick; it always find a name for whatever machine you're on.

The user is required to set up remote login permissions on all nodes
using the ``.rhosts'' file in the home directory if ``rsh'' is used for remote
login into the hosts. If ``ssh'' is used, the user will have to setup
password-less login to remote hosts using
RSA authentication based on a key-pair and adding public keys to 
``.ssh/authorized\_keys'' file. See ``ssh'' documentation for more information.

In a network environment, {\tt charmrun} must
be able to locate the directory of the executable.  If all workstations
share a common file name space this is trivial.  If they don't, {\tt charmrun}
will attempt to find the executable in a directory with the same path
from the {\bf \$HOME} directory.  Pathname resolution is performed as 
follows:
\begin{enumerate}
	\item The system computes the absolute path of {\tt pgm}.
	\item If the absolute path starts with the equivalent of {\bf \$HOME} 
	or the current working directory, the beginning part of the 
        path 
	is replaced with the environment variable {\bf \$HOME} or the 
	current working directory. However, if {\tt ++pathfix dir1 dir2} is 
        specified in the nodes file (see above), the part of
        the path matching {\tt dir1} is replaced with {\tt dir2}.
	\item The system tries to locate this program (with modified 
	pathname and appended extension if specified) on all nodes.
\end{enumerate}

\subsubsection{IO buffering options}
\label{io buffer options}
There may be circumstances where a \charmpp{} application may want to take
or relinquish control of stdout buffer flushing. Most systems default to
giving the \charmpp{} runtime control over stdout but a few default to
giving the application that control. The user can override these system
defaults with the following runtime options:

\begin{description}
\item[{\tt +io\_flush\_user}]     User (application) controls stdout flushing
\item[{\tt +io\_flush\_system}]   The \charmpp{} runtime controls flushing
\end{description}


\end{document}
