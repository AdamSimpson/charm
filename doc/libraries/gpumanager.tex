\newif\ifstandalone
%\standalonetrue   % use if built stand alone
\standalonefalse  % use if part of the manual

\ifstandalone
  \documentclass[10pt]{report}
  \usepackage[hmargin=0.85in,vmargin=0.85in]{geometry}
  \usepackage{alltt}
  \usepackage{amsmath}
  \usepackage[T1]{fontenc}
\fi

\setcounter{secnumdepth}{3}

\ifstandalone
  \begin{document}

  \chapter{GPU Manager Library}

  \newcommand{\charmpp}{Charm++}
  \newcommand{\code}[1]{\texttt{\textbf{#1}}} % \textbf doesn't work here
\else
  \renewcommand{\code}[1]{\texttt{\textbf{#1}}} % \textbf doesn't work here
\fi

\newcommand{\cuda}{\code{CUDA}}

GPU Manager is a task offload and management library for efficient use of
CUDA-enabled GPUs in \charmpp{} applications. Compared to direct use of CUDA
(through issuing kernel invocation and GPU data transfer calls in user
code) GPU Manager provides the following advantages:
\begin{enumerate}
\item Automatic management and synchronization of tasks
\item Automatic overlap of data transfer and kernel invocation for concurrent tasks
\item A simplified work flow mechanism using CkCallback to return to user code after completion of each work request
\item Reduced synchronization overhead through centralized management of all GPU tasks
\end{enumerate}

\section{Building GPU Manager}

GPU Manager is not included by default when building \charmpp{}. In order to use
GPU Manager, the user must build \charmpp{} using the \cuda{} option, e.g.

\begin{alltt}
./build charm++ netlrts-linux-x86_64 cuda -j8
\end{alltt}
Building GPU Manager requires an installation of the CUDA toolkit on the system.

\section{Overview and Work Flow}

GPUs are throughput-oriented devices with peak computational capabilities that
greatly surpass equivalent-generation CPUs but with limited control logic that
constraints them to use as accelerator devices controlled by code executing on
the CPU.

The GPU's dependence on the CPU for dispatch and synchronization of
coarse-grained data transfer and kernel execution has traditionally required
programmers to either (a) halt the execution of work on the CPU whenever issuing
GPU work to simplify synchronization or (b) issue GPU work asynchronously and
carefully manage and synchronize concurrent GPU work in order to ensure
satisfactory progress and good performance. Further, the latter option becomes
significantly more difficult in the context of a parallel program with numerous
concurrent objects that all issue kernel and data transfer calls to the same GPU.

The \charmpp{} GPU Manager is a library designed to address this issue by
automating the management of GPUs. Users of GPU Manager define \emph{work requests}
that specify the GPU kernel and any data transfer operations required before
and after completion of the kernel. The system controls the execution of the
work requests submitted by all the chares on a particular processor. This allows
it to effectively manage execution of work requests and overlap CPU-GPU data
transfer with kernel execution. In steady-state operation, GPU Manager overlaps
kernel execution of one work request with data transfer out of GPU memory for a
preceding work request and the data transfer into GPU memory for a subsequent
work request. This approach avoids blocking the CUDA DMA engine by only submitting
data transfers when they are ready to execute. When using GPU Manager, the user
does not need to poll for completion of GPU operations. The system manages
execution of a work request throughout its life cycle and returns control to the
user upon completion of a work request through a CkCallback object specified by
the user per work request. Another advantage of using GPU Manager is that the
system polls only for a handful of currently executing operations, which avoids
the problem of multiple chares all polling the GPU when using CUDA streams
directly. GPU Manager has options for recording profiling data for kernel
execution and data transfer which can be visualized using the \charmpp{}
Projections profiler.

\subsection{Execution Model and Progress Engine}

Like any \charmpp{} application, programs using GPU Manager typically consist of a
large number of concurrently executing objects. Each object executes code in
response to active messages received from some object within the parallel run,
during which it can send its own active messages or issue one or more work
requests to the GPU Manager for asynchronous execution. Work requests are always
submitted to the local GPU Manager instance at the processing element where the
call is issued. Incoming GPU work requests are simply copied into the GPU
Manager's scheduling queue, at which point the library returns and the caller
can continue with other work.

\charmpp{} employs a message driven programming model. This includes a runtime
system scheduler that is triggered every time a method finishes execution. Under
typical CPU-only execution the scheduler examines the queue of incoming messages
and selects one based on priority and location in the queue. In a \cuda{}
build of \charmpp{}, the scheduler is also programmed to periodically invoke the
GPU Manager progress engine.

GPU Manager contains a queue of all pending work requests. When its progress
function is called, GPU Manager determines whether pending GPU work has
completed since the last time the progress function was called, and whether
additional work requests can begin executing. A \code{workRequest} undergoes the
following stages during its execution:

\begin{enumerate}
\item Device memory allocation and data transfer from host to device
\item Kernel execution
\item Data transfer back to host from device
\item Invocation of a callback function (specified in the \code{workRequest})
\end{enumerate}

Based on the instructions contained in each work request, the GPU Manager will
allocate the required buffers in GPU global memory and issue asynchronous CUDA
data transfer operations directly. In order to execute kernels, the GPU Manager
calls the \code{runKernel} function that must be defined by the user. This
function specifies the CUDA kernel call for your work request.

Under steady state execution with multiple concurrent work requests, as
one \code{workRequest} progresses to the execution stage, GPU Manager will
initiate the data transfer for the second
\code{workRequest} in the queue, and so on.

In a typical application, the work request definition,
kernel run functions, CUDA kernel definitions, and code for submission
of work requests would all go in a \code{.cu} file that is compiled with
\code{nvcc} separately from the other files (e.g. \code{.C}, \code{.ci}) in the
\charmpp{} application.
We make a function call to \code{createWorkRequest} from a .C file to create
and enqueue the workRequest.
The various resulting object files of the application are then to be linked
together into the final executable.


\section{API}

Using GPU Manager involves:
\begin{enumerate}
\item Defining CUDA kernels as in a regular CUDA application
\item Defining work requests and their callback functions
\item Defining the \code{void runMyKernel(workRequest *wr, cudaStream\_t kernelStream, void **deviceBuffers)} functions, used by the GPU Manager to issue a kernel call based on the kernel identifier defined in the work request
\item Submitting work requests to the GPU Manager
\end{enumerate}

\subsection{Work Request}
\code{workRequest} is a simple structure which contains the necessary parameters
for CUDA kernel execution along with some additional members for automating
data transfer between the host and the device.
A work request consists of the following data members:

\begin{description}
\item[dim3 dimGrid]- a triple which defines the grid structure for the kernel;
in the example below \code{dimGrid.x} specifies the number of blocks.
\code{dimGrid.y} and \code{dimGrid.z} are unused.

\item[dim3 dimBlock]- a triple defining each block's structure;
specifies the number of threads in up to three dimensions.

\item[int smemSize]- the number of bytes in shared memory to be dynamically
allocated per block for kernel execution.

\item[int nBuffers]- number of buffers used by the work request.

\item[dataInfo *bufferInfo]- array of \code{dataInfo} structs containing buffer
information for the execution of the work request. This array must be of size \code{nBuffers}, e.g.
\begin{alltt}
code{wr->bufferInfo = (dataInfo *) malloc(wr->nBuffers * sizeof(dataInfo))}
\end{alltt}
We will explain the contents of \code{dataInfo} struct later.

\item[void *callbackFn]- a pointer to a \code{CkCallback} object specified by the user;
executed after the kernel and memory transfers have finished.

\item[const char *traceName]- A short identifier used for tracing and logging.

\item[function runKernel]- A user defined host function to run the kernel.
   We will pass this function three parameters:
   \begin{description}
   \item[workRequest]- The workrequest being run.
   \item[kernelStream]- The cuda stream to run the kernel in.
   \item[deviceBuffers]- An array of device pointers, indexed by bufferID.
   \end{description}

\item[int state]- the stage of a \code{workRequest}'s execution, set and used internally by the GPU Manager

\item[void *userData]-  may be used to pass scalar values to kernel
calls, such as the size of an array.

\end{description}

\subsubsection{dataInfo}
\begin{description}
\item[int bufferID]- the ID of a buffer in the runtime system's buffer table.
May be specified by the user if direct control over the buffer space is
desired. Otherwise, if it is set to a negative value, the GPU Manager will
assign a valid buffer ID.

\item[int transferToDevice, transferFromDevice]- flags to indicate if the buffer
should be transferred to the device prior to the execution of a kernel, and/or
transferred out after the kernel

\item[int freeBuffer]- a flag to indicate if the device buffer memory should be
freed after execution of \code{workRequest}.

\item[void *hostBuffer]- pointer to host data buffer. In order to allow
asynchronous memory transfer and data computation on device this buffer must
be allocated from page-locked memory.
\begin{alltt}
void *hostBuffer = hapi\_poolMalloc(size);
\end{alltt}
% code around memPool?
This returns the buffer of required size from the GPU Manager's pool of pinned memory on the host.
Direct allocation of pinned memory (e.g. using \code{cudaMallocHost}) is discouraged,
as it will block the CPU until pending GPU work has finished executing.
The user must add the \code{-DGPU\_MEMPOOL} flag while compiling CUDA files.
This is required to enable fetching of page-locked memory from GPU Manager.
You may add it with your \code{NVCC\_FLAGS}.

\item[size\_t size]- size of buffer in bytes.
\end{description}

\subsubsection{Work Request Example}

Here is an example method for creating a \code{workRequest} of the addition of
two vectors A and B.

\begin{alltt}
#include "wr.h"
#define BLOCK_SIZE 256
void createWorkRequest(int vectorSize, float *h_A, float *h_B, float **h_C, int myIndex, CkCallback *cb)
\{
    dataInfo *info;
    workRequest *vecAdd = new workRequest;
    int size = vectorSize * sizeof(float);

    vecAdd->dimGrid.x = (vectorSize - 1) / BLOCK_SIZE + 1;
    vecAdd->dimBlock.x = BLOCK_SIZE;
    vecAdd->smemSize = 0;
    vecAdd->nBuffers = 3;
    vecAdd->bufferInfo = new dataInfo[vecAdd->nBuffers];

    /* Buffer A */
    info = &(vecAdd->bufferInfo[0]);

    /* The Buffer ID will be given by the API,
       or it can be given by the user. */
    info->bufferID = -1;

    info->transferToDevice = YES;
    info->transferFromDevice = NO;
    info->freeBuffer = YES;

    /* This fetches the pinned host memory already allocated by API,
       required for asynchronous data transfers. */
    info->hostBuffer = hapi_poolMalloc(size);

    /* Copy the data to the workRequest's buffer. */
    memcpy(info->hostBuffer, h_A, size);

    info->size = size;

    /* Buffer B will be same as A.*/

    /* Buffer C */
    info = &(vecAdd->bufferInfo[2]);
    info->transferFromDevice = YES;
    info->hostBuffer = hapi_poolMalloc(size)

    / * We change the address to the address returned by the API
        to read the copied result */
    *h_C = (float *)info->hostBuffer;

    /* a CkCallback pointer */
    vecAdd->callbackFn = cb;

    vecAdd->traceName = "add";

    /* kernel run function */
    vecAdd->runKernel = run_add;

    vecAdd->userData = new int;
    memcpy(vecAdd->userData, &vectorSize, sizeof(int));

    /* enqueue the workRequest in the workRequestQueue.
    wrQueue is declared by our API during the init phase for every processor. */
    enqueue(wrQueue, vecAdd);
\}
\end{alltt}

\subsection{Writing Kernels}

Writing a kernel is unchanged from normal CUDA programs.
Kernels are written in one (or more) .cu files.
Here is an example of \code{vectorAdd.cu}.
The full example can be found in \code{examples/charm++/cuda/vectorAdd/}.

\begin{alltt}
__global__ void vecAdd(float *a, float *b, float *c, int n)
\{
    // Get our global thread ID
    int id = blockIdx.x * blockDim.x + threadIdx.x;

    // Make sure we do not go out of bounds
    if (id < n)
        c[id] = a[id] + b[id];
\}
\end{alltt}

\subsection{Launching Kernels}

Kernel launches are identical to regular kernel launches in normal CUDA
programs, run in a small dedicated function.

\begin{alltt}
void run_add(workRequest *wr, cudaStream_t kernelStream, void **deviceBuffers)
\{
    printf("Add KERNEL \textbackslash{n}");
/*
 *  devBuffers is declared by our API during the init phase on every processor.
 *  It jumps to the correct array index with the help of bufferID, 
 *  which is supplied by the API or user.
 */
    vecAdd<<< wr->dimGrid, wr->dimBlock, wr->smemSize, kernelStream>>>
        ((float *) deviceBuffers[wr->bufferInfo[0].bufferID],
         (float *) deviceBuffers[wr->bufferInfo[1].bufferID],
         (float *) deviceBuffers[wr->bufferInfo[2].bufferID],
         *((int *) wr->userData));
\}
\end{alltt}

\section{Compiling}

As mentioned earlier, there are no changes to the \code{.ci} and \code{.C}
files. Therefore there are no changes in compiling them.
CUDA code, however, must be compiled using \code{nvcc}.
You can use the following example makefile to compile a .cu file.
More example codes can be found in the \code{examples/charm++/cuda} directory.

\begin{alltt}
CUDA_LEVEL=35

NVCC = /usr/local/cuda/bin/nvcc

NVCC_FLAGS = -O3 -c -use_fast_math -DGPU_MEMPOOL

NVCC_FLAGS += -arch=compute_\$(CUDA_LEVEL) -code=sm_\$(CUDA_LEVEL)

NVCC_INC = -I/usr/local/cuda/include

CHARMINC = -I\$\{CHARMDIR\}/include

LD_LIBS= -lcublas

all: vectorAdd
\qquad\$(NVCC) \$(NVCC_FLAGS) \$(NVCC_INC) \$(CHARMINC) -o vectorAddCU.o vectorAdd.cu
\end{alltt}

GPU Manager also supports \texttt{CuBLAS} or other GPU libraries in exactly the same way.
Call \texttt{CuBLAS} or the other GPU library directly from a kernel run function;
creating the \code{workRequest} works the same as any other kernel.


\section{Debugging}

A few useful things for debugging:

\begin{enumerate}
\item Enabling the \code{GPU\_MEMPOOL\_DEBUG} flag
(using \code{-DGPU\_MEMPOOL\_DEBUG}) during execution prints debug
statements, including when buffers are allocated and freed.

\item When using \code{++debug}, add the debugging flags \code{-g} and
\code{-G} during compilation.
\end{enumerate}

\ifstandalone
  \end{document}
\fi
