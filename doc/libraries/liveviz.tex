\section{Introduction}

If array elements compute a small piece of a large 2D-image, then these 
chucks can be combined to form one large image using liveViz. In other 
words, liveViz is a reducer for 2D-image data, which combines small chunks
 of images deposited by chares into one large image.

This visualization library follows client server model, i.e. image 
assembly is done on server side and final image can be viewed using 
liveViz client avaliable at:
\begin{alltt}
           .../charm/pgms/charm++/ccs/liveViz/client
\end{alltt}

A sample liveViz server example is available at:
\begin{alltt}
           .../charm/pgms/charm++/ccs/liveViz/server
\end{alltt}

\section{How to use liveViz with \charmpp{} program}

A program must provide a chare array with one entry fuction having 
following prototype:

\begin{alltt}
  void functionName (liveVizRequestMsg *m)
\end{alltt}

This entry method is supposed to deposit its (array element's) chunk of 
image. This entry method has following structure:

\begin{alltt}
  void functionName (liveVizRequestMsg *m)
  \{
    // prepare image chunk
       ...

    liveVizDeposit (m, startX, startY, width, height, imageBuff, this);

    // delete image buffer if it was dynamically allocated
  \}
\end{alltt}

To know the width and height of image data m->req.wid and m->req.ht can 
be used, here 'm' is the pointer to 'liveVizRequestMsg'. 

\section{Format of deposit image}

'imageBuff' is run of bytes representing image. if image is gray-scale each
 byte represents one pixel otherwise 3 consecutive bytes (starting at 
array index which is a multiple of 3) represents a pixel.

\section{liveViz Initialization}

liveViz library needs to be initialized before it can be used for 
visualization. For initialization follow the following steps:
\begin{itemize}
\item create the chare array (array proxy object 'a') with the entry 
      method 'functionName' (described above).
\item create a CkCallback object ('c'), specifying 'functionName' as the 
      callback function.
\item create a liveVizConfig object ('cfg'). if image to be deposited is color 
      image (3 bytes per pixel) then liveVizConfig object need to be initialized
      as:
\begin{alltt}
          liveVizConfig cfg (true);// first parameter specifies color/gray-scale
                                   // image
\end{alltt}

\item call liveVizInit (cfg, a, c)
\end{itemize}

\section{Compilation}

\charmpp{} program using liveViz must be linked with '-module liveViz'. Before
compiling liveViz program, liveViz library needs to be compiled. To compile
liveViz library:
\begin{itemize}
\item go to .../charm/tmp/libs/ck-libs/liveVize
\item make
\end{itemize}

\section{Extensions}

In above example, liveViz client repeated (or once) asks for image, getting
this request liveViz library asks array elements for image chunks, combines
image chunks and sends the image to client.

There is another method by which server sends image to client (whether it
requested it or not). Fot this mode, second parameter passed to liveVizConfig
constructor must be true. Further details on how to use this mode will be
updated soon.

For 3-d images, third parameter to liveVizConfig constructor needs to be
'true'. Further details for this also will be updated soon. 
