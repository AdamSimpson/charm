\documentclass[10pt]{article}
\usepackage{../pplmanual}
%%% Commonly Needed packages
\usepackage{graphicx,color,calc}
\usepackage{fancyvrb}
\usepackage{makeidx}
\usepackage{alltt}
\usepackage[linkbordercolor=(0 0 1),citebordercolor=(0 1 0)]{hyperref}
%%\usepackage{xspace} <- creates problems with other hyperlink packages like "html"

%%% Commands for uniform looks of C++, Charm++, and Projections
\newcommand{\CC}{C\hbox{++}}
\newcommand{\emCC}{C\hbox{\em++}}
\newcommand{\charmpp}{\textsc{Charm++}}
\newcommand{\charmc}{\texttt{charmc}}
\newcommand{\projections}{\textsc{Projections}}
\newcommand{\converse}{\textsc{Converse}}
\newcommand{\ampi}{\textsc{AMPI}}
\newcommand{\tempo}{\textsc{TeMPO}}
\newcommand{\irecv}{\textsl{iRecv}}
\newcommand{\sdag}{\textsl{Structured Dagger}}
\newcommand{\jade}{Jade}

%%% Commands to produce margin symbols
\newcommand{\new}{\marginpar{\fbox{\bf$\mathcal{NEW}$}}}
\newcommand{\important}{\marginpar{\fbox{\bf\Huge !}}}
\newcommand{\experimental}{\marginpar{\fbox{\bf\Huge $\beta$}}}

%%% Commands for manual elements
\newcommand{\zap}[1]{ }
\newcommand{\function}[1]{{\noindent{\textsf{#1}}\\}}
\newcommand{\cmd}[1]{{\noindent{\textsf{#1}}\\}}
\newcommand{\args}[1]{\hspace*{2em}{\texttt{#1}}\\}
\newcommand{\prototype}[1]{\vspace{0.2in}\index{#1}}
\newcommand{\param}[1]{{\texttt{#1}}}
\newcommand{\kw}[1]{{\textsf{#1}\index{#1}}}
\newcommand{\uw}[1]{{\textsl{#1}}}
\newcommand{\desc}[1]{\indent{#1}}
\newcommand{\note}[1]{(\textbf{Note:} #1)}
\newcommand{\term}[1]{{\bf #1}\index{#1}}

\makeindex


\title{Adaptive MPI Manual}
\version{1.0}
\credits{
AMPI has been developed by Milind Bhandarkar with inputs from Gengbin Zheng and
Orion Lawlor. The derived data types (DDT) library, which AMPI uses for the
derived data types support, has been developed by Neelam Saboo.
}

\begin{document}
\maketitle

\section{Introduction}

This manual describes Adaptive MPI~(\ampi{}), which is an implementation of a
significant subset of MPI 1.1 standard over \charmpp{}. \charmpp{} is a
\CC{}-based parallel programming library developed by Prof. L. V. Kale and his
students over the last 10 years at University of Illinois.

We first describe our philosophy behind this work (why we do what we do).
Later we give a brief introduction to \charmpp{} and rationale for \ampi{}
(tools of the trade). We describe \ampi{} in detail. Finally we summarize the
changes required for original MPI codes to get them working with \ampi{}
(current state of our work). Appendices contain the gory details of installing
\ampi{}, building and running \ampi{} programs.

\subsection{Our Philosophy}

Developing parallel Computational Science and Engineering (CSE) applications is
a complex task. One has to implement the right physics, develop or choose and
code appropriate numerical methods, decide and implement the proper input and
output data formats, perform visualizations, and be concerned with correctness
and efficiency of the programs. It becomes even more complex for multi-physics
coupled simulations such as the solid propellant rocket simulation application.  
Our philosophy is to lessen the burden of the application developer by
providing advanced programming paradigms and versatile runtime systems that can
handle many common performance concerns automatically and let the application
programmers focus on the actual application content.

One such concern is that of load imbalance. In a dynamic simulation application
such as rocket simulation, burning  solid fuel, sub-scaling for a certain part
of the mesh, crack propagation, particle flows all contribute to load
imbalance. Centralized load balancing strategy built into an application is
impractical since each individual modules are developed almost independently by
various developers. Thus, the runtime system support for load balancing becomes
even more critical.

Automatic load balancing is infeasible for a program about which nothing is
known. Other approaches to automatic load balancing therefore require the
applications to provide hints about the load to the runtime system, or restrict
load balance to a certain kind of algorithms such as Adaptive Mesh Refinement
or  to certain architectures such as shared memory machines. Our approach is
based on actual measurement of load information at runtime, and on migrating
computations from heavily loaded to lightly loaded processors.

For this approach to be effective, we need the computation to be split into
pieces many more in number than available processors. This allows us to
flexibly map and re-map these computational pieces to available processors.
This approach is usually called ``multi-domain decomposition''.

\charmpp{}, which we use as a runtime system layer for the work described here
exemplifies our approach. It embeds an elaborate performance tracing mechanism,
a suite of plug-in load balancing strategies, infrastructure for defining and
migrating computational load, and is interoperable with other programming
paradigms.

\subsection{Terminology}

\begin{description}

\item[Module] A module refers to either a complete program or a library with an
orchestrator subroutine\footnote{Like many software engineering terms, this
term is overused, and unfortunately clashes with Fortran 90 module that denotes
a program unit. We specifically refer to the later as ``Fortran 90 module'' to
avoid confusion.} . An orchestrator subroutine specifies the main control flow
of the module by calling various subroutines from the associated library and
does not usually have much state associated with it.

\item[Thread] A thread is a lightweight process that owns a stack and machine
registers including program counter, but shares code and data with other
threads within the same address space. If the underlying operating system
recognizes a thread, it is known as kernel thread, otherwise it is known as
user-thread. A context-switch between threads refers to suspending one thread's
execution and transferring control to another thread. Kernel threads typically
have higher context switching costs than user-threads because of operating
system overheads. The policy implemented by the underlying system for
transferring control between threads is known as thread scheduling policy.
Scheduling policy for kernel threads is determined by the operating system, and
is often more inflexible than user-threads. Scheduling policy is said to be
non-preemptive if a context-switch occurs only when the currently running
thread willingly asks to be suspended, otherwise it is said to be preemptive.
\ampi{} threads are non-preemptive user-level threads.

\item[Chunk] A chunk is a combination of a user-level thread and the data it
manipulates. When a program is converted from MPI to \ampi{}, we convert an MPI
process into a chunk. This conversion is referred to as chunkification.

\item[Object] An object is just a blob of memory on which certain computations
can be performed. The memory is referred to as an object's state, and the set
of computations that can be performed on the object is called the interface of
the object.

\end{description}

\section{\charmpp{}}

\charmpp{} is an object-oriented parallel programming library for \CC{}.  It
differs from traditional message passing programming libraries (such as MPI) in
that \charmpp{} is ``message-driven''. Message-driven parallel programs do not
block the processor waiting for a message to be received.  Instead, each
message carries with itself a computation that the processor performs on
arrival of that message. The underlying runtime system of \charmpp{} is called
\converse{}, which implements a ``scheduler'' that chooses which message to
schedule next (message-scheduling in \charmpp{} involves locating the object
for which the message is intended, and executing the computation specified in
the incoming message on that object). A parallel object in \charmpp{} is a
\CC{} object on which a certain computations can be asked to performe from
remote processors.

\charmpp{} programs exhibit latency tolerance since the scheduler always picks
up the next available message rather than waiting for a particular messageto
arrive.  They also tend to be modular, because of their object-based nature.
Most importantly, \charmpp{} programs can be \emph{dynamically load balanced},
because the messages are directed at objects and not at processors; thus
allowing the runtime system to migrate the objects from heavily loaded
processors to lightly loaded processors. It is this feature of \charmpp{} that
we utilize for \ampi{}.

Since many CSE applications are originally written using MPI, one would have to
do a complete rewrite if they were to be converted to \charmpp{} to take
advantage of dynamic load balancing. This is indeed impractical. However,
\converse{} -- the runtime system of \charmpp{} -- came to our rescue here,
since it supports interoperability between different parallel programming
paradigms such as parallel objects and threads. Using this feature, we
developed \ampi{}, an implementation of a significant subset of MPI 1.1
standard over \charmpp{}.  \ampi{} is described in the next section.

\section{AMPI}

\ampi{} utilizes the dynamic load balancing capabilities of \charmpp{} by
associating a ``user-level'' thread with each \charmpp{} migratable object.
User's code runs inside this thread, so that it can issue blocking receive
calls similar to MPI, and still present the underlying scheduler an opportunity
to schedule other computations on the same processor. The runtime system keeps
track of computation loads of each thread as well as communication graph
between \ampi{} threads, and can migrate these threads in order to balance the
overall load while simultaneously minimizing communication overhead. 

For dynamic load balancing to be effective, one needs to map multiple
user-level threads onto a processor. Traditional MPI programs assume that the
entire processor is allocated to themselves, and that only one thread of
control exists within the process's address space.  Thats where the need arises
to make some transformations to the original MPI program in order to run
correctly with \ampi{}.

The basic transformation needed to port the MPI program to \ampi{} is
privatization of global variables.\footnote{Typical Fortran MPI programs
contain three types of global variables.

\begin{enumerate}

\item Global variables that are ``read-only''. These are either
\emph{parameters} that are set at compile-time. Or other variables that are
read as input or set at the beginning of the program and do not change during
execution. It is not necessary to privatize such variables.

\item Global variables that are used as temporary buffers. These are variables
that are used temporarily to store values to be accessible across subroutines.
These variables have a characteristic that there is no blocking call such as
\texttt{MPI\_recv} between the time the variable is set and the time it is ever
used. It is not necessary to privatize such variables either. 

\item True global variables. These are used across subroutines that contain
blocking receives and therefore possibility of a context switche between the
definition and use of the variable. These variables need to be privatized.

\end{enumerate}
}
With the MPI process model, each MPI node can keep a copy of its own
``permanent variables'' -- variables that are accessible from more than one
subroutines without passing them as arguments.  Module variables, ``saved''
subroutine local variables, and common blocks in Fortran 90 belong to this
category. If such a program is executed without privatization on \ampi{}, all
the \ampi{} threads that reside on one processor will access the same copy of
such variables, which is clearly not the desired semantics.  To ensure correct
execution of the original source program, it is necessary to make such
variables ``private'' to individual threads.  

we have employed two strategies to do this privatization transformation. One is
by argument passing. That is, the global variables are bunched together in a
single user-defined type, which is allocated by each thread dynamically. Then a
pointer to this type is passed from subroutine to subroutine as an argument.
Since the subroutine arguments are passed on a stack, which is not shared
across all threads, each subroutine, when executing within a thread operates on
a private copy of the global variables. 

The second method we have employed is called ``dimension increment''. Here, the
dimension of global data items is increased by one. This added dimension is
used to access thread private data by indexing it with the thread number (which
can be obtained with \texttt{MPI\_Comm\_rank} subroutine.) This scheme has a
distinct disadvantage of wastage of space. Therefore it should be used only in
case of small global data. There is another disadvantage, that of false
sharing, if the global data items are not aligned properly.  \footnote{ Another
strategy for doing privatization is also known. It employs registration of
``thread-local data'' with the system and access to those data using ``keys''.
This method is suitable for C and \CC{} programs, but the lack of consistency
in Fortran 90 pointer implementations make it complicated to implement and use
in Fortran.}

These two schemes are demonstrated in the following examples. The original
Fortran 90 code contains a module \texttt{shareddata}. This module is used in
the main program and a subroutine \texttt{subA}.

\begin{alltt}
MODULE shareddata
  INTEGER :: myrank
  DOUBLE PRECISION :: xyz(100)
END MODULE

PROGRAM MAIN
  USE shareddata
  include 'mpif.h'
  INTEGER :: i, ierr
  CALL MPI_Init(ierr)
  CALL MPI_Comm_rank(MPI_COMM_WORLD, myrank, ierr)
  DO i = 1, 100
    xyz(i) =  i + myrank
  END DO
  CALL subA
  CALL MPI_Finalize(ierr)
END PROGRAM

SUBROUTINE subA
  USE shareddata
  INTEGER :: i
  DO i = 1, 100
    xyz(i) = xyz(i) + 1.0
  END DO
END SUBROUTINE
\end{alltt}

\ampi{} executes the main program inside a user-level thread as a subroutine.
For this purpose, the main program needs to be changed to be a subroutine with
the name \verb+MPI_Main+.

Now we transform this program using the first strategy. We first group the
shared data into a user-defined type.

\begin{alltt}
MODULE shareddata
  \emph{TYPE chunk}
    INTEGER :: myrank
    DOUBLE PRECISION :: xyz(100)
  \emph{END TYPE}
END MODULE
\end{alltt}

Now we modify the main program to dynamically allocate this data and change the
references to them. Subroutine \texttt{subA} is then modified to take this data
as argument.

\begin{alltt}
SUBROUTINE MPI_Main
  USE shareddata
  USE AMPI
  INTEGER :: i, ierr
  \emph{TYPE(chunk), pointer :: c}
  CALL MPI_Init(ierr)
  \emph{ALLOCATE(c)}
  CALL MPI_Comm_rank(MPI_COMM_WORLD, c\%myrank, ierr)
  DO i = 1, 100
    \emph{c\%xyz(i) =  i + c\%myrank}
  END DO
  CALL subA(c)
  CALL MPI_Finalize(ierr)
END SUBROUTINE

SUBROUTINE subA(c)
  USE shareddata
  \emph{TYPE(chunk) :: c}
  INTEGER :: i
  DO i = 1, 100
    \emph{c\%xyz(i) = c\%xyz(i) + 1.0}
  END DO
END SUBROUTINE
\end{alltt}

With these changes, the above program can be made thread-safe. Note that it is
not really necessary to dynamically allocate \texttt{chunk}. One could have
declared it as a local variable in subroutine \texttt{MPI\_Main}.  (Or for a
small example such as this, one could have just removed the \texttt{shareddata}
module, and instead declared both variables \texttt{xyz} and \texttt{myrank} as
local variables). This is indeed a good idea if shared data are small in size.
For large shared data, it would be better to do heap allocation because in
\ampi{}, the stack sizes are fixed at the beginning (can be specified from the
command line) and stacks do not grow dynamically.

In the second scheme, we extend the dimension of the shared data, assuming that
we are going to run with maximum 16 threads.

\begin{alltt}
MODULE shareddata
  INTEGER :: myrank(16)
  DOUBLE PRECISION :: xyz(16, 100)
END MODULE
\end{alltt}

Next we change the references to these shared data, by adding an additional
level of indexing:

\begin{alltt}
SUBROUTINE MPI_Main
  USE shareddata
  USE AMPI
  INTEGER :: i, trank, ierr
  CALL MPI_Init(ierr)
  CALL MPI_Comm_rank(MPI_COMM_WORLD, trank, ierr)
  \emph{myrank(trank) = trank}
  DO i = 1, 100
    \emph{xyz(trank, i) =  i + myrank(trank)}
  END DO
  CALL subA()
  CALL MPI_Finalize(ierr)
END SUBROUTINE

SUBROUTINE subA(c)
  USE shareddata
  USE AMPI
  INTEGER :: i, trank, ierr
  CALL MPI_Comm_rank(MPI_COMM_WORLD, trank, ierr)
  DO i = 1, 100
    \emph{xyz(trank, i) = xyz(trank, i) + 1.0}
  END DO
END SUBROUTINE
\end{alltt}

Now, this program is also thread-safe and ready to run with \ampi{}.  

\subsection{AMPI Status}

Currently the following subset of MPI 1.1 standard is supported in \ampi{}.

\begin{alltt}
  MPI_Wtime           MPI_Init          MPI_Comm_rank       
  MPI_Comm_size       MPI_Finalize      MPI_Send         
  MPI_Recv            MPI_Isend         MPI_Irecv        
  MPI_Sendrecv        MPI_Barrier       MPI_Bcast
  MPI_Reduce          MPI_Allreduce     MPI_Start           
  MPI_Waitall         MPI_Send_init     MPI_Recv_init    
  MPI_Type_contiguous MPI_Type_vector   MPI_Type_hvector 
  MPI_Type_indexed    MPI_Type_hindexed MPI_Type_struct
  MPI_Type_commit     MPI_Type_free     MPI_Type_extent     
  MPI_Type_size       MPI_Allgatherv    MPI_Allgather    
  MPI_Gatherv         MPI_Gather        MPI_Alltoallv    
  MPI_Alltoall        MPI_Comm_dup      MPI_Comm_free
  MPI_Abort           MPI_Probe         MPI_Iprobe
  MPI_Testall         MPI_Get_count     MPI_Pack
  MPI_Unpack          MPI_Pack_size
\end{alltt}

Following MPI 1.1 basic datatypes are supported in \ampi{}.

\begin{alltt}
MPI_DOUBLE_PRECISION MPI_INTEGER       MPI_REAL          
MPI_COMPLEX          MPI_LOGICAL       MPI_CHARACTER     
MPI_BYTE             MPI_PACKED        MPI_SHORT            
MPI_LONG             MPI_UNSIGNED_CHAR MPI_UNSIGNED_SHORT
MPI_UNSIGNED         MPI_UNSIGNED_LONG MPI_LONG_DOUBLE
\end{alltt}

Following MPI 1.1 reduction operations are supported in \ampi{}.

\begin{alltt}
MPI_MAX  MPI_MIN  MPI_SUM  MPI_PROD
\end{alltt}

Also, unlike the MPI 1.1 standard, the Fortran90 statement ``\texttt{use mpi}'' 
should not be used.
Instead one needs to put the ``\texttt{use ampi}'' statement in Fortran
program units that use \ampi{}.

\subsection{Extensions for Migrations}

For MPI chunks to migrate, we have added a few calls to \ampi{}. These include
ability to register thread-specific data with the run-time system, to pack all
the thread's data, and to express willingness to migrate.

\subsubsection{Registering Chunk data}

When the \ampi{} runtime system decides that load imbalance exists within the
application, it will invoke one of its internal load balancing strategies,
which determines the new mapping of \ampi{} chunks so as to balance the load.
Then \ampi{} runtime has to pack up the chunk's state and move it to its new
home processor. \ampi{} packs up any internal data in use by the chunk,
including the thread's stack in use. This means that the local variables
declared in subroutines in a chunk, which are created on stack, are
automatically packed up by the \ampi{} runtime system. However, it has no way
of knowing what other data are in use by the chunk. Thus upon starting
execution, a chunk needs to notify the system about the data that it is going
to use (apart from local variables.) Even with the data registration, \ampi{}
cannot determine what size the data is, or whether the registered data contains
pointers to other places in memory. For this purpose, a packing subroutine also
needs to be provided to the \ampi{} runtime system alongwith registered data.
(See next section for writing packing subroutines.) The call provided by
\ampi{} for doing this is \texttt{MPI\_Register}. This function takes two
arguments: A data item to be transported alongwith the chunk, and the pack
subroutine, and returns an integer denoting the registration identifier. In
C/\CC{} programs, it may be necessary to use this return value after migration
completes and control returns to the chunk, using function
\texttt{MPI\_Get\_userdata}. Therefore, the return value should be stored in a
local variable.

\subsubsection{Migration}

The \ampi{} runtime system could detect load imbalance by itself and invoke the
load balancing strategy. However, since the application code is going to
pack/unpack the chunk's data, writing the pack subroutine will be complicated
if migrations occur at a stage unknown to the application. For example, if the
system decides to migrate a chunk while it is in initialization stage (say,
reading input files), application code will have to keep track of how much data
it has read, what files are open etc. Typically, since initialization occurs
only once in the beginning, load imbalance at that stage would not matter much.
Therefore, we want the demand to perform load balance check to be initiated by
the application.

\ampi{} provides a subroutine \texttt{MPI\_Migrate} for this purpose. Each
chunk periodically calls \texttt{MPI\_Migrate}. Typical CSE applications are
iterative and perform multiple time-steps. One should call
\texttt{MPI\_Migrate} in each chunk at the end of some fixed number of
timesteps. The frequency of \texttt{MPI\_Migrate} should be determined by a
tradeoff between conflicting factors such as the load balancing overhead, and
performance degradation caused by load imbalance. In some other applications,
where application suspects that load imbalance may have occurred, as in the
case of adaptive mesh refinement; it would be more effective if it performs a
couple of timesteps before telling the system to re-map chunks. This will give
the \ampi{} runtime system some time to collect the new load and communication
statistics upon which it bases its migration decisions. Note that
\texttt{MPI\_Migrate} does NOT tell the system to migrate the chunk, but
merely tells the system to check the load balance after all the chunks call
\texttt{MPI\_Migrate}. To migrate the chunk or not is decided only by the
system's load balancing strategy.

\subsubsection{Packing/Unpacking Thread Data}

Once the \ampi{} runtime system decides which chunks to send to which
processors, it calls the specified pack subroutine for that chunk, with the
chunk-specific data that was registered with the system using
\texttt{MPI\_Register}. This section explains how a subroutine should be
written for performing pack/unpack.

There are three steps to transporting the chunk's data to other processor.
First, the system calls a subroutine to get the size of the buffer required to
pack the chunk's data. This is called the ``sizing'' step. In the next step,
which is called immediately afterward on the source processor, the system
allocates the required buffer and calls the subroutine to pack the chunk's data
into that buffer. This is called the ``packing'' step. This packed data is then
sent as a message to the destination processor, where first a chunk is created
(alongwith the thread) and a subroutine is called to unpack the chunk's data
from the buffer. This is called the ``unpacking'' step.

Though the above description mentions three subroutines called by the \ampi{}
runtime system, it is possible to actually write a single subroutine that will
perform all the three tasks. This is achieved using something we call a
``pupper''. A pupper is an external subroutine that is passed to the chunk's
pack-unpack-sizing subroutine, and this subroutine, when called in different
phases performs different tasks. An example will make this clear:

Suppose the chunk data is defined as a user-defined type in Fortran 90:

\begin{alltt}
MODULE chunkmod
  TYPE, PUBLIC :: chunk
      INTEGER , parameter :: nx=4, ny=4, tchunks=16
      REAL(KIND=8) t(22,22)
      INTEGER xidx, yidx
      REAL(KIND=8), dimension(400):: bxm, bxp, bym, byp
  END TYPE chunk
END MODULE
\end{alltt}

Then the pack-unpack subroutine \texttt{chunkpup} for this chunk module is
written as:

\begin{alltt}
SUBROUTINE chunkpup(p, c)
  USE pupmod
  USE chunkmod
  IMPLICIT NONE
  INTEGER :: p
  TYPE(chunk) :: c

  call pup(p, c\%t)
  call pup(p, c\%xidx)
  call pup(p, c\%yidx)
  call pup(p, c\%bxm)
  call pup(p, c\%bxp)
  call pup(p, c\%bym)
  call pup(p, c\%byp)
end subroutine
\end{alltt}

There are several things to note in this example. First, the same subroutine
\texttt{pup} (declared in module \texttt{pupmod}) is called to size/pack/unpack
any type of data. This is possible because of procedure overloading possible in
Fortran 90. Second is the integer argument \texttt{p}. It is this argument that
specifies whether this invocation of subroutine \texttt{chunkpup} is sizing,
packing or unpacking. Third, the integer parameters declared in the type
\texttt{chunk} need not be packed or unpacked since they are guaranteed to be
constants and thus available on any processor.

A few other functions are provided in module \texttt{pupmod}. These functions
provide more control over the packing/unpacking process. Suppose one modifies
the \texttt{chunk} type to include allocatable data or pointers that are
allocated dynamically at runtime. In this case, when the chunk is packed, these
allocated data structures should be deallocated after copying them to buffers,
and when the chunk is unpacked, these data structures should be allocated
before copying them from the buffers.  For this purpose, one needs to know
whether the invocation of \texttt{chunkpup} is a packing one or unpacking one.
For this purpose, the \texttt{pupmod} module provides functions
\verb+pup_ispk+(\verb+pup_isupk+). These functions return logical value
\verb+.TRUE.+ if the invocation is for packing (unpacking), and \verb+.FALSE.+
otherwise. Following example demonstrates this:

Suppose the type \texttt{dchunk} is declared as:

\begin{alltt}
MODULE dchunkmod
  TYPE, PUBLIC :: dchunk
      INTEGER :: asize
      REAL(KIND=8), pointer :: xarr(:), yarr(:)
  END TYPE dchunk
END MODULE
\end{alltt}

Then the pack-unpack subroutine is written as:

\begin{alltt}
SUBROUTINE dchunkpup(p, c)
  USE pupmod
  USE dchunkmod
  IMPLICIT NONE
  INTEGER :: p
  TYPE(dchunk) :: c

  pup(p, c\%asize)
  \emph{
  IF (pup_isupk(p)) THEN       !! if invocation is for unpacking
    allocate(c\%xarr(asize))
    ALLOCATE(c\%yarr(asize))
  ENDIF
  }
  pup(p, c\%xarr)
  pup(p, c\%yarr)
  \emph{
  IF (pup_ispk(p)) THEN        !! if invocation is for packing
    DEALLOCATE(c\%xarr(asize))
    DEALLOCATE(c\%yarr(asize))
  ENDIF
  }

END SUBROUTINE
\end{alltt}

One more function \verb+pup_issz+ is also available in module \texttt{pupmod}
that returns \verb+.TRUE.+ when the invocation is a sizing one. In practice one
almost never needs to use it.

\subsubsection{Extensions for Checkpointing}

The pack-unpack subroutines written for migrations make sure that the current
state of the program is correctly packed (serialized) so that it can be
restarted on a different processor. Using the \emph{same} subroutines, it
is also possible to save the state of the program to disk, so that if the 
program were to crash abruptly, or if the allocated time for the program
expires before completing execution, the program can be restarted from the
previously checkpointed state. Thus, the pack-unpack subroutines act as the key facility for checkpointing in addition to their usual role for migration.

A subroutine, \texttt{MPI\_Checkpoint} has been added to AMPI. This subroutine
takes a directory name as its argument. Every chunk in the program needs to
call this subroutine and specify the same directory name. (Typically, in an
iterative AMPI program, the iteration number, converted to a character string,
can serve as a checkpoint directory name.) This directory is created, and the
entire state of the program is checkpointed to this directory.  One can restart
the program from the checkpointed state by specifying \texttt{"+restart
dirname"} on the command-line.

\subsection{Extensions for Interoperability}

Interoperability between different modules is essential for coding coupled
simulations.  In this extension to \ampi{}, each MPI application module runs
within its own group of user-level threads distributed over the physical
parallel machine.  In order to let \ampi{} know which chunks are to be created,
and in what order, a top level registration routine needs to be written. A
real-world example will make this clear. We have an MPI code for fluids and
another MPI code for solids, both with their main programs, then we first
transform each individual code to run correctly under \ampi{} as standalone
codes. This involves the usual ``chunkification'' transformation so that
multiple chunks from the application can run on the same processor without
overwriting each other's data. This also involves making the main program into
a subroutine and naming it \texttt{MPI\_Main}.

Thus now, we have two \texttt{MPI\_Main}s, one for the fluids code and one for
the solids code. We now make these codes co-exist within the same executable,
by first renaming these \texttt{MPI\_Main}s as \texttt{Fluids\_Main} and
\texttt{Solids\_Main}\footnote{Currently, we assume that the interface code,
which does mapping and interpolation among the boundary values of Fluids and
Solids domain, is integrated with one of Fluids and Solids.} writing a
subroutine called \texttt{MPI\_Setup}.

\begin{alltt}
SUBROUTINE MPI_Setup
  USE ampi

  CALL MPI_Register_main(Solids_Main)
  CALL MPI_Register_main(Fluids_Main)

END SUBROUTINE
\end{alltt}

This subroutine is called from the internal initialization routines of \ampi{}
and tells \ampi{} how many number of distinct chunk types (modules) exist, and
which orchestrator subroutines they execute.

The number of chunks to create for each chunk type is specified on the command
line when an \ampi{} program is run. Appendix B explains how \ampi{} programs
are run, and how to specify the number of chunks (\verb|+vp| option). In the
above case, suppose one wants to create 128 chunks of Solids and 64 chunks of
Fluids on 32 physical processors, one would specify those with multiple
\verb|+vp| options on the command line as:

\begin{alltt}
> charmrun gen1.x +p 32 +vp 128 +vp 64
\end{alltt}

This will ensure that multiple chunk types representing different complete
applications can co-exist within the same executable. They can also continue to
communicate among their own chunk-types using the same \ampi{} function calls
to send and receive with communicator argument as \texttt{MPI\_COMM\_WORLD}.
But this would be completely useless if these individual applications cannot
communicate with each other, which is essential for building efficient coupled
codes.  For this purpose, we have extended the \ampi{} functionality to allow
multiple ``\texttt{COMM\_WORLD}s''; one for each application. These \emph{world
communicators} form a ``communicator universe'': an array of communicators
aptly called \emph{MPI\_COMM\_UNIVERSE}. This array of communicators in
indexed $1\cdots\texttt{MPI\_MAX\_COMM}$. In the current implementation,
\texttt{MPI\_MAX\_COMM} is 8, that is, maximum of 8 applications can co-exist
within the same executable.

The order of these \texttt{COMM\_WORLD}s within \texttt{MPI\_COMM\_UNIVERSE}
is determined by the order in which individual applications are registered in
\texttt{MPI\_Setup}.

Thus, in the above example, the communicator for the Solids module would be
\texttt{MPI\_COMM\_UNIVERSE(1)} and communicator for Fluids module would be
\texttt{MPI\_COMM\_UNIVERSE(2)}.

Now any chunk within one application can communicate with any chunk in the
other application using the familiar send or receive \ampi{} calls by
specifying the appropriate communicator and the chunk number within that
communicator in the call. For example if a Solids chunk number 36 wants to send
data to chunk number 47 within the Fluids module, it calls:

\begin{alltt}
INTEGER , PARAMETER :: Fluids_Comm = 2
CALL MPI_Send(InitialTime, 1, MPI_Double_Precision, tag, &
        \emph{47, MPI_Comm_Universe(Fluids_Comm)}, ierr)
\end{alltt}

The Fluids chunk has to issue a corresponding receive call to receive this
data:

\begin{alltt}
INTEGER , PARAMETER :: Solids_Comm = 1
CALL MPI_Recv(InitialTime, 1, MPI_Double_Precision, tag, &
        \emph{36, MPI_Comm_Universe(Solids_Comm)}, stat, ierr)
\end{alltt}

\subsection{Compiling AMPI Programs}

\charmpp{} provides a cross-platform compile-and-link script called \charmc{}
to compile C, \CC{}, Fortran, \charmpp{} and \ampi{} programs.  This script
resides in the \texttt{bin} subdirectory in the \charmpp{} installation
directory. The main purpose of this script is to deal with the differences of
various compiler names and command-line options across various machines on
which \charmpp{} runs. While, \charmc{} handles C and \CC{} compiler
differences most of the time, the support for Fortran 90 is new, and may have
bugs. But \charmpp{} developers are aware of this problem and are working to
fix them. Even in its alpha stage of Fortran 90 support, \charmc{} still
handles many of the compiler differences across many machines, and it is
recommended that \charmc{} be used to compile and linking \ampi{} programs. One
major advantage of using \charmc{} is that one does not have to specify which
libraries are to be linked for ensuring that \CC{} and Fortran 90 codes are
linked correctly together. Appropriate libraries required for linking such
modules together are known to \charmc{} for various machines.

In spite of the platform-neutral syntax of \charmc{}, one may have to specify
some platform-specific options for compiling and building \ampi{} codes.
Fortunately, if \charmc{} does not recognize any particular options on its
command line, it promptly passes it to all the individual compilers and linkers
it invokes to compile the program.

Currently, \ampi{} is tested to run correctly on Origin 2000, Sun (Solaris 5.7)
workstation clusters, and Intel Linux clusters. Testing for IBM SP and Intel
Paragon (ASCI Red) is in progress.

\appendix

\section{Installing AMPI}

\ampi{} is included in the source distribution of \charmpp{}. \charmpp{} is
publicly distributed from the website \verb+http://charm.cs.uiuc.edu/+.  As of
today, the stable binary distribution of \charmpp{} available from the above
website is version 5.0. \ampi{} is a recent addition to \charmpp{}, and is NOT
included in the binary version. Even the new release of \charmpp{} in the near
future (version 5.4) is not expected to contain \ampi{} in its binary
distribution. There are several reasons for this. The most important being that
the threads implementation required for executing migratable threaded programs
(such as \ampi{} programs) is not the default threads implementation in
\charmpp{} (the default is non-migratable threads that are very lightweight.)
In order to install and use \ampi{}, one thus needs to download the publicly
available source code of \charmpp{} and install it oneself. The source
distributed in the \verb+src.tar.gz+ file on the download site will suffice.
However, since \ampi{} is under active development, and bug fixes,
optimizations and feature additions are being done frequently, it is your best
bet to use the source control system (CVS\footnote{See
\texttt{http://www.cvshome.org/} for more information on CVS.}) used by the
Parallel Programming Laboratory (PPL) to download the latest copy of \charmpp{}
source code.

To get the latest sources from PPL, issue the following commands from the shell
prompt (assuming C shell or tcsh):

\begin{alltt}
> setenv CVSROOT :pserver:checkout@thrift.cs.uiuc.edu:/expand6/cvsroot
> cvs login
(press ENTER when prompted for password)
> cvs co -P charm
\end{alltt}

This will connect to the source control server \verb+thrift.cs.uiuc.edu+ and
will download the latest version of \charmpp{} and \ampi{} into a directory
called \texttt{charm}. Now one has to build \charmpp{} and \ampi{} from source.

The build script for \charmpp{} is called \texttt{build}. The syntax for this
script is:

\begin{alltt}
> build <target> <version> <opts>
\end{alltt}

For building \ampi{} (which also includes building \charmpp{} and other
libraries needed by \ampi{}), specify \verb+<target>+ to be \verb+AMPI+. And
\verb+<opts>+ are command line options passed to the \verb+charmc+ compile
script.  Common compile time options such as \texttt{-g, -O, -Ipath, -Lpath,
-llib} are accepted. In order to build \ampi{}, one needs to specify the
following options:

\begin{alltt}
-O -DCMK_OPTIMIZE=1 -DCMK_THREADS_USE_ISOMALLOC=1
\end{alltt}

\verb+<version>+ depends on the machine, operating system, and the underlying
communication library one wants to use for running \ampi{} programs. Your
choice is determined by the following options.

\begin{enumerate}

\item The way a parallel program written in \ampi{} will communicate:

\begin{itemize}

\item ``net-'' \ampi{} communicates using the regular TCP/IP stack
(UDP packets), which works everywhere but is fairly slow.  Use this
option for networks of workstations, clusters, or single-machine 
development and testing.

\item ``mpi-'' Charm++ communicates using MPI calls.  Use this for
machines with a good MPI implementation (such as the Origin 2000).

\item ``exemplar'', ``ncube-2'', ``paragon-red'', ``sp3'', and ``t3e'' \ampi{} 
communicates using direct calls to the machine's communication primitives.

\item ``sim-'' and ``uth-'' are not actively maintained.  These are
single-processor versions: ``uth-'' simulates processors as user-level
threads; ``sim-'' switches between processors and counts communications.

\end{itemize}

\item  Your operating system and platform:

\begin{itemize}
\item ``linux''   Linux 
\item ``win32''   MS Windows NT/98/2k (and MS Visual C++ compiler)
\item ``cygwin''  MS Windows with Cygnus' Cygwin Unix layer
\item ``irix''    SGI IRIX
\item ``origin''  SGI Origin 2000 IRIX
\item ``sol''     Solaris
\item ``sun''     SunOS
\item ``rs6k''    IBM R/S 6000 A/IX 
\item ``sp''      IBM SP A/IX
\item ``hp''      Hewlett-Packard HP-UX
\item ``axp''     DEC Alpha DECUNIX
\end{itemize}

\item Your compiler and other options.  \ampi{} normally picks an
appropriate compiler for the system, but you may select another
compiler:

\begin{itemize}
\item ``-cc''      The OEM C/C++ compiler.  When given, this
will override the choice of the GNU C/C++ compiler.
\item ``-kcc''     Kuck \& Associates C++ compiler.
\item ``-acc''     Uses HP's aCC instead of CC.
\end{itemize}

Some operating systems have other options, such as:

\begin{itemize}
\item ``-x86''     For Solaris, use PC hardware (instead of Sun).
\item ``-axp''     For Linux, use Alpha hardware (instead of PC).
\item ``-64''      For IRIX, use -64 instead of -32. 
\end{itemize}

You may also choose to enable direct SMP support with a ``-smp''
version, which may result in more efficient communication in
a cluster-of-SMPs.  A ``-smp'' version will communicate using
shared memory within a machine; but message passing across machines.
``-smp'' is currently only available with ``net-'' versions.
Because of locking, ``-smp'' may slightly impact non-SMP performance.

\end{enumerate}

Your \ampi{} \verb+<version>+ is made by concatenating all three options, e.g.:

\begin{itemize}
\item ``net-linux''     \ampi{} for a network of Linux workstations, compiled
using g++.
\item ``net-linux-kcc'' \ampi{} for a network of Linux workstations, compiled
using Kuck \& Associates C++ compiler.
\item ``net-linux-smp'' \ampi{} for a network of Linux SMP workstations,
compiled using g++.
\item ``net-sol-cc''    \ampi{} for a network of Sun workstations, 
compiled using Sun CC.
\item ``mpi-origin''    \ampi{} for SGI Origin 2000, compiled using SGI CC.
\end{itemize}

\section{Running AMPI Programs}

\charmpp{} distribution contains a script called \texttt{charmrun} that makes
the job of running \ampi{} programs portable and easier across all parallel
machines supported by \charmpp{}. \texttt{charmrun} is copied to a directory
where an \ampi{} prgram is built using \charmc{}. It takes a command line
parameter specifying number of processors, and the name of the program followed
by \ampi{} options (such as number of chunks to create, and the stack size of
every chunk) and the program arguments. A typical invocation of \ampi{} program
\texttt{pgm} with \texttt{charmrun} is:

\begin{alltt}
> charmrun +p 16 pgm +vp 32 +stacksize 3276800
\end{alltt}

Here, the \ampi{} program \texttt{pgm} is run on 16 physical processors with
32 chunks (which will be mapped 2 per processor initially), where each
user-level thread associated with a chunk has the stack size of 3276800 bytes.

Running \ampi{} program over a network of workstations requires an additional
file that specifies the configuration of the network (such as which machines
are on the network, which shell to use for executing jobs etc.) The
\texttt{charmrun} command-line remains the same.  \texttt{charmrun} looks for
thhis file with name \texttt{nodelist} in the directory from where the job is
run. If it is not found in the current directory, then it tries to look for the
file with name \texttt{.nodelist} in user's home directory. The format of this
file allows you to define groups of machines, giving each group a name.  Each
line of the nodes file is a command.  The most important command is:

\begin{alltt}
host <hostname> <qualifiers>
\end{alltt}

which specifies a host.  The other commands are qualifiers: they modify
the properties of all hosts that follow them.  The qualifiers are:


\begin{tabbing}
{\tt group <groupname>}~~~\= - subsequent hosts are members of specified group\\
{\tt login <login>  }     \> - subsequent hosts use the specified login\\
{\tt shell <shell>  }     \> - subsequent hosts use the specified remote 
shell\\
{\tt setup <cmd>  }       \> - subsequent hosts should execute cmd\\
{\tt home <dir> }         \> - subsequent hosts should find programs under dir\\
{\tt cpus <n>}            \> - subsequent hosts should use N light-weight processes\\
{\tt speed <s>}           \> - subsequent hosts have relative speed rating\\
{\tt ext <extn>}          \> - subsequent hosts should append extn to the pgm name\\
\end{tabbing}

{\bf Note:}
By default, charmrun uses a remote shell ``rsh'' to spawn node processes
on the remote hosts. The {\tt shell} qualifier can be used to override
it with say, ``ssh''. One can set the {\tt CONV\_RSH} environment variable
or use charmrun option {\tt ++remote-shell} to override the default remote 
shell for all hosts with unspecified {\tt shell} qualifier.

All qualifiers accept ``*'' as an argument, this resets the modifier to
its default value.  Note that currently, the passwd, cpus, and speed
factors are ignored.  Inline qualifiers are also allowed:

\begin{alltt}
host copernicus.cse.uiuc.edu ++cpus 4 ++shell ssh
\end{alltt}

Except for ``group'', every other qualifier can be inlined, with the
restriction that if the ``setup'' qualifier is inlined, it should be
the last qualifier on the ``host'' or ``group'' statement line.

Here is a simple nodes file:

\begin{alltt}
        group kale-sun ++cpus 1
          host charm.cs.uiuc.edu ++shell ssh
          host dp.cs.uiuc.edu
          host grace.cs.uiuc.edu
          host dagger.cs.uiuc.edu
        group kale-sol
          host beauty.cs.uiuc.edu ++cpus 2
        group main
          host localhost
\end{alltt}

This defines three groups of machines: group kale-sun, group kale-sol,
and group main.  The ++nodegroup option is used to specify which group
of machines to use.  Note that there is wraparound: if you specify
more nodes than there are hosts in the group, it will reuse
hosts. Thus,

\begin{alltt}
        charmrun pgm ++nodegroup kale-sun +p6
\end{alltt}

uses hosts (charm, dp, grace, dagger, charm, dp) respectively as
nodes (0, 1, 2, 3, 4, 5).

If you don't specify a ++nodegroup, the default is ++nodegroup main.
Thus, if one specifies

\begin{alltt}
        charmrun pgm +p4
\end{alltt}

it will use ``localhost'' four times.  ``localhost'' is a Unix
trick; it always find a name for whatever machine you're on.

The user is required to set up remote login permissions on all nodes using the
``.rhosts'' file in the home directory if ``rsh'' is used for remote login into
the hosts. If ``ssh'' is used, the user will have to setup password-less login
to remote hosts either using ``.shosts'' file, or using RSA authentication
based on a key-pair and adding public keys to ``.ssh/authorized\_keys'' file.
See ``ssh'' documentation for more information.

\end{document}
