\subsection{\charmpp{} Groups and Nodegroups}

\subsubsection{What are groups and nodegroups used for?}

They are used for optimizations at the processor and node level respectively.

\subsubsection{Should I use groups?}

Probably not. People with an MPI background often overuse groups, which
results in MPI-like Charm++ programs. Arrays should generally be used
instead, because arrays can be migrated to achieve load balance.

Groups tend to be most useful in constructing communication optimization
libraries. For example, all the array elements on a processor can
contribute something to their local group, which can then send a combined
message to another processor. This can be much more efficient than
having each array element send a separate message.

\subsubsection{Is it safe to use a local pointer to a group, such as from ckLocalBranch?}

Yes. Groups never migrate, so a local pointer is safe. The only caveat
is to make sure {\em you} don't migrate without updating the pointer.

A local pointer can be used for very efficient access to data held by
a group.

%<li>
%<b>If a Group is constructed from </b><tt>main::main</tt><b>, the constructor
%gets called immediately on PE 0. Therefore, a Group constructor may create
%another Group, and get a valid group ID back. Correct?</b></li>

%<br>Yes. Groups may create other groups in their constructor.</ol>


\subsubsection{What are migratable groups?}

Migratable groups are declared so by adding the ``[migratable]'' attribute in
the .ci file. They {\em cannot} migrate from one processor to another during
normal execution, but only to disk for checkpointing purposes.

Migratable groups must declare a migration constructor (taking
{\tt CkMigrateMessage *} as a parameter) and a pup routine. The migration
constructor {\em must} call the superclass migration constructor as in this
example:
\begin{alltt}
class MyGroup : public CBase\_MyGroup \{
  ...
  MyGroup (CkMigrateMessage *msg) : CBase\_MyGroup(msg) \{ \}
  ...
\}
\end{alltt}

\subsubsection{Should I use nodegroups?}

Almost certainly not. You should use arrays for most computation, and
even quite low-level communication optimizations are often best handled
by groups. Nodegroups are very difficult to get right.

\subsubsection{What's the difference between groups and nodegroups?}

There's one group element per processor (CkNumPes() elements); and
one nodegroup element per node (CkNumNodes() elements). Because they
execute on a node, nodegroups have very different semantics from the rest
of Charm++.

Note that on a non-SMP machine, groups and nodegroups are identical.


\subsubsection{Do nodegroup entry methods execute on one fixed processor of the node,
or on the next available processor?}

Entries in node groups execute on the next available processor. Thus,
if two messages were sent to a branch of a nodegroup, two processors could
execute one each simultaneously.

\subsubsection{Are nodegroups single-threaded?}

No. They {\em can} be accessed by multiple threads at once.

\subsubsection{Do we have to worry about two entry methods in an object executing simultaneously?}

Yes, which makes nodegroups different from everything else in Charm++.

If a nodegroup method accesses a data structure in a non-threadsafe
way (such as writing to it), you need to lock it, for example using a CmiNodeLock.
